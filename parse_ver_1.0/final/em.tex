\documentclass[10pt]{article}
\usepackage{hyperref}
\usepackage{amsthm}
\usepackage{amsmath}
\usepackage{amssymb}
\usepackage{amsfonts}
\setlength\parindent{0pt}
\title{OUTPUT.TEX.}
\date{Updated: \today}
\author{}


\begin{document}


\textbf{Solution} :The charge density is given by Maxwell's equation

$$
\rho=\nabla \cdot \mathbf{D}=\varepsilon_{0} \nabla \cdot \mathbf{E} .
$$

As $\nabla \cdot u \mathbf{v}=\nabla u \cdot v+u \nabla \cdot \mathbf{v}$

$$
\nabla \cdot \mathbf{\Sigma}=A\left[\nabla\left(e^{-b r}\right) \cdot \frac{\mathbf{e}_{r}}{r^{2}}+e^{-b r} \nabla \cdot\left(\frac{e_{r}}{r^{2}}\right)\right] \text {. }
$$

Making use of Dirac's delta function $\delta(\mathbf{r})$ with properties

$$
\begin{aligned}
\delta(\mathbf{r}) &=0 \text { for } \mathbf{r} \neq 0 \\
&=\infty \text { for } r=0, \\
\int_{V} \delta(\mathbf{r}) d V &=1 \text { if } V \text { encloses } r=0, \\
&=0 \text { if otherwise, }
\end{aligned}
$$

we have

$$
\nabla^{2}\left(\frac{1}{r}\right)=\nabla \cdot \nabla\left(\frac{1}{r}\right)=\nabla \cdot\left(-\frac{e_{r}}{r^{2}}\right)=-4 \pi \delta(\mathbf{r}) .
$$

Thus

$$
\begin{aligned}
\rho &=\varepsilon_{0} A\left[-\frac{b e^{-b r}}{r^{2}} e_{r} \cdot e_{r}+4 \pi e^{-b r} \delta(r)\right] \\
&=-\frac{\varepsilon_{0} A b}{r^{2}} e^{-b r}+4 \pi \varepsilon_{0} A \delta(r) .
\end{aligned}
$$

Hence the charge distribution consists of a positive charge $4 \pi \varepsilon_{0} A$ at the origin and a spherically symmetric negative charge distribution in the surrounding space, as shown in Fig. $1.1$.

MATHPIX IMAGE

Fig. $1.1$
\textbf{Topic} :Electrostatics\\
\textbf{Book} :Problems and Solutions on Electromagnetism\\
\textbf{Final Answer} :-\frac{\varepsilon_{0} A b}{r^{2}} e^{-b r}+4 \pi \varepsilon_{0} A \delta(r)\\


\textbf{Solution} :The charge density is given by Maxwell's equation

$$
\rho=\nabla \cdot \mathbf{D}=\varepsilon_{0} \nabla \cdot \mathbf{E} .
$$

As $\nabla \cdot u \mathbf{v}=\nabla u \cdot v+u \nabla \cdot \mathbf{v}$

$$
\nabla \cdot \mathbf{\Sigma}=A\left[\nabla\left(e^{-b r}\right) \cdot \frac{\mathbf{e}_{r}}{r^{2}}+e^{-b r} \nabla \cdot\left(\frac{e_{r}}{r^{2}}\right)\right] \text {. }
$$

Making use of Dirac's delta function $\delta(\mathbf{r})$ with properties

$$
\begin{aligned}
\delta(\mathbf{r}) &=0 \text { for } \mathbf{r} \neq 0 \\
&=\infty \text { for } r=0, \\
\int_{V} \delta(\mathbf{r}) d V &=1 \text { if } V \text { encloses } r=0, \\
&=0 \text { if otherwise, }
\end{aligned}
$$

we have

$$
\nabla^{2}\left(\frac{1}{r}\right)=\nabla \cdot \nabla\left(\frac{1}{r}\right)=\nabla \cdot\left(-\frac{e_{r}}{r^{2}}\right)=-4 \pi \delta(\mathbf{r}) .
$$

Thus

$$
\begin{aligned}
\rho &=\varepsilon_{0} A\left[-\frac{b e^{-b r}}{r^{2}} e_{r} \cdot e_{r}+4 \pi e^{-b r} \delta(r)\right] \\
&=-\frac{\varepsilon_{0} A b}{r^{2}} e^{-b r}+4 \pi \varepsilon_{0} A \delta(r) .
\end{aligned}
$$

Hence the charge distribution consists of a positive charge $4 \pi \varepsilon_{0} A$ at the origin and a spherically symmetric negative charge distribution in the surrounding space, as shown in Fig. $1.1$.

MATHPIX IMAGE

Fig. $1.1$

 The total charge is

$$
\begin{aligned}
Q &=\int_{\text {all space }} \rho d V \\
&=-\int_{0}^{\infty} \frac{\varepsilon_{0} A b e^{-b r}}{r^{2}} \cdot 4 \pi r^{2} d r+\int_{\text {all space }} 4 \pi \varepsilon_{0} A \delta(\mathbf{r}) d V \\
&=4 \pi \varepsilon_{0} A\left[e^{-b r}\right]_{0}^{\infty}+4 \pi \varepsilon_{0} A \\
&=-4 \pi \varepsilon_{0} A+4 \pi \varepsilon_{0} A=0 .
\end{aligned}
$$

It can also be obtained from Gauss' flux theorem:

$$
\begin{aligned}
Q &=\lim _{r \rightarrow \infty} \oint_{S} \varepsilon_{0} \mathbf{E} \cdot d \mathbf{S} \\
&=\lim _{r \rightarrow \infty} \frac{\varepsilon_{0} A e^{-b r}}{r^{2}} \cdot 4 \pi r^{2} \\
&=\lim _{r \rightarrow \infty} 4 \pi \varepsilon_{0} A e^{-b r}=0,
\end{aligned}
$$

in agreement with the above. 

\textbf{Topic} :Electrostatics\\
\textbf{Book} :Problems and Solutions on Electromagnetism\\
\textbf{Final Answer} :0\\


\textbf{Solution} :The electric field surrounding the point charge $q$ is

$$
E(r)=\frac{q}{4 \pi \varepsilon_{0}} \frac{1}{r^{2}}(1-\sqrt{\alpha r}) \mathbf{e}_{r},
$$

where $r$ is the distance between a space point and the point charge $q$, and $\mathbf{e}_{\boldsymbol{r}}$ is a unit vector directed from $q$ to the space point.

MATHPIX IMAGE

Fig. $1.2$

 As in Fig. 1.2, for the closed path $L$ we find

$$
d l \cdot e_{r}=d l \cos \theta=d r
$$

and

$$
\begin{aligned}
\oint_{L} \mathbf{E} \cdot d \mathrm{l} &=\oint \frac{q}{4 \pi \varepsilon_{0}} \frac{1}{r^{2}}(1-\sqrt{\alpha r}) d r \\
&=\frac{q}{4 \pi \varepsilon_{0}}\left[-\oint_{L} d\left(\frac{1}{r}\right)+2 \sqrt{\alpha} \oint_{L} d\left(\frac{1}{\sqrt{r}}\right)\right]=0 .
\end{aligned}
$$

$$
\begin{aligned}
&\text { From Coulomb's law } \mathrm{F}_{12}=\frac{q 1 \varepsilon_{2}}{4 \pi \varepsilon_{0} r_{12}^{2}} \mathbf{e}_{r_{12}} \text {, we can obtain the electric field } \\
&\text { of the point charge } \\
&\qquad \mathrm{E}(r)=\frac{q}{4 \pi \varepsilon_{0} r^{2}} \mathbf{e}_{r} .
\end{aligned}
$$

Clearly, one has

$$
\oint_{L} \mathbb{L} \cdot d \mathbb{l}=0 \text {. }
$$

So the Coulomb result is the same as that of this problem.
\textbf{Topic} :Electrostatics\\
\textbf{Book} :Problems and Solutions on Electromagnetism\\
\textbf{Final Answer} :0\\


\textbf{Solution} :The electric field surrounding the point charge $q$ is

$$
E(r)=\frac{q}{4 \pi \varepsilon_{0}} \frac{1}{r^{2}}(1-\sqrt{\alpha r}) \mathbf{e}_{r},
$$

where $r$ is the distance between a space point and the point charge $q$, and $\mathbf{e}_{\boldsymbol{r}}$ is a unit vector directed from $q$ to the space point.

MATHPIX IMAGE

Fig. $1.2$

 As in Fig. 1.2, for the closed path $L$ we find

$$
d l \cdot e_{r}=d l \cos \theta=d r
$$

and

$$
\begin{aligned}
\oint_{L} \mathbf{E} \cdot d \mathrm{l} &=\oint \frac{q}{4 \pi \varepsilon_{0}} \frac{1}{r^{2}}(1-\sqrt{\alpha r}) d r \\
&=\frac{q}{4 \pi \varepsilon_{0}}\left[-\oint_{L} d\left(\frac{1}{r}\right)+2 \sqrt{\alpha} \oint_{L} d\left(\frac{1}{\sqrt{r}}\right)\right]=0 .
\end{aligned}
$$

$$
\begin{aligned}
&\text { From Coulomb's law } \mathrm{F}_{12}=\frac{q 1 \varepsilon_{2}}{4 \pi \varepsilon_{0} r_{12}^{2}} \mathbf{e}_{r_{12}} \text {, we can obtain the electric field } \\
&\text { of the point charge } \\
&\qquad \mathrm{E}(r)=\frac{q}{4 \pi \varepsilon_{0} r^{2}} \mathbf{e}_{r} .
\end{aligned}
$$

Clearly, one has

$$
\oint_{L} \mathbb{L} \cdot d \mathbb{l}=0 \text {. }
$$

So the Coulomb result is the same as that of this problem.

 Let $S$ be a spherical surface of radius $r_{1}$ with the charge $q$ at its center. Defining the surface element $d S=d S e_{r}$, we have

$$
\begin{aligned}
\oint_{8} \mathbf{E} \cdot d \mathbf{S} &=\oint_{8} \frac{q}{4 \pi \varepsilon_{0}} \frac{1}{r_{1}^{2}}\left(1-\sqrt{\alpha r_{1}}\right) d S \\
&=\frac{q}{\varepsilon_{0}}\left(1-\sqrt{\alpha r_{1}}\right)
\end{aligned}
$$

From Coulomb's law and Gauss' law, we get

$$
\oint_{s} \mathbf{E} \cdot d \mathbf{S}=\frac{q}{\varepsilon_{0}} .
$$

The two results differ by $\frac{1}{\varepsilon_{0}} \sqrt{\alpha r_{1}}$.
\textbf{Topic} :Electrostatics\\
\textbf{Book} :Problems and Solutions on Electromagnetism\\
\textbf{Final Answer} :\frac{q}{\varepsilon_{0}}\\


\textbf{Solution} :The electric field surrounding the point charge $q$ is

$$
E(r)=\frac{q}{4 \pi \varepsilon_{0}} \frac{1}{r^{2}}(1-\sqrt{\alpha r}) \mathbf{e}_{r},
$$

where $r$ is the distance between a space point and the point charge $q$, and $\mathbf{e}_{\boldsymbol{r}}$ is a unit vector directed from $q$ to the space point.

MATHPIX IMAGE

Fig. $1.2$

 As in Fig. 1.2, for the closed path $L$ we find

$$
d l \cdot e_{r}=d l \cos \theta=d r
$$

and

$$
\begin{aligned}
\oint_{L} \mathbf{E} \cdot d \mathrm{l} &=\oint \frac{q}{4 \pi \varepsilon_{0}} \frac{1}{r^{2}}(1-\sqrt{\alpha r}) d r \\
&=\frac{q}{4 \pi \varepsilon_{0}}\left[-\oint_{L} d\left(\frac{1}{r}\right)+2 \sqrt{\alpha} \oint_{L} d\left(\frac{1}{\sqrt{r}}\right)\right]=0 .
\end{aligned}
$$

$$
\begin{aligned}
&\text { From Coulomb's law } \mathrm{F}_{12}=\frac{q 1 \varepsilon_{2}}{4 \pi \varepsilon_{0} r_{12}^{2}} \mathbf{e}_{r_{12}} \text {, we can obtain the electric field } \\
&\text { of the point charge } \\
&\qquad \mathrm{E}(r)=\frac{q}{4 \pi \varepsilon_{0} r^{2}} \mathbf{e}_{r} .
\end{aligned}
$$

Clearly, one has

$$
\oint_{L} \mathbb{L} \cdot d \mathbb{l}=0 \text {. }
$$

So the Coulomb result is the same as that of this problem.

 Let $S$ be a spherical surface of radius $r_{1}$ with the charge $q$ at its center. Defining the surface element $d S=d S e_{r}$, we have

$$
\begin{aligned}
\oint_{8} \mathbf{E} \cdot d \mathbf{S} &=\oint_{8} \frac{q}{4 \pi \varepsilon_{0}} \frac{1}{r_{1}^{2}}\left(1-\sqrt{\alpha r_{1}}\right) d S \\
&=\frac{q}{\varepsilon_{0}}\left(1-\sqrt{\alpha r_{1}}\right)
\end{aligned}
$$

From Coulomb's law and Gauss' law, we get

$$
\oint_{s} \mathbf{E} \cdot d \mathbf{S}=\frac{q}{\varepsilon_{0}} .
$$

The two results differ by $\frac{1}{\varepsilon_{0}} \sqrt{\alpha r_{1}}$.

 Using the result of (c), the surface integral at $r_{1}+\Delta$ is

$$
\oint_{s} \mathbf{E} \cdot d \mathbf{S}=\frac{q}{\varepsilon_{0}}\left(1-\sqrt{\alpha\left(r_{1}+\Delta\right)}\right) \text {. }
$$

Consider a volume $V^{\prime}$ bounded by two spherical shells $S_{1}$ and $S_{2}$ with radii $r=r_{1}$ and $r=r_{1}+\Delta$ respectively. Gauss' divergence theorem gives

$$
\oint_{S_{1}+S_{2}} \mathbf{E} \cdot d S=\int_{V^{\prime}} \nabla \cdot \mathbb{\Xi} d V \text {. }
$$

As the directions of $d S$ on $S_{1}$ and $S_{2}$ are outwards from $V^{\prime}$, we have for small $\Delta$

$$
\frac{q}{\varepsilon_{0}}\left[-\sqrt{\alpha\left(r_{1}+\Delta\right)}+\sqrt{\alpha r_{1}}\right]=\left.\frac{4 \pi}{3}\left[\left(r_{1}+\Delta\right)^{3}-r_{1}^{3}\right](\Delta \cdot \mathbf{E})\right|_{r=r_{1}}
$$

As $\frac{\Delta}{r_{1}} \ll 1$, we can approximately set

$$
\left(1+\frac{\Delta}{r_{1}}\right)^{n} \approx 1+n \frac{\Delta}{r_{1}} .
$$

Thus one gets

$$
\nabla \cdot \mathrm{\Sigma}\left(r=r_{1}\right)=-\frac{\sqrt{\alpha} q}{8 \pi \varepsilon_{0} r_{1}^{5 / 2}} .
$$

On the other hand, Coulomb's law would give the divergence of the electric field produced by a point charge $q$ as

$$
\nabla \cdot \mathbf{\Sigma}(r)=\frac{q}{\varepsilon_{0}} \delta(r) .
$$


\textbf{Topic} :Electrostatics\\
\textbf{Book} :Problems and Solutions on Electromagnetism\\
\textbf{Final Answer} :\frac{q}{\varepsilon_{0}} \delta(r)\\


\textbf{Solution} :The electrostatic potential at a point on $x$-axis is

$$
\Phi(x)=\frac{1}{4 \pi \varepsilon_{0}} \int_{-a}^{a} \frac{\rho\left(x^{\prime}\right)}{\left|x-x^{\prime}\right|} d x^{\prime} .
$$

 For $x>a, a>x^{\prime}>-a$, we have

$$
\frac{1}{\left|x-x^{\prime}\right|}=\frac{1}{x}+\frac{x^{\prime}}{x^{2}}+\frac{x^{2}}{x^{3}}+\ldots
$$

Hence the multipole expansion of $\Phi(x)$ is

$$
\Phi(x)=\frac{1}{4 \pi \varepsilon_{0}}\left[\int_{-a}^{a} \frac{\rho\left(x^{\prime}\right)}{x} d x^{\prime}+\int_{-a}^{a} \frac{\rho\left(x^{\prime}\right) x^{\prime}}{x^{2}} d x^{\prime}+\int_{-a}^{a} \frac{\rho\left(x^{\prime}\right) x^{\prime 2}}{x^{3}} d x^{\prime}+\ldots\right] \text {. }
$$

 The charge configuration (I) can be represented by

$$
\rho\left(x^{\prime}\right)=q \delta\left(x^{\prime}\right),
$$

for which
\\
\textbf{Topic} :Electrostatics\\
\textbf{Book} :Problems and Solutions on Electromagnetism\\
\textbf{Final Answer} :q \delta\left(x^{\prime}\right)\\


\textbf{Solution} :First let us consider the infinite sheet of charge density $+\sigma$. The magnitude of the electric field caused by it at any space point is

$$
E=\frac{\sigma}{2 \varepsilon_{0}} .
$$

The direction of the electric field is perpendicular to the surface of the sheet. For the two orthogonal sheets of charge densities $\pm \sigma$, superposition of their electric fields yields

$$
E=\frac{\sqrt{2} \sigma}{2 \varepsilon_{0}} .
$$

The direction of $\mathbf{E}$ is as shown in Fig. 1.4.

MATHPIX IMAGE

Fig. $1.4$ 

\textbf{Topic} :Electrostatics\\
\textbf{Book} :Problems and Solutions on Electromagnetism\\
\textbf{Final Answer} :\frac{\sqrt{2} \sigma}{2 \varepsilon_{0}}\\


\textbf{Solution} :Take coordinate axes as in Fig. $1.6$ and consider a ring formed by circles with radii $\rho$ and $\rho+d \rho$ on the disc. The electrical potential at a point $(0,0, z)$ produced by the ring is given by

$$
d \varphi=\frac{1}{4 \pi \varepsilon_{0}} \cdot \frac{q}{\pi a^{2}} \cdot \frac{2 \pi \rho d \rho}{\sqrt{\rho^{2}+z^{2}}} .
$$

Integrating, we obtain the potential due to the whole ring:

$$
\begin{aligned}
\varphi(z) &=\int_{0}^{a} \frac{q}{2 \pi \varepsilon_{0} a^{2}} \cdot \frac{\rho d \rho}{\sqrt{\rho^{2}+z^{2}}} \\
&=\frac{q}{2 \pi \varepsilon_{0} a^{2}}\left(\sqrt{a^{2}+z^{2}}-|z|\right)
\end{aligned}
$$

MATHPIX IMAGE

Fig. 1.6
\textbf{Topic} :Electrostatics\\
\textbf{Book} :Problems and Solutions on Electromagnetism\\
\textbf{Final Answer} :\frac{q}{2 \pi \varepsilon_{0} a^{2}}\left(\sqrt{a^{2}+z^{2}}-|z|\right)\\


\textbf{Solution} :Take coordinate axes as in Fig. $1.6$ and consider a ring formed by circles with radii $\rho$ and $\rho+d \rho$ on the disc. The electrical potential at a point $(0,0, z)$ produced by the ring is given by

$$
d \varphi=\frac{1}{4 \pi \varepsilon_{0}} \cdot \frac{q}{\pi a^{2}} \cdot \frac{2 \pi \rho d \rho}{\sqrt{\rho^{2}+z^{2}}} .
$$

Integrating, we obtain the potential due to the whole ring:

$$
\begin{aligned}
\varphi(z) &=\int_{0}^{a} \frac{q}{2 \pi \varepsilon_{0} a^{2}} \cdot \frac{\rho d \rho}{\sqrt{\rho^{2}+z^{2}}} \\
&=\frac{q}{2 \pi \varepsilon_{0} a^{2}}\left(\sqrt{a^{2}+z^{2}}-|z|\right)
\end{aligned}
$$

MATHPIX IMAGE

Fig. 1.6

 At a point $|r|>a$, Laplace's equation $\nabla^{2} \varphi=0$ applies, with solution

$$
\varphi(r, \theta)=\sum_{n=0}^{\infty}\left(a_{n} r^{n}+\frac{b_{n}}{r^{n+1}}\right) P_{n}(\cos \theta) .
$$

As $\varphi \rightarrow 0$ for $r \rightarrow \infty$, we have $a_{n}=0$.

In the upper half-space, $z>0$, the potential on the axis is $\varphi=\varphi(r, 0)$. As $P_{n}(1)=1$, we have

$$
\varphi(r, 0)=\sum_{n=0}^{\infty} \frac{b_{n}}{r^{n+1}}
$$

In the lower half-space, $z<0$, the potential on the axis is $\varphi=\varphi(r, \pi)$. As $P_{n}(-1)=(-1)^{n}$, we have

$$
\varphi(r, \pi)=\sum_{n=0}^{\infty}(-1)^{n} \frac{b_{n}}{r^{n+1}} .
$$

Using the results of (a) and noting that for a point on the axis $|\mathbf{r}|=z$, we have for $z>0$

$$
\begin{aligned}
\sum_{n=0}^{\infty} \frac{b_{n}}{r^{n+1}} &=\frac{2 q}{4 \pi \varepsilon_{0} a^{2}}\left(\sqrt{a^{2}+r^{2}}-r\right) \\
&=\frac{q r}{2 \pi \varepsilon_{0} a^{2}}\left(\sqrt{1+\frac{a^{2}}{r^{2}}}-1\right) .
\end{aligned}
$$

However, as

$$
\begin{aligned}
\left(1+\frac{a^{2}}{r^{2}}\right)^{1 / 2}=& 1+\frac{1}{2}\left(\frac{a^{2}}{r^{2}}\right)+\frac{\frac{1}{2}\left(\frac{1}{2}-1\right)}{2 !}\left(\frac{a^{2}}{r^{2}}\right)^{2}+\ldots \\
&+\frac{\frac{1}{2}\left(\frac{1}{2}-1\right) \ldots \ldots\left(\frac{1}{2}-n+1\right)}{n !}\left(\frac{a^{2}}{r^{2}}\right)^{n}+\ldots, \cdot
\end{aligned}
$$

the equation becomes

$$
\sum_{n=0}^{\infty} \frac{b_{n}}{r^{n+1}}=\frac{q r}{2 \pi \varepsilon_{0} a^{2}} \sum_{n=1}^{\infty} \frac{\frac{1}{2}\left(\frac{1}{2}-1\right) \ldots\left(\frac{1}{2}-n+1\right)}{n !}\left(\frac{a^{2}}{r^{2}}\right)^{n} .
$$

Comparing the coefficients of powers of $r$ gives

$$
b_{2 n-1}=0, \quad b_{2 n-2}=\frac{q}{2 \pi \varepsilon_{0} a^{2}} \frac{\frac{1}{2}\left(\frac{1}{2}-1\right) \ldots\left(\frac{1}{2}-n+1\right)}{n !} a^{2 n} .
$$

Hence, the potential at any point $r$ of the half-plane $z>0$ is given by

$$
\begin{aligned}
\varphi(\mathbf{r})=& \frac{q}{2 \pi \varepsilon_{0} a} \sum_{n=1}^{\infty} \frac{\frac{1}{2}\left(\frac{1}{2}-1\right) \ldots\left(\frac{1}{2}-n+1\right)}{n !} \\
& \times\left(\frac{a}{r}\right)^{2 n-1} P_{2 n-2}(\cos \theta), \quad(z>0) .
\end{aligned}
$$

Similarly for the half-plane $z<0$, as $(-1)^{2 n-2}=1$ we have

$$
\begin{aligned}
\varphi(\mathbf{r})=& \frac{q}{2 \pi \varepsilon_{0} a} \sum_{n=1}^{\infty} \frac{\frac{1}{2}\left(\frac{1}{2}-1\right) \ldots\left(\frac{1}{2}-n+1\right)}{n !} \\
& \times\left(\frac{a}{r}\right)^{2 n-1} P_{2 n-2}(\cos \theta) \quad(z<0) .
\end{aligned}
$$

Thus the same expression for the potential applies to all points of space, which is a series in Legendre polynomials.

\textbf{Topic} :Electrostatics\\
\textbf{Book} :Problems and Solutions on Electromagnetism\\
\textbf{Final Answer} :1}^{\infty} \frac{\frac{1}{2}\left(\frac{1}{2}-1\right) \ldots\left(\frac{1}{2}-n+1\right)}{n !} \\
& \times\left(\frac{a}{r}\right)^{2 n-1} P_{2 n-2}(\cos \theta) \quad(z<0)\\


\textbf{Solution} :The electric potential $\phi_{c}$ at the center of the cube can be expressed as a linear function of the potentials of the six sides, i.e.,

$$
\phi_{\mathrm{c}}=\sum_{i} C_{i} \phi_{i},
$$

where the $C_{i}$ 's are constants. As the six sides of the cube are in the same relative geometrical position with respect to the center, the $C_{i}$ 's must have the same value, say $C$. Thus

$$
\phi_{\mathrm{c}}=C \sum_{i} \phi_{i} .
$$

If each of the six sides has potential $\phi_{0}$, the potential at the center will obviously be $\phi_{0}$ too. Hence $C=\frac{1}{6}$. Now as the potential of one side only is $\phi_{0}$ while all other sides have potential zero, the potential at the center is $\phi_{0} / 6$.

\textbf{Topic} :Electrostatics\\
\textbf{Book} :Problems and Solutions on Electromagnetism\\
\textbf{Final Answer} :C \sum_{i} \phi_{i}\\


\textbf{Solution} :The volume charge density of the sphere is

$$
\rho=\frac{Q}{\frac{4}{3} \pi R^{3}} .
$$

Take as the Gaussian surface a spherical surface of radius $r$ concentric with the charge sphere. By symmetry the magnitude of the electric field at all points of the surface is the same and the direction is radial. From Gauss' law

we immediately obtain

$$
\oint \mathbf{E} \cdot d \mathbf{S}=\frac{1}{\varepsilon_{0}} \int \rho d V
$$

$$
\begin{aligned}
& \mathbf{E}=\frac{Q \mathbf{r}}{4 \pi \varepsilon_{0} r^{3}} \quad(r \geq R), \\
& \mathbf{E}=\frac{Q \mathbf{r}}{4 \pi \varepsilon_{0} R^{3}} \quad(r \leq R) .
\end{aligned}
$$

\textbf{Topic} :Electrostatics\\
\textbf{Book} :Problems and Solutions on Electromagnetism\\
\textbf{Final Answer} :\frac{Q \mathbf{r}}{4 \pi \varepsilon_{0} R^{3}} \quad(r \leq R)\\


\textbf{Solution} :Electrostatic equilibrium requires that the total charge on inner surface of the conducting shell be $-q$. Using Gauss' law we then readily obtain

$$
\begin{array}{ll}
\mathbf{E}(r)=\frac{q}{4 \pi \varepsilon_{0} r^{2}} \mathbf{e}_{r} & \text { for } r<a, \\
\mathbf{E}=0 & \text { for } a<r<b, \\
\mathbf{E}(r)=\frac{1}{4 \pi \varepsilon_{0}} \frac{4 \pi b^{2} \sigma}{r^{2}} \mathbf{e}_{r}=\frac{\sigma b^{2}}{\varepsilon_{0} r^{2}} \mathbf{e}_{r} & \text { for } r>b .
\end{array}
$$

\textbf{Topic} :Electrostatics\\
\textbf{Book} :Problems and Solutions on Electromagnetism\\
\textbf{Final Answer} :\frac{\sigma b^{2}}{\varepsilon_{0} r^{2}} \mathbf{e}_{r} & \text { for } r>b 
\end{array}\\


\textbf{Solution} :Consider a concentric spherical surface of radius $r\left(r<R_{1}\right)$. Using Gauss' law we get

$$
\mathbf{E}=\frac{\boldsymbol{r}}{3} \frac{\rho}{\varepsilon_{0}} \mathbf{e}_{\boldsymbol{r}}
$$

As the shell is grounded, $\varphi\left(R_{1}\right)=0, E=0\left(r>R_{2}\right)$. Thus

$$
\varphi(r)=\int_{r}^{R_{1}} E d r=\frac{\rho}{6 \varepsilon_{0}}\left(R_{1}^{2}-r^{2}\right) .
$$

The potential at the center is

$$
\varphi(0)=\frac{1}{6 \varepsilon_{0}} \rho R_{1}^{2}
$$

The electrostatic energy is

$$
W=\int \frac{1}{2} \rho \varphi d V=\frac{1}{2} \int_{0}^{R_{1}} \frac{\rho}{6 \varepsilon_{0}}\left(R_{1}^{2}-r^{2}\right) \cdot \rho \cdot 4 \pi r^{2} d r=\frac{2 \rho^{2} R_{1}^{5}}{45 \varepsilon_{0}} .
$$

\textbf{Topic} :Electrostatics\\
\textbf{Book} :Problems and Solutions on Electromagnetism\\
\textbf{Final Answer} :\frac{2 \rho^{2} R_{1}^{5}}{45 \varepsilon_{0}}\\


\textbf{Solution} :Since the current is

$$
I=j \cdot S=\sigma E \cdot S=K E^{2} \cdot S=K E^{2} \cdot 4 \pi r^{2},
$$

the electric field is

$$
E=\frac{1}{r} \sqrt{\frac{I}{4 \pi K}}
$$

and the potential is

$$
V=-\int_{b}^{a} E \cdot d r=-\int_{b}^{a} \sqrt{\frac{I}{4 \pi K}} \frac{1}{r} d r=\sqrt{\frac{I}{4 \pi K}} \ln \left(\frac{b}{a}\right) .
$$

Hence the current between the spheres is given by

$$
I=4 \pi K V^{2} / \ln (b / a) \text {. }
$$



\textbf{Topic} :Electrostatics\\
\textbf{Book} :Problems and Solutions on Electromagnetism\\
\textbf{Final Answer} :4 \pi K V^{2} / \ln (b / a)\\


\textbf{Solution} :Consider an arbitrary point $P$ of the hollow region (see Fig. 1.9)

$$
O P=r, Q^{\prime} P=\mathbf{r}^{\prime}, O O^{\prime}=\mathbf{a}, \quad \mathbf{r}^{\prime}=\mathbf{r}-\mathbf{a} .
$$

MATHPIX IMAGE

Fig. $1.9$

If there were no hollow region inside the sphere, the electric field at the point $P$ would be

$$
\mathbf{\Sigma}_{1}=\frac{\rho}{3 \varepsilon_{0}} \mathbf{r} \text {. }
$$

If only the spherical hollow region has charge density $\rho$ the electric field at $P$ would be

$$
\boldsymbol{E}_{2}=\frac{\rho}{3 \varepsilon_{0}} \mathbf{r}^{\prime} .
$$

Hence the superposition theorem gives the electric field at $P$ as

$$
\mathbf{E}=\mathbf{E}_{1}-\mathbf{E}_{2}=\frac{\rho}{3 \varepsilon_{0}} \mathbf{a} .
$$

Thus the field inside the hollow region is uniform. This of course includes the center of the hollow.
\textbf{Topic} :Electrostatics\\
\textbf{Book} :Problems and Solutions on Electromagnetism\\
\textbf{Final Answer} :\frac{\rho}{3 \varepsilon_{0}} \mathbf{a}\\


\textbf{Solution} :Consider an arbitrary point $P$ of the hollow region (see Fig. 1.9)

$$
O P=r, Q^{\prime} P=\mathbf{r}^{\prime}, O O^{\prime}=\mathbf{a}, \quad \mathbf{r}^{\prime}=\mathbf{r}-\mathbf{a} .
$$

MATHPIX IMAGE

Fig. $1.9$

If there were no hollow region inside the sphere, the electric field at the point $P$ would be

$$
\mathbf{\Sigma}_{1}=\frac{\rho}{3 \varepsilon_{0}} \mathbf{r} \text {. }
$$

If only the spherical hollow region has charge density $\rho$ the electric field at $P$ would be

$$
\boldsymbol{E}_{2}=\frac{\rho}{3 \varepsilon_{0}} \mathbf{r}^{\prime} .
$$

Hence the superposition theorem gives the electric field at $P$ as

$$
\mathbf{E}=\mathbf{E}_{1}-\mathbf{E}_{2}=\frac{\rho}{3 \varepsilon_{0}} \mathbf{a} .
$$

Thus the field inside the hollow region is uniform. This of course includes the center of the hollow.

 Suppose the potential is taken to be zero at an infinite point. Consider an arbitrary sphere of radius $R$ with a uniform charge density $\rho$. We can find the electric fields inside and outside the sphere as

$$
\boldsymbol{L}(r)= \begin{cases}\frac{\rho r}{3 \epsilon_{0}}, & r<R, \\ \frac{\rho R^{3}}{3 \epsilon_{0} r^{3}} r, & r>R .\end{cases}
$$

Then the potential at an arbitrary point inside the sphere is

$$
\phi=\left(\int_{r}^{R}+\int_{R}^{\infty}\right) \mathrm{E} \cdot d \mathrm{r}=\frac{\rho}{6 \varepsilon_{0}}\left(3 R^{2}-r^{2}\right),
$$

where $r$ is the distance between this point and the spherical center.

Now consider the problem in hand. If the charges are distributed throughout the sphere of radius $R_{1}$, let $\phi_{1}$ be the potential at the center $\mathrm{O}^{\prime}$ of the hollow region. If the charge distribution is replaced by a small sphere of uniform charge density $\rho$ of radius $R_{2}$ in the hollow region, let the potential at $\mathrm{O}^{\prime}$ be $\phi_{2}$. Using (1) and the superposition theorem, we obtain

$$
\begin{aligned}
\phi_{O^{\prime}}=\phi_{1}-\phi_{2} &=\frac{\rho}{6 \varepsilon_{0}}\left(3 R_{1}^{2}-a^{2}\right)-\frac{\rho}{6 \varepsilon_{0}}\left(3 R_{2}^{2}-0\right) \\
&=\frac{\rho}{6 \varepsilon_{0}}\left[3\left(R_{1}^{2}-R_{2}^{2}\right)-a^{2}\right]
\end{aligned}
$$

\textbf{Topic} :Electrostatics\\
\textbf{Book} :Problems and Solutions on Electromagnetism\\
\textbf{Final Answer} :\frac{\rho}{6 \varepsilon_{0}}\left[3\left(R_{1}^{2}-R_{2}^{2}\right)-a^{2}\right]\\


\textbf{Solution} :By symmetry the electrostatic potential at point $P$ is only dependent on the $z$ coordinate. We choose cylindrical coordinates $(R, \theta, z)$ such that $P$ is on the $z$-axis. Then the potential at point $P$ is

$$
\phi_{P}=\frac{1}{4 \pi \varepsilon_{0}} \int \frac{\tau \cdot \mathbf{r}}{r^{3}} d S=\frac{1}{4 \pi \varepsilon_{0}} \int \frac{\tau z}{r^{3}} d S .
$$

As $r^{2}=R^{2}+z^{2}, d S=2 \pi R d R$, we get

$$
\phi_{P}=\frac{2 \pi \tau z}{4 \pi \varepsilon_{0}} \int_{0}^{\infty} \frac{R d R}{\sqrt{\left(R^{2}+z^{2}\right)^{3}}}=\left\{\begin{array}{cl}
\frac{\tau}{2 \varepsilon_{0}}, & z>0, \\
-\frac{\tau}{2 \varepsilon_{0}}, & z<0 .
\end{array}\right.
$$

Hence, the electrostaic potential is discontinous across the $x-y$ plane (for which $z=0$ ). The discontinuity is given by

$$
\Delta \phi=\frac{\tau}{2 \varepsilon_{0}}-\left(-\frac{\tau}{2 \varepsilon_{0}}\right)=\frac{\tau}{\varepsilon_{0}} .
$$
\textbf{Topic} :Electrostatics\\
\textbf{Book} :Problems and Solutions on Electromagnetism\\
\textbf{Final Answer} :\frac{\tau}{\varepsilon_{0}}\\


\textbf{Solution} :By symmetry the electrostatic potential at point $P$ is only dependent on the $z$ coordinate. We choose cylindrical coordinates $(R, \theta, z)$ such that $P$ is on the $z$-axis. Then the potential at point $P$ is

$$
\phi_{P}=\frac{1}{4 \pi \varepsilon_{0}} \int \frac{\tau \cdot \mathbf{r}}{r^{3}} d S=\frac{1}{4 \pi \varepsilon_{0}} \int \frac{\tau z}{r^{3}} d S .
$$

As $r^{2}=R^{2}+z^{2}, d S=2 \pi R d R$, we get

$$
\phi_{P}=\frac{2 \pi \tau z}{4 \pi \varepsilon_{0}} \int_{0}^{\infty} \frac{R d R}{\sqrt{\left(R^{2}+z^{2}\right)^{3}}}=\left\{\begin{array}{cl}
\frac{\tau}{2 \varepsilon_{0}}, & z>0, \\
-\frac{\tau}{2 \varepsilon_{0}}, & z<0 .
\end{array}\right.
$$

Hence, the electrostaic potential is discontinous across the $x-y$ plane (for which $z=0$ ). The discontinuity is given by

$$
\Delta \phi=\frac{\tau}{2 \varepsilon_{0}}-\left(-\frac{\tau}{2 \varepsilon_{0}}\right)=\frac{\tau}{\varepsilon_{0}} .
$$

 It is given that $\phi=0$ for $r>a$. Consequently $\mathbf{E}=0$ for $r>a$. Using Gauss' law

$$
\oint \mathbf{E} \cdot d \mathbf{S}=\frac{Q}{\varepsilon_{0}},
$$

we find that $\sigma \cdot 4 \pi a^{2}+q=0$. Thus

$$
\sigma=-\frac{q}{4 \pi a^{2}}
$$

If the potential at infinity is zero, then the potential outside the spherical surface will be zero everywhere. But the potential inside the sphere is $\varphi=\frac{q}{4 \pi \varepsilon_{0} r}$. For $r=a, \varphi=\frac{q}{4 \pi \varepsilon_{0} a}$, so that the discontinuity at the spherical surface is

$$
\Delta \phi=-\frac{q}{4 \pi \varepsilon_{0} a}
$$

We then have $\frac{\tau}{\varepsilon_{0}}=-\frac{q}{4 \pi \varepsilon_{0} a}$, giving

$$
\tau=-\frac{q}{4 \pi a} \mathbf{e}_{r}
$$




\textbf{Topic} :Electrostatics\\
\textbf{Book} :Problems and Solutions on Electromagnetism\\
\textbf{Final Answer} :-\frac{q}{4 \pi a} \mathbf{e}_{r}\\


\textbf{Solution} :Let $d_{1}$ be the distance between the two upper plates and $d_{2}$ be the distance between the two lower plates. From Fig. $1.10$ we see that

$$
\begin{gathered}
d_{1}+d_{2}=a-b, \\
C_{1}=\frac{\varepsilon_{0} A}{d_{1}}, C_{2}=\frac{\varepsilon_{0} A}{d_{2}}
\end{gathered}
$$

For the two capacitors in series, the total capacitance is

$$
C=\frac{C_{1} C_{2}}{C_{1}+C_{2}}=\frac{A \varepsilon_{0}}{d_{1}+d_{2}}=\frac{A \varepsilon_{0}}{a-b}
$$

As $C$ is independent of $d_{1}$ and $d_{2}$, the total capacitance is independent of the position of the center section. The total energy stored in the capacitor is

$$
W=\frac{1}{2} C V_{0}^{2}=\frac{A \varepsilon_{0} V_{0}^{2}}{2(a-b)} .
$$

The energy stored if the center section is removed is

$$
W^{\prime}=\frac{A \varepsilon_{0} V_{0}^{2}}{2 a},
$$

and we have

$$
W-W^{\prime}=\frac{A \varepsilon_{0} V_{0}^{2}}{2(a-b)} \frac{b}{a} .
$$

\textbf{Topic} :Electrostatics\\
\textbf{Book} :Problems and Solutions on Electromagnetism\\
\textbf{Final Answer} :\frac{A \varepsilon_{0} V_{0}^{2}}{2(a-b)} \frac{b}{a}\\


\textbf{Solution} :Choose $x$-axis perpendicular to the plates as shown in Fig. 1.11. Both the charge and current density are functions of $x$. In the steady state

$$
\frac{d \mathbf{j}(x)}{d x}=0 .
$$

MATHPIX IMAGE

Fig. 1.11

Hence $\mathbf{j}=-j_{0} \mathbf{e}_{x}$, where $j_{0}$ is a constant. Let $v(x)$ be the velocity of the electrons. Then the charge density is

$$
\rho(x)=-\frac{j_{0}}{v(x)}
$$

The potential satisfies the Poisson equation

$$
\frac{d^{2} V(x)}{d x^{2}}=-\frac{\rho(x)}{\varepsilon_{0}}=\frac{j_{0}}{\varepsilon_{0} v(x)} .
$$

Using the energy relation $\frac{1}{2} m v^{2}(x)=e V$, we get

$$
\frac{d^{2} V(x)}{d x^{2}}=\frac{j_{0}}{\varepsilon_{0}} \sqrt{\frac{m}{2 e V(x)}} .
$$

To solve this differential equation, let $u=\frac{d V}{d x}$. We then have

$$
\frac{d^{2} V}{d x^{2}}=\frac{d u}{d x}=\frac{d u}{d V} \frac{d V}{d x}=u \frac{d u}{d V},
$$

and this equation becomes

$$
u d u=A V^{-\frac{1}{2}} d V,
$$

where $A=\frac{j_{0}}{c_{0}} \sqrt{\frac{m}{2 e}}$. Note that $\frac{d V}{d x}=0$ at $x=0$, as the electrons are at rest there. Integrating the above gives

$$
\frac{1}{2} u^{2}=2 A V^{\frac{1}{2}},
$$

or

$$
V^{-\frac{4}{4}} d V=2 A^{\frac{1}{2}} d x .
$$

As $V=0$ for $x=0$ and $V=V_{0}$ for $x=d$, integrating the above leads to

$$
\frac{4}{3} V_{0}^{\frac{3}{4}}=2 A^{\frac{1}{2}} d=2\left(\frac{j_{0}}{\varepsilon_{0}} \sqrt{\frac{m}{2 e}}\right)^{\frac{1}{2}} d
$$

Finally we obtain the current density from the last equation:

$$
j=-j_{0} \mathbf{e}_{x}=-\frac{4 \varepsilon_{0} V_{0}}{9 d^{2}} \sqrt{\frac{2 e V_{0}}{m}} \mathbf{e}_{x}
$$


\textbf{Topic} :Electrostatics\\
\textbf{Book} :Problems and Solutions on Electromagnetism\\
\textbf{Final Answer} :-\frac{4 \varepsilon_{0} V_{0}}{9 d^{2}} \sqrt{\frac{2 e V_{0}}{m}} \mathbf{e}_{x}\\


\textbf{Solution} :We choose cylindrical coordinates with the $z$-axis along the axis of the cylinder and the origin at the center of the rod. Noting $l \gg d$ and using Gauss' theorem, we can find the electric field near the cylindrical surface and far from its ends as

$$
\mathbf{E}_{0}=\frac{\lambda}{\pi \varepsilon_{0} d} \mathbf{e}_{\rho},
$$

where $\lambda$ is the charge per unit length of the cylinder and $e_{\rho}$ is a unit vector in the radial direction. For $r \gg l$, we can regard the conducting rod as a point charge with $Q=\lambda l$. So the electric field intensity at a distant point on the axis is approximately

$$
E=\frac{Q}{4 \pi \varepsilon_{0} r^{2}}=\frac{E_{0} d l}{4 r^{2}} .
$$

The direction of $\mathbf{E}$ is along the axis away from the cylinder.

\textbf{Topic} :Electrostatics\\
\textbf{Book} :Problems and Solutions on Electromagnetism\\
\textbf{Final Answer} :\frac{E_{0} d l}{4 r^{2}}\\


\textbf{Solution} :Let the linear charge density for the inner conductor be $\lambda$. By symmetry we see that the field intensity at a point distance $r$ from the axis in the cable between the conductors is radial and its magnitude is given by Gauss' theorem as

$$
E=\frac{\lambda}{2 \pi \varepsilon_{0} r} .
$$

Then the potential difference between the inner and outer conductors is

$$
V=\int_{a}^{b} E d r=\frac{\lambda}{2 \pi \varepsilon_{0}} \ln (b / a)
$$

with $a=1.5 \mathrm{~cm}, b=0.5 \mathrm{~cm}$, which gives

$$
\begin{aligned}
\lambda &=\frac{2 \pi \varepsilon_{0} V}{\ln (b / a)}=\frac{2 \pi \times 8.9 \times 10^{-12} \times 8000}{\ln (1.5 / 0.5)} \\
&=4.05 \times 10^{-7} \mathrm{C} / \mathrm{m} .
\end{aligned}
$$
\textbf{Topic} :Electrostatics\\
\textbf{Book} :Problems and Solutions on Electromagnetism\\
\textbf{Final Answer} :405 \times 10^{-7} \mathrm{C} / \mathrm{m}\\


\textbf{Solution} :Let the linear charge density for the inner conductor be $\lambda$. By symmetry we see that the field intensity at a point distance $r$ from the axis in the cable between the conductors is radial and its magnitude is given by Gauss' theorem as

$$
E=\frac{\lambda}{2 \pi \varepsilon_{0} r} .
$$

Then the potential difference between the inner and outer conductors is

$$
V=\int_{a}^{b} E d r=\frac{\lambda}{2 \pi \varepsilon_{0}} \ln (b / a)
$$

with $a=1.5 \mathrm{~cm}, b=0.5 \mathrm{~cm}$, which gives

$$
\begin{aligned}
\lambda &=\frac{2 \pi \varepsilon_{0} V}{\ln (b / a)}=\frac{2 \pi \times 8.9 \times 10^{-12} \times 8000}{\ln (1.5 / 0.5)} \\
&=4.05 \times 10^{-7} \mathrm{C} / \mathrm{m} .
\end{aligned}
$$

 The point $r=1 \mathrm{~cm}$ is outside the cable. Gauss' law gives that its electric intensity is zero.

\textbf{Topic} :Electrostatics\\
\textbf{Book} :Problems and Solutions on Electromagnetism\\
\textbf{Final Answer} :405 \times 10^{-7} \mathrm{C} / \mathrm{m}\\


\textbf{Solution} :Let $E_{b}$ be the breakdown field intensity in air and let $R_{1}$ and $R_{2}$ be the radii of the inner and outer conductors respectively. Letting $\tau$ be the charge per unit length on each conductor and using Gauss' theorem, we obtain the electric field intensity in the capacitor and the potential difference between the two conductors respectively as

$$
\mathbf{E}_{r}=\frac{\tau}{2 \pi \varepsilon_{0} r} \mathbf{e}_{r}, \quad V=\int_{R_{1}}^{R_{2}} \frac{\tau}{2 \pi \varepsilon_{0} r} d r=\frac{\tau}{2 \pi \varepsilon_{0}} \ln \frac{R_{2}}{R_{1}} .
$$

As the electric field close to the surface of the inner conductor is strongest we have

$$
E_{b}=\frac{\tau}{2 \pi \varepsilon_{0} R_{1}} .
$$

Accordingly, we have

$$
\begin{aligned}
V_{b} &=E_{b} R_{1} \ln \frac{R_{2}}{R_{1}}, \\
\frac{d V_{b}}{d R_{1}} &=E_{b}\left[\ln \frac{R_{2}}{R_{1}}+R_{1} \frac{R_{1}}{R_{2}}\left(-\frac{R_{2}}{R_{1}^{2}}\right)\right]=E_{b}\left(\ln \frac{R_{2}}{R_{1}}-1\right) .
\end{aligned}
$$

In order to obtain the maximum potential difference, $R_{1}$ should be such that $\frac{d V_{b}}{d R_{1}}=0$, i.e., $\ln \frac{R_{2}}{R_{1}}=1$ or $R_{1}=\frac{R_{2}}{e}$. The maximum potential difference is then

$$
V_{\max }=\frac{R_{2}}{e} E_{b} .
$$

 The energy stored per unit length of the capacitor is

$$
W=\frac{1}{2} \tau V=\pi \varepsilon_{0} E_{b}^{2} R_{1}^{2} \ln \frac{R_{2}}{R_{1}}
$$

and

$$
\begin{aligned}
\frac{d W}{d R_{1}} &=\pi \varepsilon_{0} E_{b}^{2}\left[2 R_{1} \ln \frac{R_{2}}{R_{1}}+R_{1}^{2} \frac{R_{1}}{R_{2}}\left(-\frac{R_{2}}{R_{1}^{2}}\right)\right] \\
&=\pi \varepsilon_{0} E_{b}^{2} R_{1}\left(2 \ln \frac{R_{2}}{R_{1}}-1\right)
\end{aligned}
$$

For maximum energy storage, we require $\frac{d W}{d R_{1}}=0$, i.e., $2 \ln \frac{R_{2}}{R_{1}}=1$ or $R_{1}=\frac{R_{2}}{\sqrt{e}}$. In this case the potential difference is

$$
V=\frac{1}{2 \sqrt{e}} R_{2} E_{b} .
$$

 For (a),

$$
V_{\max }=\frac{R_{2}}{e} E_{b}=\frac{0.01}{e} \times 3 \times 10^{6}=1.1 \times 10^{4} \mathrm{~V}
$$

For (b),

$$
V_{\max }=\frac{1}{2 \sqrt{e}} R_{2} E_{b}=\frac{0.01 \times 3 \times 10^{6}}{2 \sqrt{e}}=9.2 \times 10^{3} \mathrm{~V}
$$


\textbf{Topic} :Electrostatics\\
\textbf{Book} :Problems and Solutions on Electromagnetism\\
\textbf{Final Answer} :92 \times 10^{3} \mathrm{~V}\\


\textbf{Solution} :Assume that the length of the cable is $2 l$ and that the inner and outer cylinders are connected at the end surface $z=-l$.
(The surface $z=l$ may be connected to a battery.) The outer cylindrical shell is an ideal conductor, whose potential is the same everywhere, taken to be zero. The inner cylinder has a current density $\mathbf{j}=\sigma \mathbf{E}$, i.e., $\mathbf{E}=\frac{\mathbf{j}}{\sigma}=\frac{j}{\sigma} \mathbf{e}_{z}$, so that its cross section $z=$ const. is an equipotential surface with potential

$$
V(z)=-\frac{j}{\sigma}(z+l) .
$$

In cylindrical coordinates the electric field intensity at a point $(r, \varphi, z)$ inside the cable can be expressed as

$$
\mathbf{\Sigma}(r, \varphi, z)=E_{r}(r, z) \mathbf{e}_{r}+E_{z}(r, z) \mathbf{e}_{z} .
$$

As the current does not change with $z, E_{z}(r, z)$ is independent of $z$ also. Take for the Gaussian surface a cylindrical surface of radius $r$ and length $d z$ with $z$-axis as the axis. We note that the electric fluxes through its two end surfaces have the same magnitude and direction so that their contributions cancel out. Gauss' law then becomes

$$
E_{r}(r, z) \cdot 2 \pi r d z=\lambda(z) d z / \varepsilon_{0},
$$

where $\lambda(z)$ is the charge per unit length of the inner cylinder, and gives

$$
E_{r}(r, z)=\frac{\lambda(z)}{2 \pi r \varepsilon_{0}} .
$$

Hence, we obtain the potential difference between the inner and outer conductors as

$$
V(z)=\int_{a}^{b} E_{r}(r, z) d r=\frac{\lambda(z)}{2 \pi \varepsilon_{0}} \ln \frac{b}{a} .
$$

As $V(z)=-\frac{j}{\sigma}(z+l)$, the above gives

$$
\lambda(z)=\frac{2 \pi \varepsilon_{0} V(z)}{\ln (b / a)}=-\frac{2 \pi \varepsilon_{0} j(z+l)}{\sigma \ln (b / a)} .
$$

The surface charge density at $z$ is then

$$
\sigma_{s}(z)=\frac{\lambda(z)}{2 \pi a}=-\frac{\varepsilon_{0} j(z+l)}{a \sigma \ln (b / a)} .
$$

Choosing the origin at the end surface with $z=-l$, we can write

$$
\sigma_{s}(z)=-\frac{\varepsilon_{0} j z}{a \sigma \ln (b / a)}
$$


\textbf{Topic} :Electrostatics\\
\textbf{Book} :Problems and Solutions on Electromagnetism\\
\textbf{Final Answer} :-\frac{\varepsilon_{0} j z}{a \sigma \ln (b / a)}\\


\textbf{Solution} :Use spherical coordinates $(r, \theta, \varphi)$. The electric field outside the

$$
\mathbf{E}(r)=\frac{Q}{4 \pi \varepsilon_{0} r^{2}} \mathbf{e}_{r} .
$$

Let the potential at infinity be zero, then the potential at $r$ is

$$
V(r)=\int_{r}^{\infty} \frac{Q}{4 \pi \varepsilon_{0} r^{\prime 2}} d r^{\prime}=\frac{Q}{4 \pi \varepsilon_{0}} \cdot \frac{1}{r} .
$$



Hence the capacitance is

$$
C=\frac{Q}{V(b)}=4 \pi \varepsilon_{0} b .
$$
\textbf{Topic} :Electrostatics\\
\textbf{Book} :Problems and Solutions on Electromagnetism\\
\textbf{Final Answer} :4 \pi \varepsilon_{0} b\\


\textbf{Solution} :Use spherical coordinates $(r, \theta, \varphi)$. The electric field outside the

$$
\mathbf{E}(r)=\frac{Q}{4 \pi \varepsilon_{0} r^{2}} \mathbf{e}_{r} .
$$

Let the potential at infinity be zero, then the potential at $r$ is

$$
V(r)=\int_{r}^{\infty} \frac{Q}{4 \pi \varepsilon_{0} r^{\prime 2}} d r^{\prime}=\frac{Q}{4 \pi \varepsilon_{0}} \cdot \frac{1}{r} .
$$



Hence the capacitance is

$$
C=\frac{Q}{V(b)}=4 \pi \varepsilon_{0} b .
$$

 $w_{e}(r)=\frac{1}{2} \mathbf{D} \cdot \mathbf{E}=\frac{1}{2} \varepsilon_{0} \mathbf{E}^{2}=\frac{Q^{2}}{32 \pi^{2} \varepsilon_{0} r^{4}}$.
\textbf{Topic} :Electrostatics\\
\textbf{Book} :Problems and Solutions on Electromagnetism\\
\textbf{Final Answer} :4 \pi \varepsilon_{0} b\\


\textbf{Solution} :Use spherical coordinates $(r, \theta, \varphi)$. The electric field outside the

$$
\mathbf{E}(r)=\frac{Q}{4 \pi \varepsilon_{0} r^{2}} \mathbf{e}_{r} .
$$

Let the potential at infinity be zero, then the potential at $r$ is

$$
V(r)=\int_{r}^{\infty} \frac{Q}{4 \pi \varepsilon_{0} r^{\prime 2}} d r^{\prime}=\frac{Q}{4 \pi \varepsilon_{0}} \cdot \frac{1}{r} .
$$



Hence the capacitance is

$$
C=\frac{Q}{V(b)}=4 \pi \varepsilon_{0} b .
$$

 $w_{e}(r)=\frac{1}{2} \mathbf{D} \cdot \mathbf{E}=\frac{1}{2} \varepsilon_{0} \mathbf{E}^{2}=\frac{Q^{2}}{32 \pi^{2} \varepsilon_{0} r^{4}}$.

 $W_{e}=\frac{1}{2} V(b) Q=\frac{Q^{2}}{8 \pi \varepsilon_{0} b}$.

It may also be calculated from the field energy density $w_{e}(r)$ :

$$
W_{e}=\int_{r>b} w_{e}\left(r^{\prime}\right) d V^{\prime}=\int_{b}^{\infty} \frac{Q^{2}}{2 \times 16 \pi^{2} \varepsilon_{0}} \frac{1}{r^{\prime 4}} \cdot 4 \pi r^{\prime 2} d r^{\prime}=\frac{Q^{2}}{8 \pi \varepsilon_{0} b} .
$$
\textbf{Topic} :Electrostatics\\
\textbf{Book} :Problems and Solutions on Electromagnetism\\
\textbf{Final Answer} :\frac{Q^{2}}{8 \pi \varepsilon_{0} b}\\


\textbf{Solution} :Use spherical coordinates $(r, \theta, \varphi)$. The electric field outside the

$$
\mathbf{E}(r)=\frac{Q}{4 \pi \varepsilon_{0} r^{2}} \mathbf{e}_{r} .
$$

Let the potential at infinity be zero, then the potential at $r$ is

$$
V(r)=\int_{r}^{\infty} \frac{Q}{4 \pi \varepsilon_{0} r^{\prime 2}} d r^{\prime}=\frac{Q}{4 \pi \varepsilon_{0}} \cdot \frac{1}{r} .
$$



Hence the capacitance is

$$
C=\frac{Q}{V(b)}=4 \pi \varepsilon_{0} b .
$$

 $w_{e}(r)=\frac{1}{2} \mathbf{D} \cdot \mathbf{E}=\frac{1}{2} \varepsilon_{0} \mathbf{E}^{2}=\frac{Q^{2}}{32 \pi^{2} \varepsilon_{0} r^{4}}$.

 $W_{e}=\frac{1}{2} V(b) Q=\frac{Q^{2}}{8 \pi \varepsilon_{0} b}$.

It may also be calculated from the field energy density $w_{e}(r)$ :

$$
W_{e}=\int_{r>b} w_{e}\left(r^{\prime}\right) d V^{\prime}=\int_{b}^{\infty} \frac{Q^{2}}{2 \times 16 \pi^{2} \varepsilon_{0}} \frac{1}{r^{\prime 4}} \cdot 4 \pi r^{\prime 2} d r^{\prime}=\frac{Q^{2}}{8 \pi \varepsilon_{0} b} .
$$

 The work expended in charging the sphere by carrying infinitesimal charges in from infinity is

$$
W=\int_{0}^{Q} V\left(Q^{\prime}\right) d Q^{\prime}=\int_{0}^{Q} \frac{Q^{\prime}}{4 \pi \varepsilon_{0}} \frac{d Q^{\prime}}{b}=\frac{Q^{2}}{8 \pi \varepsilon_{0} b}=W_{e}
$$

as expected.
\textbf{Topic} :Electrostatics\\
\textbf{Book} :Problems and Solutions on Electromagnetism\\
\textbf{Final Answer} :W_{e}\\


\textbf{Solution} :Use spherical coordinates $(r, \theta, \varphi)$. The electric field outside the

$$
\mathbf{E}(r)=\frac{Q}{4 \pi \varepsilon_{0} r^{2}} \mathbf{e}_{r} .
$$

Let the potential at infinity be zero, then the potential at $r$ is

$$
V(r)=\int_{r}^{\infty} \frac{Q}{4 \pi \varepsilon_{0} r^{\prime 2}} d r^{\prime}=\frac{Q}{4 \pi \varepsilon_{0}} \cdot \frac{1}{r} .
$$



Hence the capacitance is

$$
C=\frac{Q}{V(b)}=4 \pi \varepsilon_{0} b .
$$

 $w_{e}(r)=\frac{1}{2} \mathbf{D} \cdot \mathbf{E}=\frac{1}{2} \varepsilon_{0} \mathbf{E}^{2}=\frac{Q^{2}}{32 \pi^{2} \varepsilon_{0} r^{4}}$.

 $W_{e}=\frac{1}{2} V(b) Q=\frac{Q^{2}}{8 \pi \varepsilon_{0} b}$.

It may also be calculated from the field energy density $w_{e}(r)$ :

$$
W_{e}=\int_{r>b} w_{e}\left(r^{\prime}\right) d V^{\prime}=\int_{b}^{\infty} \frac{Q^{2}}{2 \times 16 \pi^{2} \varepsilon_{0}} \frac{1}{r^{\prime 4}} \cdot 4 \pi r^{\prime 2} d r^{\prime}=\frac{Q^{2}}{8 \pi \varepsilon_{0} b} .
$$

 The work expended in charging the sphere by carrying infinitesimal charges in from infinity is

$$
W=\int_{0}^{Q} V\left(Q^{\prime}\right) d Q^{\prime}=\int_{0}^{Q} \frac{Q^{\prime}}{4 \pi \varepsilon_{0}} \frac{d Q^{\prime}}{b}=\frac{Q^{2}}{8 \pi \varepsilon_{0} b}=W_{e}
$$

as expected.

 Suppose that the inner sphere carries a charge $Q$. For $a<r<b$ the field intensity is

$$
\mathbf{E}(r)=\frac{Q}{4 \pi \varepsilon_{0} r^{2}} \mathbf{e}_{r} .
$$

The potential difference between the concentric spheres is

$$
V=\int_{a}^{b} \mathrm{E}(r) \cdot d \mathbf{r}=\int_{a}^{b} \frac{Q}{4 \pi \varepsilon_{0}} \frac{1}{r^{2}} d r=\frac{Q}{4 \pi \varepsilon_{0}}\left(\frac{1}{a}-\frac{1}{b}\right)
$$

In terms of $V$ we have

$$
Q=\frac{4 \pi \varepsilon_{0} V}{\left(\frac{1}{a}-\frac{1}{b}\right)}
$$

and

$$
E(r)=\frac{4 \pi \varepsilon_{0} V}{4 \pi \varepsilon_{0} r^{2}\left(\frac{1}{a}-\frac{1}{b}\right)}=\frac{V}{r^{2}\left(\frac{1}{a}-\frac{1}{b}\right)} .
$$

In particular, we have

$$
E(a)=\frac{V}{a^{2}\left(\frac{1}{a}-\frac{1}{b}\right)}=\frac{V b}{a b-a^{2}} .
$$

From $\frac{d E(a)}{d a}=0$, we see that $E(a)$ is a minimum at $a=\frac{b}{2}$, and the minimum value is

$$
E_{\min }(a)=\frac{4 V}{b} .
$$

\textbf{Topic} :Electrostatics\\
\textbf{Book} :Problems and Solutions on Electromagnetism\\
\textbf{Final Answer} :\frac{4 V}{b}\\


\textbf{Solution} :Suppose that the charge of the inner spherical shell is $Q_{1}$ and the charge of the outer shell is $-Q_{2}$. Then the charges on the inner and outer surfaces of the middle spherical shell are $-Q_{1}$ and $+Q_{2}\left(Q_{1}, Q_{2}>0\right)$ as shown in Fig. 1.16. The electric field intensities are as follows:

$$
\begin{aligned}
&\mathbf{E}=\frac{Q_{1} \mathbf{r}}{4 \pi \varepsilon_{0} r^{3}}, \quad(a<r<b), \\
&\mathbf{E}=\frac{Q_{2} \mathbf{r}}{4 \pi \varepsilon_{0} r^{3}}, \quad(b<r<d), \\
&\mathbf{E}=0, \quad(r<a, r>d) .
\end{aligned}
$$

The potential at a point $P$ is given by

$$
\varphi(p)=\int_{P}^{\infty} \mathbf{E} \cdot d \mathbf{r}
$$

MATHPIX IMAGE

Fig. $1.16$

with $\varphi(\infty)=0$. Thus we have

$$
\varphi(d)=0, \quad \varphi(b)=\frac{Q_{2}}{4 \pi \varepsilon_{0}}\left(\frac{1}{b}-\frac{1}{d}\right)
$$

As the inner and outer spherical shells are connected their potentials should be equal. Hence

$$
\varphi(a)=\frac{Q_{1}}{4 \pi \varepsilon_{0}}\left(\frac{1}{a}-\frac{1}{b}\right)+\frac{Q_{2}}{4 \pi \varepsilon_{0}}\left(\frac{1}{b}-\frac{1}{d}\right)=0,
$$

whence

$$
Q_{1}\left(\frac{1}{a}-\frac{1}{b}\right)=-Q_{2}\left(\frac{1}{b}-\frac{1}{d}\right) \text {. }
$$

The potential differences of the spherical shells are

$$
\begin{aligned}
&V_{a b}=\varphi(a)-\varphi(b)=-\varphi(b), \\
&V_{d b}=\varphi(d)-\varphi(b)=-\varphi(b) .
\end{aligned}
$$

Thus the capacitance between the inner sphere and the inner surface of the middle spherical shell is

$$
C_{a b}=\frac{Q_{1}}{V_{a b}}=-\frac{Q_{1}}{\varphi(b)},
$$

and the capacitance between the outer surface of the middle shell and the outer shell is

$$
C_{b d}=\frac{Q_{2}}{V_{b d}}=\frac{Q_{2}}{\varphi(b)} .
$$

The capacitance of the whole system can be considered as $C_{a b}$ and $C_{b d}$ in series, namely

$$
C=\left(\frac{1}{C_{a b}}+\frac{1}{C_{b d}}\right)^{-1}=\frac{1}{\varphi(b)}\left(\frac{1}{Q_{2}}-\frac{1}{Q_{1}}\right)^{-1}=\frac{4 \pi \varepsilon_{0} a d}{d-a} .
$$
\textbf{Topic} :Electrostatics\\
\textbf{Book} :Problems and Solutions on Electromagnetism\\
\textbf{Final Answer} :\frac{4 \pi \varepsilon_{0} a d}{d-a}\\


\textbf{Solution} :Suppose that the charge of the inner spherical shell is $Q_{1}$ and the charge of the outer shell is $-Q_{2}$. Then the charges on the inner and outer surfaces of the middle spherical shell are $-Q_{1}$ and $+Q_{2}\left(Q_{1}, Q_{2}>0\right)$ as shown in Fig. 1.16. The electric field intensities are as follows:

$$
\begin{aligned}
&\mathbf{E}=\frac{Q_{1} \mathbf{r}}{4 \pi \varepsilon_{0} r^{3}}, \quad(a<r<b), \\
&\mathbf{E}=\frac{Q_{2} \mathbf{r}}{4 \pi \varepsilon_{0} r^{3}}, \quad(b<r<d), \\
&\mathbf{E}=0, \quad(r<a, r>d) .
\end{aligned}
$$

The potential at a point $P$ is given by

$$
\varphi(p)=\int_{P}^{\infty} \mathbf{E} \cdot d \mathbf{r}
$$

MATHPIX IMAGE

Fig. $1.16$

with $\varphi(\infty)=0$. Thus we have

$$
\varphi(d)=0, \quad \varphi(b)=\frac{Q_{2}}{4 \pi \varepsilon_{0}}\left(\frac{1}{b}-\frac{1}{d}\right)
$$

As the inner and outer spherical shells are connected their potentials should be equal. Hence

$$
\varphi(a)=\frac{Q_{1}}{4 \pi \varepsilon_{0}}\left(\frac{1}{a}-\frac{1}{b}\right)+\frac{Q_{2}}{4 \pi \varepsilon_{0}}\left(\frac{1}{b}-\frac{1}{d}\right)=0,
$$

whence

$$
Q_{1}\left(\frac{1}{a}-\frac{1}{b}\right)=-Q_{2}\left(\frac{1}{b}-\frac{1}{d}\right) \text {. }
$$

The potential differences of the spherical shells are

$$
\begin{aligned}
&V_{a b}=\varphi(a)-\varphi(b)=-\varphi(b), \\
&V_{d b}=\varphi(d)-\varphi(b)=-\varphi(b) .
\end{aligned}
$$

Thus the capacitance between the inner sphere and the inner surface of the middle spherical shell is

$$
C_{a b}=\frac{Q_{1}}{V_{a b}}=-\frac{Q_{1}}{\varphi(b)},
$$

and the capacitance between the outer surface of the middle shell and the outer shell is

$$
C_{b d}=\frac{Q_{2}}{V_{b d}}=\frac{Q_{2}}{\varphi(b)} .
$$

The capacitance of the whole system can be considered as $C_{a b}$ and $C_{b d}$ in series, namely

$$
C=\left(\frac{1}{C_{a b}}+\frac{1}{C_{b d}}\right)^{-1}=\frac{1}{\varphi(b)}\left(\frac{1}{Q_{2}}-\frac{1}{Q_{1}}\right)^{-1}=\frac{4 \pi \varepsilon_{0} a d}{d-a} .
$$

 The net charge $Q_{B}$ carried by the middle shell must be equal to $Q_{2}-Q_{1}$, so that

$$
Q_{1}=-\frac{a(d-b)}{b(d-a)} Q_{B}, \quad Q_{2}=\frac{d(b-a)}{b(d-a)} Q_{B}
$$

This is to say, the inner surface of the middle shell will carry a total charge $\frac{a(d-b)}{b(d-a)} Q_{B}$ while the outer surface, $\frac{d(b-a)}{b(d-a)} Q_{B}$.

\textbf{Topic} :Electrostatics\\
\textbf{Book} :Problems and Solutions on Electromagnetism\\
\textbf{Final Answer} :\frac{d(b-a)}{b(d-a)} Q_{B}\\


\textbf{Solution} :z-plane

MATHPIX IMAGE
\textbf{Topic} :Electrostatics\\
\textbf{Book} :Problems and Solutions on Electromagnetism\\
\textbf{Final Answer} :\frac{d(b-a)}{b(d-a)} Q_{B}\\


\textbf{Solution} :z-plane

MATHPIX IMAGE

 For $r \gg a$, we have

$$
V \approx \frac{V_{0}}{\pi}\left[\frac{\pi}{2}+\frac{2 a \sin \theta}{r}\right]=\frac{V_{0}}{2}+\frac{2 V_{0} a \sin \theta}{\pi r},
$$

and hence

$$
\begin{aligned}
&E_{r}=-\frac{\partial V}{\partial r}=\frac{2 V_{0} a \sin \theta}{\pi r^{2}}, \\
&E_{\theta}=-\frac{1}{r} \frac{\partial V}{\partial \theta}=-\frac{2 V a}{\pi r^{2}} \cos \theta .
\end{aligned}
$$

MATHPIX IMAGE

Fig. $1.18$
\textbf{Topic} :Electrostatics\\
\textbf{Book} :Problems and Solutions on Electromagnetism\\
\textbf{Final Answer} :-\frac{2 V a}{\pi r^{2}} \cos \theta\\


\textbf{Solution} :z-plane

MATHPIX IMAGE

 For $r \gg a$, we have

$$
V \approx \frac{V_{0}}{\pi}\left[\frac{\pi}{2}+\frac{2 a \sin \theta}{r}\right]=\frac{V_{0}}{2}+\frac{2 V_{0} a \sin \theta}{\pi r},
$$

and hence

$$
\begin{aligned}
&E_{r}=-\frac{\partial V}{\partial r}=\frac{2 V_{0} a \sin \theta}{\pi r^{2}}, \\
&E_{\theta}=-\frac{1}{r} \frac{\partial V}{\partial \theta}=-\frac{2 V a}{\pi r^{2}} \cos \theta .
\end{aligned}
$$

MATHPIX IMAGE

Fig. $1.18$

 For $r \ll a$, we have

$$
V \approx \frac{V_{0}}{\pi}\left[\frac{\pi}{2}+\frac{2 r \sin \theta}{a}\right]=\frac{V_{0}}{2}+\frac{2 V_{0} r \sin \theta}{\pi a},
$$

and hence

$$
\begin{aligned}
&E_{r}=-\frac{\partial V}{\partial r}=-\frac{2 V_{0} \sin \theta}{\pi a}, \\
&E_{\theta}=-\frac{1}{r} \frac{\partial V}{\partial \theta}=-\frac{2 V_{0}}{\pi a} \cos \theta .
\end{aligned}
$$
\textbf{Topic} :Electrostatics\\
\textbf{Book} :Problems and Solutions on Electromagnetism\\
\textbf{Final Answer} :-\frac{2 V_{0}}{\pi a} \cos \theta\\


\textbf{Solution} :Letting $V$ be the voltage difference between the inner and outer conductors, we can express the electric field intensity between the two conductors as

$$
\mathrm{E}(r)=\frac{V}{r \ln \left(\frac{b}{a}\right)} \mathbf{e}_{r} .
$$

Ohm's law $J=\sigma \mathbb{E}$ then gives the current between the two conductors as

$$
I=2 \pi r l J=\frac{2 \pi \sigma l V}{\ln \left(\frac{b}{a}\right)} .
$$

The resistance between the inner and outer conductors is thus

$$
R=\frac{V}{I}=\frac{\ln \left(\frac{b}{a}\right)}{2 \pi l \sigma} .
$$

Since the field is zero inside a conductor, we find the surface charge density $\omega$ of the inner conductor from the boundary relation $E=\frac{\omega}{c}$, i.e.,

$$
\omega=\varepsilon \frac{V}{a \ln \left(\frac{b}{a}\right)} \text {. }
$$

Thus the inner conductor carries a total charge $Q=2 \pi a l \omega$. Hence the capacitance between the two conductors is

$$
C=\frac{Q}{V}=\frac{2 \pi \varepsilon l}{\ln \left(\frac{b}{a}\right)} .
$$

\textbf{Topic} :Electrostatics\\
\textbf{Book} :Problems and Solutions on Electromagnetism\\
\textbf{Final Answer} :\frac{2 \pi \varepsilon l}{\ln \left(\frac{b}{a}\right)}\\


\textbf{Solution} :Suppose that the two conductors carry free charges $Q$ and $-Q$. Consider a closed surface enclosing the conductor with the charge $Q$ (but not the other conductor). We have, using Ohm's and Gauss' laws,

$$
I=\oint \mathbf{j} \cdot d \mathbf{S}=\oint \sigma \mathbf{E} \cdot d \mathbf{S}=\sigma \oint \mathbf{E} \cdot d \mathbf{S}=\sigma \frac{Q}{\varepsilon} .
$$

If the potential difference between the two conductors is $V$, we have $V=$ $I R=\frac{\sigma Q}{\varepsilon} R$, whence

$$
C=\frac{Q}{V}=\frac{\varepsilon}{\sigma R} .
$$

Numerically the capacitance between the conductors is

$$
C=\frac{80 \times 8.85 \times 10^{-12}}{10^{-4} \times 10^{5}}=7.08 \times 10^{-11} \mathrm{~F} .
$$


\textbf{Topic} :Electrostatics\\
\textbf{Book} :Problems and Solutions on Electromagnetism\\
\textbf{Final Answer} :708 \times 10^{-11} \mathrm{~F}\\


\textbf{Solution} :Let $\lambda$ be the charge per unit length carried by the inner conductor. Gauss' law gives

$$
D(r)=\frac{\lambda}{2 \pi r},
$$

as $D$ is along the radial direction on account of symmetry.

The energy density at $r$ is

$$
U(r)=\frac{1}{2} E D=\frac{D^{2}}{2 \varepsilon_{0} K(r)}=\frac{\lambda^{2}}{8 \pi^{2} \varepsilon_{0} r^{2} K(r)} .
$$

If this is to be independent of $r$, we require $r^{2} K(r)=$ constant $=k$, say, i.e., $K(r)=k r^{-2}$. The voltage across the two conductors is

$$
\begin{aligned}
V &=-\int_{a}^{b} E d r=-\frac{\lambda}{2 \pi \varepsilon_{0} k} \int_{a}^{b} r d r \\
&=-\frac{\lambda}{4 \pi \varepsilon_{0} k}\left(b^{2}-a^{2}\right)
\end{aligned}
$$

Hence

giving

$$
\lambda=-\frac{4 \pi \varepsilon_{0} k V}{b^{2}-a^{2}},
$$

$$
E(r)=-\frac{2 r V}{b^{2}-a^{2}} .
$$

\textbf{Topic} :Electrostatics\\
\textbf{Book} :Problems and Solutions on Electromagnetism\\
\textbf{Final Answer} :-\frac{2 r V}{b^{2}-a^{2}}\\


\textbf{Solution} :Use cylindrical coordinates $(r, \varphi, z)$ with the surface of the semi-infinite medium as the $z=0$ plane and the $z$-axis passing through the point charge $q$, which is located at $z=x$. Let $\sigma_{p}(r)$ be the bound surface charge density of the dielectric medium on the $z=0$ plane, assuming the medium to carry no free charge.

The normal component of the electric intensity at a point $(r, \varphi, 0)$ is

$$
E_{z 1}(r)=-\frac{q x}{4 \pi \varepsilon_{0}\left(r^{2}+x^{2}\right)^{3 / 2}}+\frac{\sigma_{p}(r)}{2 \varepsilon_{0}}
$$

on the upper side of the interface $\left(z=0_{+}\right)$. However, the normal component of the electeric field is given by

$$
E_{z 2}(r)=-\frac{q x}{4 \pi \varepsilon_{0}\left(r^{2}+x^{2}\right)^{3 / 2}}-\frac{\sigma_{p}(r)}{2 \varepsilon_{0}}
$$

on the lower side of the interface $\left(z=0_{-}\right)$. The boundary condition of the displacement vector at $z=0$ yields

$$
\varepsilon_{0} E_{z 1}(r)=\varepsilon_{0} K E_{z 2}(r) .
$$



Hence

$$
\sigma_{p}(r)=\frac{(1-K) q x}{2 \pi(1+K)\left(r^{2}+x^{2}\right)^{3 / 2}} .
$$

The electric field at the point $(0,0, x)$, the location of $q$, produced by the distribution of the bound charges has only the normal component because of symmetry, whose value is obtained by

$$
E=\int \frac{\sigma_{p}(r) x d S}{4 \pi \varepsilon_{0}\left(r^{2}+x^{2}\right)^{3 / 2}}=\frac{(1-K) q x^{2}}{4 \pi(1+K) \varepsilon_{0}} \int_{0}^{\infty} \frac{r d r}{\left(r^{2}+x^{2}\right)^{3}}=\frac{(1-K) q}{16 \pi(1+K) \varepsilon_{0} x^{2}},
$$

where the surface element $d S$ has been taken to be $2 \pi r d r$. Hence the force acted on the point charge is

$$
F=q E=\frac{(1-K) q^{2}}{16 \pi(1+K) \varepsilon_{0} x^{2}} .
$$

The potential energy $W$ of the point charge $q$ equals the work done by an external force in moving $q$ from infinity to the position $x$, i.e.,

$$
W=-\int_{\infty}^{x} F d x^{\prime}=-\int_{\infty}^{x} \frac{(1-K) q^{2}}{16 \pi(1+K) \varepsilon_{0} x^{\prime 2}} d x^{\prime}=\frac{(1-K) q^{2}}{16 \pi(1+K) \varepsilon_{0} x}
$$


\textbf{Topic} :Electrostatics\\
\textbf{Book} :Problems and Solutions on Electromagnetism\\
\textbf{Final Answer} :\frac{(1-K) q^{2}}{16 \pi(1+K) \varepsilon_{0} x}\\


\textbf{Solution} :As shown in Fig. 1.20, before filling in the dielectric material, one of the thin conductors carries charge $+Q$, while the other carries charge $-Q$. The potential difference between the two conductors is $V$ and the capacitance of the system is $C=Q / V$. The electric field intensity in space is $\mathbf{E}$. After the half space is filled with the dielectric, let $\mathbf{E}^{\prime}$ be the field intensity in space. This field is related to the original one by the equation $\mathbf{E}^{\prime}=K \mathbf{E}$, where $K$ is a constant to be determined below. 

MATHPIX IMAGE

Fig. $1.20$

We consider a short right cylinder across the interface $z=0$ with its cross-section at $z=0$ just contains the area enclosed by the wire carrying charge $+Q$ and the wire itself. The upper end surface $S_{1}$ of this cylinder is in the space $z>0$ and the lower end surface $S_{2}$ is in the space $z<0$. Apply Gauss' law to this cylinder. The contribution from the curved surface may be neglected if we make the cylinder sufficiently short. Thus we have, before the introduction of the dielectric,

$$
\oint_{s} D \cdot d S=\varepsilon_{0} \int_{s_{1}} \mathbf{E} \cdot d S+\varepsilon_{0} \int_{s_{2}} \mathbf{E} \cdot d S=Q
$$

and, after introducing the dielectric,

$$
\oint_{3} D^{\prime} \cdot d S=\varepsilon_{0} \int_{3_{1}} \mathbf{E}^{\prime} \cdot d S+\varepsilon \int_{3_{2}} \mathbf{L}^{\prime} \cdot d S=Q
$$

Note that the vector areas $S_{1}$ and $S_{2}$ are equal in magnitude and opposite in direction. In Eq.
(1) as the designation of 1 and 2 is interchangeable the two contributions must be equal. We therefore have

$$
\int_{s_{1}} \mathbf{E} \cdot d \mathbf{S}=\int_{s_{2}} \mathbf{E} \cdot d \mathbf{S}=\frac{Q}{2 \varepsilon_{0}} .
$$

Equation (2) can be written as

$$
K\left(\varepsilon_{0} \int_{s_{1}} \mathbf{E} \cdot d \mathbf{S}+\varepsilon \int_{s_{2}} \mathbf{E} \cdot d \mathbf{S}\right)=Q
$$

Or

$$
\frac{\left(\varepsilon_{0}+\varepsilon\right) K}{2 \varepsilon_{0}} Q=Q
$$

whence we get

$$
K=\frac{2 \varepsilon_{0}}{\varepsilon_{0}+\varepsilon^{\prime}}, \quad \Gamma^{\prime}=\frac{2 \varepsilon_{0} \overline{3}}{\varepsilon+\varepsilon_{0}} .
$$

To calculate the potential difference between the two conductors, we may select an arbitrary path of integration $L$ from one conductor to the other. Before filling in the dielectric material, the potential is

$$
V=-\int_{L} \boldsymbol{N} \cdot d \mathbb{L}
$$

while after filling in the dielectric the potential will become

$$
V^{\prime}=-\int_{L} \boldsymbol{\Sigma}^{\prime} \cdot d \boldsymbol{d}=-K \int_{L} \boldsymbol{\Sigma} \cdot d \boldsymbol{d}=K V
$$

Hence, the capacitance after introducing the dielectric is

$$
C^{\prime}=\frac{Q}{V^{\prime}}=\frac{Q}{K V}=\frac{\varepsilon+\varepsilon_{0}}{2 \varepsilon_{0}} C
$$


\textbf{Topic} :Electrostatics\\
\textbf{Book} :Problems and Solutions on Electromagnetism\\
\textbf{Final Answer} :\frac{\varepsilon+\varepsilon_{0}}{2 \varepsilon_{0}} C\\


\textbf{Solution} :Neglecting edge effects, the electric fields $E_{1}$ and $E_{2}$ in material (1) an (2) are both uniform fields and their directions are perpendicular to the parallel plates. Thus we have

$$
V=E_{1} d_{1}+E_{2} d_{2} .
$$

As the currents flowing through material (1) and (2) must be equal, we have

$$
\sigma_{1} E_{1}=\sigma_{2} E_{2} \text {. }
$$

Combination of Eqs.
(1) and (2) gives

$$
E_{1}=\frac{V \sigma_{2}}{d_{1} \sigma_{2}+d_{2} \sigma_{1}}, \quad E_{2}=\frac{V \sigma_{1}}{d_{1} \sigma_{2}+d_{2} \sigma_{1}} .
$$
\textbf{Topic} :Electrostatics\\
\textbf{Book} :Problems and Solutions on Electromagnetism\\
\textbf{Final Answer} :\frac{V \sigma_{1}}{d_{1} \sigma_{2}+d_{2} \sigma_{1}}\\


\textbf{Solution} :Neglecting edge effects, the electric fields $E_{1}$ and $E_{2}$ in material (1) an (2) are both uniform fields and their directions are perpendicular to the parallel plates. Thus we have

$$
V=E_{1} d_{1}+E_{2} d_{2} .
$$

As the currents flowing through material (1) and (2) must be equal, we have

$$
\sigma_{1} E_{1}=\sigma_{2} E_{2} \text {. }
$$

Combination of Eqs.
(1) and (2) gives

$$
E_{1}=\frac{V \sigma_{2}}{d_{1} \sigma_{2}+d_{2} \sigma_{1}}, \quad E_{2}=\frac{V \sigma_{1}}{d_{1} \sigma_{2}+d_{2} \sigma_{1}} .
$$

 The current density flowing through the capacitor is

$$
J=\sigma_{1} E_{1}=\frac{\sigma_{1} \sigma_{2} V}{d_{1} \sigma_{2}+d_{2} \sigma_{1}}
$$

Its direction is perpendicular to the plates.
\textbf{Topic} :Electrostatics\\
\textbf{Book} :Problems and Solutions on Electromagnetism\\
\textbf{Final Answer} :\frac{\sigma_{1} \sigma_{2} V}{d_{1} \sigma_{2}+d_{2} \sigma_{1}}\\


\textbf{Solution} :Neglecting edge effects, the electric fields $E_{1}$ and $E_{2}$ in material (1) an (2) are both uniform fields and their directions are perpendicular to the parallel plates. Thus we have

$$
V=E_{1} d_{1}+E_{2} d_{2} .
$$

As the currents flowing through material (1) and (2) must be equal, we have

$$
\sigma_{1} E_{1}=\sigma_{2} E_{2} \text {. }
$$

Combination of Eqs.
(1) and (2) gives

$$
E_{1}=\frac{V \sigma_{2}}{d_{1} \sigma_{2}+d_{2} \sigma_{1}}, \quad E_{2}=\frac{V \sigma_{1}}{d_{1} \sigma_{2}+d_{2} \sigma_{1}} .
$$

 The current density flowing through the capacitor is

$$
J=\sigma_{1} E_{1}=\frac{\sigma_{1} \sigma_{2} V}{d_{1} \sigma_{2}+d_{2} \sigma_{1}}
$$

Its direction is perpendicular to the plates.

 By using the boundary condition (see Fig. 1.21)

$$
\mathbf{n} \cdot\left(\boldsymbol{\Sigma}_{2}-\boldsymbol{\Sigma}_{1}\right)=\sigma_{t} / \varepsilon_{0},
$$

we find the total surface charge density on the interface between material (1) and (2)

$$
\sigma_{t}=\varepsilon_{0}\left(E_{2}-E_{1}\right)=\frac{\varepsilon_{0}\left(\sigma_{1}-\sigma_{2}\right) V}{d_{1} \sigma_{2}+d_{2} \sigma_{1}} .
$$
\textbf{Topic} :Electrostatics\\
\textbf{Book} :Problems and Solutions on Electromagnetism\\
\textbf{Final Answer} :\frac{\varepsilon_{0}\left(\sigma_{1}-\sigma_{2}\right) V}{d_{1} \sigma_{2}+d_{2} \sigma_{1}}\\


\textbf{Solution} :Neglecting edge effects, the electric fields $E_{1}$ and $E_{2}$ in material (1) an (2) are both uniform fields and their directions are perpendicular to the parallel plates. Thus we have

$$
V=E_{1} d_{1}+E_{2} d_{2} .
$$

As the currents flowing through material (1) and (2) must be equal, we have

$$
\sigma_{1} E_{1}=\sigma_{2} E_{2} \text {. }
$$

Combination of Eqs.
(1) and (2) gives

$$
E_{1}=\frac{V \sigma_{2}}{d_{1} \sigma_{2}+d_{2} \sigma_{1}}, \quad E_{2}=\frac{V \sigma_{1}}{d_{1} \sigma_{2}+d_{2} \sigma_{1}} .
$$

 The current density flowing through the capacitor is

$$
J=\sigma_{1} E_{1}=\frac{\sigma_{1} \sigma_{2} V}{d_{1} \sigma_{2}+d_{2} \sigma_{1}}
$$

Its direction is perpendicular to the plates.

 By using the boundary condition (see Fig. 1.21)

$$
\mathbf{n} \cdot\left(\boldsymbol{\Sigma}_{2}-\boldsymbol{\Sigma}_{1}\right)=\sigma_{t} / \varepsilon_{0},
$$

we find the total surface charge density on the interface between material (1) and (2)

$$
\sigma_{t}=\varepsilon_{0}\left(E_{2}-E_{1}\right)=\frac{\varepsilon_{0}\left(\sigma_{1}-\sigma_{2}\right) V}{d_{1} \sigma_{2}+d_{2} \sigma_{1}} .
$$

 From the boundary condition

$$
\mathbf{n} \cdot\left(\mathbf{D}_{2}-\mathbf{D}_{1}\right)=\mathbf{n} \cdot\left(\varepsilon_{2} \mathbf{E}_{2}-\varepsilon_{1} \mathbf{E}_{1}\right)=\sigma_{f},
$$

we find the free surface charge density on the interface

$$
\sigma_{f}=\frac{\left(\sigma_{1} \varepsilon_{2}-\sigma_{2} \varepsilon_{1}\right) V}{d_{1} \sigma_{2}+d_{2} \sigma_{1}}
$$

\textbf{Topic} :Electrostatics\\
\textbf{Book} :Problems and Solutions on Electromagnetism\\
\textbf{Final Answer} :\frac{\left(\sigma_{1} \varepsilon_{2}-\sigma_{2} \varepsilon_{1}\right) V}{d_{1} \sigma_{2}+d_{2} \sigma_{1}}\\


\textbf{Solution} :When half the condenser is filled with the material, the capacitance of the condenser (two condensers in parallel) becomes

$$
C^{\prime}=\frac{C}{2}+\frac{\varepsilon C}{2 \varepsilon_{0}}=\frac{1}{2}\left(1+\frac{\varepsilon}{\varepsilon_{0}}\right) C .
$$

The voltage across $R$ is $V R / Z$, where $V$ is the voltage of the ac source and $Z$ is the total impedance of the circuit. Thus

$$
\left|\frac{R}{R+\frac{1}{j \omega C^{\prime}}}\right|=2\left|\frac{R}{R+\frac{1}{j \omega C}}\right|,
$$

where $j=\sqrt{-1}$. Therefore we get

$$
4 R^{2}+\frac{16}{\omega^{2} C^{2}\left(1+\frac{\varepsilon}{\varepsilon_{0}}\right)^{2}}=R^{2}+\frac{1}{\omega^{2} C^{2}} .
$$

Solving this equation, we obtain

$$
\varepsilon=\left(\frac{4}{\sqrt{1-3 R^{2} C^{2} \omega^{2}}}-1\right) \varepsilon_{0} .
$$

\textbf{Topic} :Electrostatics\\
\textbf{Book} :Problems and Solutions on Electromagnetism\\
\textbf{Final Answer} :\left(\frac{4}{\sqrt{1-3 R^{2} C^{2} \omega^{2}}}-1\right) \varepsilon_{0}\\


\textbf{Solution} :As $V=$ constant, $Q=C V$ gives

$$
V d Q=V^{2} d C \text {. }
$$

Hence

$$
F=\frac{1}{2} V^{2} \frac{d C}{d x}=\frac{\varepsilon_{0}(K-1) a V^{2}}{2 d} .
$$

Since $K>1, F>0$. This means that $F$ tends to increase $x$, i.e., to pull the slab back into the plates.
\textbf{Topic} :Electrostatics\\
\textbf{Book} :Problems and Solutions on Electromagnetism\\
\textbf{Final Answer} :\frac{\varepsilon_{0}(K-1) a V^{2}}{2 d}\\


\textbf{Solution} :As $V=$ constant, $Q=C V$ gives

$$
V d Q=V^{2} d C \text {. }
$$

Hence

$$
F=\frac{1}{2} V^{2} \frac{d C}{d x}=\frac{\varepsilon_{0}(K-1) a V^{2}}{2 d} .
$$

Since $K>1, F>0$. This means that $F$ tends to increase $x$, i.e., to pull the slab back into the plates.

 Since the plates are isolated electrically, $d Q=0$. Let the initial voltage be $V_{0}$. As initially $x=b, C_{0}=\varepsilon_{0} \frac{K b a}{d}$ and $Q=C_{0} V_{0}$. The energy principle now gives

As

$$
F=-\frac{d}{d x}\left(\frac{1}{2} V^{2} C\right)=-V C \frac{d V}{d x}-\frac{V^{2}}{2} \frac{d C}{d x} .
$$

$$
\frac{d V}{d x}=\frac{d}{d x}\left(\frac{Q}{C}\right)=-\frac{Q}{C^{2}} \frac{d C}{d x},
$$

the above becomes

$$
\begin{aligned}
F &=Q \frac{d V}{d x}-\frac{Q^{2}}{2 C^{2}} \frac{d C}{d x}=\frac{Q^{2}}{2 C^{2}} \frac{d C}{d x} \\
&=\frac{\varepsilon_{0} K^{2}(K-1)}{[(K-1) x+b]^{2}} \frac{a b^{2}}{2 d} V_{0}^{2} .
\end{aligned}
$$

Again, as $F>0$ the force will tend to pull back the slab into the plates.


\textbf{Topic} :Electrostatics\\
\textbf{Book} :Problems and Solutions on Electromagnetism\\
\textbf{Final Answer} :\frac{\varepsilon_{0} K^{2}(K-1)}{[(K-1) x+b]^{2}} \frac{a b^{2}}{2 d} V_{0}^{2}\\


\textbf{Solution} :Supposing that the charge per unit length of the inner wire is $-\lambda$ and using cylindrical coordinates $(r, \varphi, z)$, we find the electric field intensity in the capacitor by Gauss' theorem to be

$$
\mathbf{E}=-\frac{\lambda}{2 \pi \varepsilon r} \mathbf{e}_{r}=\frac{-Q}{2 \pi \varepsilon L r} \mathbf{e}_{r} .
$$
\textbf{Topic} :Electrostatics\\
\textbf{Book} :Problems and Solutions on Electromagnetism\\
\textbf{Final Answer} :\frac{-Q}{2 \pi \varepsilon L r} \mathbf{e}_{r}\\


\textbf{Solution} :Supposing that the charge per unit length of the inner wire is $-\lambda$ and using cylindrical coordinates $(r, \varphi, z)$, we find the electric field intensity in the capacitor by Gauss' theorem to be

$$
\mathbf{E}=-\frac{\lambda}{2 \pi \varepsilon r} \mathbf{e}_{r}=\frac{-Q}{2 \pi \varepsilon L r} \mathbf{e}_{r} .
$$

 The potential difference between the inner and outer capacitors is

$$
V=-\int_{a}^{b} \mathrm{E} \cdot d \mathrm{r}=\frac{\lambda}{2 \pi \varepsilon} \ln \left(\frac{b}{a}\right) .
$$

Hence the capacitance is

$$
C=\frac{\lambda L}{V}=\frac{2 \pi \varepsilon L}{\ln \left(\frac{b}{a}\right)}
$$
\textbf{Topic} :Electrostatics\\
\textbf{Book} :Problems and Solutions on Electromagnetism\\
\textbf{Final Answer} :\frac{2 \pi \varepsilon L}{\ln \left(\frac{b}{a}\right)}\\


\textbf{Solution} :Supposing that the charge per unit length of the inner wire is $-\lambda$ and using cylindrical coordinates $(r, \varphi, z)$, we find the electric field intensity in the capacitor by Gauss' theorem to be

$$
\mathbf{E}=-\frac{\lambda}{2 \pi \varepsilon r} \mathbf{e}_{r}=\frac{-Q}{2 \pi \varepsilon L r} \mathbf{e}_{r} .
$$

 The potential difference between the inner and outer capacitors is

$$
V=-\int_{a}^{b} \mathrm{E} \cdot d \mathrm{r}=\frac{\lambda}{2 \pi \varepsilon} \ln \left(\frac{b}{a}\right) .
$$

Hence the capacitance is

$$
C=\frac{\lambda L}{V}=\frac{2 \pi \varepsilon L}{\ln \left(\frac{b}{a}\right)}
$$

 When the capacitor is connected to a battery, the potential difference between the inner and outer conductors remains a constant. The dielectric is now pulled a length $x$ out of the capacitor, so that a length $L-x$ of the material remains inside the capacitor, as shown schematically in Fig. 1.24. The total capacitance of the capacitor becomes

MATHPIX IMAGE

Fig. $1.24$

Pulling out the material changes the energy stored in the capacitor and thus a force must be exerted on the material. Consider the energy equation

$$
F d x=V d Q-\frac{1}{2} V^{2} d C .
$$

As $V$ is kept constant, $d Q=V d C$ and we have

$$
F=\frac{1}{2} V^{2} \frac{d C}{d x}=\frac{\pi \varepsilon_{0} V^{2}}{\ln \left(\frac{b}{a}\right)}\left(1-\frac{\varepsilon}{\varepsilon_{0}}\right)
$$

as the force acting on the material.

As $\varepsilon>\varepsilon_{0}, F<0$. Hence $F$ will tend to decrease $x$, i.e., $F$ is attractive. Then to hold the dielectric in this position, a force must be applied with magnitude $F$ and a direction away from the capacitor.


\textbf{Topic} :Electrostatics\\
\textbf{Book} :Problems and Solutions on Electromagnetism\\
\textbf{Final Answer} :\frac{\pi \varepsilon_{0} V^{2}}{\ln \left(\frac{b}{a}\right)}\left(1-\frac{\varepsilon}{\varepsilon_{0}}\right)\\


\textbf{Solution} :Neglecting edge effects, this problem becomes a 2-dimensional one. Take the $z$-axis normal to the diagram and pointing into the page as shown in Fig. 1.25. The electric field is parallel to the $x y$ plane, and independent of $z$.

Suppose that the intersection line of the planes of the two plates crosses the $x$-axis at $\mathrm{O}^{\prime}$, using the coordinate system shown in Fig. 1.26. Then

$$
\overline{\mathrm{OO}^{\prime}}=\frac{b d}{a}, \quad \theta_{0}=\arctan \frac{a}{b},
$$

where $\theta_{0}$ is the angle between the two plates. Now use cylindrical coordinates $\left(r, \theta, z^{\prime}\right)$ with the $z^{\prime}$-axis passing through point $O^{\prime}$ and parallel to the $z$-axis. Any plane through the $z^{\prime}$-axis is an equipotential surface according to the symmetry of this problem. So the potential inside the capacitor will depend only on $\theta$ :

$$
\varphi\left(r, \theta, z^{\prime}\right)=\varphi(\theta) .
$$

MATHPIX IMAGE

Fig. $1.26$

The potential $\varphi$ satisfies the Laplace equation

$$
\nabla^{2} \varphi=\frac{1}{r^{2}} \frac{d^{2} \varphi}{d \theta^{2}}=0,
$$



whose general solution is

$$
\varphi(\theta)=A+B \theta .
$$

Since both the upper and lower plates are equipotential surfaces, the boundary conditions are

$$
\varphi(0)=0, \quad \varphi\left(\theta_{0}\right)=V,
$$

whence $A=0, B=V / \theta_{0}$. For a point $(x, y)$ inside the capacitor,

$$
\theta=\arctan \left[y /\left(x+\frac{b d}{a}\right)\right]
$$

Hence

$$
\varphi(x, y)=\frac{V \theta}{\theta_{0}}=\frac{V \arctan \left[y /\left(x+\frac{b d}{a}\right)\right]}{\arctan \left(\frac{a}{b}\right)} .
$$
\textbf{Topic} :Electrostatics\\
\textbf{Book} :Problems and Solutions on Electromagnetism\\
\textbf{Final Answer} :\frac{V \arctan \left[y /\left(x+\frac{b d}{a}\right)\right]}{\arctan \left(\frac{a}{b}\right)}\\


\textbf{Solution} :Neglecting edge effects, this problem becomes a 2-dimensional one. Take the $z$-axis normal to the diagram and pointing into the page as shown in Fig. 1.25. The electric field is parallel to the $x y$ plane, and independent of $z$.

Suppose that the intersection line of the planes of the two plates crosses the $x$-axis at $\mathrm{O}^{\prime}$, using the coordinate system shown in Fig. 1.26. Then

$$
\overline{\mathrm{OO}^{\prime}}=\frac{b d}{a}, \quad \theta_{0}=\arctan \frac{a}{b},
$$

where $\theta_{0}$ is the angle between the two plates. Now use cylindrical coordinates $\left(r, \theta, z^{\prime}\right)$ with the $z^{\prime}$-axis passing through point $O^{\prime}$ and parallel to the $z$-axis. Any plane through the $z^{\prime}$-axis is an equipotential surface according to the symmetry of this problem. So the potential inside the capacitor will depend only on $\theta$ :

$$
\varphi\left(r, \theta, z^{\prime}\right)=\varphi(\theta) .
$$

MATHPIX IMAGE

Fig. $1.26$

The potential $\varphi$ satisfies the Laplace equation

$$
\nabla^{2} \varphi=\frac{1}{r^{2}} \frac{d^{2} \varphi}{d \theta^{2}}=0,
$$



whose general solution is

$$
\varphi(\theta)=A+B \theta .
$$

Since both the upper and lower plates are equipotential surfaces, the boundary conditions are

$$
\varphi(0)=0, \quad \varphi\left(\theta_{0}\right)=V,
$$

whence $A=0, B=V / \theta_{0}$. For a point $(x, y)$ inside the capacitor,

$$
\theta=\arctan \left[y /\left(x+\frac{b d}{a}\right)\right]
$$

Hence

$$
\varphi(x, y)=\frac{V \theta}{\theta_{0}}=\frac{V \arctan \left[y /\left(x+\frac{b d}{a}\right)\right]}{\arctan \left(\frac{a}{b}\right)} .
$$

 Let $Q$ be the total charge on the lower plate. The electric field inside the capacitor is

$$
\mathbf{E}=-\nabla \varphi=-\frac{\partial \varphi}{\partial \theta} \frac{\mathbf{e}_{\theta}}{r}=-\frac{V}{\theta_{0} r} \mathbf{e}_{\theta} .
$$

For a point $(x, 0)$ on the lower plate, $\theta=0, r=\frac{b d}{a}+x$ and $\mathbf{E}$ is normal to the plate. The surface charge density $\sigma$ on the lower plate is obtained from the boundary condition for the displacement vector:

$$
\sigma=\varepsilon E=-\frac{V \varepsilon}{\theta_{0}\left(\frac{b d}{a}+x\right)} .
$$

Integrating over the lower plate surface, we obtain

$$
Q=\int \sigma d S=-\int_{0}^{w} d z \int_{0}^{b} \frac{V \varepsilon}{\theta_{0}\left(\frac{b d}{a}+x\right)} d x=-\frac{\varepsilon V w}{\arctan \frac{a}{b}} \ln \left(\frac{d+a}{d}\right) .
$$

Hence, the capacitance of the capacitor is

$$
C=\frac{|Q|}{V}=\frac{\varepsilon w}{\arctan \frac{a}{b}} \ln \left(\frac{d+a}{d}\right) .
$$



\textbf{Topic} :Electrostatics\\
\textbf{Book} :Problems and Solutions on Electromagnetism\\
\textbf{Final Answer} :\frac{\varepsilon w}{\arctan \frac{a}{b}} \ln \left(\frac{d+a}{d}\right)\\


\textbf{Solution} :Let $n$ be a unit normal vector to the left plate. As $\mathbf{E}=0$ inside the plates, the tangential component of the electric field inside the dielectric is also zero because of the continuity of the tangential component of $\mathbf{E}$. Hence, the electric field intensity inside the dielectric can be expressed as

$$
\mathbf{E}=\boldsymbol{E n} .
$$

Resolving $\mathbf{E}$ along the principal axes we have

$$
E_{1}=E \cos \theta, \quad E_{2}=E \sin \theta, \quad E_{3}=0 .
$$

In the coordinates $\left(\hat{\mathrm{e}}_{1}, \hat{\mathrm{e}}_{2}, \hat{\mathrm{e}}_{3}\right)$ based on the principal axes, tensor $\varepsilon_{i j}$ is a diagonal matrix

$$
\left(\varepsilon_{i j}\right)=\left(\begin{array}{ccc}
\varepsilon_{1} & 0 & 0 \\
0 & \varepsilon_{2} & 0 \\
0 & 0 & \varepsilon_{3}
\end{array}\right)
$$

and along these axes the electric displacement in the capacitor has components

$$
D_{1}=\varepsilon_{1} E_{1}=\varepsilon_{1} E \cos \theta, \quad D_{2}=\varepsilon_{2} E \sin \theta, \quad D_{3}=0 \text {. }
$$

The boundary condition of $D$ on the surface of the left plate yields

$$
D_{n}=\sigma_{f}=Q_{F} / A .
$$

That is, the normal component of the electric displacement is a constant. Thus

$$
D_{1} \cos \theta+D_{2} \sin \theta=D_{n}=\frac{Q_{F}}{A} .
$$

Combining Eqs.
(1) and (2), we get

$$
E=\frac{Q_{F}}{A\left(\varepsilon_{1} \cos ^{2} \theta+\varepsilon_{2} \sin ^{2} \theta\right)}
$$

Hence the horizontal and vertical components of $\mathbf{E}$ and $\mathbf{D}$ are

$$
\begin{aligned}
&E_{n}=E=\frac{Q_{F}}{A\left(\varepsilon_{1} \cos ^{2} \theta+\varepsilon_{2} \sin ^{2} \theta\right)}, \quad E_{t}=0, \\
&D_{n}=\frac{Q_{F}}{A}, \quad D_{t}=D_{1} \sin \theta-D_{2} \cos \theta=\frac{Q_{F}\left(\varepsilon_{1}-\varepsilon_{2}\right) \sin \theta \cos \theta}{A\left(\varepsilon_{1} \cos ^{2} \theta+\varepsilon_{2} \sin ^{2} \theta\right)},
\end{aligned}
$$

where the subscript $t$ denotes components tangential to the plates.
\textbf{Topic} :Electrostatics\\
\textbf{Book} :Problems and Solutions on Electromagnetism\\
\textbf{Final Answer} :\frac{Q_{F}\left(\varepsilon_{1}-\varepsilon_{2}\right) \sin \theta \cos \theta}{A\left(\varepsilon_{1} \cos ^{2} \theta+\varepsilon_{2} \sin ^{2} \theta\right)}\\


\textbf{Solution} :Let $n$ be a unit normal vector to the left plate. As $\mathbf{E}=0$ inside the plates, the tangential component of the electric field inside the dielectric is also zero because of the continuity of the tangential component of $\mathbf{E}$. Hence, the electric field intensity inside the dielectric can be expressed as

$$
\mathbf{E}=\boldsymbol{E n} .
$$

Resolving $\mathbf{E}$ along the principal axes we have

$$
E_{1}=E \cos \theta, \quad E_{2}=E \sin \theta, \quad E_{3}=0 .
$$

In the coordinates $\left(\hat{\mathrm{e}}_{1}, \hat{\mathrm{e}}_{2}, \hat{\mathrm{e}}_{3}\right)$ based on the principal axes, tensor $\varepsilon_{i j}$ is a diagonal matrix

$$
\left(\varepsilon_{i j}\right)=\left(\begin{array}{ccc}
\varepsilon_{1} & 0 & 0 \\
0 & \varepsilon_{2} & 0 \\
0 & 0 & \varepsilon_{3}
\end{array}\right)
$$

and along these axes the electric displacement in the capacitor has components

$$
D_{1}=\varepsilon_{1} E_{1}=\varepsilon_{1} E \cos \theta, \quad D_{2}=\varepsilon_{2} E \sin \theta, \quad D_{3}=0 \text {. }
$$

The boundary condition of $D$ on the surface of the left plate yields

$$
D_{n}=\sigma_{f}=Q_{F} / A .
$$

That is, the normal component of the electric displacement is a constant. Thus

$$
D_{1} \cos \theta+D_{2} \sin \theta=D_{n}=\frac{Q_{F}}{A} .
$$

Combining Eqs.
(1) and (2), we get

$$
E=\frac{Q_{F}}{A\left(\varepsilon_{1} \cos ^{2} \theta+\varepsilon_{2} \sin ^{2} \theta\right)}
$$

Hence the horizontal and vertical components of $\mathbf{E}$ and $\mathbf{D}$ are

$$
\begin{aligned}
&E_{n}=E=\frac{Q_{F}}{A\left(\varepsilon_{1} \cos ^{2} \theta+\varepsilon_{2} \sin ^{2} \theta\right)}, \quad E_{t}=0, \\
&D_{n}=\frac{Q_{F}}{A}, \quad D_{t}=D_{1} \sin \theta-D_{2} \cos \theta=\frac{Q_{F}\left(\varepsilon_{1}-\varepsilon_{2}\right) \sin \theta \cos \theta}{A\left(\varepsilon_{1} \cos ^{2} \theta+\varepsilon_{2} \sin ^{2} \theta\right)},
\end{aligned}
$$

where the subscript $t$ denotes components tangential to the plates.

 The potential difference between the left and right plates is

$$
V=\int E_{n} d x=\frac{Q_{F} d}{A\left(\varepsilon_{1} \cos ^{2} \theta+\varepsilon_{2} \sin ^{2} \theta\right)}
$$

Therefore, the capacitance of the system is

$$
C=\frac{Q_{F}}{V}=\frac{A\left(\varepsilon_{1} \cos ^{2} \theta+\varepsilon_{2} \sin ^{2} \theta\right)}{d} .
$$



\textbf{Topic} :Electrostatics\\
\textbf{Book} :Problems and Solutions on Electromagnetism\\
\textbf{Final Answer} :\frac{A\left(\varepsilon_{1} \cos ^{2} \theta+\varepsilon_{2} \sin ^{2} \theta\right)}{d}\\


\textbf{Solution} :The electric field inside the sphere is a uniform field $\mathbf{E}_{0}$, as shown in Fig. 1.28. The field at point $p$ of the outer surface of the sphere is $\mathbf{E}=E_{r} \mathbf{e}_{r}+E_{t} \mathbf{e}_{\theta}$, using polar coordinates. Similarly $\mathbf{E}_{0}$ may be expressed as

$$
\mathbf{E}_{0}=E_{0} \cos \theta \mathbf{e}_{\boldsymbol{r}}-E_{0} \sin \theta \mathbf{e}_{\boldsymbol{\theta}} .
$$

MATHPIX IMAGE

Fig. $1.28$

From the boundary conditions for the electric vectors at $p$ we obtain

$$
\varepsilon E_{0} \cos \theta=\varepsilon_{0} E_{r}, \quad-E_{0} \sin \theta=E_{t}
$$

Hence

$$
\mathbf{E}=K_{e} E_{0} \cos \theta \mathbf{e}_{r}-E_{0} \sin \theta \mathbf{e}_{\theta} .
$$

The bound surface charge density at point $p$ is $\sigma_{b}=\mathbf{P} \cdot \mathbf{e}_{r}$, where $\mathbf{P}$ is the polarization vector. As $\mathbf{P}=\left(\varepsilon-\varepsilon_{0}\right) \mathbf{E}_{0}$, we find

$$
\sigma_{p}=\left(\varepsilon-\varepsilon_{0}\right) E_{0} \cos \theta=\varepsilon_{0}\left(K_{e}-1\right) E_{0} \cos \theta .
$$



\textbf{Topic} :Electrostatics\\
\textbf{Book} :Problems and Solutions on Electromagnetism\\
\textbf{Final Answer} :\varepsilon_{0}\left(K_{e}-1\right) E_{0} \cos \theta\\


\textbf{Solution} :At $t=0$, when the inner sphere carries electric charge $q$, the field intensity inside the medium is

$$
E_{0}=\frac{q}{4 \pi \varepsilon r^{2}}
$$

and directs radially outwords. At time $t$ when the inner sphere has charge $q(t)$, the field intensity is

$$
E(t)=\frac{q(t)}{4 \pi \varepsilon r^{2}} .
$$

Ohm's law gives the current density $\mathbf{j}=\sigma \mathbf{E}$. Considering a concentric spherical surface of radius $r$ enclosing the inner sphere, we have from charge conservation

$$
-\frac{d}{d t} q(t)=4 \pi r^{2} j(t)=4 \pi r^{2} \sigma E(t)=\frac{\sigma}{\varepsilon} q(t) .
$$

The differential equation has solution

$$
q(t)=q e^{-\frac{\varepsilon}{\epsilon} t} .
$$

Hence

$$
\begin{aligned}
&E(t, r)=\frac{q}{4 \pi \varepsilon r^{2}} e^{-\frac{\sigma}{\epsilon} t}, \\
&j(t, r)=\frac{\sigma q}{4 \pi \varepsilon r^{2}} e^{-\frac{\pi}{\varepsilon} t}
\end{aligned}
$$

The total current flowing through the medium at time $t$ is

$$
I(t)=4 \pi r^{2} j(t, r)=\frac{\sigma q}{\varepsilon} e^{-\frac{\varepsilon}{\varepsilon} t} .
$$
is
\textbf{Topic} :Electrostatics\\
\textbf{Book} :Problems and Solutions on Electromagnetism\\
\textbf{Final Answer} :\frac{\sigma q}{\varepsilon} e^{-\frac{\varepsilon}{\varepsilon} t}\\


\textbf{Solution} :At $t=0$, when the inner sphere carries electric charge $q$, the field intensity inside the medium is

$$
E_{0}=\frac{q}{4 \pi \varepsilon r^{2}}
$$

and directs radially outwords. At time $t$ when the inner sphere has charge $q(t)$, the field intensity is

$$
E(t)=\frac{q(t)}{4 \pi \varepsilon r^{2}} .
$$

Ohm's law gives the current density $\mathbf{j}=\sigma \mathbf{E}$. Considering a concentric spherical surface of radius $r$ enclosing the inner sphere, we have from charge conservation

$$
-\frac{d}{d t} q(t)=4 \pi r^{2} j(t)=4 \pi r^{2} \sigma E(t)=\frac{\sigma}{\varepsilon} q(t) .
$$

The differential equation has solution

$$
q(t)=q e^{-\frac{\varepsilon}{\epsilon} t} .
$$

Hence

$$
\begin{aligned}
&E(t, r)=\frac{q}{4 \pi \varepsilon r^{2}} e^{-\frac{\sigma}{\epsilon} t}, \\
&j(t, r)=\frac{\sigma q}{4 \pi \varepsilon r^{2}} e^{-\frac{\pi}{\varepsilon} t}
\end{aligned}
$$

The total current flowing through the medium at time $t$ is

$$
I(t)=4 \pi r^{2} j(t, r)=\frac{\sigma q}{\varepsilon} e^{-\frac{\varepsilon}{\varepsilon} t} .
$$
is

 The Joule heat loss per unit volume per unit time in the medium

$$
w(t, r)=\mathbf{j} \cdot \mathbf{E}=\sigma E^{2}=\frac{\sigma q^{2}}{(4 \pi \varepsilon)^{2} r^{4}} e^{-\frac{2 \sigma}{\epsilon} t}
$$

and the total Joule heat produced is

$$
W=\int_{0}^{+\infty} d t \int_{a}^{b} d r \cdot 4 \pi r^{2} w(t, r)=\frac{q^{2}}{8 \pi \varepsilon}\left(\frac{1}{a}-\frac{1}{b}\right) .
$$

The electrostatic energy in the medium before discharging is

$$
W_{0}=\int_{a}^{b} d r \cdot 4 \pi r^{2} \cdot \frac{\varepsilon E_{0}^{2}}{2}=\frac{q^{2}}{8 \pi \varepsilon}\left(\frac{1}{a}-\frac{1}{b}\right) .
$$

Hence $W=W_{0}$. 

\textbf{Topic} :Electrostatics\\
\textbf{Book} :Problems and Solutions on Electromagnetism\\
\textbf{Final Answer} :\frac{q^{2}}{8 \pi \varepsilon}\left(\frac{1}{a}-\frac{1}{b}\right)\\


\textbf{Solution} :Suppose the inner sphere carries total free charge $Q$. Then the outer sphere will carry total free charge $-Q$ as it is grounded.
\textbf{Topic} :Electrostatics\\
\textbf{Book} :Problems and Solutions on Electromagnetism\\
\textbf{Final Answer} :\frac{q^{2}}{8 \pi \varepsilon}\left(\frac{1}{a}-\frac{1}{b}\right)\\


\textbf{Solution} :Suppose the inner sphere carries total free charge $Q$. Then the outer sphere will carry total free charge $-Q$ as it is grounded.

 Using Gauss' law and the spherical symmetry, we find the following results:

$$
\begin{array}{rlr}
\mathbf{E} & =\frac{Q}{4 \pi K_{1} \varepsilon_{0} r^{2}} \mathbf{e}_{r}, & (a<r<b), \\
\mathbf{E} & =\frac{Q}{4 \pi \varepsilon_{0} r^{2}} \mathbf{e}_{r}, & (b<r<c), \\
\mathbf{E} & =\frac{Q}{4 \pi \varepsilon_{0} K_{2} r^{2}} \mathbf{e}_{r}, & (c<r<d) .
\end{array}
$$
\textbf{Topic} :Electrostatics\\
\textbf{Book} :Problems and Solutions on Electromagnetism\\
\textbf{Final Answer} :\frac{Q}{4 \pi \varepsilon_{0} K_{2} r^{2}} \mathbf{e}_{r} & (c<r<d) 
\end{array}\\


\textbf{Solution} :Suppose the inner sphere carries total free charge $Q$. Then the outer sphere will carry total free charge $-Q$ as it is grounded.

 Using Gauss' law and the spherical symmetry, we find the following results:

$$
\begin{array}{rlr}
\mathbf{E} & =\frac{Q}{4 \pi K_{1} \varepsilon_{0} r^{2}} \mathbf{e}_{r}, & (a<r<b), \\
\mathbf{E} & =\frac{Q}{4 \pi \varepsilon_{0} r^{2}} \mathbf{e}_{r}, & (b<r<c), \\
\mathbf{E} & =\frac{Q}{4 \pi \varepsilon_{0} K_{2} r^{2}} \mathbf{e}_{r}, & (c<r<d) .
\end{array}
$$

 Using the equations

$$
\sigma_{P}=\mathbf{n} \cdot\left(\mathbf{P}_{1}-\mathbf{P}_{2}\right), \quad \mathbf{P}=\varepsilon_{0}(K-1) \mathbf{E},
$$

we obtain the polarization charge densities

$$
\begin{aligned}
\sigma_{P} &=\frac{Q}{4 \pi a^{2}} \frac{1-K_{1}}{K_{1}} \text { at } r=a, \\
\sigma_{P} &=\frac{Q}{4 \pi b^{2}} \frac{K_{1}-1}{K_{1}} \text { at } r=b \\
\sigma_{P} &=\frac{Q}{4 \pi c^{2}} \frac{1-K_{2}}{K_{2}} \text { at } r=c \\
\sigma_{P} &=\frac{Q}{4 \pi d^{2}} \frac{K_{2}-1}{K_{2}} \text { at } r=d .
\end{aligned}
$$
\textbf{Topic} :Electrostatics\\
\textbf{Book} :Problems and Solutions on Electromagnetism\\
\textbf{Final Answer} :d\\


\textbf{Solution} :Suppose the inner sphere carries total free charge $Q$. Then the outer sphere will carry total free charge $-Q$ as it is grounded.

 Using Gauss' law and the spherical symmetry, we find the following results:

$$
\begin{array}{rlr}
\mathbf{E} & =\frac{Q}{4 \pi K_{1} \varepsilon_{0} r^{2}} \mathbf{e}_{r}, & (a<r<b), \\
\mathbf{E} & =\frac{Q}{4 \pi \varepsilon_{0} r^{2}} \mathbf{e}_{r}, & (b<r<c), \\
\mathbf{E} & =\frac{Q}{4 \pi \varepsilon_{0} K_{2} r^{2}} \mathbf{e}_{r}, & (c<r<d) .
\end{array}
$$

 Using the equations

$$
\sigma_{P}=\mathbf{n} \cdot\left(\mathbf{P}_{1}-\mathbf{P}_{2}\right), \quad \mathbf{P}=\varepsilon_{0}(K-1) \mathbf{E},
$$

we obtain the polarization charge densities

$$
\begin{aligned}
\sigma_{P} &=\frac{Q}{4 \pi a^{2}} \frac{1-K_{1}}{K_{1}} \text { at } r=a, \\
\sigma_{P} &=\frac{Q}{4 \pi b^{2}} \frac{K_{1}-1}{K_{1}} \text { at } r=b \\
\sigma_{P} &=\frac{Q}{4 \pi c^{2}} \frac{1-K_{2}}{K_{2}} \text { at } r=c \\
\sigma_{P} &=\frac{Q}{4 \pi d^{2}} \frac{K_{2}-1}{K_{2}} \text { at } r=d .
\end{aligned}
$$

 The potential is

$$
V=-\int_{d}^{a} \frac{D}{a} \cdot d x=\frac{Q}{4 \pi \varepsilon_{0}}\left[\left(\frac{1}{a}-\frac{1}{b}\right) \frac{1}{K_{1}}+\left(\frac{1}{b}-\frac{1}{c}\right)+\left(\frac{1}{c}-\frac{1}{d}\right) \frac{1}{K_{2}}\right] \text {. }
$$

Therefore, the charge in the inner sphere is

$$
Q=\frac{4 \pi \varepsilon_{0} K_{1} K_{2} a b c d V}{K_{1} a b(d-c)+K_{1} K_{2} a d(c-b)+K_{2} c d(b-a)},
$$

and the capacitance is

$$
C=\frac{Q}{V}=\frac{4 \pi \varepsilon_{0} K_{1} K_{2} a b c d}{K_{1} a b(d-c)+K_{1} K_{2} a d(c-b)+K_{2} c d(b-a)}
$$

\textbf{Topic} :Electrostatics\\
\textbf{Book} :Problems and Solutions on Electromagnetism\\
\textbf{Final Answer} :\frac{4 \pi \varepsilon_{0} K_{1} K_{2} a b c d}{K_{1} a b(d-c)+K_{1} K_{2} a d(c-b)+K_{2} c d(b-a)}\\


\textbf{Solution} :Gauss' law and spherical symmetry give

$$
\mathrm{D}=\frac{Q}{4 \pi r^{2}} \mathbf{e}_{r}, \quad(a<r<b) .
$$
\textbf{Topic} :Electrostatics\\
\textbf{Book} :Problems and Solutions on Electromagnetism\\
\textbf{Final Answer} :\frac{Q}{4 \pi r^{2}} \mathbf{e}_{r} \quad(a<r<b)\\


\textbf{Solution} :Gauss' law and spherical symmetry give

$$
\mathrm{D}=\frac{Q}{4 \pi r^{2}} \mathbf{e}_{r}, \quad(a<r<b) .
$$

 The electric field intensity is

$$
\mathbf{E}=\frac{Q}{4 \pi \varepsilon_{0} r^{2}}(1+K r) e_{r}, \quad(a<r<b) .
$$

Hence, the potential difference between the inner and outer spheres is

$$
V=\int_{a}^{b} \mathbf{\Sigma} \cdot d \mathbf{r}=\frac{Q}{4 \pi \varepsilon_{0}}\left(\frac{1}{a}-\frac{1}{b}+K \ln \frac{b}{a}\right) .
$$

The capacitance of the device is then

$$
C=\frac{Q}{V}=\frac{4 \pi \varepsilon_{0} a b}{(b-a)+a b K \ln (b / a)}
$$
\textbf{Topic} :Electrostatics\\
\textbf{Book} :Problems and Solutions on Electromagnetism\\
\textbf{Final Answer} :\frac{4 \pi \varepsilon_{0} a b}{(b-a)+a b K \ln (b / a)}\\


\textbf{Solution} :Gauss' law and spherical symmetry give

$$
\mathrm{D}=\frac{Q}{4 \pi r^{2}} \mathbf{e}_{r}, \quad(a<r<b) .
$$

 The electric field intensity is

$$
\mathbf{E}=\frac{Q}{4 \pi \varepsilon_{0} r^{2}}(1+K r) e_{r}, \quad(a<r<b) .
$$

Hence, the potential difference between the inner and outer spheres is

$$
V=\int_{a}^{b} \mathbf{\Sigma} \cdot d \mathbf{r}=\frac{Q}{4 \pi \varepsilon_{0}}\left(\frac{1}{a}-\frac{1}{b}+K \ln \frac{b}{a}\right) .
$$

The capacitance of the device is then

$$
C=\frac{Q}{V}=\frac{4 \pi \varepsilon_{0} a b}{(b-a)+a b K \ln (b / a)}
$$

 The polarization is

$$
\mathbf{P}=\left(\varepsilon-\varepsilon_{0}\right) \mathbb{E}=-\frac{Q K}{4 \pi r} \mathbf{e}_{r} .
$$

Therefore, the volume polarization charge density at $a<r<b$ is given by

$$
\rho_{P}=-\nabla \cdot \mathbf{P}=\frac{1}{r^{2}} \frac{\partial}{\partial r}\left(\frac{Q K r}{4 \pi}\right)=\frac{Q K}{4 \pi r^{2}} .
$$
\textbf{Topic} :Electrostatics\\
\textbf{Book} :Problems and Solutions on Electromagnetism\\
\textbf{Final Answer} :\frac{Q K}{4 \pi r^{2}}\\


\textbf{Solution} :Gauss' law and spherical symmetry give

$$
\mathrm{D}=\frac{Q}{4 \pi r^{2}} \mathbf{e}_{r}, \quad(a<r<b) .
$$

 The electric field intensity is

$$
\mathbf{E}=\frac{Q}{4 \pi \varepsilon_{0} r^{2}}(1+K r) e_{r}, \quad(a<r<b) .
$$

Hence, the potential difference between the inner and outer spheres is

$$
V=\int_{a}^{b} \mathbf{\Sigma} \cdot d \mathbf{r}=\frac{Q}{4 \pi \varepsilon_{0}}\left(\frac{1}{a}-\frac{1}{b}+K \ln \frac{b}{a}\right) .
$$

The capacitance of the device is then

$$
C=\frac{Q}{V}=\frac{4 \pi \varepsilon_{0} a b}{(b-a)+a b K \ln (b / a)}
$$

 The polarization is

$$
\mathbf{P}=\left(\varepsilon-\varepsilon_{0}\right) \mathbb{E}=-\frac{Q K}{4 \pi r} \mathbf{e}_{r} .
$$

Therefore, the volume polarization charge density at $a<r<b$ is given by

$$
\rho_{P}=-\nabla \cdot \mathbf{P}=\frac{1}{r^{2}} \frac{\partial}{\partial r}\left(\frac{Q K r}{4 \pi}\right)=\frac{Q K}{4 \pi r^{2}} .
$$

 The surface polarization charge densities at $r=a, b$ are

$$
\sigma_{P}=\frac{Q K}{4 \pi \sigma} \quad \text { at } \quad r=a ; \quad \sigma_{P}=-\frac{Q K}{4 \pi b} \quad \text { at } \quad r=b .
$$


\textbf{Topic} :Electrostatics\\
\textbf{Book} :Problems and Solutions on Electromagnetism\\
\textbf{Final Answer} :b\\


\textbf{Solution} :Suppose the conductors and the material are homogeneous so that the total charge $Q$ carried by the inner sphere is uniformly distributed over its surface. Gauss' law and spherical symmetry give

$$
\mathbf{E}(r)=\frac{Q}{4 \pi \varepsilon r^{2}} \mathbf{e}_{r},
$$

where $\varepsilon$ is the dielectric constant of the material. From Ohm's law $\mathbf{j}=\sigma \mathbf{E}$, one has

$$
\mathrm{j}=\frac{\sigma Q}{4 \pi \varepsilon r^{2}} \mathbf{e}_{r} .
$$

Then the total current is

$$
I=\oint \mathbf{j} \cdot d \mathbf{S}=\frac{\sigma}{\varepsilon} Q .
$$

The potential difference between the two conductors is

$$
V=-\int_{b}^{a} \mathrm{E} \cdot d \mathrm{r}=-\int_{b}^{a} \frac{I}{4 \pi \sigma r^{2}} d r=\frac{I}{4 \pi \sigma}\left(\frac{1}{a}-\frac{1}{b}\right),
$$

giving the resistance as

$$
R=\frac{V}{I}=\frac{1}{4 \pi \sigma}\left(\frac{1}{a}-\frac{1}{b}\right)
$$


\textbf{Topic} :Electrostatics\\
\textbf{Book} :Problems and Solutions on Electromagnetism\\
\textbf{Final Answer} :\frac{1}{4 \pi \sigma}\left(\frac{1}{a}-\frac{1}{b}\right)\\


\textbf{Solution} :Let the origin be at the spherical center and take the direction of the original field $\mathbf{E}$ to define the polar axis $z$, as shown in Fig. 1.30. Let the electrostatic potential at a point inside the sphere be $\Phi_{1}$, and the potential at a point outside the sphere be $\Phi_{2}$. By symmetry we can write $\Phi_{1}$ and $\Phi_{2}$ as

$$
\begin{aligned}
&\Phi_{1}=\sum_{n=0}\left(A_{n} r^{n}+\frac{B_{n}}{r^{n+1}}\right) P_{n}(\cos \theta) \\
&\Phi_{2}=\sum_{n=0}\left(C_{n} r^{n}+\frac{D_{n}}{r^{n+1}}\right) P_{n}(\cos \theta)
\end{aligned}
$$

where $A_{n}, B_{n}, C_{n}, D_{n}$ are constants, and $P_{n}$ are Legendre polynomials. The boundary conditions are as follows:

(1) $\Phi_{1}$ is finite at $r=0$.

(2) $\left.\Phi_{2}\right|_{r \rightarrow \infty}=-E r \cos \theta=-E r P_{1}(\cos \theta)$.

(3) $\Phi_{1}=\left.\Phi_{2}\right|_{r=a}, \varepsilon_{1} \frac{\partial \Phi_{1}}{\partial r}=\left.\varepsilon_{2} \frac{\partial \Phi_{2}}{\partial r}\right|_{r=a}$.

MATHPIX IMAGE

Fig. $1.30$

From conditions (1) and (2), we obtain

$$
B_{n}=0, \quad C_{1}=-E, \quad C_{n}=0(n \neq 1) .
$$

Then from condition (3), we obtain

$$
\begin{aligned}
&-E a P_{1}(\cos \theta)+\sum_{n} \frac{D_{n}}{a^{n+1}} P_{n}(\cos \theta)=\sum_{n} A_{n} a^{n} P_{n}(\cos \theta), \\
&-\varepsilon_{2}\left[E P_{1}(\cos \theta)+\sum_{n}(n+1) \frac{D^{n}}{a^{n+2}} P_{n}(\cos \theta)\right]=\varepsilon_{1} \sum_{n} A_{n} a^{n-1} P_{n}(\cos \theta) .
\end{aligned}
$$

These equations are to be satisfied for each of the possible angles $\theta$. That is, the coefficients of $P_{n}(\cos \theta)$ on the two sides of each equation must be equal for every $n$. This gives

$$
\begin{array}{r}
A_{1}=-\frac{3 \varepsilon_{2}}{\varepsilon_{1}+2 \varepsilon_{2}} E, \quad D_{1}=\frac{\varepsilon_{1}-\varepsilon_{2}}{\varepsilon_{1}+2 \varepsilon_{2}} E a^{3}, \\
A_{n}=D_{n}=0, \quad(n \neq 1) .
\end{array}
$$

Hence, the electric potentials inside and outside the sphere can be expressed as

$$
\begin{aligned}
\Phi_{1} &=-\frac{3 \varepsilon_{2}}{\varepsilon_{1}+2 \varepsilon_{2}} E r \cos \theta \\
\Phi_{2} &=-\left[1-\frac{\varepsilon_{1}-\varepsilon_{2}}{\varepsilon_{1}+2 \varepsilon_{2}}\left(\frac{a}{r}\right)^{3}\right] \operatorname{Er} \cos \theta,
\end{aligned}
$$

and the electric fields inside and outside the sphere by

$$
\begin{gathered}
\mathbf{E}_{1}=-\nabla \Phi_{1}=\frac{3 \varepsilon_{2}}{\varepsilon_{1}+2 \varepsilon_{2}} \mathbf{E}, \quad(r<a), \\
\mathbf{\Sigma}_{2}=-\nabla \Phi_{2}=\mathbf{E}+\frac{\varepsilon_{1}-\varepsilon_{2}}{\varepsilon_{1}+2 \varepsilon_{2}} a^{3}\left[\frac{3(\mathbf{E} \cdot \mathbf{r}) \mathbf{r}}{r^{5}}-\frac{\mathbf{E}}{r^{3}}\right], \quad(r>a) .
\end{gathered}
$$

\textbf{Topic} :Electrostatics\\
\textbf{Book} :Problems and Solutions on Electromagnetism\\
\textbf{Final Answer} :\mathbf{E}+\frac{\varepsilon_{1}-\varepsilon_{2}}{\varepsilon_{1}+2 \varepsilon_{2}} a^{3}\left[\frac{3(\mathbf{E} \cdot \mathbf{r}) \mathbf{r}}{r^{5}}-\frac{\mathbf{E}}{r^{3}}\right] \quad(r>a) 
\end{gathered}\\


\textbf{Solution} :The boundary conditions on the conductor surface are

$$
\begin{aligned}
&\Phi=\text { constant }=\Phi_{s}, \text { say, } \\
&\varepsilon_{0} \frac{\partial \Phi}{\partial r}=-\sigma
\end{aligned}
$$

where $\Phi_{3}$ is the potential of the conducting sphere and $\sigma$ is its surface charge density. On account of symmetry, the potential at a point $(r, \theta, \varphi)$ outside the sphere is, in spherical coordinates with origin at the center of the sphere,

$$
\Phi=\sum_{n=0}\left(C_{n} r^{n}+\frac{D_{n}}{r^{n+1}}\right) P_{n}(\cos \theta)
$$

Let $E_{0}$ be the original uniform electric intensity. As $r \rightarrow \infty$,

$$
\Phi=-E_{0} r \cos \theta=-E_{0} r P_{1}(\cos \theta) .
$$

By equating the coefficients of $P_{n}(\cos \theta)$ on the two sides of Eq.
(1), we have

$$
C_{0}=0, \quad C_{1}=-E_{0}, \quad D_{1}=E_{0} a^{3}, \quad C_{n}=D_{n}=0 \text { for } n>1 .
$$

Hence

$$
\Phi=-E_{0} r \cos \theta+\frac{E_{0} a^{3}}{r^{2}} \cos \theta,
$$

where $a$ is the radius of the sphere. The second boundary condition and Eq.
(2) give

$$
\sigma=3 \varepsilon_{0} E_{0} \cos \theta .
$$
\textbf{Topic} :Electrostatics\\
\textbf{Book} :Problems and Solutions on Electromagnetism\\
\textbf{Final Answer} :3 \varepsilon_{0} E_{0} \cos \theta\\


\textbf{Solution} :The boundary conditions on the conductor surface are

$$
\begin{aligned}
&\Phi=\text { constant }=\Phi_{s}, \text { say, } \\
&\varepsilon_{0} \frac{\partial \Phi}{\partial r}=-\sigma
\end{aligned}
$$

where $\Phi_{3}$ is the potential of the conducting sphere and $\sigma$ is its surface charge density. On account of symmetry, the potential at a point $(r, \theta, \varphi)$ outside the sphere is, in spherical coordinates with origin at the center of the sphere,

$$
\Phi=\sum_{n=0}\left(C_{n} r^{n}+\frac{D_{n}}{r^{n+1}}\right) P_{n}(\cos \theta)
$$

Let $E_{0}$ be the original uniform electric intensity. As $r \rightarrow \infty$,

$$
\Phi=-E_{0} r \cos \theta=-E_{0} r P_{1}(\cos \theta) .
$$

By equating the coefficients of $P_{n}(\cos \theta)$ on the two sides of Eq.
(1), we have

$$
C_{0}=0, \quad C_{1}=-E_{0}, \quad D_{1}=E_{0} a^{3}, \quad C_{n}=D_{n}=0 \text { for } n>1 .
$$

Hence

$$
\Phi=-E_{0} r \cos \theta+\frac{E_{0} a^{3}}{r^{2}} \cos \theta,
$$

where $a$ is the radius of the sphere. The second boundary condition and Eq.
(2) give

$$
\sigma=3 \varepsilon_{0} E_{0} \cos \theta .
$$

 Suppose that an electric dipole $\mathbf{P}=P \mathbf{e}_{z}$ is placed at the origin, instead of the sphere. The potential at $r$ produced by the dipole is

$$
\Phi_{P}=-\frac{1}{4 \pi \varepsilon_{0}} \mathbf{P} \cdot \nabla\left(\frac{1}{r}\right)=\frac{P \cos \theta}{4 \pi \varepsilon_{0} r^{2}} .
$$

Comparing this with the second term of Eq.
(2) shows that the latter corresponds to the contribution of a dipole having a moment

$$
\mathbf{P}=4 \pi \varepsilon_{0} a^{3} \mathbf{E}_{0},
$$

which can be considered as the induced dipole moment of the sphere.

\textbf{Topic} :Electrostatics\\
\textbf{Book} :Problems and Solutions on Electromagnetism\\
\textbf{Final Answer} :4 \pi \varepsilon_{0} a^{3} \mathbf{E}_{0}\\


\textbf{Solution} :Let $\Phi_{+}, \Phi_{-}$be respectively the potentials outside and inside the shell. Both $\Phi_{+}$and $\Phi_{-}$satistify Laplace's equation and, on account of cylindrical symmetry, they have the expressions

$$
\begin{aligned}
&\Phi_{+}=\sum_{n=0} b_{n} r^{-n-1} P_{n}(\cos \theta), \quad(r>R) \\
&\Phi_{-}=\sum_{n=0} a_{n} r^{n} P_{n}(\cos \theta), \quad(r<R)
\end{aligned}
$$

The boundary conditions at $r=R$ for the potential and displacement vector are

$$
\begin{aligned}
&\Phi_{-}=\Phi_{+} \\
&\sigma(\theta)=\sigma_{0} P_{1}(\cos \theta)=\varepsilon_{0}\left(\frac{\partial \Phi_{-}}{\partial r}-\frac{\partial \Phi_{+}}{\partial r}\right)
\end{aligned}
$$

Substituting in the above the expressions for the potentials and equating the coefficients of $P_{n}(\cos \theta)$ on the two sides of the equations, we obtain

$$
\begin{array}{ll}
a_{n}=b_{n}=0 & \text { for } \quad n \neq 1, \\
a_{1}=\frac{\sigma_{0}}{3 \varepsilon_{0}}, \quad b_{1}=\frac{\sigma_{0}}{3 \varepsilon_{0}} R^{3} & \text { for } \quad n=1
\end{array}
$$

Hence

$$
\begin{aligned}
\Phi_{+} &=\frac{\sigma_{0} R^{3}}{3 \varepsilon_{0} r^{2}} \cos \theta, \quad r>R \\
\Phi_{-} &=\frac{\sigma_{0} r}{3 \varepsilon_{0}} \cos \theta, \quad r<R
\end{aligned}
$$

From $\mathbf{E}=-\nabla \Phi$ we obtain

$$
\begin{aligned}
&\mathbf{E}_{+}=\frac{2 \sigma_{0} R^{3}}{3 \varepsilon_{0} r^{3}} \cos \theta \mathbf{e}_{r}+\frac{\sigma_{0} R^{3}}{3 \varepsilon_{0} r^{3}} \sin \theta \mathbf{e}_{\theta}, \quad r>R, \\
&\mathbf{E}_{-}=-\frac{\sigma_{0}}{3 \varepsilon_{0}} \mathbf{e}_{z}, \quad r<R .
\end{aligned}
$$



\textbf{Topic} :Electrostatics\\
\textbf{Book} :Problems and Solutions on Electromagnetism\\
\textbf{Final Answer} :-\frac{\sigma_{0}}{3 \varepsilon_{0}} \mathbf{e}_{z} \quad r<R\\


\textbf{Solution} :The potential is given by either Poisson's or Laplace's equation:

$$
\begin{array}{ll}
\nabla^{2} \Phi_{-}=-\frac{q}{\varepsilon_{0}} \delta(r), & r<R \\
\nabla^{2} \Phi_{+}=0, & r>R .
\end{array}
$$

The general solutions finite in the respective regions, taking account of the symmetry, are

$$
\begin{aligned}
&\Phi_{-}=\frac{q}{4 \pi \varepsilon_{0} r}+\sum_{n=0}^{\infty} A_{n} r^{n} P_{n}(\cos \theta), \quad r<R, \\
&\Phi_{+}=\sum_{n=0}^{\infty} \frac{B_{n}}{r^{n+1}} P_{n}(\cos \theta), \quad r>R .
\end{aligned}
$$

Then from the condition $\Phi_{-}=\Phi_{+}=V_{0} \cos \theta$ at $r=R$ we obtain $A_{0}=$ $-\frac{q}{4 \pi \varepsilon_{0} R}, A_{1}=\frac{V_{0}}{R}, B_{1}=V_{0} R^{2}, B_{0}=0, A_{n}=B_{n}=0$ for $n \neq 0,1$, and hence

$$
\begin{aligned}
&\Phi_{-}=\frac{q}{4 \pi \varepsilon_{0} r}-\frac{q}{4 \pi \varepsilon_{0} R}+\frac{V_{0} \cos \theta}{R} r, \quad r<R \\
&\Phi_{+}=\frac{V_{0} R^{2}}{r^{2}} \cos \theta, \quad r>R .
\end{aligned}
$$

\textbf{Topic} :Electrostatics\\
\textbf{Book} :Problems and Solutions on Electromagnetism\\
\textbf{Final Answer} :\frac{V_{0} R^{2}}{r^{2}} \cos \theta \quad r>R\\


\textbf{Solution} :The field in this problem is the superposition of three fields: a uniform field $\mathbf{E}_{0}$, a dipole field due to the induced charges of the conducting sphere, and a field due to a charge $q$ uniformly distributed over the conducting sphere.

Let $\Phi_{1}$ and $\Phi_{2}$ be the total potentials inside and outside the sphere respectively. Then we have

$$
\nabla^{2} \Phi_{1}=\nabla^{2} \Phi_{2}=0, \quad \Phi_{1}=\Phi_{0},
$$

where $\Phi_{0}$ is a constant. The boundary conditions are

$$
\begin{aligned}
&\Phi_{1}=\Phi_{2}, \text { for } r=a \\
&\Phi_{2}=-E_{0} r P_{1}(\cos \theta) \text { for } r \rightarrow \infty
\end{aligned}
$$

On account of cylindrical symmetry the general solution of Laplace's equation is

$$
\Phi_{2}=\sum_{n}\left(a_{n} r^{n}+\frac{b_{n}}{r^{n+1}}\right) P_{n}(\cos \theta) .
$$

Inserting the above boundary conditions, we find

$$
a_{1}=-E_{0}, \quad b_{0}=a \Phi_{0}, \quad b_{1}=E_{0} a^{3},
$$

while all other coefficients are zero. As $\sigma=-\varepsilon_{0}\left(\frac{\partial \Phi_{2}}{\partial r}\right)_{r=a}$, we have

$$
q=\int_{0}^{\pi}\left(3 \varepsilon_{0} E_{0} \cos \theta+\frac{\varepsilon_{0} \Phi_{0}}{a}\right) 2 \pi a^{2} \sin \theta d \theta=4 \pi a \varepsilon_{0} \Phi_{0}
$$

or

$$
\Phi_{0}=\frac{q}{4 \pi \varepsilon_{0} a} .
$$

So the potentials inside and outside the sphere are

$$
\begin{aligned}
&\Phi_{1}=\frac{q}{4 \pi \varepsilon_{0} a}, \quad(r<a), \\
&\Phi_{2}=-E_{0} r \cos \theta+\frac{q}{4 \pi \varepsilon_{0} r}+\frac{E_{0} a^{3}}{r^{2}} \cos \theta, \quad(r>a) .
\end{aligned}
$$

The field outside the sphere may be considered as the superposition of three fields with contributions to the potential $\Phi_{2}$ equal to the three terms on the right-hand side of the last expression: the uniform field $\mathbf{E}_{0}$, a field due to the charge $q$ uniformly distributed over the sphere, and a dipole field due to charges induced on the surface of the sphere. The last is that which would be produced by a dipole of moment $P=4 \pi \varepsilon_{0} a^{3} E_{0}$ located at the spherical center.

The energies of these three fields may be divided into two kinds: electrostatic energy produced by each field alone, interaction energies among the fields.

The energy density of the uniform external field $\mathbf{E}_{0}$ is $\frac{c_{0}}{2} E_{0}^{2}$. Its total energy $\int \frac{\varepsilon_{0}}{2} E_{0}^{2} d V$ is infinite, i.e. $W_{1} \rightarrow \infty$, since $\mathbf{E}_{0}$ extends over the entire space.

The total electrostatic energy of an isolated conducting sphere with charge $q$ is

$$
W_{2}=\int_{a}^{\infty} \frac{\varepsilon_{0}}{2}\left(\frac{q}{4 \pi \varepsilon_{0} r^{2}}\right)^{2} 4 \pi r^{2} d r=\frac{q^{2}}{8 \pi \varepsilon_{0} a}
$$

which is finite.

The electric intensity outside the sphere due to the dipole $P$ is

$$
\mathbf{E}_{3}=-\nabla\left(\frac{E_{0} a^{3}}{r^{2}} \cos \theta\right)=\frac{2 a^{3} E_{0} \cos \theta}{r^{3}} \mathbf{e}_{r}+\frac{a^{3} E_{0} \sin \theta}{r^{3}} \mathbf{e}_{\theta} .
$$

The corresponding energy density is

$$
\begin{aligned}
w_{3} &=\frac{1}{2} \varepsilon_{0} E_{3}^{2}=\frac{\varepsilon_{0}}{2} \frac{a^{6} E_{0}^{2}}{r^{6}}\left(4 \cos ^{2} \theta+\sin ^{2} \theta\right) \\
&=\frac{\varepsilon_{0}}{2} \frac{a^{6} E_{0}^{2}}{r^{6}}\left(1+3 \cos ^{2} \theta\right)
\end{aligned}
$$

As the dipole does not give rise to a freld inside the sphere the total electrostatic energy of $\mathbf{P}$ is

$$
\begin{aligned}
W_{3} &=\int w_{3} d V=\frac{\varepsilon_{0} a^{6} E_{0}^{2}}{2} \int_{a}^{\infty} \frac{1}{r^{4}} d r \int_{0}^{2 \pi} d \varphi \int_{-1}^{1}\left(1+3 \cos ^{2} \theta\right) d \cos \theta \\
&=\frac{4 \pi \varepsilon_{0} a^{3}}{3} E_{0}^{2}
\end{aligned}
$$

which is also finite. For the conducting sphere with total charge $q$, its suface charge density is $\sigma=q / 4 \pi a^{2}$. The interaction energy between the sphere and the external field $\mathbb{E}_{0}$ is then

$$
\begin{aligned}
W_{12} &=\int \sigma \cdot\left(-E_{0} a \cos \theta\right) 2 \pi a^{2} \sin \theta d \theta \\
&=\frac{q a E_{0}}{2} \int_{0}^{\pi} \cos \theta d \cos \theta=0
\end{aligned}
$$

Similarly, the interaction energy of the conducting sphere with the field of dipole $\mathbf{P}$ is

$$
W_{23}=\int_{0}^{\pi} \sigma \cdot\left(\frac{E_{0} a^{3}}{r^{2}} \cos \theta\right) 2 \pi a^{2} \sin \theta d \theta=0 .
$$

The interaction energy between dipole $P$ and external field $\mathbf{E}_{0}$ is

$$
W_{13}=-\frac{1}{2} \mathbf{P} \cdot \mathbf{E}_{0}=-2 \pi \varepsilon_{0} a^{3} E_{0}^{2}
$$

which is finite. The appearance of the factor $\frac{1}{2}$ in the expression is due to the fact that the dipole $P$ is just an equivalent dipole induced by the external field $\mathbf{E}_{0}$.

\textbf{Topic} :Electrostatics\\
\textbf{Book} :Problems and Solutions on Electromagnetism\\
\textbf{Final Answer} :-2 \pi \varepsilon_{0} a^{3} E_{0}^{2}\\


\textbf{Solution} :Use spherical coordinates $(r, \theta, \phi)$ with the origin at the spherical center. The $z$-axis is taken perpendicular to the cutting seam of the two hemi-spheres (see Fig. 1.31). It is readily seen that the potential $\phi$ is a function of $r$ and $\theta$ only and satisfies the following 2-dimensional Laplace's equation,

$$
\frac{1}{r^{2}} \frac{\partial}{\partial r}\left(r^{2} \frac{\partial \phi}{\partial r}\right)+\frac{1}{r^{2} \sin \theta} \frac{\partial}{\partial \theta}\left(\sin \theta \frac{\partial \phi}{\partial \theta}\right)=0, \quad(r \geq R) .
$$

MATHPIX IMAGE

Fig. 1.31

The general solution of this equation is

$$
\phi=\sum_{l=0}^{\infty} \frac{A_{l}}{r^{l+1}} P_{l}(\cos \theta), \quad(r \geq R) .
$$

Keeping only terms up to $l=3$ as required, we have

$$
\phi \approx \sum_{l=0}^{3} \frac{A_{l}}{r^{l+1}} P_{l}(\cos \theta) .
$$

The boundary condition at $r=R$ is

$$
\left.\phi\right|_{r=R}=f(\theta)= \begin{cases}\phi_{0}, & \text { for } 0 \leq \theta<\frac{\pi}{2}, \\ 0, & \text { for } \frac{\pi}{2}<\theta \leq \pi .\end{cases}
$$

$f(\theta)$ can be expanded as a series of Legendre polynomials, retaining terms up to $l=3$ :

$$
f(\theta) \approx \sum_{l=0}^{3} B_{l} P_{l}(\cos \theta),
$$

where, making use of the orthogonality of the Legendre polynomials,

$$
\begin{aligned}
B_{l} &=\frac{2 l+1}{2} \int_{0}^{\pi} f(\theta) P_{l}(\cos \theta) \sin \theta d \theta \\
&=\frac{(2 l+1) \phi_{0}}{2} \int_{0}^{1} P_{l}(x) d x .
\end{aligned}
$$

Integrating the first few Legendre polynomials as given, we obtain

$$
\begin{aligned}
&\int_{0}^{1} P_{0}(x) d x=\int_{0}^{1} d x=1 \\
&\int_{0}^{1} P_{2}(x) d x=\int_{0}^{1} x d x=\frac{1}{2}, \\
&\int_{0}^{1} P_{2}(x) d x=\int_{0}^{1} \frac{1}{2}\left(3 x^{2}-1\right) d x=0, \\
&\int_{0}^{1} P_{3}(x) d x=\int_{0}^{1} \frac{1}{2}\left(5 x^{3}-3 x\right) d x=-\frac{1}{8},
\end{aligned}
$$

which in turn give

$$
B_{0}=\frac{1}{2} \phi_{0}, \quad B_{1}=\frac{3}{4} \phi_{0}, \quad B_{2}=0, \quad B_{3}=-\frac{7}{16} \phi_{0} .
$$

From

$$
\left.\phi\right|_{r=R} \approx \sum_{l=0}^{3} \frac{A_{l}}{R^{l+1}} P_{l}=\sum_{l=0}^{3} B_{l} P_{l}
$$

we further get

$$
A_{l}=R^{l+1} B_{l} .
$$

Hence

$$
\begin{aligned}
\phi \approx & \sum_{l=0}^{3} B_{l}\left(\frac{R}{r}\right)^{l+1} P_{l}(\cos \theta)=\phi_{0}\left\{\frac{R}{2 r}+\frac{3}{4}\left(\frac{R}{r}\right)^{2} \cos \theta\right.\\
&\left.-\frac{7}{32}\left(\frac{R}{r}\right)^{4}\left(5 \cos ^{2} \theta-3\right) \cos \theta\right\}
\end{aligned}
$$



\textbf{Topic} :Electrostatics\\
\textbf{Book} :Problems and Solutions on Electromagnetism\\
\textbf{Final Answer} :\phi_{0}\left\{\frac{R}{2 r}+\frac{3}{4}\left(\frac{R}{r}\right)^{2} \cos \theta\right\\
&\left-\frac{7}{32}\left(\frac{R}{r}\right)^{4}\left(5 \cos ^{2} \theta-3\right) \cos \theta\right\}\\


\textbf{Solution} :Applying the cosine theorem to the triangle of Fig. $1.32$ we have to first order in $c$

$$
b^{2}=c^{2}+r^{2}-2 c r \cos \theta \simeq r^{2}-2 c r \cos \theta,
$$

or

$$
r \approx \frac{1}{2}\left(2 c \cos \theta+\sqrt{4 c^{2} \cos ^{2} \theta+b^{2}}\right) \approx b+c \cos \theta .
$$

 Using Laplace's equation $\nabla^{2} \Phi=0$ and the axial symmetry, we can express the potential at a point between the two spheres as

$$
\Phi=\sum_{l=0}^{\infty}\left(A_{r} r^{l}+\frac{B_{l}}{r^{l+1}}\right) P_{l}(\cos \theta) .
$$

Then retaining only the $l=0,1$ angular components, we have,

$$
\Phi=A_{0}+\frac{B_{0}}{r}+\left(A_{1} r+\frac{B_{1}}{r^{2}}\right) \cos \theta .
$$

As the surface of the inner conductor is an equipotential, $\Phi$ for $r=a$ should not depend on $\theta$. Hence

$$
A_{1} a+\frac{B_{1}}{a^{2}}=0 \text {. }
$$

The charge density on the surface of the inner sphere is

$$
\sigma=-\varepsilon_{0}\left(\frac{\partial \Phi}{\partial r}\right)_{r=a}
$$

and we have

$$
\int_{0}^{\pi} \sigma 2 \pi a^{2} \sin \theta d \theta=Q
$$

This gives

$$
B_{0}=\frac{Q}{4 \pi \varepsilon_{0}} .
$$

Then as the outer sphere is grounded, $\Phi=0$ for $r \approx b+c \cos \theta$. This gives

$$
A_{0}+\frac{B_{0}}{b+c \cos \theta}+\left[A_{1}(b+c \cos \theta)+\frac{B_{1}}{(b+c \cos \theta)^{2}}\right] \cos \theta=0 \text {. }
$$

To first order in $c$, we have the approximations

$$
\begin{aligned}
&(b+c \cos \theta)^{-1}=b^{-1}\left(1+\frac{c}{b} \cos \theta\right)^{-1} \approx \frac{1}{b}\left(1-\frac{c}{b} \cos \theta\right) \\
&(b+c \cos \theta)^{-2}=b^{-2}\left(1+\frac{c}{b} \cos \theta\right)^{-2} \approx \frac{1}{b^{2}}\left(1-\frac{2 c}{b} \cos \theta\right)
\end{aligned}
$$

Substituting these expressions in Eq.
(3) gives

$$
A_{0}+\frac{B_{0}}{b}+\left(-\frac{B_{0} c}{b^{2}}+A_{1} b+\frac{B_{1}}{b^{2}}\right) \cos \theta \approx 0,
$$

neglecting $c \cos ^{2} \theta$ and higher order terms. As (4) is valid for whatever value of $\theta$, we require

$$
\begin{array}{r}
A_{0}+\frac{B_{0}}{b}=0 \\
-\frac{B_{0} c}{b^{2}}+A_{1} b+\frac{B_{1}}{b^{2}}=0
\end{array}
$$

The last two equations,
(1) and (2) together give

$$
\begin{gathered}
A_{0}=-\frac{Q}{4 \pi \varepsilon_{0} b}, \\
A_{1}=\frac{Q c}{4 \pi \varepsilon_{0}\left(b^{3}-a^{3}\right)}, \quad B_{1}=-\frac{Q c a^{3}}{4 \pi \varepsilon_{0}\left(b^{3}-a^{3}\right)} .
\end{gathered}
$$

Hence the potential between the two spherical shells is

$$
\Phi=\frac{Q}{4 \pi \varepsilon_{0}}\left\{\frac{1}{r}-\frac{1}{b}+\frac{c r}{b^{3}-a^{3}}\left[1-\left(\frac{a}{r}\right)^{3}\right] \cos \theta\right\} \text {. }
$$

\textbf{Topic} :Electrostatics\\
\textbf{Book} :Problems and Solutions on Electromagnetism\\
\textbf{Final Answer} :\frac{Q}{4 \pi \varepsilon_{0}}\left\{\frac{1}{r}-\frac{1}{b}+\frac{c r}{b^{3}-a^{3}}\left[1-\left(\frac{a}{r}\right)^{3}\right] \cos \theta\right\}\\


\textbf{Solution} :The difference between this case and the case of the cylindrical conductor lies in the fact that $\sigma(\theta)$ can be positive or negative for a conductor, while in this case $\sigma(\theta) \leq 0$. However, when $\left|E_{a}\right|<\left|\frac{\sigma_{n}}{2 \varepsilon_{0}}\right|$, the two problems have the same solution.

 For the case of a conducting cylinder the electrostatic field must satisfy the following:

(1) Inside the conductor $\mathrm{E}_{I}=0$ and $\varphi_{I}$ is a constant.

(2) Outside the conductor

$$
\begin{gathered}
\nabla^{2} \varphi_{\mathrm{II}}=0, \\
\left(\frac{\partial \varphi_{\mathrm{II}}}{\partial \theta}\right)_{\rho=r}=0,\left.\quad \varphi_{\mathrm{II}}\right|_{\rho \rightarrow \infty} \rightarrow-E_{a} \rho \cos \theta .
\end{gathered}
$$

The solution for $\varphi_{I I}$ is the same as before. For the solution of the conductor to fit the case of an insulating cylinder, the necessary condition is $\left|E_{a}\right| \leq\left|\frac{\sigma_{a}}{2 \varepsilon_{0}}\right|$, which ensures that the surface charge density on the cylinder is negative everywhere.

\textbf{Topic} :Electrostatics\\
\textbf{Book} :Problems and Solutions on Electromagnetism\\
\textbf{Final Answer} :0\left\quad \varphi_{\mathrm{II}}\right|_{\rho \rightarrow \infty} \rightarrow-E_{a} \rho \cos \theta 
\end{gathered}\\


\textbf{Solution} :In order that the potential of the metal plane is maintained at zero, we imagine that an infinite straight wire with linear charge density $-\sigma$ is symmetrically placed on the other side of the plane. Then the capacitance between the original wire and the metal plane is that between the two straight wires separated at $2 b$.

The potential $\varphi(r)$ at a point between the two wires at distance $r$ from the original wire (and at distance $2 b-r$ from the image wire) is then

$$
\varphi(r)=\frac{\sigma}{2 \pi} \ln \frac{1}{r}-\frac{\sigma}{2 \pi} \ln \frac{1}{2 b-r} .
$$

So the potential difference between the two wires is

$$
V=\varphi(a)-\varphi(2 b-a)=\frac{\sigma}{\pi} \ln \left(\frac{2 b-a}{a}\right) \approx \frac{\sigma}{\pi} \ln \frac{2 b}{a} .
$$

Thus the capacitance of this system per unit length of the wire is

$$
C=\frac{\sigma}{V}=\pi / \ln \frac{2 b}{a} .
$$

\textbf{Topic} :Electrostatics\\
\textbf{Book} :Problems and Solutions on Electromagnetism\\
\textbf{Final Answer} :\pi / \ln \frac{2 b}{a}\\


\textbf{Solution} :The method of images requires that an image charge $-q$ is placed symmetrically with respect to the plane sheet. This means that the total induced charge on the surface of the conductor is $-q$.
\\
\textbf{Topic} :Electrostatics\\
\textbf{Book} :Problems and Solutions on Electromagnetism\\
\textbf{Final Answer} :\pi / \ln \frac{2 b}{a}\\


\textbf{Solution} :The method of images requires that an image charge $-q$ is placed symmetrically with respect to the plane sheet. This means that the total induced charge on the surface of the conductor is $-q$.
 The force acting on $+q$ is

$$
F=\frac{1}{4 \pi \varepsilon_{0}} \frac{q^{2}}{(2 a)^{2}}=9 \times 10^{9} \times \frac{\left(2 \times 10^{-6}\right)^{2}}{0.2^{2}}=0.9 \mathrm{~N}
$$

where we have used $\varepsilon_{0}=\frac{1}{4 \pi \times 9 \times 10^{9}} C^{2} /\left(\mathrm{N} \cdot \mathrm{m}^{2}\right)$.
\textbf{Topic} :Electrostatics\\
\textbf{Book} :Problems and Solutions on Electromagnetism\\
\textbf{Final Answer} :09 \mathrm{~N}\\


\textbf{Solution} :The method of images requires that an image charge $-q$ is placed symmetrically with respect to the plane sheet. This means that the total induced charge on the surface of the conductor is $-q$.
 The force acting on $+q$ is

$$
F=\frac{1}{4 \pi \varepsilon_{0}} \frac{q^{2}}{(2 a)^{2}}=9 \times 10^{9} \times \frac{\left(2 \times 10^{-6}\right)^{2}}{0.2^{2}}=0.9 \mathrm{~N}
$$

where we have used $\varepsilon_{0}=\frac{1}{4 \pi \times 9 \times 10^{9}} C^{2} /\left(\mathrm{N} \cdot \mathrm{m}^{2}\right)$.

 The total work required to remove the charge to infinity is

$$
W=\int_{a}^{\infty} F d r=\int_{a}^{\infty} \frac{1}{4 \pi \varepsilon_{0}} \frac{q^{2}}{(2 r)^{2}} d r=\frac{q^{2}}{16 \pi \varepsilon_{0} a}=0.09 \mathrm{~J}
$$

\textbf{Topic} :Electrostatics\\
\textbf{Book} :Problems and Solutions on Electromagnetism\\
\textbf{Final Answer} :009 \mathrm{~J}\\


\textbf{Solution} :The method of images requires image charges $+q$ at $(-a, 0,-a)$ and $-q$ at $(a, 0,-a)$ (see Fig. 1.34). The resultant force exerted on $+q$ at $(a, 0, a)$ is thus

$$
\begin{aligned}
\mathbf{F} &=\frac{q^{2}}{4 \pi \varepsilon_{0}}\left[-\frac{1}{(2 a)^{2}} \mathbf{e}_{x}-\frac{1}{(2 a)^{2}} \mathbf{e}_{z}+\frac{1}{(2 \sqrt{2} a)^{2}}\left(\frac{1}{\sqrt{2}} \mathbf{e}_{x}+\frac{1}{\sqrt{2}} \mathbf{e}_{z}\right)\right] \\
&=\frac{q^{2}}{4 \pi \varepsilon_{0} a^{2}}\left[\left(-\frac{1}{4}+\frac{1}{8 \sqrt{2}}\right) \mathbf{e}_{x}+\left(-\frac{1}{4}+\frac{1}{8 \sqrt{2}}\right) \mathbf{e}_{z}\right] .
\end{aligned}
$$

This force has magnitude

$$
F=\frac{(\sqrt{2}-1) q^{2}}{32 \pi \varepsilon_{0} a^{2}} .
$$



It is in the $x z$-plane and points to the origin along a direction at angle $45^{\circ}$ to the $x$-axis as shown in Fig. 1.34.

MATHPIX IMAGE

Fig. 1.34
\textbf{Topic} :Electrostatics\\
\textbf{Book} :Problems and Solutions on Electromagnetism\\
\textbf{Final Answer} :\frac{(\sqrt{2}-1) q^{2}}{32 \pi \varepsilon_{0} a^{2}}\\


\textbf{Solution} :The method of images requires image charges $+q$ at $(-a, 0,-a)$ and $-q$ at $(a, 0,-a)$ (see Fig. 1.34). The resultant force exerted on $+q$ at $(a, 0, a)$ is thus

$$
\begin{aligned}
\mathbf{F} &=\frac{q^{2}}{4 \pi \varepsilon_{0}}\left[-\frac{1}{(2 a)^{2}} \mathbf{e}_{x}-\frac{1}{(2 a)^{2}} \mathbf{e}_{z}+\frac{1}{(2 \sqrt{2} a)^{2}}\left(\frac{1}{\sqrt{2}} \mathbf{e}_{x}+\frac{1}{\sqrt{2}} \mathbf{e}_{z}\right)\right] \\
&=\frac{q^{2}}{4 \pi \varepsilon_{0} a^{2}}\left[\left(-\frac{1}{4}+\frac{1}{8 \sqrt{2}}\right) \mathbf{e}_{x}+\left(-\frac{1}{4}+\frac{1}{8 \sqrt{2}}\right) \mathbf{e}_{z}\right] .
\end{aligned}
$$

This force has magnitude

$$
F=\frac{(\sqrt{2}-1) q^{2}}{32 \pi \varepsilon_{0} a^{2}} .
$$



It is in the $x z$-plane and points to the origin along a direction at angle $45^{\circ}$ to the $x$-axis as shown in Fig. 1.34.

MATHPIX IMAGE

Fig. 1.34

 We can construct the system by slowly bringing the charges $+q$ and $-q$ from infinity by the paths

$$
\begin{aligned}
&L_{1}: z=x, y=0, \\
&L_{2}: z=-x, y=0,
\end{aligned}
$$

symmetrically to the points $(a, 0, a)$ and $(-a, 0, a)$ respectively. When the charges are at $(l, 0, l)$ on path $L_{1}$ and $(-l, 0, l)$ on path $L_{2}$ respectively, each suffers a force $\frac{(\sqrt{2}-1) q^{2}}{32 \pi \epsilon^{2}}$ whose direction is parallel to the direction of the path so that the total work done by the external forces is

$$
W=-2 \int_{\infty}^{a} F d l=2 \int_{a}^{\infty} \frac{(\sqrt{2}-1) q^{2}}{32 \pi \varepsilon_{0} l^{2}} d l=\frac{(\sqrt{2}-1) q^{2}}{16 \pi \varepsilon_{0} a} .
$$
\textbf{Topic} :Electrostatics\\
\textbf{Book} :Problems and Solutions on Electromagnetism\\
\textbf{Final Answer} :\frac{(\sqrt{2}-1) q^{2}}{16 \pi \varepsilon_{0} a}\\


\textbf{Solution} :The method of images requires image charges $+q$ at $(-a, 0,-a)$ and $-q$ at $(a, 0,-a)$ (see Fig. 1.34). The resultant force exerted on $+q$ at $(a, 0, a)$ is thus

$$
\begin{aligned}
\mathbf{F} &=\frac{q^{2}}{4 \pi \varepsilon_{0}}\left[-\frac{1}{(2 a)^{2}} \mathbf{e}_{x}-\frac{1}{(2 a)^{2}} \mathbf{e}_{z}+\frac{1}{(2 \sqrt{2} a)^{2}}\left(\frac{1}{\sqrt{2}} \mathbf{e}_{x}+\frac{1}{\sqrt{2}} \mathbf{e}_{z}\right)\right] \\
&=\frac{q^{2}}{4 \pi \varepsilon_{0} a^{2}}\left[\left(-\frac{1}{4}+\frac{1}{8 \sqrt{2}}\right) \mathbf{e}_{x}+\left(-\frac{1}{4}+\frac{1}{8 \sqrt{2}}\right) \mathbf{e}_{z}\right] .
\end{aligned}
$$

This force has magnitude

$$
F=\frac{(\sqrt{2}-1) q^{2}}{32 \pi \varepsilon_{0} a^{2}} .
$$



It is in the $x z$-plane and points to the origin along a direction at angle $45^{\circ}$ to the $x$-axis as shown in Fig. 1.34.

MATHPIX IMAGE

Fig. 1.34

 We can construct the system by slowly bringing the charges $+q$ and $-q$ from infinity by the paths

$$
\begin{aligned}
&L_{1}: z=x, y=0, \\
&L_{2}: z=-x, y=0,
\end{aligned}
$$

symmetrically to the points $(a, 0, a)$ and $(-a, 0, a)$ respectively. When the charges are at $(l, 0, l)$ on path $L_{1}$ and $(-l, 0, l)$ on path $L_{2}$ respectively, each suffers a force $\frac{(\sqrt{2}-1) q^{2}}{32 \pi \epsilon^{2}}$ whose direction is parallel to the direction of the path so that the total work done by the external forces is

$$
W=-2 \int_{\infty}^{a} F d l=2 \int_{a}^{\infty} \frac{(\sqrt{2}-1) q^{2}}{32 \pi \varepsilon_{0} l^{2}} d l=\frac{(\sqrt{2}-1) q^{2}}{16 \pi \varepsilon_{0} a} .
$$

 Consider the electric field at a point $\left(a, 0,0^{+}\right)$just above the conducting plane. The resultant field intensity $\mathbf{E}_{1}$ produced by $+q$ at $(a, 0, a)$ and $-q$ at $(a, 0,-a)$ is

$$
\mathbf{E}_{1}=-\frac{2 q}{4 \pi \varepsilon_{0} a^{2}} \mathbf{e}_{z} .
$$

The resultant field $\mathrm{E}_{2}$ produced by $-q$ at $(-a, 0, a)$ and $+q$ at $(-a, 0,-a)$ is

$$
\mathbf{E}_{2}=\frac{2 q}{4 \pi \varepsilon_{0} a^{2}} \frac{1}{5 \sqrt{5}} \mathbf{e}_{z} .
$$

Hence the total field at $\left(a, 0,0^{+}\right)$is

$$
\mathbf{E}=\mathbf{E}_{1}+\mathbf{E}_{2}=\frac{q}{2 \pi \varepsilon_{0} a^{2}}\left(\frac{1}{5 \sqrt{5}}-1\right) \mathbf{e}_{z},
$$

and the surface charge density at this point is

$$
\sigma=\varepsilon_{0} E=\frac{q}{2 \pi \varepsilon_{0} a^{2}}\left(\frac{1}{5 \sqrt{5}}-1\right)
$$

\textbf{Topic} :Electrostatics\\
\textbf{Book} :Problems and Solutions on Electromagnetism\\
\textbf{Final Answer} :\frac{q}{2 \pi \varepsilon_{0} a^{2}}\left(\frac{1}{5 \sqrt{5}}-1\right)\\


\textbf{Solution} :Consider first the simple case where a point charge $q_{1}$ is placed at $(0,0, a)$. The method of images requires image charges $q_{1}^{\prime}$ at $(0,0,-a)$ and $q_{1}^{\prime \prime}$ at $(0,0, a)$. Then the potential (in Gaussian units) at a point $(x, y, z)$ is given by

$$
\varphi_{1}=\frac{q_{1}}{\varepsilon_{1} r_{1}}+\frac{q_{1}^{\prime}}{\varepsilon_{1} r_{2}}, \quad(z \geq 0), \quad \varphi_{2}=\frac{q_{1}^{\prime \prime}}{\varepsilon_{2} r_{1}}, \quad(z<0)
$$

where

$$
r_{1}=\sqrt{x^{2}+y^{2}+(z-a)^{2}}, \quad r_{2}=\sqrt{x^{2}+y^{2}+(z+a)^{2}} .
$$

Applying the boundary conditions at $(x, y, 0)$ :

$$
\varphi_{1}=\varphi_{2}, \quad \varepsilon_{1} \frac{\partial \varphi_{1}}{\partial z}=\varepsilon_{2} \frac{\partial \varphi_{2}}{\partial z},
$$

we obtain

$$
q_{1}^{\prime}=q_{1}^{\prime \prime}=\frac{\varepsilon_{1}-\varepsilon_{0}}{\varepsilon_{1}\left(\varepsilon_{1}+\varepsilon_{2}\right)} q_{1} .
$$

Similarly, if a point charge $q_{2}$ is placed at $(0,0,-a)$ inside the dielectic $\varepsilon_{2}$, its image charges will be $q_{2}^{\prime}$ at $(0,0, a)$ and $q_{2}^{\prime \prime}$ at $(0,0,-a)$ with magnitudes

$$
q_{2}^{\prime}=q_{2}^{\prime \prime}=\frac{\varepsilon_{2}-\varepsilon_{1}}{\varepsilon_{2}\left(\varepsilon_{1}+\varepsilon_{2}\right)} q_{2} .
$$

When both $q_{1}$ and $q_{2}$ exist, the force on $q_{1}$ will be the resultant due to $q_{2}, q_{1}^{\prime}$ and $q_{1}^{\prime \prime}$. It follows that

$$
\begin{aligned}
F &=\frac{q_{1} q_{1}^{\prime}}{4 a^{2} \varepsilon_{1}}+\frac{q_{1} q_{2}}{4 a^{2} \varepsilon_{2}}+\frac{q_{1} q_{1}^{\prime \prime}}{4 a^{2} \varepsilon_{2}} \\
&=\frac{\varepsilon_{1}-\varepsilon_{2}}{\varepsilon_{1}\left(\varepsilon_{1}+\varepsilon_{2}\right)} \cdot \frac{q_{1}^{2}}{4 a^{2}}+\frac{q_{1} q_{2}}{2\left(\varepsilon_{1}+\varepsilon_{2}\right) a^{2}} .
\end{aligned}
$$

In the present problem $q_{1}=-q, q_{2}=+q$, and one has

$$
F=\frac{\varepsilon_{1}-\varepsilon_{2}}{\varepsilon_{1}\left(\varepsilon_{1}+\varepsilon_{2}\right)} \frac{q^{2}}{4 a^{2}}-\frac{q^{2}}{2\left(\varepsilon_{1}+\varepsilon_{2}\right) a^{2}}=-\frac{q^{2}}{4 \varepsilon_{1} a^{2}} .
$$

Hence, a force $-F$ is required to keep on $-q$ at rest.

\textbf{Topic} :Electrostatics\\
\textbf{Book} :Problems and Solutions on Electromagnetism\\
\textbf{Final Answer} :-\frac{q^{2}}{4 \varepsilon_{1} a^{2}}\\


\textbf{Solution} :We use the method of images. The electric field outside the sphere corresponds to the resultant field of the two given charges $+q$ and two image charges $+q^{\prime}$. The magnitudes of the image charges are both $q^{\prime}=-q \frac{a}{b}$, and they are to be placed at two sides of the center of the sphere at the same distance $b^{\prime}=\frac{a^{2}}{b}$ from it (see Fig. 1.37). 

MATHPIX IMAGE

Fig. 1.37

For each given charge $+q$, apart from the electric repulsion acted on it by the other given charge $+q$, there is also the attraction exerted by the two image charges. For the resultant force to vanish we require

$$
\begin{aligned}
\frac{q^{2}}{4 b^{2}} &=\frac{q^{2} \frac{a}{b}}{\left(b-\frac{a^{2}}{b}\right)^{2}}+\frac{q^{2} \frac{a}{b}}{\left(b+\frac{a^{2}}{b}\right)^{2}} \\
&=\frac{2 q^{2} a}{b^{3}}\left[1+3\left(\frac{a}{b}\right)^{4}+5\left(\frac{a}{b}\right)^{8}+\ldots\right] \approx \frac{2 q^{2} a}{b^{3}},
\end{aligned}
$$

The value of $a(a<b)$ that satisfies the above requirement is therefore approximately

$$
a \approx \frac{b}{8} .
$$

\textbf{Topic} :Electrostatics\\
\textbf{Book} :Problems and Solutions on Electromagnetism\\
\textbf{Final Answer} :-\frac{q^{2}}{4 \varepsilon_{1} a^{2}}\\


\textbf{Solution} :Apply the method of images and let the distance between $q$ and the center of the shell be $a$. Then an image charge $q^{\prime}=-\frac{r_{1}}{a} q$ is to be placed at $b=\frac{r_{1}^{2}}{a}$ (see Fig. 1.38). Since the conductor is isolated and uncharged, it is an equipotential body with potential $\varphi=\varphi_{0}$, say. Then the electric field inside the shell $\left(r<r_{1}\right)$ equals the field created by $q$ and $q^{\prime}$.

MATHPIX IMAGE

Fig. $1.38$

The force on the charge $q$ is that exerted by $q^{\prime}$ :

$$
\begin{aligned}
F &=\frac{q q^{\prime}}{4 \pi \varepsilon_{0}(b-a)^{2}}=-\frac{\frac{r_{1}}{a} q^{2}}{4 \pi \varepsilon_{0}\left(\frac{r_{1}^{2}}{a}-a\right)^{2}} \\
&=-\frac{a r_{1} q^{2}}{4 \pi \varepsilon_{0}\left(r_{1}^{2}-a^{2}\right)^{2}}
\end{aligned}
$$

In zone $r>r_{2}$ the potential is $\varphi_{\text {out }}=\frac{q}{4 \pi \varepsilon_{0} r}$. In particular, the potential of the conducting sphere at $r=r_{2}$ is

$$
\varphi_{\text {sphere }}=\frac{q}{4 \pi \varepsilon_{0} r_{2}} \text {. }
$$

Owing to the conductor being an equipotential body, the potential of the inner surface of the conducting shell is also $\frac{q}{4 \pi \varepsilon_{0} r_{2}}$.


\textbf{Topic} :Electrostatics\\
\textbf{Book} :Problems and Solutions on Electromagnetism\\
\textbf{Final Answer} :\frac{q}{4 \pi \varepsilon_{0} r_{2}}\\


\textbf{Solution} :We use the method of images. The positions of the image charges are shown in Fig. 1.41. Consider an arbitrary point $A$ on the lower plate. Choose the $x z$-plane to contain $A$. It can be seen that the electric field at $A$, which is at the surface of a conductor, has only the $z$-component and its magnitude is (letting $d=\frac{D}{2}$ )

$$
\begin{aligned}
E_{z}=& \frac{Q}{4 \pi \varepsilon_{0}\left(d^{2}+x^{2}\right)} \cdot \frac{2 d}{\left(d^{2}+x^{2}\right)^{1 / 2}} \\
&-\frac{Q}{4 \pi \varepsilon_{0}\left[(3 d)^{2}+x^{2}\right]} \cdot \frac{2 \cdot 3 d}{\left[(3 d)^{2}+x^{2}\right]^{1 / 2}} \\
&+\frac{Q}{4 \pi \varepsilon_{0}\left[(5 d)^{2}+x^{2}\right]} \cdot \frac{2 \cdot 5 d}{\left[(5 d)^{2}+x^{2}\right]^{1 / 2}}-\cdots \\
=& \frac{Q D}{4 \pi \varepsilon_{0}} \sum_{n=0}^{\infty} \frac{(-1)^{n}(2 n+1)}{\left[\left(n+\frac{1}{2}\right)^{2} D^{2}+x^{2}\right]^{3 / 2}} .
\end{aligned}
$$

Accordingly,

$$
\sigma(x)=-\varepsilon_{0} E_{z}=-\frac{Q D}{4 \pi} \sum_{n=0}^{\infty} \frac{(-1)^{n}(2 n+1)}{\left[\left(n+\frac{1}{2}\right)^{2} D^{2}+x^{2}\right]^{3 / 2}} .
$$



MATHPIX IMAGE

Fig. $1.41$

\textbf{Topic} :Electrostatics\\
\textbf{Book} :Problems and Solutions on Electromagnetism\\
\textbf{Final Answer} :0}^{\infty} \frac{(-1)^{n}(2 n+1)}{\left[\left(n+\frac{1}{2}\right)^{2} D^{2}+x^{2}\right]^{3 / 2}}\\


\textbf{Solution} :The potential is found by the method of images, which requires image charges $+Q$ at $\cdots-9 x,-5 x, 3 x, 7 x, 11 x \cdots$ and image charges $-Q$ at $\cdots-7 x,-3 x, 5 x, 9 x, 13 x, \cdots$ along the $x$-axis as shown in Fig. 1.43. Then the charge density of the system of real and image charges can be expressed as

$$
\rho=\sum_{k=-\infty}^{\infty}(-1)^{k+1} Q \delta[x-(2 k+1) x]
$$

where $\delta$ is the one-dimensional Dirac delta function. 

MATHPIX IMAGE

Fig. $1.42$

MATHPIX IMAGE

Fig. $1.43$

The electrostatic field energy of the system is

$$
W=\frac{1}{2} \sum Q U=\frac{1}{2} Q U_{+}-\frac{1}{2} Q U_{-},
$$

where $U_{+}$is the potential at the $+Q$ charge produced by the other real and image charges not including the $+Q$ itself, while $U_{-}$is the potential at the $-Q$ charge produced by the other real and image charges not including the $-Q$ itself. As

$$
\begin{aligned}
U_{+} &=\frac{1}{4 \pi \varepsilon_{0}}\left[-\frac{2 Q}{(2 x)}+\frac{2 Q}{(4 x)}-\frac{2 Q}{(6 x)}+\cdots\right] \\
&=\frac{-Q}{4 \pi \varepsilon_{0} x} \cdot \sum_{k=1}^{+\infty}(-1)^{k-1} \cdot \frac{1}{k}=\frac{-Q \ln 2}{4 \pi \varepsilon_{0} x}, \\
U_{-} &=-U_{+}=\frac{Q \ln 2}{4 \pi \varepsilon_{0} x},
\end{aligned}
$$

we have

$$
W=-\frac{Q^{2}}{4 \pi \varepsilon_{0} x} \ln 2 .
$$

Hence the energy required to remove the two charges to infinite distances from the plates and from each other is $-W$.

 The force acting on $+Q$ is just that exerted by the fields of all the other real and image charges produced by $Q$. Because of symmetry this force is equal to zero. Similarly the force on $-Q$ is also zero.
\textbf{Topic} :Electrostatics\\
\textbf{Book} :Problems and Solutions on Electromagnetism\\
\textbf{Final Answer} :-\frac{Q^{2}}{4 \pi \varepsilon_{0} x} \ln 2\\


\textbf{Solution} :The potential is found by the method of images, which requires image charges $+Q$ at $\cdots-9 x,-5 x, 3 x, 7 x, 11 x \cdots$ and image charges $-Q$ at $\cdots-7 x,-3 x, 5 x, 9 x, 13 x, \cdots$ along the $x$-axis as shown in Fig. 1.43. Then the charge density of the system of real and image charges can be expressed as

$$
\rho=\sum_{k=-\infty}^{\infty}(-1)^{k+1} Q \delta[x-(2 k+1) x]
$$

where $\delta$ is the one-dimensional Dirac delta function. 

MATHPIX IMAGE

Fig. $1.42$

MATHPIX IMAGE

Fig. $1.43$

The electrostatic field energy of the system is

$$
W=\frac{1}{2} \sum Q U=\frac{1}{2} Q U_{+}-\frac{1}{2} Q U_{-},
$$

where $U_{+}$is the potential at the $+Q$ charge produced by the other real and image charges not including the $+Q$ itself, while $U_{-}$is the potential at the $-Q$ charge produced by the other real and image charges not including the $-Q$ itself. As

$$
\begin{aligned}
U_{+} &=\frac{1}{4 \pi \varepsilon_{0}}\left[-\frac{2 Q}{(2 x)}+\frac{2 Q}{(4 x)}-\frac{2 Q}{(6 x)}+\cdots\right] \\
&=\frac{-Q}{4 \pi \varepsilon_{0} x} \cdot \sum_{k=1}^{+\infty}(-1)^{k-1} \cdot \frac{1}{k}=\frac{-Q \ln 2}{4 \pi \varepsilon_{0} x}, \\
U_{-} &=-U_{+}=\frac{Q \ln 2}{4 \pi \varepsilon_{0} x},
\end{aligned}
$$

we have

$$
W=-\frac{Q^{2}}{4 \pi \varepsilon_{0} x} \ln 2 .
$$

Hence the energy required to remove the two charges to infinite distances from the plates and from each other is $-W$.

 The force acting on $+Q$ is just that exerted by the fields of all the other real and image charges produced by $Q$. Because of symmetry this force is equal to zero. Similarly the force on $-Q$ is also zero.

 Consider the force exerted on the left conducting plate. This is the resultant of all the forces acting on the image charges of the left plate (i.e., image charges to the left of the left plate) by the real charges $+Q,-Q$ and all the image charges of the right plate (i.e., image charges to the right side of right plate).

Let us consider first the force $F_{1}$ acting on the image charges of the left plate by the real charge $+Q$

$$
F_{1}=\frac{Q^{2}}{4 \pi \varepsilon_{0}(2 x)^{2}}-\frac{Q^{2}}{4 \pi \varepsilon_{0}(4 x)^{2}}+\cdots=\frac{Q^{2}}{16 \pi \varepsilon_{0} x^{2}} \sum_{n=1}^{\infty} \frac{(-1)^{n-1}}{n^{2}}
$$

taking the direction along $+x$ as positive.

We next find the force $F_{2}$ between the real charge $-Q$ and the image charges of the left plate:

$$
F_{2}=-\frac{Q^{2}}{4 \pi \varepsilon_{0}(4 x)^{2}}+\frac{Q^{2}}{4 \pi \varepsilon_{0}(6 x)^{2}}-\cdots=\frac{Q^{2}}{16 \pi \varepsilon_{0} x^{2}} \sum_{n=2}^{\infty} \frac{(-1)^{n-1}}{n^{2}} .
$$

Finally consider the force $F_{3}$ acting on the image charges of the left plate by the image charges of the right plate:

$$
F_{3}=\frac{Q^{2}}{16 \pi \varepsilon_{0} x^{2}} \sum_{m=3}^{\infty} \sum_{n=m}^{\infty} \frac{(-1)^{n-1}}{n^{2}} .
$$

Thus the total force exerted on the left plate is

$$
\begin{aligned}
F &=F_{1}+F_{2}+F_{3}=\frac{Q^{2}}{16 \pi \varepsilon_{0} x^{2}} \sum_{m=1}^{\infty} \sum_{n=m}^{\infty} \frac{(-1)^{n}}{n^{2}} \\
&=\frac{Q^{2}}{16 \pi \varepsilon_{0} x^{2}}\left(1-\frac{2}{2^{2}}+\frac{3}{3^{2}}-\frac{4}{4^{2}}+\cdots\right)=\frac{Q^{2}}{16 \pi \varepsilon_{0} x^{2}} \sum_{n=1}^{\infty} \frac{(-1)^{n-1}}{n^{2}} .
\end{aligned}
$$

Using the identity

$$
\sum_{n=1}^{\infty} \frac{(-1)^{n-1}}{n}=\ln 2
$$

we obtain

$$
F=\frac{Q^{2}}{16 \pi \varepsilon_{0} x^{2}} \ln 2 .
$$

This force directs to the right. In a similar manner, we can show that the magnitude of the force exerted on the right plate is also equal to $\frac{Q^{2}}{16 \pi \varepsilon_{0} x^{2}} \ln 2$, its direction being towards the left.
\\
\textbf{Topic} :Electrostatics\\
\textbf{Book} :Problems and Solutions on Electromagnetism\\
\textbf{Final Answer} :\frac{Q^{2}}{16 \pi \varepsilon_{0} x^{2}} \ln 2\\


\textbf{Solution} :The potential is found by the method of images, which requires image charges $+Q$ at $\cdots-9 x,-5 x, 3 x, 7 x, 11 x \cdots$ and image charges $-Q$ at $\cdots-7 x,-3 x, 5 x, 9 x, 13 x, \cdots$ along the $x$-axis as shown in Fig. 1.43. Then the charge density of the system of real and image charges can be expressed as

$$
\rho=\sum_{k=-\infty}^{\infty}(-1)^{k+1} Q \delta[x-(2 k+1) x]
$$

where $\delta$ is the one-dimensional Dirac delta function. 

MATHPIX IMAGE

Fig. $1.42$

MATHPIX IMAGE

Fig. $1.43$

The electrostatic field energy of the system is

$$
W=\frac{1}{2} \sum Q U=\frac{1}{2} Q U_{+}-\frac{1}{2} Q U_{-},
$$

where $U_{+}$is the potential at the $+Q$ charge produced by the other real and image charges not including the $+Q$ itself, while $U_{-}$is the potential at the $-Q$ charge produced by the other real and image charges not including the $-Q$ itself. As

$$
\begin{aligned}
U_{+} &=\frac{1}{4 \pi \varepsilon_{0}}\left[-\frac{2 Q}{(2 x)}+\frac{2 Q}{(4 x)}-\frac{2 Q}{(6 x)}+\cdots\right] \\
&=\frac{-Q}{4 \pi \varepsilon_{0} x} \cdot \sum_{k=1}^{+\infty}(-1)^{k-1} \cdot \frac{1}{k}=\frac{-Q \ln 2}{4 \pi \varepsilon_{0} x}, \\
U_{-} &=-U_{+}=\frac{Q \ln 2}{4 \pi \varepsilon_{0} x},
\end{aligned}
$$

we have

$$
W=-\frac{Q^{2}}{4 \pi \varepsilon_{0} x} \ln 2 .
$$

Hence the energy required to remove the two charges to infinite distances from the plates and from each other is $-W$.

 The force acting on $+Q$ is just that exerted by the fields of all the other real and image charges produced by $Q$. Because of symmetry this force is equal to zero. Similarly the force on $-Q$ is also zero.

 Consider the force exerted on the left conducting plate. This is the resultant of all the forces acting on the image charges of the left plate (i.e., image charges to the left of the left plate) by the real charges $+Q,-Q$ and all the image charges of the right plate (i.e., image charges to the right side of right plate).

Let us consider first the force $F_{1}$ acting on the image charges of the left plate by the real charge $+Q$

$$
F_{1}=\frac{Q^{2}}{4 \pi \varepsilon_{0}(2 x)^{2}}-\frac{Q^{2}}{4 \pi \varepsilon_{0}(4 x)^{2}}+\cdots=\frac{Q^{2}}{16 \pi \varepsilon_{0} x^{2}} \sum_{n=1}^{\infty} \frac{(-1)^{n-1}}{n^{2}}
$$

taking the direction along $+x$ as positive.

We next find the force $F_{2}$ between the real charge $-Q$ and the image charges of the left plate:

$$
F_{2}=-\frac{Q^{2}}{4 \pi \varepsilon_{0}(4 x)^{2}}+\frac{Q^{2}}{4 \pi \varepsilon_{0}(6 x)^{2}}-\cdots=\frac{Q^{2}}{16 \pi \varepsilon_{0} x^{2}} \sum_{n=2}^{\infty} \frac{(-1)^{n-1}}{n^{2}} .
$$

Finally consider the force $F_{3}$ acting on the image charges of the left plate by the image charges of the right plate:

$$
F_{3}=\frac{Q^{2}}{16 \pi \varepsilon_{0} x^{2}} \sum_{m=3}^{\infty} \sum_{n=m}^{\infty} \frac{(-1)^{n-1}}{n^{2}} .
$$

Thus the total force exerted on the left plate is

$$
\begin{aligned}
F &=F_{1}+F_{2}+F_{3}=\frac{Q^{2}}{16 \pi \varepsilon_{0} x^{2}} \sum_{m=1}^{\infty} \sum_{n=m}^{\infty} \frac{(-1)^{n}}{n^{2}} \\
&=\frac{Q^{2}}{16 \pi \varepsilon_{0} x^{2}}\left(1-\frac{2}{2^{2}}+\frac{3}{3^{2}}-\frac{4}{4^{2}}+\cdots\right)=\frac{Q^{2}}{16 \pi \varepsilon_{0} x^{2}} \sum_{n=1}^{\infty} \frac{(-1)^{n-1}}{n^{2}} .
\end{aligned}
$$

Using the identity

$$
\sum_{n=1}^{\infty} \frac{(-1)^{n-1}}{n}=\ln 2
$$

we obtain

$$
F=\frac{Q^{2}}{16 \pi \varepsilon_{0} x^{2}} \ln 2 .
$$

This force directs to the right. In a similar manner, we can show that the magnitude of the force exerted on the right plate is also equal to $\frac{Q^{2}}{16 \pi \varepsilon_{0} x^{2}} \ln 2$, its direction being towards the left.
 The potential on the plane $x=0$ is zero, so only half of the lines of force emerging from the $+Q$ charge reach the left plate, while those emerging from the $-Q$ charge cannot reach the left plate at all. Therefore, the total induced charge on the left plate is $-\frac{Q}{2}$, and similarly that of the right plate is $\frac{Q}{2}$.
\textbf{Topic} :Electrostatics\\
\textbf{Book} :Problems and Solutions on Electromagnetism\\
\textbf{Final Answer} :\frac{Q^{2}}{16 \pi \varepsilon_{0} x^{2}} \ln 2\\


\textbf{Solution} :The potential is found by the method of images, which requires image charges $+Q$ at $\cdots-9 x,-5 x, 3 x, 7 x, 11 x \cdots$ and image charges $-Q$ at $\cdots-7 x,-3 x, 5 x, 9 x, 13 x, \cdots$ along the $x$-axis as shown in Fig. 1.43. Then the charge density of the system of real and image charges can be expressed as

$$
\rho=\sum_{k=-\infty}^{\infty}(-1)^{k+1} Q \delta[x-(2 k+1) x]
$$

where $\delta$ is the one-dimensional Dirac delta function. 

MATHPIX IMAGE

Fig. $1.42$

MATHPIX IMAGE

Fig. $1.43$

The electrostatic field energy of the system is

$$
W=\frac{1}{2} \sum Q U=\frac{1}{2} Q U_{+}-\frac{1}{2} Q U_{-},
$$

where $U_{+}$is the potential at the $+Q$ charge produced by the other real and image charges not including the $+Q$ itself, while $U_{-}$is the potential at the $-Q$ charge produced by the other real and image charges not including the $-Q$ itself. As

$$
\begin{aligned}
U_{+} &=\frac{1}{4 \pi \varepsilon_{0}}\left[-\frac{2 Q}{(2 x)}+\frac{2 Q}{(4 x)}-\frac{2 Q}{(6 x)}+\cdots\right] \\
&=\frac{-Q}{4 \pi \varepsilon_{0} x} \cdot \sum_{k=1}^{+\infty}(-1)^{k-1} \cdot \frac{1}{k}=\frac{-Q \ln 2}{4 \pi \varepsilon_{0} x}, \\
U_{-} &=-U_{+}=\frac{Q \ln 2}{4 \pi \varepsilon_{0} x},
\end{aligned}
$$

we have

$$
W=-\frac{Q^{2}}{4 \pi \varepsilon_{0} x} \ln 2 .
$$

Hence the energy required to remove the two charges to infinite distances from the plates and from each other is $-W$.

 The force acting on $+Q$ is just that exerted by the fields of all the other real and image charges produced by $Q$. Because of symmetry this force is equal to zero. Similarly the force on $-Q$ is also zero.

 Consider the force exerted on the left conducting plate. This is the resultant of all the forces acting on the image charges of the left plate (i.e., image charges to the left of the left plate) by the real charges $+Q,-Q$ and all the image charges of the right plate (i.e., image charges to the right side of right plate).

Let us consider first the force $F_{1}$ acting on the image charges of the left plate by the real charge $+Q$

$$
F_{1}=\frac{Q^{2}}{4 \pi \varepsilon_{0}(2 x)^{2}}-\frac{Q^{2}}{4 \pi \varepsilon_{0}(4 x)^{2}}+\cdots=\frac{Q^{2}}{16 \pi \varepsilon_{0} x^{2}} \sum_{n=1}^{\infty} \frac{(-1)^{n-1}}{n^{2}}
$$

taking the direction along $+x$ as positive.

We next find the force $F_{2}$ between the real charge $-Q$ and the image charges of the left plate:

$$
F_{2}=-\frac{Q^{2}}{4 \pi \varepsilon_{0}(4 x)^{2}}+\frac{Q^{2}}{4 \pi \varepsilon_{0}(6 x)^{2}}-\cdots=\frac{Q^{2}}{16 \pi \varepsilon_{0} x^{2}} \sum_{n=2}^{\infty} \frac{(-1)^{n-1}}{n^{2}} .
$$

Finally consider the force $F_{3}$ acting on the image charges of the left plate by the image charges of the right plate:

$$
F_{3}=\frac{Q^{2}}{16 \pi \varepsilon_{0} x^{2}} \sum_{m=3}^{\infty} \sum_{n=m}^{\infty} \frac{(-1)^{n-1}}{n^{2}} .
$$

Thus the total force exerted on the left plate is

$$
\begin{aligned}
F &=F_{1}+F_{2}+F_{3}=\frac{Q^{2}}{16 \pi \varepsilon_{0} x^{2}} \sum_{m=1}^{\infty} \sum_{n=m}^{\infty} \frac{(-1)^{n}}{n^{2}} \\
&=\frac{Q^{2}}{16 \pi \varepsilon_{0} x^{2}}\left(1-\frac{2}{2^{2}}+\frac{3}{3^{2}}-\frac{4}{4^{2}}+\cdots\right)=\frac{Q^{2}}{16 \pi \varepsilon_{0} x^{2}} \sum_{n=1}^{\infty} \frac{(-1)^{n-1}}{n^{2}} .
\end{aligned}
$$

Using the identity

$$
\sum_{n=1}^{\infty} \frac{(-1)^{n-1}}{n}=\ln 2
$$

we obtain

$$
F=\frac{Q^{2}}{16 \pi \varepsilon_{0} x^{2}} \ln 2 .
$$

This force directs to the right. In a similar manner, we can show that the magnitude of the force exerted on the right plate is also equal to $\frac{Q^{2}}{16 \pi \varepsilon_{0} x^{2}} \ln 2$, its direction being towards the left.
 The potential on the plane $x=0$ is zero, so only half of the lines of force emerging from the $+Q$ charge reach the left plate, while those emerging from the $-Q$ charge cannot reach the left plate at all. Therefore, the total induced charge on the left plate is $-\frac{Q}{2}$, and similarly that of the right plate is $\frac{Q}{2}$.

 When the $+Q$ charge alone exists, the sum of the total induced charges on the two plates is $-Q$. If the total induced charge is $-Q_{x}$ on the left plate, then the total reduced charge is $-Q+Q_{s}$ on the right plate. Similarly if $-Q$ alone exists, the total induced charge on the left plate is $Q-Q_{x}$ and that on the right plate is $+Q_{x}$, by reason of symmetry. If the two charges exist at the same time, the induced charge on the left plate is the superposition of the induced charges produced by both $+Q$ and $-Q$. Hence we have, using the result of $(d)$,

or

$$
Q-2 Q_{x}=-\frac{Q}{2}
$$

$$
Q_{x}=\frac{3}{4} Q .
$$

Thus after $-Q$ has been removed, the total induced charge on the inside surface of the left plate is $-3 Q / 4$ and that of the right plate is $-Q / 4$.

\textbf{Topic} :Electrostatics\\
\textbf{Book} :Problems and Solutions on Electromagnetism\\
\textbf{Final Answer} :\frac{3}{4} Q\\


\textbf{Solution} :Use Cartesian coordinates with the origin at the center of the sphere and the $z$-axis along the line joining the spherical center and the charge q. It is obvious that the greatest induced surface charge density, which is negative, on the sphere will occur at $(0,0, a)$.

The action of the conducting spherical surface may be replaced by that of a point charge $\left(-\frac{a}{r} q\right)$ at $\left(0,0, \frac{a^{2}}{r}\right)$ and a point charge $\left(\frac{a}{r} q\right)$ at the spherical center $(0,0,0)$. Then, the field $\boldsymbol{E}$ at $\left(0,0, a_{+}\right)$is

$$
\begin{aligned}
\mathbf{\Sigma} &=\frac{1}{4 \pi \varepsilon_{0}}\left[\frac{\frac{a}{r} q}{a^{2}}-\frac{q}{(r-a)^{2}}-\frac{\frac{a}{r} q}{\left(a-\frac{a^{2}}{r}\right)^{2}}\right] \mathbf{e}_{z} \\
&=\frac{1}{4 \pi \varepsilon_{0}}\left[\frac{q}{a r}-\frac{q}{(r-a)^{2}}-\frac{\frac{r}{a} q}{(r-a)^{2}}\right] \mathbf{e}_{z}
\end{aligned}
$$

Hence, the maximum negative induced surface charge density is

$$
\sigma_{e}=\varepsilon_{0} E_{n}=\frac{q}{4 \pi}\left[\frac{1}{a r}-\frac{1}{(r-a)^{2}}-\frac{r / a}{(r-a)^{2}}\right] .
$$

If a positive charge $Q$ is given to the sphere, it will distribute uniformly on the spherical surface with a surface density $Q / 4 \pi a^{2}$. In order that the total surface charge density is everywhere positive, we require that

$$
\begin{aligned}
Q \geq-\sigma \cdot 4 \pi a^{2} &=a^{2} q\left[\left(1+\frac{r}{a}\right) \frac{1}{(r-a)^{2}}-\frac{1}{a r}\right] \\
&=\frac{a^{2}(3 r-a)}{r(r-a)^{2}} q .
\end{aligned}
$$

On the other hand, the field at point $\left(0,0,-a_{-}\right)$is

$$
\mathbf{E}=-\frac{q}{4 \pi \varepsilon_{0}}\left[\frac{1}{r a}+\frac{1}{(r+a)^{2}}-\frac{r / a}{(r+a)^{2}}\right] \mathbf{e}_{z}
$$

If we replace $q$ by $-q$, the maximum negative induced surface charge density will occur at $(0,0,-a)$. Then as above the required positive charge is

$$
\begin{aligned}
Q \geq-\sigma \cdot 4 \pi a^{2} &=-\varepsilon_{0} \cdot 4 \pi a^{2}\left(-\frac{q}{4 \pi \varepsilon_{0}}\right) \cdot\left[\frac{1}{r a}+\frac{1}{(r+a)^{2}}-\frac{r / a}{(r+a)^{2}}\right] \\
&=q\left[\frac{a}{r}+\frac{a^{2}}{(r+a)^{2}}-\frac{a r}{(r+a)^{2}}\right]=\frac{q a^{2}(3 r+a)}{r(r+a)^{2}} .
\end{aligned}
$$


\textbf{Topic} :Electrostatics\\
\textbf{Book} :Problems and Solutions on Electromagnetism\\
\textbf{Final Answer} :\frac{q a^{2}(3 r+a)}{r(r+a)^{2}}\\


\textbf{Solution} :Taking the spherical center as the origin and the axis of symmetry of the system as the $z$-axis, then we can write $P=d e_{z}$. Regarding $P$ as a positive charge $q$ and a negative charge $-q$ separated by a distance $2 l$ such that $d=\lim _{l \rightarrow 0} 2 q l$, we use the method of images. As shown in Fig. 1.44, the coordinates of $q$ and $-q$ are respectively given by

$$
q: \quad z=-L+l, \quad-q: \quad z=-L-1 .
$$

Let $q_{1}$ and $q_{2}$ be the image charges of $q$ and $-q$ respectively. For the spherical surface to be of equipotential, the magnitudes and positions of $q_{1}$ and $q_{2}$ are as follows (Fig. 1.44):

$$
\begin{aligned}
&q_{1}=-\frac{a}{L-l} q \text { at }\left(0,0,-\frac{a^{2}}{L-l}\right) \\
&q_{2}=\frac{a}{L+l} q \text { at }\left(0,0,-\frac{a^{2}}{L+l}\right) .
\end{aligned}
$$

MATHPIX IMAGE

Fig. $1.44$

As $L \gg 1$, by the approximation

$$
\frac{1}{L \pm l} \approx \frac{1}{L} \mp \frac{l}{L^{2}}
$$

the magnitudes and positions may be expressed as

$$
\begin{aligned}
&q_{1}=-\frac{a}{L} q-\frac{a d}{2 L^{2}} \quad \text { at } \quad\left(0,0, \frac{-a^{2}}{L}-\frac{a^{2} l}{L^{2}}\right) \\
&q_{2}=\frac{a}{L} q-\frac{a d}{2 L^{2}} \quad \text { at } \quad\left(0,0,-\frac{a^{2}}{L}+\frac{a^{2} l}{L^{2}}\right)
\end{aligned}
$$

where we have used $d=2 q l$. Hence, an image dipole with dipole moment $\mathbf{P}^{\prime}=\frac{a}{L} q \cdot \frac{2 a^{2} l}{L^{2}} \mathbf{e}_{z}=\frac{a^{3}}{L^{3}} \mathbf{P}$ and an image charge $q^{\prime}=-\frac{a d}{L^{2}}$ may be used in place of the action of $q_{1}$ and $q_{2}$. Both $P^{\prime}$ and $q^{\prime}$ are located at $r^{\prime}=\left(0,0, \frac{-a^{2}}{L}\right)$ (see Fig. 1.45). Therefore, the potential at $r$ outside the sphere is the superposition of the potentials produced by $\mathbf{P}, \mathbf{P}^{\prime}$, and $q^{\prime}$, i.e.,

$$
\begin{aligned}
\varphi(\mathbf{r})=& \frac{1}{4 \pi \varepsilon_{0}}\left[\frac{q^{\prime}}{\left|\mathbf{r}-\mathbf{r}^{\prime}\right|}+\frac{\mathbf{P}^{\prime} \cdot\left(\mathbf{r}-\mathbf{r}^{\prime}\right)}{\left|\mathbf{r}-\mathbf{r}^{\prime}\right|^{3}}+\frac{\mathbf{P} \cdot\left(\mathbf{r}+L \mathbf{e}_{z}\right)}{\left|\mathbf{r}+L \mathbf{e}_{z}\right|^{3}}\right] \\
&=\frac{1}{4 \pi \varepsilon_{0}}\left[-\frac{a d}{L^{2}\left(r^{2}+\frac{a^{2} r}{L} \cos \theta+\frac{a^{4}}{L^{2}}\right)^{1 / 2}}\right.\\
&\left.+\frac{a^{3} d\left(r \cos \theta+\frac{a^{2}}{L}\right)}{L^{3}\left(r^{2}+\frac{a^{2} r}{L} \cos \theta+\frac{a^{4}}{L^{2}}\right)^{3 / 2}}+\frac{d(r \cos \theta+L)}{\left(r^{2}+2 r L \cos \theta+L^{2}\right)^{3 / 2}}\right] .
\end{aligned}
$$

MATHPIX IMAGE

Fig. $1.45$
\textbf{Topic} :Electrostatics\\
\textbf{Book} :Problems and Solutions on Electromagnetism\\
\textbf{Final Answer} :\frac{1}{4 \pi \varepsilon_{0}}\left[-\frac{a d}{L^{2}\left(r^{2}+\frac{a^{2} r}{L} \cos \theta+\frac{a^{4}}{L^{2}}\right)^{1 / 2}}\right\\
&\left+\frac{a^{3} d\left(r \cos \theta+\frac{a^{2}}{L}\right)}{L^{3}\left(r^{2}+\frac{a^{2} r}{L} \cos \theta+\frac{a^{4}}{L^{2}}\right)^{3 / 2}}+\frac{d(r \cos \theta+L)}{\left(r^{2}+2 r L \cos \theta+L^{2}\right)^{3 / 2}}\right]\\


\textbf{Solution} :Poisson's equation for the potential and the boundary conditions can be written as follows:

$$
\left.\begin{array}{l}
\nabla^{2} \varphi=-\frac{Q}{\varepsilon_{0}} \delta(x) \delta(y) \delta(z), \\
\left.\varphi\right|_{x=\pm D / 2}=0, \\
\left.\varphi\right|_{y=\pm D / 2}=0 .
\end{array}\right\}
$$

By Fourior transform

$$
\bar{\varphi}(x, y, k)=\int_{-\infty}^{\infty} \varphi(x, y, z) e^{i k z} d z,
$$

the above become

$$
\begin{aligned}
&\left(\frac{\partial^{2}}{\partial x^{2}}+\frac{\partial^{2}}{\partial y^{2}}-k^{2}\right) \bar{\varphi}(x, y, k)=-\frac{Q}{c_{0}} \delta(x) \delta(y), \\
&\left.\bar{\varphi}(x, y, k)\right|_{x=\pm D / 2}=0, \\
&\left.\bar{\varphi}(x, y, k)\right|_{y=\pm D / 2}=0 .
\end{aligned}
$$

Use $F(\Omega)$ to denote the functional space of the functions which are equal to zero at $x=\pm \frac{D}{2}$ or $y=\pm \frac{D}{2}$. A set of unitary and complete basis in this functional space is

$$
\left.\begin{array}{c}
\frac{2}{D} \cos \frac{(2 m+1) \pi x}{D} \cos \frac{\left(2 m^{\prime}+1\right) \pi y}{D}, \frac{2}{D} \cos \frac{(2 m+1) \pi x}{D} \sin \frac{2 n^{\prime} \pi y}{D}, \\
\frac{2}{D} \sin \frac{2 n \pi x}{D} \cos \frac{\left(2 m^{\prime}+1\right) \pi y}{D}, \frac{2}{D} \sin \frac{2 n \pi x}{D} \sin \frac{2 n^{\prime} \pi x}{D} . \\
m, m^{\prime} \geq 0, n, n^{\prime} \geq 1 .
\end{array}\right\}
$$

In this functional space $\delta(x) \delta(y)$ may be expanded as

$$
\delta(x) \delta(y)=\left(\frac{2}{D}\right)^{2} \sum_{m, m^{\prime}=0}^{\infty} \cos \frac{(2 m+1) \pi x}{D} \cos \frac{\left(2 m^{\prime}+1\right) \pi y}{D} .
$$

Letting $\bar{\varphi}(x, y, k)$ be the general solution in the following form,

$$
\bar{\varphi}(x, y, k)=\sum_{m, m^{\prime}=0}^{\infty} \bar{\varphi}_{m m^{\prime}}(k) \cos \frac{(2 m+1) \pi x}{D} \cos \frac{\left(2 m^{\prime}+1\right) \pi y}{D},
$$

and substituting Eq.
(3) into Eq.
(1), we find from Eq.
(2) that

$$
\bar{\varphi}(x, y, k)=\frac{4 Q}{\varepsilon D^{2}} \sum_{m, m^{\prime}=0}^{\infty} \frac{\cos \frac{(2 m+1) \pi x}{D} \cos \frac{\left(2 m^{\prime}+1\right) \pi y}{D}}{k^{2}+\left(\frac{(2 m+1) \pi}{D}\right)^{2}+\left(\frac{\left(2 m^{\prime}+1\right) \pi}{D}\right)^{2}} .
$$

Applying the integral formula

$$
\int_{-\infty}^{\infty} \frac{e^{i k z}}{k^{2}+\lambda^{2}} d k=\frac{\pi}{\lambda} e^{-\lambda|z|}, \quad(\lambda>0)
$$

we finally obtain

$$
\begin{aligned}
\varphi(x, y, z)=& \frac{2 Q}{\pi \varepsilon_{0} D} \sum_{m, m^{\prime}=0}^{\infty} \frac{\cos \frac{(2 m+1) \pi x}{D} \cos \frac{\left(2 m^{\prime}+1\right) \pi y}{D}}{\sqrt{(2 m+1)^{2}+\left(2 m^{\prime}+1\right)^{2}}} \\
& \cdot e^{-\frac{\pi|z|}{D} \sqrt{(2 m+1)^{2}+\left(2 m^{\prime}+1\right)^{2}}} .
\end{aligned}
$$
\textbf{Topic} :Electrostatics\\
\textbf{Book} :Problems and Solutions on Electromagnetism\\
\textbf{Final Answer} :0}^{\infty} \frac{\cos \frac{(2 m+1) \pi x}{D} \cos \frac{\left(2 m^{\prime}+1\right) \pi y}{D}}{\sqrt{(2 m+1)^{2}+\left(2 m^{\prime}+1\right)^{2}}} \\
& \cdot e^{-\frac{\pi|z|}{D} \sqrt{(2 m+1)^{2}+\left(2 m^{\prime}+1\right)^{2}}}\\


\textbf{Solution} :In Fig. $1.49$ the radius vector $r$ is directed from $P_{1}$ to $P_{2}$. Taking the electric field produced by $P_{1}$ as the external field, its intensity at the position of $\mathbf{P}_{2}$ is given by

$$
\mathbf{E}_{e}=\frac{3\left(\mathbf{P}_{1} \cdot \mathbf{r}\right) \mathbf{r}-r^{2} \mathbf{P}_{1}}{4 \pi \varepsilon_{0} r^{5}} .
$$

Hence the force on $\mathbf{P}_{2}$ is

$$
\begin{aligned}
\mathbf{F} &=\left(\mathbf{P}_{2} \cdot \nabla\right) \mathbf{E}_{e} \\
&=\frac{3}{4 \pi \varepsilon_{0} r^{7}}\left\{r^{2}\left[\left(\mathbf{P}_{1} \cdot \mathbf{P}_{2}\right) \mathbf{r}+\left(\mathbf{P}_{1} \cdot \mathbf{r}\right) \mathbf{P}_{2}+\left(\mathbf{P}_{2} \cdot \mathbf{r}\right) \mathbf{P}_{1}\right]-5\left(\mathbf{P}_{1} \cdot \mathbf{r}\right)\left(\mathbf{P}_{2} \cdot \mathbf{r}\right) \mathbf{r}\right\}
\end{aligned}
$$

MATHPIX IMAGE

Fig. $1.49$

If $\mathbf{P}_{1} \| \mathbf{r}$, let $\mathbf{P}_{2} \cdot \mathbf{r}=P_{2} r \cos \theta$. Then $\mathbf{P}_{1} \cdot \mathbf{P}_{2}=P_{1} P_{2} \cos \theta$ and the force between $P_{1}$ and $P_{2}$ becomes

$$
\mathbf{F}=\frac{3}{4 \pi \varepsilon_{0} r^{5}}\left\{-3 P_{1} P_{2} \cos \theta \mathbf{r}+P_{1} r \mathbf{P}_{2}\right\} .
$$

The maximum attraction is obviously given by $\theta=0^{\circ}$, when $\mathrm{P}_{2}$ is also parallel to $r$. This maximum is

$$
\mathbf{F}_{\max }=-\frac{3 P_{1} P_{2} \mathbf{r}}{2 \pi \varepsilon_{0} r^{5}} .
$$

Note that the negative sign signifies attraction. 

\textbf{Topic} :Electrostatics\\
\textbf{Book} :Problems and Solutions on Electromagnetism\\
\textbf{Final Answer} :-\frac{3 P_{1} P_{2} \mathbf{r}}{2 \pi \varepsilon_{0} r^{5}}\\


\textbf{Solution} :In this case $\theta=0$ and we have

$$
F_{\theta}=0, \quad F_{r}=-\frac{3 P_{1} P_{2}}{2 \pi \varepsilon_{0} r^{4}} .
$$

The negative sign denotes an attractive force.
\textbf{Topic} :Electrostatics\\
\textbf{Book} :Problems and Solutions on Electromagnetism\\
\textbf{Final Answer} :-\frac{3 P_{1} P_{2}}{2 \pi \varepsilon_{0} r^{4}}\\


\textbf{Solution} :In this case $\theta=0$ and we have

$$
F_{\theta}=0, \quad F_{r}=-\frac{3 P_{1} P_{2}}{2 \pi \varepsilon_{0} r^{4}} .
$$

The negative sign denotes an attractive force.

 In this case $\theta=\frac{\pi}{2}$ and we have

$$
F_{\theta}=0, \quad F_{r}=\frac{3 P_{1} P_{2}}{4 \pi \varepsilon_{0} r^{4}} .
$$

The positive sign denotes a repulsive force.


\textbf{Topic} :Electrostatics\\
\textbf{Book} :Problems and Solutions on Electromagnetism\\
\textbf{Final Answer} :\frac{3 P_{1} P_{2}}{4 \pi \varepsilon_{0} r^{4}}\\


\textbf{Solution} :Any plane passing through the $z$-axis is an equipotential surface whose potential only depends on the angle $\theta$ it makes with the $y=0$ plane:

$$
\phi(x, y)=\phi(\theta) .
$$

Accordingly, Laplace's equation is reduced to one dimension only:

$$
\frac{d^{2} \phi}{d \theta^{2}}=0,
$$

with the solution

$$
\phi(\theta)=\frac{2 V_{0}}{\pi} \theta,
$$

taking into account the boundary conditions $\phi=0$ for $\theta=0$ and $\phi=V_{0}$ for $\theta=\frac{\pi}{2}$. This can also be written in Cartesian coordinates as

$$
\phi(x, y)=\frac{2 V_{0}}{\pi} \arctan \left(\frac{y}{x}\right) .
$$

 The field is then

$$
\mathbf{E}=-\nabla \phi=\frac{2 V_{0}}{\pi}\left(\frac{y}{x^{2}+y^{2}} \mathbf{e}_{x}-\frac{x}{x^{2}+y^{2}} \mathbf{e}_{y}\right) .
$$

Hence, the force acting on the dipole $\left(P_{x}, 0,0\right)$ is

$$
\begin{aligned}
\boldsymbol{F} &=(\mathbf{P} \cdot \nabla) \mathbf{E}=\left.P_{x} \frac{\partial \mathbf{E}}{\partial x}\right|_{x=x_{0}, y=y_{0}} \\
&=\frac{2 V_{0} P_{x}}{\pi}\left(-\frac{2 x_{0} y_{0}}{\left(x_{0}^{2}+y_{0}^{2}\right)^{2}} \mathbf{e}_{x}+\frac{x_{0}^{2}-y_{0}^{2}}{\left(x_{0}^{2}+y_{0}^{2}\right)^{2}} \mathbf{e}_{y}\right) .
\end{aligned}
$$

\textbf{Topic} :Electrostatics\\
\textbf{Book} :Problems and Solutions on Electromagnetism\\
\textbf{Final Answer} :\frac{2 V_{0} P_{x}}{\pi}\left(-\frac{2 x_{0} y_{0}}{\left(x_{0}^{2}+y_{0}^{2}\right)^{2}} \mathbf{e}_{x}+\frac{x_{0}^{2}-y_{0}^{2}}{\left(x_{0}^{2}+y_{0}^{2}\right)^{2}} \mathbf{e}_{y}\right)\\


\textbf{Solution} :Initially, as the voltages on the two sides of the central plate are the same, we can consider the three plates as forming two parallel capacitors with capacitances $C_{1}$ and $C_{2}$. When the central plate is located at position $x$, the total capacitance of the parallel capacitors is

$$
C=C_{1}+C_{2}=\frac{A}{\varepsilon_{0}(L+x)}+\frac{A}{\varepsilon_{0}(L-x)}=\frac{2 A L}{\varepsilon_{0}\left(L^{2}-x^{2}\right)} .
$$

Hence the electrostatic energy of the system is

$$
W_{e}=\frac{1}{2} \frac{Q^{2}}{C}=\frac{\varepsilon_{0} Q^{2}\left(L^{2}-x^{2}\right)}{4 A L} .
$$

As the charge $Q$ is distributed over the central plate, when the plate moves work is done against the electrostatic force. Hence the latter is given by

$$
F_{e}=-\frac{\partial W_{e}}{\partial x}=\frac{Q^{2} \varepsilon_{0} x}{2 A L} .
$$

As $F>0$, the force is in the direction of increasing $x$. As an electric conductor is also a good thermal conductor, the motion of the central plate can be considered isothermal. Let the pressures exerted by air on the left and right sides on the central plate be $p_{1}$ and $p_{2}$ respectively. We have by Boyle's law

$$
p_{1}=\frac{p_{0} L}{L+x}, \quad p_{2}=\frac{p_{0} L}{L-x} .
$$

When the central plate is in the equilibrium position, the electrostatic force is balanced by the force produced by the pressure difference, i.e.,

$$
F_{e}=\left(p_{2}-p_{1}\right) A,
$$

or

$$
\frac{Q^{2} \varepsilon_{0} x}{2 A L}=\frac{2 A p_{0} L x}{L^{2}-x^{2}} .
$$

This determines the equilibrium positions of the central plate as

$$
x=\pm L\left(1-\frac{4 p_{0} A^{2}}{\varepsilon_{0} Q^{2}}\right)^{\frac{1}{2}} .
$$


\textbf{Topic} :Electrostatics\\
\textbf{Book} :Problems and Solutions on Electromagnetism\\
\textbf{Final Answer} :\pm L\left(1-\frac{4 p_{0} A^{2}}{\varepsilon_{0} Q^{2}}\right)^{\frac{1}{2}}\\


\textbf{Solution} :To leading order, we can regard the distance between the conducting sphere and the conductor plane as infinite. Then the capacitance of the whole system corresponds to that of an isolated conducting sphere of radius $a$, its value being

$$
C=4 \pi \varepsilon_{0} a .
$$
\textbf{Topic} :Electrostatics\\
\textbf{Book} :Problems and Solutions on Electromagnetism\\
\textbf{Final Answer} :4 \pi \varepsilon_{0} a\\


\textbf{Solution} :To leading order, we can regard the distance between the conducting sphere and the conductor plane as infinite. Then the capacitance of the whole system corresponds to that of an isolated conducting sphere of radius $a$, its value being

$$
C=4 \pi \varepsilon_{0} a .
$$

 To find the first correction, we consider the field established by a point charge $Q$ at the spherical center and its image charge $-Q$ at distance $z$ from and on the other side of the plane. At a point on the line passing through the spherical center and normal to the plane the magnitude of this field is

$$
E=\frac{Q}{4 \pi \varepsilon_{0}(z-h)^{2}}-\frac{Q}{4 \pi \varepsilon_{0}(z+h)^{2}},
$$

where $h$ is the distance from this point to the plane. The potential of the sphere is then

$$
\begin{aligned}
V &=-\int_{0}^{z-a} E d h=\left.\frac{Q}{4 \pi \varepsilon_{0}(z-h)}\right|_{0} ^{z-a}-\left.\frac{Q}{4 \pi \varepsilon_{0}(z+h)}\right|_{0} ^{z-a} \\
&=\frac{Q}{4 \pi \varepsilon_{0} a}\left[1-\frac{a}{2 z-a}\right] \approx \frac{Q}{4 \pi \varepsilon_{0} a}\left(1-\frac{a}{2 z}\right)
\end{aligned}
$$

Hence the capacitance is

$$
C=\frac{Q}{V} \approx 4 \pi \varepsilon_{0} a\left(1+\frac{a}{2 z}\right)
$$

and the first correction is $2 \pi \varepsilon_{0} a^{2} / z$.
\textbf{Topic} :Electrostatics\\
\textbf{Book} :Problems and Solutions on Electromagnetism\\
\textbf{Final Answer} :\frac{Q}{V} \approx 4 \pi \varepsilon_{0} a\left(1+\frac{a}{2 z}\right)\\


\textbf{Solution} :To leading order, we can regard the distance between the conducting sphere and the conductor plane as infinite. Then the capacitance of the whole system corresponds to that of an isolated conducting sphere of radius $a$, its value being

$$
C=4 \pi \varepsilon_{0} a .
$$

 To find the first correction, we consider the field established by a point charge $Q$ at the spherical center and its image charge $-Q$ at distance $z$ from and on the other side of the plane. At a point on the line passing through the spherical center and normal to the plane the magnitude of this field is

$$
E=\frac{Q}{4 \pi \varepsilon_{0}(z-h)^{2}}-\frac{Q}{4 \pi \varepsilon_{0}(z+h)^{2}},
$$

where $h$ is the distance from this point to the plane. The potential of the sphere is then

$$
\begin{aligned}
V &=-\int_{0}^{z-a} E d h=\left.\frac{Q}{4 \pi \varepsilon_{0}(z-h)}\right|_{0} ^{z-a}-\left.\frac{Q}{4 \pi \varepsilon_{0}(z+h)}\right|_{0} ^{z-a} \\
&=\frac{Q}{4 \pi \varepsilon_{0} a}\left[1-\frac{a}{2 z-a}\right] \approx \frac{Q}{4 \pi \varepsilon_{0} a}\left(1-\frac{a}{2 z}\right)
\end{aligned}
$$

Hence the capacitance is

$$
C=\frac{Q}{V} \approx 4 \pi \varepsilon_{0} a\left(1+\frac{a}{2 z}\right)
$$

and the first correction is $2 \pi \varepsilon_{0} a^{2} / z$.

 When the sphere carries charge $Q$, the leading term of the force between it and the conducting plane is just the attraction between two point charges $Q$ and $-Q$ separated by $2 z$. It follows that

$$
F=-\frac{Q^{2}}{16 \pi \varepsilon_{0} z^{2}} \text {. }
$$

The energy required to completely separate the sphere from the plane is

$$
W_{1}=-\int_{z}^{\infty} F d r=\int_{z}^{\infty} \frac{Q^{2}}{16 \pi \varepsilon_{0} r^{2}} d r=\frac{Q^{2}}{16 \pi \varepsilon_{0} z} .
$$

On the other hand, the energy of complete separation of the two charges $Q$ and $-Q$, initially spaced by a distance $2 z$, is

$$
W_{2}=-\int_{2 z}^{\infty} F d r=\int_{2 z}^{\infty} \frac{Q^{2}}{4 \pi \varepsilon_{0} r^{2}} d r=\frac{Q^{2}}{8 \pi \varepsilon_{0} z}=2 W_{1}
$$

The difference in the required energy is due to the fact that in the first case one has to move $Q$ from $z$ to $\infty$ while in the second case one has to move $Q$ from $z$ to $\infty$ and $-Q$ from $-z$ to $-\infty$, with the same force $-Q^{2} / 4 \pi \varepsilon_{0} r^{2}$ applying to all the three charges.

\textbf{Topic} :Electrostatics\\
\textbf{Book} :Problems and Solutions on Electromagnetism\\
\textbf{Final Answer} :2 W_{1}\\


\textbf{Solution} :The angle between the dipole and the polar axis is $(\theta+\alpha)$, so the angular velocity of the dipole about its center of mass is $(\dot{\theta}+\dot{\alpha})$. The kinetic energy of the dipole is then

$$
\begin{aligned}
T &=\frac{1}{2} \cdot 2 m \cdot\left(\dot{r}^{2}+r^{2} \dot{\theta}^{2}\right)+\frac{1}{2} \cdot 2 m R^{2} \cdot(\dot{\theta}+\dot{\alpha})^{2} \\
&=m \dot{r}^{2}+m\left(r^{2}+R^{2}\right) \dot{\theta}^{2}+m R^{2} \dot{\alpha}^{2}+2 m R^{2} \dot{\theta} \dot{\alpha}
\end{aligned}
$$

Moreover, the potential energy of the dipole is

$$
V=\frac{1}{4 \pi \varepsilon_{0}} \frac{Q_{1} Q_{2}}{r_{+}}-\frac{1}{4 \pi \varepsilon_{0}} \frac{Q_{1} Q_{2}}{r_{-}} .
$$

As

$$
\begin{aligned}
r_{\pm} &=\sqrt{r^{2}+R^{2} \pm 2 r R \cos \alpha}=r \sqrt{1 \pm 2 \frac{R}{r} \cos \alpha+\left(\frac{R}{r}\right)^{2}} \\
& \approx r \sqrt{1 \pm 2 \frac{R}{r} \cos \alpha} \approx r\left(1 \pm \frac{1}{2} \cdot 2 \frac{R}{r} \cos \alpha\right)=r \pm R \cos \alpha, \\
& \frac{1}{r_{+}}-\frac{1}{r_{-}}=-\frac{2 R \cos \alpha}{r^{2}-R^{2} \cos ^{2} \alpha} \approx-\frac{2 R \cos \alpha}{r^{2}},
\end{aligned}
$$

and the potential energy is

$$
V=-\frac{Q_{1} Q_{2}}{4 \pi \varepsilon_{0}} \cdot \frac{2 R \cos \alpha}{r^{2}} .
$$

The above give the Lagrangian $L=T-V$. Lagrange's equation

$$
\frac{d}{d t}\left(\frac{\partial L}{\partial \dot{r}}\right)-\frac{\partial L}{\partial r}=0
$$

gives

$$
m \ddot{r}-m r \dot{\theta}^{2}+\frac{Q_{1} Q_{2}}{4 \pi \varepsilon_{0}} \cdot \frac{2 R \cos \alpha}{r^{2}}=0 ;
$$

of the other Lagrange's equations:

gives

$$
\frac{d}{d t}\left(\frac{\partial L}{\partial \dot{\theta}}\right)-\frac{\partial L}{\partial \theta}=0
$$

$$
\left(r^{2}+R^{2}\right) \ddot{\theta}+R^{2} \ddot{\alpha}+2 m r \dot{r} \dot{\theta}=0,
$$

and $\frac{d}{d t}\left(\frac{\partial L}{\partial \dot{\alpha}}\right)-\frac{\partial L}{\partial \alpha}=0$ gives

$$
m R(\ddot{\alpha}+\ddot{\theta})+\frac{Q_{1} Q_{2}}{4 \pi \varepsilon_{0}} \cdot \frac{\sin \alpha}{r^{2}}=0 .
$$

Equations (1)-(3) are the equations of the motion of the dipole.
\\
\textbf{Topic} :Electrostatics\\
\textbf{Book} :Problems and Solutions on Electromagnetism\\
\textbf{Final Answer} :0\\


\textbf{Solution} :The angle between the dipole and the polar axis is $(\theta+\alpha)$, so the angular velocity of the dipole about its center of mass is $(\dot{\theta}+\dot{\alpha})$. The kinetic energy of the dipole is then

$$
\begin{aligned}
T &=\frac{1}{2} \cdot 2 m \cdot\left(\dot{r}^{2}+r^{2} \dot{\theta}^{2}\right)+\frac{1}{2} \cdot 2 m R^{2} \cdot(\dot{\theta}+\dot{\alpha})^{2} \\
&=m \dot{r}^{2}+m\left(r^{2}+R^{2}\right) \dot{\theta}^{2}+m R^{2} \dot{\alpha}^{2}+2 m R^{2} \dot{\theta} \dot{\alpha}
\end{aligned}
$$

Moreover, the potential energy of the dipole is

$$
V=\frac{1}{4 \pi \varepsilon_{0}} \frac{Q_{1} Q_{2}}{r_{+}}-\frac{1}{4 \pi \varepsilon_{0}} \frac{Q_{1} Q_{2}}{r_{-}} .
$$

As

$$
\begin{aligned}
r_{\pm} &=\sqrt{r^{2}+R^{2} \pm 2 r R \cos \alpha}=r \sqrt{1 \pm 2 \frac{R}{r} \cos \alpha+\left(\frac{R}{r}\right)^{2}} \\
& \approx r \sqrt{1 \pm 2 \frac{R}{r} \cos \alpha} \approx r\left(1 \pm \frac{1}{2} \cdot 2 \frac{R}{r} \cos \alpha\right)=r \pm R \cos \alpha, \\
& \frac{1}{r_{+}}-\frac{1}{r_{-}}=-\frac{2 R \cos \alpha}{r^{2}-R^{2} \cos ^{2} \alpha} \approx-\frac{2 R \cos \alpha}{r^{2}},
\end{aligned}
$$

and the potential energy is

$$
V=-\frac{Q_{1} Q_{2}}{4 \pi \varepsilon_{0}} \cdot \frac{2 R \cos \alpha}{r^{2}} .
$$

The above give the Lagrangian $L=T-V$. Lagrange's equation

$$
\frac{d}{d t}\left(\frac{\partial L}{\partial \dot{r}}\right)-\frac{\partial L}{\partial r}=0
$$

gives

$$
m \ddot{r}-m r \dot{\theta}^{2}+\frac{Q_{1} Q_{2}}{4 \pi \varepsilon_{0}} \cdot \frac{2 R \cos \alpha}{r^{2}}=0 ;
$$

of the other Lagrange's equations:

gives

$$
\frac{d}{d t}\left(\frac{\partial L}{\partial \dot{\theta}}\right)-\frac{\partial L}{\partial \theta}=0
$$

$$
\left(r^{2}+R^{2}\right) \ddot{\theta}+R^{2} \ddot{\alpha}+2 m r \dot{r} \dot{\theta}=0,
$$

and $\frac{d}{d t}\left(\frac{\partial L}{\partial \dot{\alpha}}\right)-\frac{\partial L}{\partial \alpha}=0$ gives

$$
m R(\ddot{\alpha}+\ddot{\theta})+\frac{Q_{1} Q_{2}}{4 \pi \varepsilon_{0}} \cdot \frac{\sin \alpha}{r^{2}}=0 .
$$

Equations (1)-(3) are the equations of the motion of the dipole.
 As $\dot{r} \approx \ddot{r} \approx 0, r$ is a constant. Also with $\ddot{\theta}=0, \alpha \ll 1$ (i.e., $\sin \alpha \approx \alpha)$, Eq.
(3) becomes

$$
m R \ddot{\alpha}+\frac{Q_{1} Q_{2}}{4 \pi \varepsilon_{0}} \cdot \frac{\alpha}{r^{2}}=0 .
$$

This shows that the motion in $\alpha$ is simple harmonic with angular frequency

$$
\omega=\sqrt{\frac{Q_{1} Q_{2}}{4 \pi \varepsilon_{0}} \cdot \frac{1}{m R r^{2}}} .
$$

The period of such small oscillations is

$$
T=\frac{2 \pi}{\omega}=2 \pi \sqrt{\frac{4 \pi \varepsilon_{0}}{Q_{1} Q_{2}} \cdot m R r^{2}}
$$


\textbf{Topic} :Electrostatics\\
\textbf{Book} :Problems and Solutions on Electromagnetism\\
\textbf{Final Answer} :2 \pi \sqrt{\frac{4 \pi \varepsilon_{0}}{Q_{1} Q_{2}} \cdot m R r^{2}}\\


\textbf{Solution} :When the two plates are in contact, they can be taken as a parallelplate capacitor. Letting $\delta$ be their separation and $V$ be the potential difference, the magnitude of charge on each plate is

$$
Q=C V=\frac{\pi \varepsilon_{0}\left(\frac{d}{2}\right)^{2} V}{\delta} .
$$

As $d=0.05 \mathrm{~m}$, taking the contact potential as $V \sim 10^{-3} \mathrm{~V}$ and $\delta \sim 10^{-10} \mathrm{~m}$ we obtain

$$
Q \approx 1.7 \times 10^{-7} \mathrm{C} \text {. }
$$
\textbf{Topic} :Electrostatics\\
\textbf{Book} :Problems and Solutions on Electromagnetism\\
\textbf{Final Answer} :2 \pi \sqrt{\frac{4 \pi \varepsilon_{0}}{Q_{1} Q_{2}} \cdot m R r^{2}}\\


\textbf{Solution} :Use cylindrical coordinates $(r, \varphi, z)$ with the $z$-axis along the cylindrical axis. The electric field at a point $(r, \varphi, z)$ satisfies $\mathbf{E} \propto \frac{1}{r} \mathbf{e}_{r}$ according to Gauss' flux theorem. From

$$
-\int_{b}^{a} E(r) d r=-V_{0}
$$

we get

$$
\mathbf{E}(r)=\frac{V_{0}}{r \ln \left(\frac{a}{b}\right)} \mathbf{e}_{r} .
$$

If $Q_{0}$ is the charge on the wire, Gauss' theorem gives

$$
\mathbf{E}(r)=\frac{Q_{0}}{2 \pi \varepsilon_{0} L r} \mathbf{e}_{r} .
$$

The capacitance of the chamber is accordingly

$$
\begin{aligned}
C &=Q_{0} / V_{0}=2 \pi \varepsilon_{0} L / \ln (a / b) \\
&=2 \pi \times 8.85 \times 10^{-12} \times 0.5 / \ln \left(\frac{1}{0.01}\right) \\
&=6 \times 10^{-12} \mathrm{~F} .
\end{aligned}
$$

Hence the time constant of the circuit is

$$
R C=10^{5} \times 6 \times 10^{-12}=6 \times 10^{-7} \mathrm{~s} .
$$

The mobility of a charged particle is defined as $\mu=\frac{1}{E} \frac{d r}{d t}$, or $d t=\frac{d r}{\mu E}$. Hence the time taken for the particle to drift from $r_{1}$ to $r_{2}$ is

$$
\Delta t=\int_{r_{1}}^{r_{2}} \frac{d r}{\mu_{0} \frac{V_{0}}{r \ln (a / b)}}=\frac{\ln (a / b)}{2 \mu V_{0}}\left(r_{2}^{2}-r_{1}^{2}\right) .
$$

For an electron to drift from $r=a / 2$ to the wire, we have

$$
\begin{aligned}
\Delta t_{-} & \approx \frac{\ln \left(\frac{a}{b}\right)}{2 \mu_{-} V_{0}}\left[\left(\frac{a}{2}\right)^{2}-b^{2}\right] \approx \frac{\ln \left(\frac{a}{b}\right)}{2 \mu_{-} V_{0}} \cdot \frac{a^{2}}{4} \\
&=\frac{\ln 100}{2 \times 6 \times 10^{3} \times 10^{-4} \times 1000} \times \frac{10^{-4}}{4}=9.6 \times 10^{-8} \mathrm{~s}
\end{aligned}
$$

and for a positive ion to reach the cylinder wall, we have

$$
\begin{aligned}
\Delta t_{+} &=\frac{\ln \left(\frac{a}{b}\right)}{2 \mu_{+} V_{0}}\left[a^{2}-\left(\frac{a}{2}\right)^{2}\right]=\frac{\ln \frac{a}{b}}{2 \mu_{+} V_{0}} \cdot \frac{3 a^{2}}{4} \\
&=\frac{\ln 100}{2 \times 1.3 \times 10^{-4} \times 1000} \times \frac{3 \times 10^{-4}}{4}=1.3 \times 10^{-3} \mathrm{~s} .
\end{aligned}
$$

It follows that $\Delta t_{-} \ll R C \ll \Delta t_{+}$. When the electrons are drifting from $r=a / 2$ to the anode wire at $r=b$, the positive ions remain essentially stationary at $r=a / 2$, and the discharge through resistor $R$ is also negligible. The output voltage $\Delta V$ of the anode wire at $t \leq \Delta t_{-}$(taking $t=0$ at the instant when the ionizing particle enters the chamber) can be derived from energy conservation. When a charge $q$ in the chamber displaces by $d \mathbf{r}$, the work done by the field is $q \mathbf{E} \cdot d \mathbf{r}$ corresponding to a decrease of the energy stored in the capacitance of $d\left(C V^{2} / 2\right)$. Thus $C V d V=-q \mathbf{E} \cdot d \mathbf{r}=-q E d r$. Since $\Delta V \ll V_{0}, V \approx V_{0}$, and we can write

$$
C V_{0} d V=-q E d r .
$$



Integrating, we have

$$
\begin{aligned}
C V_{0} \Delta V &=-q \int_{a / 2}^{r} E d r=-q \int_{a / 2}^{r} \frac{V_{0}}{r \ln \left(\frac{a}{b}\right)} d r \\
&=-q \frac{V_{0}}{\ln \left(\frac{a}{b}\right)} \ln \left(\frac{2 r}{a}\right)
\end{aligned}
$$

Noting that

we have

$$
\Delta t_{-} \approx \frac{\ln \left(\frac{a}{b}\right)}{2 \mu_{-} V_{0}}\left[\left(\frac{a}{2}\right)^{2}-b^{2}\right]
$$

or

$$
t=\frac{\ln \left(\frac{a}{b}\right)}{2 \mu_{-} V_{0}}\left[\left(\frac{a}{2}\right)^{2}-r^{2}\right]
$$

$$
r=\frac{a}{2}\left\{1-\frac{t}{\Delta t_{-}}\left(1-\frac{2 b}{a}\right)^{2}\right\}^{1 / 2}
$$

and, as $q=-N e$

$$
\Delta V=\frac{N e}{C} \ln \left\{1-\frac{\left[1-\left(\frac{2 b}{a}\right)^{2}\right] t}{\Delta t_{-}}\right\}^{1 / 2} / \ln \left(\frac{a}{b}\right), \quad 0 \leq t \leq \Delta t_{-} .
$$

At $t=\Delta t_{-}$,

$$
\begin{aligned}
\Delta V &=\frac{N e}{C} \ln \left(\frac{2 b}{a}\right) / \ln \left(\frac{a}{b}\right) \\
&=-\frac{10^{5} \times 1.6 \times 10^{-19}}{6 \times 10^{-12}} \times \frac{\ln 50}{\ln 100} \\
&=-2.3 \times 10^{-3} \mathrm{~V} .
\end{aligned}
$$

This voltage is then discharged through the $R C$ circuit. Therefore, the variation of $\Delta V$ with time is as follows:

$$
\begin{aligned}
&\Delta V=5.86 \times 10^{-3} \ln \left[1-\frac{\left(1-\frac{1}{50^{2}}\right)^{1 / 2} t}{9.6 \times 10^{-8}}\right] \mathrm{V}, \quad \text { for } \quad 0 \leq t \leq 9.6 \times 10^{-8} \mathrm{~s} \\
&\Delta V=-2.3 \times 10^{-3} \exp \left(-\frac{t}{6 \times 10^{-7}}\right) \mathrm{V}, \quad \text { for } \quad t>9.6 \times 10^{-8} \mathrm{~s}
\end{aligned}
$$

This means that the voltage across the two ends of $R$ decreases to $-2.3 \mathrm{mV}$ in the time $\Delta t_{-}$, and then increases to zero with the time constant $R C . A$ final remark is that, as the ions drift only slowly and the induced charges on the two electrodes of the chamber are discharged quickly through the $R C$ circuit, their influence on the wave form of $\Delta V$ can be completely ignored.

\textbf{Topic} :Electrostatics\\
\textbf{Book} :Problems and Solutions on Electromagnetism\\
\textbf{Final Answer} :-23 \times 10^{-3} \exp \left(-\frac{t}{6 \times 10^{-7}}\right) \mathrm{V} \quad \text { for } \quad t>96 \times 10^{-8} \mathrm{~s}\\


\textbf{Solution} :At $t=10^{-8} \mathrm{~s}$, the beam forms a charge cylinder on the right side of the foil with a cross sectional area $A=1000 \mathrm{~cm}^{2}$ as shown in Fig. 1.62. The length $h$ of the cylinder is $c t=3 \times 10^{8} \times 10^{-8}=3 \mathrm{~m}$, assuming the electrons to have sufficiently high energy so that their speed is close to the velocity of light. We may consider a total charge of

$$
-Q=-I t=-3 \times 10^{6} \times 10^{-8}=-3 \times 10^{-2} \mathrm{C}
$$

being uniformly distributed in this cylinder. As the charge on the left side of the foil does not contribute to the electric field at point $P$ (Shielding effect), the action of the grounded metal foil can be replaced by an image charge cylinder. This image cylinder and the real cylinder are symmetrical with respect to the metal foil and their charges are opposite in sign (see Fig. 1.62).

MATHPIX IMAGE

Fig. $1.62$

MATHPIX IMAGE

Fig. $1.63$ We first calculate the electric field at point $P$ on the axis of a uniformly charge disc of surface charge density $\sigma$ as shown in Fig. 1.63. The potential is

$$
\begin{aligned}
\varphi_{P} &=\frac{1}{4 \pi \varepsilon_{0}} \int_{0}^{R} \frac{\sigma \cdot 2 \pi r d r}{\sqrt{z^{2}+r^{2}}}=\frac{1}{4 \pi \varepsilon_{0}} \int_{0}^{R} \frac{\sigma \pi d r^{2}}{\sqrt{z^{2}+r^{2}}} \\
&=\left.\frac{\pi \sigma}{4 \pi \varepsilon_{0}} 2 \sqrt{z^{2}+r^{2}}\right|_{0} ^{R}=\frac{\sigma}{2 \varepsilon_{0}}\left[\sqrt{z^{2}+R^{2}}-z\right],
\end{aligned}
$$

and the field intensity is

$$
E_{P}=-\frac{\partial \varphi_{P}}{\partial z}=-\frac{\sigma}{2 \varepsilon_{0}}\left(\frac{z}{\sqrt{z^{2}+R^{2}}}-1\right) .
$$

Refer now to Fig. 1.62. The field at the point $P$ produced by the right charge cylinder is

$$
\begin{aligned}
E_{P} &=\frac{1}{2 \varepsilon_{0}} \int_{0}^{h}\left(\frac{Q}{h \pi R^{2}} d z\right)\left(\frac{z}{\sqrt{z^{2}+R^{2}}}-1\right) \\
&=\frac{Q}{2 \pi \varepsilon_{0} h R^{2}} \int_{0}^{h}\left(\frac{\frac{1}{2} d z^{2}}{\sqrt{z^{2}+R^{2}}}-d z\right) \\
&=\frac{Q}{2 \pi \varepsilon_{0} h R^{2}}\left[\left.\sqrt{z^{2}+R^{2}}\right|_{0} ^{h}-h\right] \\
&=\frac{Q}{2 \pi \varepsilon_{0} h R^{2}}\left[\sqrt{h^{2}+R^{2}}-R-h\right] .
\end{aligned}
$$

Hence the total electric field at $P$ is

$$
\begin{aligned}
E_{P} &=\frac{-Q}{\pi \varepsilon_{0} h R^{2}}\left[R+h-\sqrt{R^{2}+h^{2}}\right] \\
&=\frac{-3 \times 10^{-2}}{\pi \times 8.85 \times 10^{-12} \times 3 \times \frac{0.1}{\pi}} \times\left[3+\sqrt{\frac{0.1}{\pi}}-\sqrt{3^{2}+\frac{0.1}{\pi}}\right] \\
&=-1.42 \times 10^{9} \mathrm{~V} / \mathrm{m} .
\end{aligned}
$$

The minus sign indicates that the field intensity points to the right.


\textbf{Topic} :Electrostatics\\
\textbf{Book} :Problems and Solutions on Electromagnetism\\
\textbf{Final Answer} :-142 \times 10^{9} \mathrm{~V} / \mathrm{m}\\


\textbf{Solution} :Use cylindrical coordinates with the $z$-axis along the axis of the wire and the positive direction along the current flow, as shown in Fig. 2.1. On account of the uniformity of the current the current density is

$$
\mathbf{j}=\frac{I}{\pi R^{2}} \mathbf{e}_{z} .
$$

MATHPIX IMAGE

Fig. $2.1$

Consider a point at distance $r$ from the axis of the wire. Ampère's circuital law

$$
\oint_{L} \mathbf{H} \cdot d \mathbf{l}=I,
$$

where $L$ is a circle of radius $r$ with centre on the $z$-axis, gives for $r>R$,

or

$$
\mathbf{H}(r)=\frac{I}{2 \pi r} \mathbf{e}_{\theta},
$$

$$
\mathrm{B}(r)=\frac{\mu_{0} I}{2 \pi r} \mathbf{e}_{\theta}
$$

since by symmetry $\mathrm{H}(r)$ and $\mathbf{B}(r)$ are independent of $\theta$. For $r<R$,

$$
I(r)=\pi r^{2} j=\frac{I r^{2}}{R^{2}}
$$

and the circuital law gives

$$
\mathbf{H}(r)=\frac{I r}{2 \pi R^{2}} e_{\theta}, \quad B(r)=\frac{\mu I r}{2 \pi R^{2}} e_{\theta}
$$

\textbf{Topic} :Magnetostatic Field and Quasi-Stationary Electromagnetic Fields\\
\textbf{Book} :Problems and Solutions on Electromagnetism\\
\textbf{Final Answer} :\frac{\mu I r}{2 \pi R^{2}} e_{\theta}\\


\textbf{Solution} :Assume the three wires are coplanar, then the points of zero magnetic field must also be located in the same plane. Let the distance of such a point from the middle wire be $x$. Then the distance of this point from the other two wires are $d \pm x$. Applying Ampère's circuital law we obtain for a point of zero magnetic field

$$
\frac{\mu_{0} I}{2 \pi(d-x)}=\frac{\mu_{0} I}{2 \pi x}+\frac{\mu_{0} I}{2 \pi(d+x)} .
$$

Two solutions are possible, namely

$$
x=\pm \frac{1}{\sqrt{3}} d,
$$

corresponding to two points located between the middle wire and each of the other 2 wires, both having distance $\frac{1}{\sqrt{3}} d$ from the middle wire.
\textbf{Topic} :Magnetostatic Field and Quasi-Stationary Electromagnetic Fields\\
\textbf{Book} :Problems and Solutions on Electromagnetism\\
\textbf{Final Answer} :\pm \frac{1}{\sqrt{3}} d\\


\textbf{Solution} :The straight parts of the wire do not contribute to the magnetic field at $\mathrm{O}$ since for them $I d \mathbf{l} \times \mathbf{r}=0$. We need only to consider the contribution of the semi-circular part. The magnetic field at $O$ produced by a current element $I d I$ is

$$
d \mathbf{B}=\frac{\mu_{0}}{4 \pi} \frac{I d \times \mathbf{r}}{r^{3}} .
$$

As $I d \mathbf{l}$ and $\mathbf{r}$ are mutually perpendicular for the semi-circular wire, $d \mathbf{B}$ is always pointing into the page. The total magnetic field of the semi-circular wire is then

$$
B=\int d B=\frac{\mu_{0} I}{4 \pi r} \int_{0}^{\pi} d \theta=\frac{\mu_{0} I}{4 r} .
$$

With $I=1 \mathrm{~A}, r=10^{-2} \mathrm{~m}$, the magnetic induction at $\mathrm{O}$ is

$$
B=3.14 \times 10^{-5} \mathrm{~T} \text {, }
$$

pointing perpendicularly into the page.

\textbf{Topic} :Magnetostatic Field and Quasi-Stationary Electromagnetic Fields\\
\textbf{Book} :Problems and Solutions on Electromagnetism\\
\textbf{Final Answer} :314 \times 10^{-5} \mathrm{~T} \text { }\\


\textbf{Solution} :We first find an expression for the magnetic induction at a point on the axis of the solenoid. As shown in Fig. 2.4, the field at point $z_{0}$ on the axis is given by

$$
B\left(z_{0}\right)=\frac{\mu_{0}}{4 \pi} \int_{0}^{\infty} \frac{2 \pi R^{2} n I d z}{\left[R^{2}+\left(z-z_{0}\right)^{2}\right]^{3 / 2}} .
$$

Let $z-z_{0}=R \tan \theta$. Then $d z=R \sec ^{2} \theta d \theta$ and we get

$$
\begin{aligned}
B\left(z_{0}\right) &=\frac{\mu_{0}}{4 \pi} \int_{-\theta_{0}}^{\frac{\pi}{2}} \frac{2 \pi R^{2} n I \cdot R \sec ^{2} \theta d \theta}{R^{3} \sec ^{3} \theta} \\
&=\frac{\mu_{0}}{4 \pi} \int_{-\theta_{0}}^{\frac{\pi}{2}} 2 \pi n I \cos \theta d \theta=\left.\frac{\mu_{0}}{4 \pi} \cdot 2 \pi n I \sin \theta\right|_{-\theta_{0}} ^{\frac{\pi}{2}} .
\end{aligned}
$$

As

$$
R \tan \theta_{0}=z_{0}
$$

we have

$$
\sin ^{2} \theta_{0}=\frac{1}{\cot ^{2} \theta_{0}+1}=\frac{1}{\left(\frac{R}{z_{0}}\right)^{2}+1}=\frac{z_{0}^{2}}{R^{2}+z_{0}^{2}},
$$

or

$$
\sin \theta_{0}=\frac{z_{0}}{\sqrt{R^{2}+z_{0}^{2}}} .
$$

Hence

$$
B\left(z_{0}\right)=\frac{1}{2} \mu_{0} n I\left(1+\frac{z_{0}}{\sqrt{R^{2}+z_{0}^{2}}}\right) .
$$

Next, we imagine a short cylinder of thickness $d z_{0}$ and radius $r$ along the $z$-axis as shown in Fig. 2.5. Applying Maxwell's equation

$$
\oint \mathbf{B} \cdot d \mathbf{S}=0
$$

to its surface $S$ we obtain

$$
\left[B_{z}\left(z_{0}+d z\right)-B_{z}\left(z_{0}\right)\right] \cdot \pi r^{2}=B_{r}\left(z_{0}\right) 2 \pi r d z_{0} .
$$

MATHPIX IMAGE

Fig. $2.4$

MATHPIX IMAGE

Fig. $2.5$

For $r \ll R$, we can take $B_{z}\left(z_{0}\right)=B\left(z_{0}\right)$. The above equation then gives

$$
\frac{d B\left(z_{0}\right)}{d z_{0}} \pi r^{2}=B_{r}\left(z_{0}\right) \cdot 2 \pi r,
$$

or

$$
B_{r}\left(z_{0}\right)=\frac{r}{2} \frac{d B\left(z_{0}\right)}{d z_{0}}=\frac{\mu_{0} n \operatorname{Ir} R^{2}}{4\left(R^{2}+z_{0}^{2}\right)^{3 / 2}} .
$$

At the end of the solenoid, where $z_{0}=0$, the radial component of the magnetic field is

$$
B_{r}(0)=\frac{\mu_{0} n I r}{4 R} .
$$



\textbf{Topic} :Magnetostatic Field and Quasi-Stationary Electromagnetic Fields\\
\textbf{Book} :Problems and Solutions on Electromagnetism\\
\textbf{Final Answer} :\frac{\mu_{0} n I r}{4 R}\\


\textbf{Solution} :Assume that the axis of the current loop, of small radius a, coincides with the axis of rotation of the earth, which is taken to be the $z$-axis as shown in Fig. 2.8. The contribution of a current element $I d I$ to the magnetic induction $B$ at an axial point $z$ is, according to the Biot-Savart law,

$$
d \mathrm{~B}=\frac{\mu_{0}}{4 \pi} \frac{I d \rrbracket \times \mathrm{r}}{r^{3}} .
$$

$d \mathrm{~B}$ is in the plane containing the $z$-axis and $\mathbf{r}$ and is perpendicular to r. Summing up the contributions of all current elements of the loop, by symmetry the resultant $B$ will be along the $z$-axis, i.e.,

$$
\begin{aligned}
&B=B_{z} \mathbf{e}_{z}, \text { or } \\
&d B_{z}=d B \cdot \frac{a}{r} .
\end{aligned}
$$

At the pole, $z=R$. As $R \gg a, r=\sqrt{R^{2}+a^{2}} \approx R$ and

$$
\begin{aligned}
B_{z} &=\frac{\mu_{0}}{4 \pi} \frac{I a}{R^{3}} \oint d l=\frac{\mu_{0}}{4 \pi} \frac{I a}{R^{3}} \cdot 2 \pi a \\
&=\frac{\mu_{0}}{2 \pi} \frac{I S}{R^{3}},
\end{aligned}
$$

where $S=\pi a^{2}$ is the area of the current loop. 

MATHPIX IMAGE

Fig. $2.8$

The magnetic moment of the loop is $\mathbf{m}=I S \mathbf{e}_{z}$, thus

$$
m=\frac{2 \pi R^{3}}{\mu_{0}} B_{z} .
$$

Using the given data $R=6 \times 10^{6} \mathrm{~m}, B_{z}=0.8 \mathrm{Gs}$, we obtain

$$
m \approx 8.64 \times 10^{-26} \mathrm{Am}^{2} \text {. }
$$

\textbf{Topic} :Magnetostatic Field and Quasi-Stationary Electromagnetic Fields\\
\textbf{Book} :Problems and Solutions on Electromagnetism\\
\textbf{Final Answer} :\frac{\mu_{0} n I r}{4 R}\\


\textbf{Solution} :Because $d \ll a$, the electric field in region $I$ is approximately $\mathbb{E}^{(I)}=$ $E_{z}^{(\mathrm{I})} \mathbf{e}_{z}$, where

$$
E_{z}^{(\mathrm{l})}=-\frac{V_{0}}{d} \cos \omega t
$$

at time $t$.

Apply Maxwell's equation

$$
\oint_{L} \mathbf{B} \cdot d \mathbf{l}=\mu_{0} \varepsilon_{0} \int_{S} \frac{\partial \mathbf{E}}{\partial t} \cdot d \mathbf{S}
$$

to a circle $L$ of radius $r$ centered at the line joining the centers of the two plates. By symmetry, $\mathbf{B}^{(\mathrm{I})}=B_{\phi}^{(\mathrm{I})} \mathbf{e}_{\phi}$. Thus one has

$$
2 \pi r B_{\phi}^{(\mathrm{I})}=\mu_{0} \varepsilon_{0} \pi r^{2}\left(\frac{V_{0} \omega}{d} \sin \omega t\right)
$$

or

$$
B_{\phi}^{(\mathrm{I})}=\frac{\mu_{0} \varepsilon_{0} V_{0} \omega}{2 d} r \sin \omega t
$$
\textbf{Topic} :Magnetostatic Field and Quasi-Stationary Electromagnetic Fields\\
\textbf{Book} :Problems and Solutions on Electromagnetism\\
\textbf{Final Answer} :\frac{\mu_{0} \varepsilon_{0} V_{0} \omega}{2 d} r \sin \omega t\\


\textbf{Solution} :Because $d \ll a$, the electric field in region $I$ is approximately $\mathbb{E}^{(I)}=$ $E_{z}^{(\mathrm{I})} \mathbf{e}_{z}$, where

$$
E_{z}^{(\mathrm{l})}=-\frac{V_{0}}{d} \cos \omega t
$$

at time $t$.

Apply Maxwell's equation

$$
\oint_{L} \mathbf{B} \cdot d \mathbf{l}=\mu_{0} \varepsilon_{0} \int_{S} \frac{\partial \mathbf{E}}{\partial t} \cdot d \mathbf{S}
$$

to a circle $L$ of radius $r$ centered at the line joining the centers of the two plates. By symmetry, $\mathbf{B}^{(\mathrm{I})}=B_{\phi}^{(\mathrm{I})} \mathbf{e}_{\phi}$. Thus one has

$$
2 \pi r B_{\phi}^{(\mathrm{I})}=\mu_{0} \varepsilon_{0} \pi r^{2}\left(\frac{V_{0} \omega}{d} \sin \omega t\right)
$$

or

$$
B_{\phi}^{(\mathrm{I})}=\frac{\mu_{0} \varepsilon_{0} V_{0} \omega}{2 d} r \sin \omega t
$$

 Let $\sigma$ be the surface charge density of the upper plate which is the interface between regions I and II. We have

$$
\sigma=-\varepsilon_{0} E_{z}^{(I)}=\frac{\varepsilon_{0} V_{0}}{d} \cos \omega t
$$

Then the total charge on the plate is

$$
Q=\pi a^{2} \sigma=\frac{\pi a^{2} \varepsilon_{0} V_{0}}{d} \cos \omega t
$$

Note that $\sigma$ is uniform because $E_{z}^{(\mathrm{I})}$ is uniform for any instant $t$. The time variation of $Q$ shows that an alternating current $I$ passes through the lead wires:

$$
I=\frac{d Q}{d t}=-\frac{\pi a^{2} \varepsilon_{0} V_{0} \omega}{d} \sin \omega t .
$$

As the charge $Q$ on the plate changes continuously with time, there will be surface current flowing in the plate. As shown in Fig. 2.11, this current flows towards the center of the plate along radial directions. The total current flowing through the shaded loop is

$$
i=-\frac{d}{d t}\left[\pi\left(a^{2}-r^{2}\right) \sigma\right]=\frac{\pi\left(a^{2}-r^{2}\right) \varepsilon_{0} V_{0} \omega}{d} \sin \omega t
$$

Hence, the linear current density (current per unit width) in the plate is

$$
\mathrm{j}_{l}(r)=\frac{i}{2 \pi r} \mathbf{e}_{r}=\frac{\left(a^{2}-r^{2}\right) \varepsilon_{0} V_{0} \omega}{2 d r} \sin (\omega t) \mathbf{e}_{r} .
$$

MATHPIX IMAGE

Fig. $2.11$ 
\textbf{Topic} :Magnetostatic Field and Quasi-Stationary Electromagnetic Fields\\
\textbf{Book} :Problems and Solutions on Electromagnetism\\
\textbf{Final Answer} :\frac{\left(a^{2}-r^{2}\right) \varepsilon_{0} V_{0} \omega}{2 d r} \sin (\omega t) \mathbf{e}_{r}\\


\textbf{Solution} :Because $d \ll a$, the electric field in region $I$ is approximately $\mathbb{E}^{(I)}=$ $E_{z}^{(\mathrm{I})} \mathbf{e}_{z}$, where

$$
E_{z}^{(\mathrm{l})}=-\frac{V_{0}}{d} \cos \omega t
$$

at time $t$.

Apply Maxwell's equation

$$
\oint_{L} \mathbf{B} \cdot d \mathbf{l}=\mu_{0} \varepsilon_{0} \int_{S} \frac{\partial \mathbf{E}}{\partial t} \cdot d \mathbf{S}
$$

to a circle $L$ of radius $r$ centered at the line joining the centers of the two plates. By symmetry, $\mathbf{B}^{(\mathrm{I})}=B_{\phi}^{(\mathrm{I})} \mathbf{e}_{\phi}$. Thus one has

$$
2 \pi r B_{\phi}^{(\mathrm{I})}=\mu_{0} \varepsilon_{0} \pi r^{2}\left(\frac{V_{0} \omega}{d} \sin \omega t\right)
$$

or

$$
B_{\phi}^{(\mathrm{I})}=\frac{\mu_{0} \varepsilon_{0} V_{0} \omega}{2 d} r \sin \omega t
$$

 Let $\sigma$ be the surface charge density of the upper plate which is the interface between regions I and II. We have

$$
\sigma=-\varepsilon_{0} E_{z}^{(I)}=\frac{\varepsilon_{0} V_{0}}{d} \cos \omega t
$$

Then the total charge on the plate is

$$
Q=\pi a^{2} \sigma=\frac{\pi a^{2} \varepsilon_{0} V_{0}}{d} \cos \omega t
$$

Note that $\sigma$ is uniform because $E_{z}^{(\mathrm{I})}$ is uniform for any instant $t$. The time variation of $Q$ shows that an alternating current $I$ passes through the lead wires:

$$
I=\frac{d Q}{d t}=-\frac{\pi a^{2} \varepsilon_{0} V_{0} \omega}{d} \sin \omega t .
$$

As the charge $Q$ on the plate changes continuously with time, there will be surface current flowing in the plate. As shown in Fig. 2.11, this current flows towards the center of the plate along radial directions. The total current flowing through the shaded loop is

$$
i=-\frac{d}{d t}\left[\pi\left(a^{2}-r^{2}\right) \sigma\right]=\frac{\pi\left(a^{2}-r^{2}\right) \varepsilon_{0} V_{0} \omega}{d} \sin \omega t
$$

Hence, the linear current density (current per unit width) in the plate is

$$
\mathrm{j}_{l}(r)=\frac{i}{2 \pi r} \mathbf{e}_{r}=\frac{\left(a^{2}-r^{2}\right) \varepsilon_{0} V_{0} \omega}{2 d r} \sin (\omega t) \mathbf{e}_{r} .
$$

MATHPIX IMAGE

Fig. $2.11$ 

 In Ampere's circuital law

$$
\oint_{L} \mathrm{~B}^{(\mathrm{II})} \cdot d \mathbf{l}=\mu_{0} I
$$

the direction of flow of $I$ and the sense of traversing $L$ follow the righthanded screw rule. At time $t, I$ flows along the $-z$-direction and by axial symmetry

$$
\mathbf{B}^{(\text {II })}=B_{\phi}^{(\mathrm{II})} \mathbf{e}_{\phi}
$$

Hence

$$
B_{\phi}^{(\mathrm{II})}=-\frac{\mu_{0} I}{2 \pi r}=\frac{\mu_{0} \varepsilon_{0} a^{2} V_{0} \omega}{2 d r} \sin \omega t
$$

Thus

$$
\mathbf{B}_{\phi}^{(I I)}-B_{\phi}^{(I)}=\frac{\mu_{0} \varepsilon_{0}\left(a^{2}-r^{2}\right) V_{0} \omega}{2 d r} \sin (\omega t) e_{r}=\mu_{0} j_{l}
$$

or

$$
\mathbf{n} \times\left(\mathbf{B}^{(\mathrm{II})}-\mathbf{B}^{(\mathrm{I})}\right)=\mu_{0} \mathbf{j} \text {. }
$$

This is just the boundary condition for the magnetic field intensity

$$
H_{t}^{(\mathrm{II})}-H_{t}^{(\mathrm{I})}=j_{l} .
$$

\textbf{Topic} :Magnetostatic Field and Quasi-Stationary Electromagnetic Fields\\
\textbf{Book} :Problems and Solutions on Electromagnetism\\
\textbf{Final Answer} :j_{l}\\


\textbf{Solution} :It is $10^{-2} \mathrm{~cm} / \mathrm{sec}$.

\textbf{Topic} :Magnetostatic Field and Quasi-Stationary Electromagnetic Fields\\
\textbf{Book} :Problems and Solutions on Electromagnetism\\
\textbf{Final Answer} :j_{l}\\


\textbf{Solution} :It is $10^{6} \mathrm{~cm} / \mathrm{sec}$.

\textbf{Topic} :Magnetostatic Field and Quasi-Stationary Electromagnetic Fields\\
\textbf{Book} :Problems and Solutions on Electromagnetism\\
\textbf{Final Answer} :j_{l}\\


\textbf{Solution} :As the magnetic permeability of the conductor is the same as that of the vacuum, we can think of the lens-shaped region as being filled with the same conductor without affecting either the magnetic property of the conducting system or the distribution of the magnetic field. We can then consider this region as being traversed by two currents of densities $\pm J$, i.e., having the same magnitude but opposite directions. Thus we have two cylindrical conductors, each having a uniform current distribution, and the magnetic induction in the region is the sum of their contributions. In their own cylindrical coordinates Ampère's circuital law yields

$$
\begin{aligned}
& \mathbf{B}_{1}=-\frac{\mu_{0}}{2} J r_{1} \mathbf{e}_{\varphi_{1}}, \quad\left(r_{1} \leq b\right) \\
& \mathrm{B}_{2}=\frac{\mu_{0}}{2} J r_{2} \mathbf{e}_{\varphi_{2}} . \quad\left(r_{2} \leq b\right)
\end{aligned}
$$

As

$$
\mathbf{e}_{\varphi_{1}}=\left(-\sin \varphi_{1}, \cos \varphi_{1}\right)\left(\begin{array}{l}
\mathbf{e}_{x} \\
\mathbf{e}_{y}
\end{array}\right), \quad \mathbf{e}_{\varphi_{2}}=\left(-\sin \varphi_{2}, \cos \varphi_{2}\right)\left(\begin{array}{l}
\mathbf{e}_{x} \\
\mathbf{e}_{y}
\end{array}\right)
$$

we have

$$
\begin{aligned}
&\mathbf{B}_{1}=\frac{\mu_{0}}{2} J\left(y_{1} \mathbf{e}_{x}-x_{1} \mathbf{e}_{y}\right), \\
&\mathbf{B}_{2}=\frac{\mu_{0}}{2} J\left(-y_{2} \mathbf{e}_{x}+x_{2} \mathbf{e}_{y}\right) .
\end{aligned}
$$

Using the transformation

$$
\left\{\begin{array}{l}
x_{2}=x_{1}-2 a \\
y_{2}=y_{1}
\end{array}\right.
$$

we have

$$
\mathbf{B}_{2}=\frac{\mu_{0}}{2} J\left[-y_{1} \mathbf{e}_{x}+\left(x_{1}-2 a\right) \mathbf{e}_{y}\right] \text {. }
$$

Hence the magnetic field induction in the lens-shaped region is

$$
\mathbf{B}=\mathbf{B}_{1}+\mathbf{B}_{2}=\frac{\mu_{0}}{2} J\left[\left(y_{1}-y_{2}\right) \mathbf{e}_{x}+\left(x_{1}-x_{2}-2 a\right) \mathbf{e}_{y}\right]=-\mu_{0} a J \mathbf{e}_{y} \text {. }
$$

This means that the field is uniform and is in the $-e_{y}$ direction.

\textbf{Topic} :Magnetostatic Field and Quasi-Stationary Electromagnetic Fields\\
\textbf{Book} :Problems and Solutions on Electromagnetism\\
\textbf{Final Answer} :-\mu_{0} a J \mathbf{e}_{y}\\


\textbf{Solution} :Using cylindrical coordinates $(r, \theta, z)$, we have $P=P_{r}=P_{0} r / 2$. The bound charge density is

$$
\rho=-\nabla \cdot \mathbf{P}=-\frac{1}{r} \frac{\partial}{\partial r}\left(r \cdot \frac{P_{0} r}{2}\right)=-P_{0} .
$$
\textbf{Topic} :Magnetostatic Field and Quasi-Stationary Electromagnetic Fields\\
\textbf{Book} :Problems and Solutions on Electromagnetism\\
\textbf{Final Answer} :-P_{0}\\


\textbf{Solution} :Using cylindrical coordinates $(r, \theta, z)$, we have $P=P_{r}=P_{0} r / 2$. The bound charge density is

$$
\rho=-\nabla \cdot \mathbf{P}=-\frac{1}{r} \frac{\partial}{\partial r}\left(r \cdot \frac{P_{0} r}{2}\right)=-P_{0} .
$$

 As $\omega=\omega e_{z}$, the volume current density at a point $r=r e_{r}+z e_{z}$ in the cylinder is

$$
\mathbf{j}(\mathbf{r})=\rho \mathbf{v}=\rho \boldsymbol{\omega} \times \mathbf{r}=-P_{0} \omega \mathbf{e}_{z} \times\left(r \mathbf{e}_{r}+z \mathbf{e}_{z}\right)=-\mathbf{P}_{0} \omega r \mathbf{e}_{\theta} .
$$

On the surface of the cylinder there is also a surface charge distribution, of density

$$
\sigma=\mathbf{n} \cdot \mathbf{P}=\left.\mathbf{e}_{r} \cdot \frac{P_{0} \mathbf{r}}{2}\right|_{r=a}=\frac{P_{0} a}{2} .
$$

This produces a surface current density of

$$
\alpha=\sigma v=\frac{P_{0}}{2} \omega a^{2} e_{\theta} .
$$

To find the magnetic field at a point on the axis of the cylinder not too near its ends, as the cylinder is very long we can take this point as the origin and regard the cylinder as infinitely long. Then the magnetic induction at the origin is given by

$$
\mathbf{B}=-\frac{\mu_{0}}{4 \pi}\left(\int_{V} \frac{\mathrm{j}\left(\mathbf{r}^{\prime}\right) \times \mathbf{r}^{\prime}}{r^{\prime 3}} d V^{\prime}+\int_{S} \frac{\alpha\left(\mathbf{r}^{\prime}\right) \times \mathbf{r}^{\prime}}{r^{\prime 3}} d S^{\prime}\right),
$$

where $V$ and $S$ are respectively the volume and curved surface area of the cylinder and $\mathbf{r}^{\prime}=(r, \theta, z)$ is a source point. Note the minus sign arises because $\mathbf{r}^{\prime}$ directs from the field point to a source point, rather than the other way around. Consider the volume integral

$$
\begin{aligned}
\int_{V} \frac{\mathbf{j}\left(\mathbf{r}^{\prime}\right) \times \mathbf{r}^{\prime}}{r^{3}} d V^{\prime} &=\int_{V} \frac{-P_{0} \omega r \mathbf{e}_{\theta} \times\left(r \mathbf{e}_{r}+z \mathbf{e}_{z}\right)}{\left(r^{2}+z^{2}\right)^{3 / 2}} r d r d \theta d z \\
&=P_{0} \omega\left[\int_{V} \frac{r^{3} d r d \theta d z}{\left(r^{2}+z^{2}\right)^{3 / 2}} \mathbf{e}_{z}-\int_{V} \frac{r^{2} d r d \theta d z}{\left(r^{2}+z^{2}\right)^{3 / 2}} \mathbf{e}_{r}\right]
\end{aligned}
$$

As the cylinder can be considered infinitely long, by symmetry the second integral vanishes. For the first integral we put $z=r \tan \beta$. We then have

$$
\int_{V} \frac{j \times \mathbf{r}^{\prime}}{r^{\prime 3}} d V^{\prime}=P_{0} \omega \int_{0}^{2 \pi} d \theta \int_{0}^{a} r d r \int_{-\frac{\pi}{2}}^{\frac{\pi}{2}} \cos \beta d \beta \mathbf{e}_{x}=2 \pi P_{0} \omega a^{2} \mathbf{e}_{z}
$$

Similary, the surface integral gives

$$
\begin{aligned}
\int_{S} \frac{\alpha\left(r^{\prime}\right) \times r^{\prime}}{r^{\prime 3}} d S^{\prime} &=\int_{S} \frac{\frac{P_{0}}{2} \omega a^{2} \mathbf{e}_{\theta} \times\left(a \mathbf{e}_{r}+z \mathbf{e}_{z}\right)}{\left(a^{2}+z^{2}\right)^{3 / 2}} d S^{\prime} \\
&=-\frac{P_{0}}{2} \omega a^{4} \int_{S} \frac{d \theta d z}{\left(a^{2}+z^{2}\right)^{3 / 2}} \mathbf{e}_{z}=-2 \pi P_{0} \omega a^{2} \mathbf{e}_{z} .
\end{aligned}
$$

Hence, the magnetic induction $B$ vanishes at points of the cylindrical axis not too near the ends.

\textbf{Topic} :Magnetostatic Field and Quasi-Stationary Electromagnetic Fields\\
\textbf{Book} :Problems and Solutions on Electromagnetism\\
\textbf{Final Answer} :-2 \pi P_{0} \omega a^{2} \mathbf{e}_{z}\\


\textbf{Solution} :Use cylindrical coordinates $(r, \theta, z)$ where the $z$ axis is along the axis of the cable and its positive direction is the same as that of the current in the inner conductor. From $\oint_{C} \mathrm{~B} \cdot d \mathrm{l}=\mu_{0} i$ and the axial symmetry we have

$$
B=\frac{\mu_{0} i}{2 \pi r} \mathbf{e}_{\theta} .
$$

From $\oint_{S} \mathbf{E} \cdot d \mathbf{S}=\frac{Q}{\varepsilon_{0}}$ and the axial symmetry we have

$$
\mathbf{E}=\frac{\lambda}{2 \pi \varepsilon_{0} r} \mathbf{e}_{r},
$$

where $\lambda$ is the charge per unit length of the inner conductor. The voltage between the conductors is $V=-\int_{R_{2}}^{R_{1}} \mathbf{E} \cdot d \mathbf{r}$, giving

$$
\lambda=2 \pi \varepsilon_{0} V / \ln \frac{R_{2}}{R_{1}} .
$$

Accordingly,

$$
\mathbf{E}=\frac{V}{r \ln \frac{R_{2}}{R_{1}}} \mathrm{e}_{r} .
$$
\textbf{Topic} :Magnetostatic Field and Quasi-Stationary Electromagnetic Fields\\
\textbf{Book} :Problems and Solutions on Electromagnetism\\
\textbf{Final Answer} :\frac{V}{r \ln \frac{R_{2}}{R_{1}}} \mathrm{e}_{r}\\


\textbf{Solution} :Use cylindrical coordinates $(r, \theta, z)$ where the $z$ axis is along the axis of the cable and its positive direction is the same as that of the current in the inner conductor. From $\oint_{C} \mathrm{~B} \cdot d \mathrm{l}=\mu_{0} i$ and the axial symmetry we have

$$
B=\frac{\mu_{0} i}{2 \pi r} \mathbf{e}_{\theta} .
$$

From $\oint_{S} \mathbf{E} \cdot d \mathbf{S}=\frac{Q}{\varepsilon_{0}}$ and the axial symmetry we have

$$
\mathbf{E}=\frac{\lambda}{2 \pi \varepsilon_{0} r} \mathbf{e}_{r},
$$

where $\lambda$ is the charge per unit length of the inner conductor. The voltage between the conductors is $V=-\int_{R_{2}}^{R_{1}} \mathbf{E} \cdot d \mathbf{r}$, giving

$$
\lambda=2 \pi \varepsilon_{0} V / \ln \frac{R_{2}}{R_{1}} .
$$

Accordingly,

$$
\mathbf{E}=\frac{V}{r \ln \frac{R_{2}}{R_{1}}} \mathrm{e}_{r} .
$$

 The magnetic energy density is $w_{m}=\frac{B_{2}}{2 \mu_{0}}=\frac{\mu_{0}}{2}\left(\frac{i}{2 \pi r}\right)^{2}$. Hence the magnetic energy per unit length is

$$
\frac{d W_{m}}{d z}=\int w_{m} d S=\int_{R_{1}}^{R_{2}} \frac{\mu_{0}}{2}\left(\frac{i}{2 \pi r}\right)^{2} \cdot 2 \pi r d r=\frac{\mu_{0} i^{2}}{4 \pi} \ln \frac{R_{2}}{R_{1}} .
$$

The electric energy density is $w_{e}=\frac{\varepsilon_{0} E^{2}}{2}=\frac{\varepsilon_{0}}{2}\left(\frac{V}{2 \ln \frac{R_{2}}{R_{1}}}\right)^{2}$. Hence the electric energy per unit length is

$$
\frac{d W_{e}}{d z}=\int w_{e} d S=\frac{\pi \varepsilon_{0} V^{2}}{\ln \frac{R_{2}}{R_{1}}} .
$$
\textbf{Topic} :Magnetostatic Field and Quasi-Stationary Electromagnetic Fields\\
\textbf{Book} :Problems and Solutions on Electromagnetism\\
\textbf{Final Answer} :\frac{\pi \varepsilon_{0} V^{2}}{\ln \frac{R_{2}}{R_{1}}}\\


\textbf{Solution} :Use cylindrical coordinates $(r, \theta, z)$ where the $z$ axis is along the axis of the cable and its positive direction is the same as that of the current in the inner conductor. From $\oint_{C} \mathrm{~B} \cdot d \mathrm{l}=\mu_{0} i$ and the axial symmetry we have

$$
B=\frac{\mu_{0} i}{2 \pi r} \mathbf{e}_{\theta} .
$$

From $\oint_{S} \mathbf{E} \cdot d \mathbf{S}=\frac{Q}{\varepsilon_{0}}$ and the axial symmetry we have

$$
\mathbf{E}=\frac{\lambda}{2 \pi \varepsilon_{0} r} \mathbf{e}_{r},
$$

where $\lambda$ is the charge per unit length of the inner conductor. The voltage between the conductors is $V=-\int_{R_{2}}^{R_{1}} \mathbf{E} \cdot d \mathbf{r}$, giving

$$
\lambda=2 \pi \varepsilon_{0} V / \ln \frac{R_{2}}{R_{1}} .
$$

Accordingly,

$$
\mathbf{E}=\frac{V}{r \ln \frac{R_{2}}{R_{1}}} \mathrm{e}_{r} .
$$

 The magnetic energy density is $w_{m}=\frac{B_{2}}{2 \mu_{0}}=\frac{\mu_{0}}{2}\left(\frac{i}{2 \pi r}\right)^{2}$. Hence the magnetic energy per unit length is

$$
\frac{d W_{m}}{d z}=\int w_{m} d S=\int_{R_{1}}^{R_{2}} \frac{\mu_{0}}{2}\left(\frac{i}{2 \pi r}\right)^{2} \cdot 2 \pi r d r=\frac{\mu_{0} i^{2}}{4 \pi} \ln \frac{R_{2}}{R_{1}} .
$$

The electric energy density is $w_{e}=\frac{\varepsilon_{0} E^{2}}{2}=\frac{\varepsilon_{0}}{2}\left(\frac{V}{2 \ln \frac{R_{2}}{R_{1}}}\right)^{2}$. Hence the electric energy per unit length is

$$
\frac{d W_{e}}{d z}=\int w_{e} d S=\frac{\pi \varepsilon_{0} V^{2}}{\ln \frac{R_{2}}{R_{1}}} .
$$

 From $\frac{d W_{m}}{d z}=\frac{1}{2}\left(\frac{d L}{d z}\right) i^{2}$, the inductance per unit length is

$$
\frac{d L}{d z}=\frac{\mu_{0}}{2 \pi} \ln \frac{R_{2}}{R_{1}} .
$$

From $\frac{d W_{c}}{d z}=\frac{1}{2}\left(\frac{d C}{d z}\right) V^{2}$, the capacitance per unit length is

$$
\frac{d C}{d z}=\frac{2 \pi \varepsilon_{0}}{\ln \frac{R_{2}}{R_{1}}} .
$$

\textbf{Topic} :Magnetostatic Field and Quasi-Stationary Electromagnetic Fields\\
\textbf{Book} :Problems and Solutions on Electromagnetism\\
\textbf{Final Answer} :\frac{2 \pi \varepsilon_{0}}{\ln \frac{R_{2}}{R_{1}}}\\


\textbf{Solution} :As in Fig. 2.16, consider a small cylinder of thickness $d z$ and radius $r$ at and perpendicular to the $z$-axis and apply Maxwell's equation $\oint_{S} \mathbf{B}$. $d \mathbf{S}=0$. As $r$ is very small, we have

$$
B_{z}(r, z) \approx B_{z}(0, z) .
$$

Hence

$$
\left[B_{z}(0, z+d z)-B_{z}(0, z)\right] \pi r^{2}+B_{r}(r, z) 2 \pi r d z=0,
$$

or

$$
-\frac{\partial B(0, z)}{\partial z} d z \cdot \pi r^{2}=B_{r}(r, z) \cdot 2 \pi r d z,
$$

giving

$$
B_{r}(r, z)=-\frac{r}{2} \frac{\partial B(0, z)}{\partial z}=-\frac{r}{2} \frac{d}{d z}\left[B_{0}\left(1+\nu z^{2}\right)\right]=-\nu B_{0} r z .
$$



MATHPIX IMAGE

Fig. $2.16$
\textbf{Topic} :Magnetostatic Field and Quasi-Stationary Electromagnetic Fields\\
\textbf{Book} :Problems and Solutions on Electromagnetism\\
\textbf{Final Answer} :-\nu B_{0} r z\\


\textbf{Solution} :According to Ampère's circuital law

$$
\begin{aligned}
\oint \mathbf{H} \cdot d \boldsymbol{d} &=N I, \\
H &=\frac{N I}{2 \pi r},
\end{aligned}
$$

where $r$ is the distance from the axis of the toroid.

The magnetization $M$ inside the iron is

$$
M=\frac{B}{\mu_{0}}-H=\frac{\mu H}{\mu_{0}}-H=\frac{\mu-1}{\mu_{0}} \cdot \frac{N I}{2 \pi r} .
$$

\textbf{Topic} :Magnetostatic Field and Quasi-Stationary Electromagnetic Fields\\
\textbf{Book} :Problems and Solutions on Electromagnetism\\
\textbf{Final Answer} :\frac{\mu-1}{\mu_{0}} \cdot \frac{N I}{2 \pi r}\\


\textbf{Solution} :$L$ is the periphery of a cross section of the magnet parallel to the plane of the diagram as shown in Fig. 2.20. Ampère's circuital law becomes

$$
\oint_{L} \mathrm{H} \cdot d \mathrm{l}=\frac{B}{\mu_{0}} x+\frac{B}{\mu}(L-x)=N I,
$$

where $x$ is the width of the gap. As $\mu \gg \mu_{0}$, the second term in the middle may be neglected. Denoting the cross section of the copper wire by $S$, the current crossing $S$ is $I=j S$. Together we have

$$
N=\frac{B x}{\mu_{0} j S} .
$$

The power dissipated in the wire, which is the power required, is

$$
P=I^{2} R=I^{2} \rho \frac{2 N(a+b)}{S}=2 j \rho(a+b) \frac{B}{\mu_{0}} x
$$

where $\rho$ is the resistivity of copper. Using the given data, we get

$$
P=9.5 \times 10^{4} \mathrm{~W} \text {. }
$$

MATHPIX IMAGE

Fig. $2.20$ Let $\delta$ be the density of copper, then the necessary weight of the copper is

$$
2 N(a+b) S \delta=2(a+b) \frac{B}{j \mu_{0}} x \delta=3.8 \mathrm{~kg} .
$$

The cross section of the gap is $A=a \cdot b$. Hence the force of attraction between the plates is

$$
F=\frac{A B^{2}}{2 \mu_{0}}=8 \times 10^{5} \mathrm{~N} .
$$

\textbf{Topic} :Magnetostatic Field and Quasi-Stationary Electromagnetic Fields\\
\textbf{Book} :Problems and Solutions on Electromagnetism\\
\textbf{Final Answer} :8 \times 10^{5} \mathrm{~N}\\


\textbf{Solution} :As $s \ll d \ll R$, magnetic leakage in the gap may be neglected. The magnetic field in the gap is then the same as that in the rod. From Ampère's circuital law

$$
\oint \mathbf{H} \cdot d \mathbf{l}=N I
$$

we obtain

$$
B=\frac{\mu_{r} \mu_{0} N I}{2 \pi R+\left(\mu_{r}-1\right) s}
$$

\textbf{Topic} :Magnetostatic Field and Quasi-Stationary Electromagnetic Fields\\
\textbf{Book} :Problems and Solutions on Electromagnetism\\
\textbf{Final Answer} :\frac{\mu_{r} \mu_{0} N I}{2 \pi R+\left(\mu_{r}-1\right) s}\\


\textbf{Solution} :According to the principle of superposition the field can be regarded as the difference of two fields $\mathbf{H}_{2}$ and $\mathbf{H}_{1}$, where $\mathbf{H}_{2}$ is the field produced by a solid (without the hole) cylinder of radius $3 a$ and $\mathbf{H}_{1}$ is that produced by a cylinder of radius $a$ at the position of the hole. The current in each of these two cylinders is uniformly distributed over the cross section. The currents $I_{1}$ and $I_{2}$ in the small and large cylinders have current densities $-j$ and $+j$ respectively. Then as $I=I_{2}-I_{1}=9 \pi a^{2} j-\pi a^{2} j=8 \pi a^{2} j$, we have $j=\frac{I}{8 \pi a^{2}}$ and

$$
I_{1}=\pi a^{2} j=\frac{I}{8}, \quad I_{2}=9 \pi a^{2} j=\frac{9}{8} I .
$$

Take the $z$-axis along the axis of the large cylinder with its positive direction in the direction of $I_{2}$, which we assume to be upwards from the plane of the paper. Take the $x$-axis crossing the axis of the small cylinder as shown in Fig. 2.21. Then the plane $P$ is the $x z$ plane, i.e., the plane $y=0$. Ampere's law gives $\mathbf{H}_{1}$ and $\mathbf{H}_{2}$ as follows (noting $r=\sqrt{x^{2}+y^{2}}, r_{1}=$ $\sqrt{(x-a)^{2}+y^{2}}$, being the distances of the field point from the cylinder and hole respectively):

$$
\begin{array}{lll}
H_{2 x}=-\frac{I y}{16 \pi a^{2}}, & H_{2 y}=\frac{I x}{16 \pi a^{2}}, & (r \leq 3 a) \\
H_{2 x}=-\frac{9 I y}{16 \pi\left(x^{2}+y^{2}\right)}, & H_{2 y}=\frac{9 I x}{16 \pi\left(x^{2}+y^{2}\right)}, & (r>3 a)
\end{array}
$$



$$
\begin{aligned}
& H_{1 x}=-\frac{I y}{16 \pi a^{2}}, \\
& H_{1 y}=\frac{I(x-a)}{16 \pi a^{2}} \text {, } \\
& \left(p_{1} \leq a\right) \\
& H_{1 x}=-\frac{I y}{16 \pi\left[(x-a)^{2}+y^{2}\right]}, \quad H_{1 y}=\frac{I(x-a)}{16 \pi\left[(x-a)^{2}+y^{2}\right]}, \quad\left(r_{1}>a\right) .
\end{aligned}
$$

On the plane $P, H_{2 x}=H_{1 x}=0$. Hence $H_{x}=0, H_{y}=H_{2 y}-H_{1 y}$. We therefore have the following:

(1) Inside the hole $(0<x<2 a)$,

$$
H_{y}=\frac{I x}{16 \pi a^{2}}-\frac{I(x-a)}{16 \pi a^{2}}=\frac{I a}{16 \pi a^{2}}=\frac{I}{16 \pi a} .
$$

(2) Inside the solid part $(2 a \leq x \leq 3 a$ or $-3 a \leq x \leq 0)$,

$$
H_{y}=\frac{I x}{16 \pi a^{2}}-\frac{I(x-a)}{16 \pi\left[(x-a)^{2}+y^{2}\right]}=\frac{I\left(x^{2}-a x-a^{2}\right)}{16 \pi a^{2}(x-a)} .
$$

(3) Outside the cylinder $(|x|>3 a)$,

$$
H_{y}=\frac{9 I x}{16 \pi\left(x^{2}+y^{2}\right)}-\frac{I(x-a)}{16 \pi\left[(x-a)^{2}+y^{2}\right]}=\frac{(8 x-9 a) I}{16 \pi x(x-a)} .
$$
\textbf{Topic} :Magnetostatic Field and Quasi-Stationary Electromagnetic Fields\\
\textbf{Book} :Problems and Solutions on Electromagnetism\\
\textbf{Final Answer} :\frac{(8 x-9 a) I}{16 \pi x(x-a)}\\


\textbf{Solution} :According to the principle of superposition the field can be regarded as the difference of two fields $\mathbf{H}_{2}$ and $\mathbf{H}_{1}$, where $\mathbf{H}_{2}$ is the field produced by a solid (without the hole) cylinder of radius $3 a$ and $\mathbf{H}_{1}$ is that produced by a cylinder of radius $a$ at the position of the hole. The current in each of these two cylinders is uniformly distributed over the cross section. The currents $I_{1}$ and $I_{2}$ in the small and large cylinders have current densities $-j$ and $+j$ respectively. Then as $I=I_{2}-I_{1}=9 \pi a^{2} j-\pi a^{2} j=8 \pi a^{2} j$, we have $j=\frac{I}{8 \pi a^{2}}$ and

$$
I_{1}=\pi a^{2} j=\frac{I}{8}, \quad I_{2}=9 \pi a^{2} j=\frac{9}{8} I .
$$

Take the $z$-axis along the axis of the large cylinder with its positive direction in the direction of $I_{2}$, which we assume to be upwards from the plane of the paper. Take the $x$-axis crossing the axis of the small cylinder as shown in Fig. 2.21. Then the plane $P$ is the $x z$ plane, i.e., the plane $y=0$. Ampere's law gives $\mathbf{H}_{1}$ and $\mathbf{H}_{2}$ as follows (noting $r=\sqrt{x^{2}+y^{2}}, r_{1}=$ $\sqrt{(x-a)^{2}+y^{2}}$, being the distances of the field point from the cylinder and hole respectively):

$$
\begin{array}{lll}
H_{2 x}=-\frac{I y}{16 \pi a^{2}}, & H_{2 y}=\frac{I x}{16 \pi a^{2}}, & (r \leq 3 a) \\
H_{2 x}=-\frac{9 I y}{16 \pi\left(x^{2}+y^{2}\right)}, & H_{2 y}=\frac{9 I x}{16 \pi\left(x^{2}+y^{2}\right)}, & (r>3 a)
\end{array}
$$



$$
\begin{aligned}
& H_{1 x}=-\frac{I y}{16 \pi a^{2}}, \\
& H_{1 y}=\frac{I(x-a)}{16 \pi a^{2}} \text {, } \\
& \left(p_{1} \leq a\right) \\
& H_{1 x}=-\frac{I y}{16 \pi\left[(x-a)^{2}+y^{2}\right]}, \quad H_{1 y}=\frac{I(x-a)}{16 \pi\left[(x-a)^{2}+y^{2}\right]}, \quad\left(r_{1}>a\right) .
\end{aligned}
$$

On the plane $P, H_{2 x}=H_{1 x}=0$. Hence $H_{x}=0, H_{y}=H_{2 y}-H_{1 y}$. We therefore have the following:

(1) Inside the hole $(0<x<2 a)$,

$$
H_{y}=\frac{I x}{16 \pi a^{2}}-\frac{I(x-a)}{16 \pi a^{2}}=\frac{I a}{16 \pi a^{2}}=\frac{I}{16 \pi a} .
$$

(2) Inside the solid part $(2 a \leq x \leq 3 a$ or $-3 a \leq x \leq 0)$,

$$
H_{y}=\frac{I x}{16 \pi a^{2}}-\frac{I(x-a)}{16 \pi\left[(x-a)^{2}+y^{2}\right]}=\frac{I\left(x^{2}-a x-a^{2}\right)}{16 \pi a^{2}(x-a)} .
$$

(3) Outside the cylinder $(|x|>3 a)$,

$$
H_{y}=\frac{9 I x}{16 \pi\left(x^{2}+y^{2}\right)}-\frac{I(x-a)}{16 \pi\left[(x-a)^{2}+y^{2}\right]}=\frac{(8 x-9 a) I}{16 \pi x(x-a)} .
$$

 The magnetic field at all points inside the hole $\left(r_{1} \leq a\right)$ is

$$
\begin{aligned}
&H_{x}=-\frac{I y}{16 \pi a^{2}}+\frac{I y}{16 \pi a^{2}}=0, \\
&H_{y}=\frac{I x}{16 \pi a^{2}}-\frac{I(x-a)}{16 \pi a^{2}}=\frac{I}{16 \pi a} .
\end{aligned}
$$

This field is uniform inside the hole and is along the positive $y$-direction.

\textbf{Topic} :Magnetostatic Field and Quasi-Stationary Electromagnetic Fields\\
\textbf{Book} :Problems and Solutions on Electromagnetism\\
\textbf{Final Answer} :\frac{I}{16 \pi a}\\


\textbf{Solution} :If the sphere carries charge $Q$, the potential on its surface is

$$
V=\frac{Q}{4 \pi \varepsilon_{0} r},
$$

i.e., $Q=4 \pi \varepsilon_{0} r V$. When the sphere is immersed in a conducting medium of conductivity $\sigma$, the current that starts to flow out from the sphere is

$$
I=\oint_{S} \mathbf{J} \cdot d \mathbf{S}=\sigma \oint_{S} \mathbf{E} \cdot d \mathbf{S}=\sigma \frac{Q}{\varepsilon_{0}}=4 \pi \sigma r V,
$$

where $S$ is the spherical surface and we have assumed the medium to be Ohmic. If the potential $V$ is maintained, the current $I$ is steady.
\textbf{Topic} :Magnetostatic Field and Quasi-Stationary Electromagnetic Fields\\
\textbf{Book} :Problems and Solutions on Electromagnetism\\
\textbf{Final Answer} :4 \pi \sigma r V\\


\textbf{Solution} :If the sphere carries charge $Q$, the potential on its surface is

$$
V=\frac{Q}{4 \pi \varepsilon_{0} r},
$$

i.e., $Q=4 \pi \varepsilon_{0} r V$. When the sphere is immersed in a conducting medium of conductivity $\sigma$, the current that starts to flow out from the sphere is

$$
I=\oint_{S} \mathbf{J} \cdot d \mathbf{S}=\sigma \oint_{S} \mathbf{E} \cdot d \mathbf{S}=\sigma \frac{Q}{\varepsilon_{0}}=4 \pi \sigma r V,
$$

where $S$ is the spherical surface and we have assumed the medium to be Ohmic. If the potential $V$ is maintained, the current $I$ is steady.

 As $d \gg r$, we can regard the spheres as point charges. Suppose the sphere with potential $V$ carries a net charge $+Q$ and that with potential $0,-Q$. Take the line joining the two spherical centers as the $x$-axis and the mid-point of this line as the origin. Then the potential of an arbitrary point $x$ on the line is

$$
V(x)=\frac{Q}{4 \pi \varepsilon_{0}}\left(\frac{1}{d-x}-\frac{1}{d+x}\right) .
$$

The potential difference between the two spherical surfaces is then

$$
\begin{aligned}
V &=\left.\frac{Q}{4 \pi \varepsilon_{0}}\left(\frac{1}{d-x}-\frac{1}{d+x}\right)\right|_{-d+r} ^{d-r} \\
&=\frac{Q}{4 \pi \varepsilon_{0}} \cdot \frac{4(d-r)}{r(2 d-r)} \approx \frac{Q}{2 \pi \varepsilon_{0} r}
\end{aligned}
$$

as $d \gg r$. Hence

$$
Q=2 \pi \varepsilon_{0} r V .
$$

On the $y z$ plane the points which are equidistant from the two spheres will constitute a circle with center at the origin. By symmetry the magnitudes of the electric and magnetic fields at these points are the same, so we need only calculate them for a point, say the intersection of the circle and the $z$-axis (see Fig. 2.22). Let $R$ be the radius of the circle, $\mathbf{E}_{1}$ and $\mathbf{E}_{2}$ be the electric fields produced by $+Q$ and $-Q$ respectively. The resultant of these fields is along the $-x$ direction:

$$
\mathbf{E}=-\frac{2 Q}{4 \pi \varepsilon_{0}\left(R^{2}+d^{2}\right)} \cos \theta \mathbf{e}_{x}=-\frac{Q d}{2 \pi \varepsilon_{0}\left(R^{2}+d^{2}\right)^{3 / 2}} \mathbf{e}_{x}
$$



MATHPIX IMAGE

Fig. $2.22$

The current density at this point is then

$$
J=\sigma \mathbf{E}=-\frac{\sigma Q d}{2 \pi \varepsilon_{0}\left(R^{2}+d^{2}\right)^{3 / 2}} \mathbf{e}_{x}=-\frac{V r d}{\left(R^{2}+d^{2}\right)^{3 / 2}} \mathbf{e}_{x}
$$

As the choice of $z$-axis is arbitrary, the above results apply to all points of the circle.
\textbf{Topic} :Magnetostatic Field and Quasi-Stationary Electromagnetic Fields\\
\textbf{Book} :Problems and Solutions on Electromagnetism\\
\textbf{Final Answer} :-\frac{V r d}{\left(R^{2}+d^{2}\right)^{3 / 2}} \mathbf{e}_{x}\\


\textbf{Solution} :If the sphere carries charge $Q$, the potential on its surface is

$$
V=\frac{Q}{4 \pi \varepsilon_{0} r},
$$

i.e., $Q=4 \pi \varepsilon_{0} r V$. When the sphere is immersed in a conducting medium of conductivity $\sigma$, the current that starts to flow out from the sphere is

$$
I=\oint_{S} \mathbf{J} \cdot d \mathbf{S}=\sigma \oint_{S} \mathbf{E} \cdot d \mathbf{S}=\sigma \frac{Q}{\varepsilon_{0}}=4 \pi \sigma r V,
$$

where $S$ is the spherical surface and we have assumed the medium to be Ohmic. If the potential $V$ is maintained, the current $I$ is steady.

 As $d \gg r$, we can regard the spheres as point charges. Suppose the sphere with potential $V$ carries a net charge $+Q$ and that with potential $0,-Q$. Take the line joining the two spherical centers as the $x$-axis and the mid-point of this line as the origin. Then the potential of an arbitrary point $x$ on the line is

$$
V(x)=\frac{Q}{4 \pi \varepsilon_{0}}\left(\frac{1}{d-x}-\frac{1}{d+x}\right) .
$$

The potential difference between the two spherical surfaces is then

$$
\begin{aligned}
V &=\left.\frac{Q}{4 \pi \varepsilon_{0}}\left(\frac{1}{d-x}-\frac{1}{d+x}\right)\right|_{-d+r} ^{d-r} \\
&=\frac{Q}{4 \pi \varepsilon_{0}} \cdot \frac{4(d-r)}{r(2 d-r)} \approx \frac{Q}{2 \pi \varepsilon_{0} r}
\end{aligned}
$$

as $d \gg r$. Hence

$$
Q=2 \pi \varepsilon_{0} r V .
$$

On the $y z$ plane the points which are equidistant from the two spheres will constitute a circle with center at the origin. By symmetry the magnitudes of the electric and magnetic fields at these points are the same, so we need only calculate them for a point, say the intersection of the circle and the $z$-axis (see Fig. 2.22). Let $R$ be the radius of the circle, $\mathbf{E}_{1}$ and $\mathbf{E}_{2}$ be the electric fields produced by $+Q$ and $-Q$ respectively. The resultant of these fields is along the $-x$ direction:

$$
\mathbf{E}=-\frac{2 Q}{4 \pi \varepsilon_{0}\left(R^{2}+d^{2}\right)} \cos \theta \mathbf{e}_{x}=-\frac{Q d}{2 \pi \varepsilon_{0}\left(R^{2}+d^{2}\right)^{3 / 2}} \mathbf{e}_{x}
$$



MATHPIX IMAGE

Fig. $2.22$

The current density at this point is then

$$
J=\sigma \mathbf{E}=-\frac{\sigma Q d}{2 \pi \varepsilon_{0}\left(R^{2}+d^{2}\right)^{3 / 2}} \mathbf{e}_{x}=-\frac{V r d}{\left(R^{2}+d^{2}\right)^{3 / 2}} \mathbf{e}_{x}
$$

As the choice of $z$-axis is arbitrary, the above results apply to all points of the circle.

 Using a circle of radius $R$ as the loop $L$, in Ampère's circuital law

$$
\oint_{L} \mathbf{B} \cdot d l^{\prime}=\mu_{0} \int_{S} \mathbf{J} \cdot d S^{\prime}, \quad d S^{\prime}=r^{\prime} d r^{\prime} d \theta^{\prime} \mathbf{e}_{x}
$$

we have

$$
\begin{aligned}
2 \pi R B &=-\mu_{0} \int_{0}^{2 \pi} d \theta^{\prime} \int_{0}^{R} \frac{V r d}{\left(r^{\prime 2}+d^{2}\right)^{3 / 2}} d r^{\prime} \\
&=2 \pi \mu_{0} V r d\left(\frac{1}{\sqrt{R^{2}+d^{2}}}-\frac{1}{d}\right)
\end{aligned}
$$

For distant points, $R \gg d$ and we obtain to good approximation

$$
B=\frac{\mu_{0} V r}{R} \text {. }
$$

Note $\mathrm{B}$ is tangential to the circle $R$ and is clockwise when viewed from the side of positive $x$.

\textbf{Topic} :Magnetostatic Field and Quasi-Stationary Electromagnetic Fields\\
\textbf{Book} :Problems and Solutions on Electromagnetism\\
\textbf{Final Answer} :\frac{\mu_{0} V r}{R}\\


\textbf{Solution} :Suppose current $I_{1}$ passes through the outer coil, then the magnetic induction produced by it is

$$
B_{1}=\mu_{0} \frac{N_{1} I_{1}}{l_{1}} .
$$

As $r_{2} \ll r_{1}, l_{2} \ll l_{1}$, we may consider the magnetic field $B_{1}$ as uniform across the inner coil. Thus the magnetic flux crossing the inner coil is

$$
\psi_{12}=N_{2} B_{1} S_{2}=\mu_{0} \frac{N_{1} N_{2} I_{1}}{l_{1}} \pi r_{2}^{2},
$$

which gives the mutual inductance of the two coils as

$$
M=\frac{\psi_{12}}{I_{1}}=\mu_{0} \frac{N_{1} N_{2}}{l_{1}} \pi r_{2}^{2} \approx 39.5 \mu \mathrm{H} .
$$

\textbf{Topic} :Magnetostatic Field and Quasi-Stationary Electromagnetic Fields\\
\textbf{Book} :Problems and Solutions on Electromagnetism\\
\textbf{Final Answer} :\mu_{0} \frac{N_{1} N_{2}}{l_{1}} \pi r_{2}^{2} \approx 395 \mu \mathrm{H}\\


\textbf{Solution} :At a point distance $r$ from the spherical center the magnetic field established by the small magnet is

$$
\mathbf{B}=\frac{\mu_{0}}{4 \pi}\left[\frac{3(\mathbf{M} \cdot \mathbf{r}) \mathbf{r}}{r^{5}}-\frac{M}{r^{3}}\right], \quad \mathbf{M}=M \mathbf{e}_{z}
$$

Let $C$ be the mid-point of arc $\widehat{P Q}$. The velocity of a point on arc $\widehat{P C}$ is $\mathbf{v}=\omega R \sin \theta e_{\varphi}$. The induced emf between the points $P$ and $C$ along arc $\widehat{P Q}$ is given by

$$
\varepsilon_{P C}=\int_{P}^{C}(\mathbf{v} \times \mathrm{B}) \cdot d l, \quad \text { with } \quad d l=R d \theta \mathrm{e}_{\theta}
$$

As $\mathbf{e}_{z}=\cos \theta \mathbf{e}_{r}-\sin \theta \mathbf{e}_{\theta}, \mathbf{e}_{\varphi} \times \mathbf{e}_{r}=\mathbf{e}_{\theta}, \mathbf{e}_{\varphi} \times \mathbf{e}_{\theta}=-\mathbf{e}_{r}$, we have

$$
\mathbf{v} \times \mathrm{B}=\frac{\mu_{0} \omega R M \sin \theta}{4 \pi R^{3}}\left(2 \cos \theta \mathbf{e}_{\theta}+\sin \theta \mathbf{e}_{r}\right)
$$

and

$$
\varepsilon_{P C}=\frac{\mu_{0} M \omega}{4 \pi R} \int_{0}^{\frac{\pi}{2}} 2 \cos \theta \sin \theta d \theta=\frac{\mu_{0} M \omega}{4 \pi R}
$$

\textbf{Topic} :Magnetostatic Field and Quasi-Stationary Electromagnetic Fields\\
\textbf{Book} :Problems and Solutions on Electromagnetism\\
\textbf{Final Answer} :\frac{\mu_{0} M \omega}{4 \pi R}\\


\textbf{Solution} :The magnetic field produced by an infinite straight wire carrying current $I$ at a point distance $r$ from the wire is given by Ampère's circuital law as

$$
B=\frac{\mu_{0} I}{2 \pi r},
$$

its direction being perpendicular to the wire. Thus the magnetic flux crossing the loop due to the wire farther away from the loop is

$$
\phi_{1}=\int_{2 d}^{3 d} \frac{\mu_{0} I d}{2 \pi r} d r=\frac{\mu_{0} I d}{2 \pi} \ln \frac{3}{2}
$$

directing into the page. The other wire, which is nearer the loop, gives rise to the magnetic flux

$$
\phi_{2}=\int_{d}^{2 d} \frac{\mu_{0} I d}{2 \pi r} d r=\frac{\mu_{0} I d}{2 \pi} \ln 2
$$

pointing out from the page. Hence the total flux is

$$
\phi=\phi_{2}-\phi_{1}=\frac{\mu_{0} I d}{2 \pi} \ln \frac{4}{3}
$$

pointing out from the page. The emf induced in the square loop is therefore

$$
\varepsilon=-\frac{d \phi}{d t}=-\frac{\mu_{0} d}{2 \pi} \ln \left(\frac{4}{3}\right) \frac{d I}{d t} .
$$
\textbf{Topic} :Magnetostatic Field and Quasi-Stationary Electromagnetic Fields\\
\textbf{Book} :Problems and Solutions on Electromagnetism\\
\textbf{Final Answer} :-\frac{\mu_{0} d}{2 \pi} \ln \left(\frac{4}{3}\right) \frac{d I}{d t}\\


\textbf{Solution} :The magnetic field at a point between the two conductors at distance $r$ from one conductor is

$$
\mathrm{B}(r)=\frac{\mu_{0} I}{2 \pi}\left(\frac{1}{r}+\frac{1}{2 a-r}\right) \mathbf{e}_{\theta} .
$$

So the magnetic flux crossing the area of the ring is given by

Let $x=a-r$ and integrate:

$$
\begin{aligned}
\phi &=\int \mathrm{B} \cdot d \mathrm{~S}=2 \int_{0}^{a} B(r) \cdot 2 y d r \\
&=2 \int_{0}^{a} \frac{\mu_{0} I}{2 \pi}\left(\frac{1}{r}+\frac{1}{2 a-r}\right) \cdot 2 \sqrt{a^{2}-(a-r)^{2}} d r .
\end{aligned}
$$

$$
\begin{aligned}
\phi &=2 \int_{0}^{a} \frac{\mu_{0} I}{\pi}\left(\frac{1}{a-x}+\frac{1}{a+x}\right) \sqrt{a^{2}-x^{2}} d x \\
&=\left.\frac{4 \mu_{0} I a}{\pi} \arcsin \frac{x}{a}\right|_{0} ^{a}=2 \mu_{0} I a .
\end{aligned}
$$



Hence the coefficient of mutual inductance is

$$
M=\frac{\phi}{I}=2 \mu_{0} a .
$$

\textbf{Topic} :Magnetostatic Field and Quasi-Stationary Electromagnetic Fields\\
\textbf{Book} :Problems and Solutions on Electromagnetism\\
\textbf{Final Answer} :2 \mu_{0} a\\


\textbf{Solution} :The magnetic field at a point of radial distance $r$ is

$$
B=\frac{\mu_{0} I}{2 \pi r},
$$

and its direction is perpendicular to and pointing into the paper. The induced emf in the rectangular loop (i.e. reading of the voltmeter) is

$$
V=\oint(\mathbf{u} \times \mathbf{B}) \cdot d l=\frac{\mu_{0} I u l}{2 \pi}\left(\frac{1}{r_{1}}-\frac{1}{r_{2}}\right)
$$

if we integrate in the clockwise sense. Note that $\mathbf{u} \times \mathbf{B}$ is in the $+z$ direction. As $V>0$, terminal $a$ is positive. 
\textbf{Topic} :Magnetostatic Field and Quasi-Stationary Electromagnetic Fields\\
\textbf{Book} :Problems and Solutions on Electromagnetism\\
\textbf{Final Answer} :\frac{\mu_{0} I u l}{2 \pi}\left(\frac{1}{r_{1}}-\frac{1}{r_{2}}\right)\\


\textbf{Solution} :As $b \ll a$, the magnetic field at the small loop created by the large loop can be considered approximately as the magnetic field on the axis of the large loop, namely

$$
B=\frac{\mu_{0} a^{2} I}{2\left(a^{2}+c^{2}\right)^{3 / 2}},
$$

where $I$ is the current in the large loop. Hence the magnetic flux crossing the small loop is

$$
\psi_{12}=\frac{\mu_{0} a^{2} I}{2\left(a^{2}+c^{2}\right)^{3 / 2}} \cdot \pi b^{2}
$$

and the mutual inductance is

$$
M_{12}=\frac{\psi_{12}}{I}=\frac{\pi \mu_{0} a^{2} b^{2}}{2\left(a^{2}+c^{2}\right)^{3 / 2}}
$$

MATHPIX IMAGE

Fig. $2.33$

\textbf{Topic} :Magnetostatic Field and Quasi-Stationary Electromagnetic Fields\\
\textbf{Book} :Problems and Solutions on Electromagnetism\\
\textbf{Final Answer} :\frac{\pi \mu_{0} a^{2} b^{2}}{2\left(a^{2}+c^{2}\right)^{3 / 2}}\\


\textbf{Solution} :Suppose the coil rotates from a position parallel to the uniform magnetic field to one perpendicular in time $\Delta t$. Since $\Delta t$ is very short, we have

$$
\varepsilon=\frac{\Delta \phi}{\Delta t}=i(R+r)
$$

As $q=i \Delta t$, the increase of the magnetic flux is

$$
\Delta \phi=q(R+r)=B A N,
$$

since the coil is now perpendicular to the field. Hence the magnetic flux density is

$$
\begin{aligned}
B &=\frac{(R+r) q}{A N}=\frac{(50+30) \times\left(4 \times 10^{-5}\right)}{4 \times 10^{-4} \times 160} \\
&=0.05 \mathrm{~T}=50 \mathrm{Gs} .
\end{aligned}
$$

\textbf{Topic} :Magnetostatic Field and Quasi-Stationary Electromagnetic Fields\\
\textbf{Book} :Problems and Solutions on Electromagnetism\\
\textbf{Final Answer} :50 \mathrm{Gs}\\


\textbf{Solution} :Applying Ampère's circuital law $\oint \mathbf{H} \cdot d \mathbf{l}=i$ to a rectangle with the long sides parallel to the $z$-axis, one inside and one outside the solenoid, we obtain $H=n i$, or

$$
\mathrm{B}=\mu_{0} n K t \mathrm{e}_{z} .
$$
\textbf{Topic} :Magnetostatic Field and Quasi-Stationary Electromagnetic Fields\\
\textbf{Book} :Problems and Solutions on Electromagnetism\\
\textbf{Final Answer} :\mu_{0} n K t \mathrm{e}_{z}\\


\textbf{Solution} :Applying Ampère's circuital law $\oint \mathbf{H} \cdot d \mathbf{l}=i$ to a rectangle with the long sides parallel to the $z$-axis, one inside and one outside the solenoid, we obtain $H=n i$, or

$$
\mathrm{B}=\mu_{0} n K t \mathrm{e}_{z} .
$$

 Maxwell's equation $\nabla \times \mathbf{E}=-\dot{\mathbf{B}}$ gives

$$
\frac{1}{r}\left[\frac{\partial}{\partial r}\left(r E_{\theta}\right)-\frac{\partial E_{r}}{\partial \theta_{r}}\right]=-\mu_{0} n K .
$$

Noting that by symmetry $E$ does not depend on $\theta$ and integrating, we have

$$
\mathbf{E}=-\frac{\mu_{0} n K \boldsymbol{r}}{2} \mathbf{e}_{\boldsymbol{\theta}} .
$$


\textbf{Topic} :Magnetostatic Field and Quasi-Stationary Electromagnetic Fields\\
\textbf{Book} :Problems and Solutions on Electromagnetism\\
\textbf{Final Answer} :-\frac{\mu_{0} n K \boldsymbol{r}}{2} \mathbf{e}_{\boldsymbol{\theta}}\\


\textbf{Solution} :Applying Ampère's circuital law $\oint \mathbf{H} \cdot d \mathbf{l}=i$ to a rectangle with the long sides parallel to the $z$-axis, one inside and one outside the solenoid, we obtain $H=n i$, or

$$
\mathrm{B}=\mu_{0} n K t \mathrm{e}_{z} .
$$

 Maxwell's equation $\nabla \times \mathbf{E}=-\dot{\mathbf{B}}$ gives

$$
\frac{1}{r}\left[\frac{\partial}{\partial r}\left(r E_{\theta}\right)-\frac{\partial E_{r}}{\partial \theta_{r}}\right]=-\mu_{0} n K .
$$

Noting that by symmetry $E$ does not depend on $\theta$ and integrating, we have

$$
\mathbf{E}=-\frac{\mu_{0} n K \boldsymbol{r}}{2} \mathbf{e}_{\boldsymbol{\theta}} .
$$



 The Poynting vector is

$$
\mathrm{N}=\mathrm{E} \times \mathrm{H}=-\frac{\mu_{0} n^{2} K^{2}}{2} r t \mathrm{e}_{r}
$$

So energy flows into the cylinder along the radial directions. The energy flowing in per unit time is then

$$
\frac{d W}{d t}=2 \pi r l N=\mu_{0} V n^{2} K^{2} t
$$

where $V$ is the volume of the cylinder. The self-inductance per unit length of the solenoid is

Hence

$$
L=\frac{n B \pi r^{2}}{i}=\mu_{0} n \pi r^{2} .
$$

$$
\frac{d}{d t}\left(\frac{1}{2} l L i^{2}\right)=\mu_{0} V n^{2} K^{2} t=\frac{d W}{d t} .
$$

\textbf{Topic} :Magnetostatic Field and Quasi-Stationary Electromagnetic Fields\\
\textbf{Book} :Problems and Solutions on Electromagnetism\\
\textbf{Final Answer} :\frac{d W}{d t}\\


\textbf{Solution} :The magnetic flux crossing the loop is

$$
\phi=\mathbf{B} \cdot \mathbf{S}=B_{0} a b \sin (\omega t) \cos (\omega t)=\frac{1}{2} B_{0} a b \sin (2 \omega t) .
$$

So the induced emf is

$$
\varepsilon=-\frac{d \phi}{d t}=-B_{0} a b \omega \cos (2 \omega t)
$$

Its alternating frequency is $\frac{2 \omega}{2 \pi}=2 \cdot \frac{\omega}{2 \pi}=2 f$.

\textbf{Topic} :Magnetostatic Field and Quasi-Stationary Electromagnetic Fields\\
\textbf{Book} :Problems and Solutions on Electromagnetism\\
\textbf{Final Answer} :-B_{0} a b \omega \cos (2 \omega t)\\


\textbf{Solution} :As the wire cuts across the lines of induction an emf is induced and produces current. Suppose the length of the wire is $l$ and the terminal velocity is $\mathbf{v}=-v e_{z}$. The current so induced is given by

$$
-\int_{l} \mathrm{~B} \times \mathbf{v} \cdot d \mathbf{l}=i \lambda l .
$$

Thus

$$
i=-\frac{v B l}{\lambda l}=-\frac{v B}{\lambda},
$$

flowing in the $-y$ direction. The magnetic force acting on the wire is

$$
\mathbf{F}=\int i d l \times \mathbf{B}=i B l \mathbf{e}_{z}=\frac{v B^{2} l}{\lambda} \mathbf{e}_{z} .
$$

When the terminal velocity is reached this force is in equilibrium with the gravitation. Hence the terminal velocity of the wire is given by

$$
\frac{v B^{2} l}{\lambda}=\rho l g \text {, }
$$

i.e.,

$$
v=\frac{\rho g \lambda}{B^{2}}
$$

or

$$
\mathbf{v}=-\frac{\rho g \lambda}{B^{2}} \mathbf{e}_{z} .
$$

\textbf{Topic} :Magnetostatic Field and Quasi-Stationary Electromagnetic Fields\\
\textbf{Book} :Problems and Solutions on Electromagnetism\\
\textbf{Final Answer} :-\frac{\rho g \lambda}{B^{2}} \mathbf{e}_{z}\\


\textbf{Solution} :The emf induced in the coil is given by

$$
\varepsilon=-\frac{d}{d t} \int_{S} B \cdot d S .
$$

Noting that, as the vector $d S$ is normal to the plane of the coil, $B \cdot d S=$ $B \cos \left(\frac{\pi}{2}-\theta\right) d S$, we have, with $\theta=\omega t$,

$$
\begin{aligned}
\varepsilon &=-\frac{d}{d t} \int_{S} B \sin (\omega t) d S=-\frac{d}{d t}\left[\pi a^{2} N B \sin (\omega t)\right] \\
&=-\pi a^{2} \omega N B \cos (\omega t) \\
&=-\operatorname{Re}\left[\dot{\pi} a^{2} \omega N B \exp (i \omega t)\right] .
\end{aligned}
$$

The current in the circuit is given by

$$
L \frac{d I}{d t}+I R=\varepsilon .
$$

Let $I=I_{0} \exp (i \omega t)$. The above gives

$$
I_{0}=\frac{-\pi a^{2} \omega N B}{i \omega L+R}=\frac{\pi a^{2} \omega N B}{\sqrt{\omega^{2} L^{2}+R^{2}}} e^{-i\left(\frac{5}{2}+\varphi\right)},
$$

where $\varphi=\arctan \left(\frac{\omega L}{R}\right)$.

MATHPIX IMAGE

Fig. $2.47$

Thus we have

$$
\begin{aligned}
I(t) &=\frac{\pi a^{2} \omega N B}{\sqrt{\omega^{2} L^{2}+R^{2}}} \cos \left(\omega t-\varphi-\frac{\pi}{2}\right) \\
&=\frac{\pi a^{2} \omega N B}{\sqrt{\omega^{2} L^{2}+R^{2}}} \sin (\omega t-\varphi) .
\end{aligned}
$$
\textbf{Topic} :Magnetostatic Field and Quasi-Stationary Electromagnetic Fields\\
\textbf{Book} :Problems and Solutions on Electromagnetism\\
\textbf{Final Answer} :\frac{\pi a^{2} \omega N B}{\sqrt{\omega^{2} L^{2}+R^{2}}} \sin (\omega t-\varphi)\\


\textbf{Solution} :The emf induced in the coil is given by

$$
\varepsilon=-\frac{d}{d t} \int_{S} B \cdot d S .
$$

Noting that, as the vector $d S$ is normal to the plane of the coil, $B \cdot d S=$ $B \cos \left(\frac{\pi}{2}-\theta\right) d S$, we have, with $\theta=\omega t$,

$$
\begin{aligned}
\varepsilon &=-\frac{d}{d t} \int_{S} B \sin (\omega t) d S=-\frac{d}{d t}\left[\pi a^{2} N B \sin (\omega t)\right] \\
&=-\pi a^{2} \omega N B \cos (\omega t) \\
&=-\operatorname{Re}\left[\dot{\pi} a^{2} \omega N B \exp (i \omega t)\right] .
\end{aligned}
$$

The current in the circuit is given by

$$
L \frac{d I}{d t}+I R=\varepsilon .
$$

Let $I=I_{0} \exp (i \omega t)$. The above gives

$$
I_{0}=\frac{-\pi a^{2} \omega N B}{i \omega L+R}=\frac{\pi a^{2} \omega N B}{\sqrt{\omega^{2} L^{2}+R^{2}}} e^{-i\left(\frac{5}{2}+\varphi\right)},
$$

where $\varphi=\arctan \left(\frac{\omega L}{R}\right)$.

MATHPIX IMAGE

Fig. $2.47$

Thus we have

$$
\begin{aligned}
I(t) &=\frac{\pi a^{2} \omega N B}{\sqrt{\omega^{2} L^{2}+R^{2}}} \cos \left(\omega t-\varphi-\frac{\pi}{2}\right) \\
&=\frac{\pi a^{2} \omega N B}{\sqrt{\omega^{2} L^{2}+R^{2}}} \sin (\omega t-\varphi) .
\end{aligned}
$$

 The magnetic dipole moment of the coil is

$$
\mathbf{m}=I \pi a^{2} N_{\mathbf{n}},
$$

where $\mathbf{n}$ is a unit vector normal to the coil. At time $t$ the external torque on the coil $\tau=m \times B$ has magnitude

$$
\tau=|\mathbf{m} \times \mathbf{B}|=I \pi a^{2} N B \sin \left(\frac{\pi}{2}-\theta\right)=\frac{\left(\pi a^{2} N B\right)^{2} \omega}{\sqrt{R^{2}+L^{2} \omega^{2}}} \cos (\omega t) \sin (\omega t-\varphi) .
$$

\textbf{Topic} :Magnetostatic Field and Quasi-Stationary Electromagnetic Fields\\
\textbf{Book} :Problems and Solutions on Electromagnetism\\
\textbf{Final Answer} :\frac{\left(\pi a^{2} N B\right)^{2} \omega}{\sqrt{R^{2}+L^{2} \omega^{2}}} \cos (\omega t) \sin (\omega t-\varphi)\\


\textbf{Solution} :Inside the slab only the $y$-component is present. It is

$$
\begin{aligned}
H_{y}(z) &=A e^{-k z}+B e^{k z} \\
&=\frac{H_{2} \sinh [k(z+d)]-H_{1} \sinh [k(z-d)]}{\sinh (2 k d)} .
\end{aligned}
$$
\textbf{Topic} :Magnetostatic Field and Quasi-Stationary Electromagnetic Fields\\
\textbf{Book} :Problems and Solutions on Electromagnetism\\
\textbf{Final Answer} :\frac{H_{2} \sinh [k(z+d)]-H_{1} \sinh [k(z-d)]}{\sinh (2 k d)}\\


\textbf{Solution} :Inside the slab only the $y$-component is present. It is

$$
\begin{aligned}
H_{y}(z) &=A e^{-k z}+B e^{k z} \\
&=\frac{H_{2} \sinh [k(z+d)]-H_{1} \sinh [k(z-d)]}{\sinh (2 k d)} .
\end{aligned}
$$

 From Maxwell's equation $\mathrm{j}=\frac{c}{4 \pi} \nabla \times \mathrm{H}$ and $\mathrm{H}=H_{y}(z) \mathrm{e}_{y}$ we have $\mathbf{j}=j_{x} \mathbf{e}_{x}$ with

$$
j_{x}=-\frac{c}{4 \pi} \frac{\partial H_{y}}{\partial z}=-\frac{c}{4 \pi} \cdot \frac{k\left\{H_{2} \cosh [k(z+d)]-H_{1} \cosh [k(z-d)]\right\}}{\sinh (2 k d)} .
$$
\textbf{Topic} :Magnetostatic Field and Quasi-Stationary Electromagnetic Fields\\
\textbf{Book} :Problems and Solutions on Electromagnetism\\
\textbf{Final Answer} :-\frac{c}{4 \pi} \cdot \frac{k\left\{H_{2} \cosh [k(z+d)]-H_{1} \cosh [k(z-d)]\right\}}{\sinh (2 k d)}\\


\textbf{Solution} :Inside the slab only the $y$-component is present. It is

$$
\begin{aligned}
H_{y}(z) &=A e^{-k z}+B e^{k z} \\
&=\frac{H_{2} \sinh [k(z+d)]-H_{1} \sinh [k(z-d)]}{\sinh (2 k d)} .
\end{aligned}
$$

 From Maxwell's equation $\mathrm{j}=\frac{c}{4 \pi} \nabla \times \mathrm{H}$ and $\mathrm{H}=H_{y}(z) \mathrm{e}_{y}$ we have $\mathbf{j}=j_{x} \mathbf{e}_{x}$ with

$$
j_{x}=-\frac{c}{4 \pi} \frac{\partial H_{y}}{\partial z}=-\frac{c}{4 \pi} \cdot \frac{k\left\{H_{2} \cosh [k(z+d)]-H_{1} \cosh [k(z-d)]\right\}}{\sinh (2 k d)} .
$$

 The force on the slab is

$$
\begin{aligned}
\mathbf{F} &=\frac{1}{c} \int \mathbf{j} \times \mathbf{H} d V \\
&=\frac{\mathrm{e}_{z}}{c} \int\left(-\frac{c}{4 \pi}\right) \frac{\partial H_{\mathrm{y}}}{\partial z} H_{y} d z d S .
\end{aligned}
$$

Hence the force per unit area on the surface is

$$
\begin{aligned}
\mathbf{f} &=-\frac{\mathbf{e}_{z}}{4 \pi} \int H_{y} \frac{\partial H_{y}}{\partial z} d z=-\frac{\mathbf{e}_{z}}{4 \pi} \int H_{y} d H_{y} \\
&=-\frac{1}{8 \pi}\left(H_{2}^{2}-H_{1}^{2}\right) \mathbf{e}_{z} .
\end{aligned}
$$

\textbf{Topic} :Magnetostatic Field and Quasi-Stationary Electromagnetic Fields\\
\textbf{Book} :Problems and Solutions on Electromagnetism\\
\textbf{Final Answer} :-\frac{1}{8 \pi}\left(H_{2}^{2}-H_{1}^{2}\right) \mathbf{e}_{z}\\


\textbf{Solution} :The induced emf of the loop is given by

$$
\varepsilon=-L \frac{d I}{d t}=-\frac{\partial}{\partial t} \int \mathbf{B} \cdot d \mathbf{S} .
$$

Integrating over time we have

$$
L[I(f)-I(i)]=\int[\mathbf{B}(f)-\mathbf{B}(i)] \cdot d \mathbf{S} .
$$

Initially, when the dipole is far away,

$$
I(i)=0, \quad \mathbf{B}(i)=0 .
$$

Writing for the final position $I=I(f), \mathbf{B}=\mathbf{B}(f)$, we have

$$
L I=\int \mathbf{B} \cdot d \mathbf{S} .
$$

Consider a point $P$ in the plane of the loop. Use cylindrical coordinates $(\rho, \theta, z)$ such that $P$ has radius vector $\rho e_{\rho}$. Then the radius vector from $\mathbf{m}$ to $P$ is $\mathbf{r}=\rho \mathbf{e}_{\rho}-z \mathbf{e}_{z}$. The magnetic induction at $P$ due to $\mathbf{m}$ is

$$
\mathbf{B}=\frac{\mu_{0}}{4 \pi}\left[\frac{3(\mathbf{m} \cdot \mathbf{r}) \mathbf{r}}{r^{5}}-\frac{\mathbf{m}}{r^{3}}\right]
$$

where $m=m \mathbf{e}_{z}$. As $d S=\rho d \rho d \theta \mathbf{e}_{z}$ we have

$$
\begin{aligned}
\int \mathrm{B} \cdot d \mathrm{~S} &=\frac{\mu_{0}}{4 \pi} \iint\left(\frac{3(\mathbf{m} \cdot \mathbf{r})\left(\mathbf{r} \cdot \mathbf{e}_{z}\right)}{r^{5}}-\frac{\mathbf{m} \cdot \mathbf{e}_{z}}{r^{3}}\right) \rho d \rho d \theta \\
&=\frac{\mu_{0}}{4 \pi} \cdot 2 \pi \int_{0}^{b}\left[\frac{3 m z^{2}}{\left(\rho^{2}+z^{2}\right)^{5 / 2}}-\frac{m}{\left(\rho^{2}+z^{2}\right)^{3 / 2}}\right] \rho d \rho \\
&=\frac{\mu_{0} m}{2}\left[\left(b^{2}+z^{2}\right)^{-\frac{1}{2}}-z^{2}\left(b^{2}+z^{2}\right)^{-\frac{3}{2}}\right]
\end{aligned}
$$

and the induced current in the loop is

$$
I=\frac{\mu_{0} m}{2 L}\left[\left(b^{2}+z^{2}\right)^{-\frac{1}{2}}-z^{2}\left(b^{2}+z^{2}\right)^{-\frac{3}{2}}\right] .
$$

By Lenz's law the direction of flow is clockwise when looking from the location of $m$ positioned as shown in Fig. 2.57.
\textbf{Topic} :Magnetostatic Field and Quasi-Stationary Electromagnetic Fields\\
\textbf{Book} :Problems and Solutions on Electromagnetism\\
\textbf{Final Answer} :\frac{\mu_{0} m}{2 L}\left[\left(b^{2}+z^{2}\right)^{-\frac{1}{2}}-z^{2}\left(b^{2}+z^{2}\right)^{-\frac{3}{2}}\right]\\


\textbf{Solution} :The induced emf of the loop is given by

$$
\varepsilon=-L \frac{d I}{d t}=-\frac{\partial}{\partial t} \int \mathbf{B} \cdot d \mathbf{S} .
$$

Integrating over time we have

$$
L[I(f)-I(i)]=\int[\mathbf{B}(f)-\mathbf{B}(i)] \cdot d \mathbf{S} .
$$

Initially, when the dipole is far away,

$$
I(i)=0, \quad \mathbf{B}(i)=0 .
$$

Writing for the final position $I=I(f), \mathbf{B}=\mathbf{B}(f)$, we have

$$
L I=\int \mathbf{B} \cdot d \mathbf{S} .
$$

Consider a point $P$ in the plane of the loop. Use cylindrical coordinates $(\rho, \theta, z)$ such that $P$ has radius vector $\rho e_{\rho}$. Then the radius vector from $\mathbf{m}$ to $P$ is $\mathbf{r}=\rho \mathbf{e}_{\rho}-z \mathbf{e}_{z}$. The magnetic induction at $P$ due to $\mathbf{m}$ is

$$
\mathbf{B}=\frac{\mu_{0}}{4 \pi}\left[\frac{3(\mathbf{m} \cdot \mathbf{r}) \mathbf{r}}{r^{5}}-\frac{\mathbf{m}}{r^{3}}\right]
$$

where $m=m \mathbf{e}_{z}$. As $d S=\rho d \rho d \theta \mathbf{e}_{z}$ we have

$$
\begin{aligned}
\int \mathrm{B} \cdot d \mathrm{~S} &=\frac{\mu_{0}}{4 \pi} \iint\left(\frac{3(\mathbf{m} \cdot \mathbf{r})\left(\mathbf{r} \cdot \mathbf{e}_{z}\right)}{r^{5}}-\frac{\mathbf{m} \cdot \mathbf{e}_{z}}{r^{3}}\right) \rho d \rho d \theta \\
&=\frac{\mu_{0}}{4 \pi} \cdot 2 \pi \int_{0}^{b}\left[\frac{3 m z^{2}}{\left(\rho^{2}+z^{2}\right)^{5 / 2}}-\frac{m}{\left(\rho^{2}+z^{2}\right)^{3 / 2}}\right] \rho d \rho \\
&=\frac{\mu_{0} m}{2}\left[\left(b^{2}+z^{2}\right)^{-\frac{1}{2}}-z^{2}\left(b^{2}+z^{2}\right)^{-\frac{3}{2}}\right]
\end{aligned}
$$

and the induced current in the loop is

$$
I=\frac{\mu_{0} m}{2 L}\left[\left(b^{2}+z^{2}\right)^{-\frac{1}{2}}-z^{2}\left(b^{2}+z^{2}\right)^{-\frac{3}{2}}\right] .
$$

By Lenz's law the direction of flow is clockwise when looking from the location of $m$ positioned as shown in Fig. 2.57.

 For the loop, with the current $I$ as above, the magnetic field at a point on its axis is

$$
B^{\prime}=-\frac{\mu_{0} I}{2} \frac{b^{2}}{\left(b^{2}+z^{2}\right)^{3 / 2}} \mathbf{e}_{z}=-\frac{\mu_{0}^{2} m}{4 L} \frac{b^{4}}{\left(b^{2}+z^{2}\right)^{3}} \mathbf{e}_{z} .
$$

The energy of the magnetic dipole $m$ in the field $B^{\prime}$ is

$$
W=\mathbf{m} \cdot \mathbf{B}^{\prime}
$$

and the force between the dipole and the loop is

$$
F=-\frac{\partial W}{\partial z}=-\frac{3 \mu_{0}^{2} m^{2} b^{4} z}{2 L\left(b^{2}+z^{2}\right)^{4}} .
$$

\textbf{Topic} :Magnetostatic Field and Quasi-Stationary Electromagnetic Fields\\
\textbf{Book} :Problems and Solutions on Electromagnetism\\
\textbf{Final Answer} :-\frac{3 \mu_{0}^{2} m^{2} b^{4} z}{2 L\left(b^{2}+z^{2}\right)^{4}}\\


\textbf{Solution} :We expand the expression for the force on a current loop:

$$
\mathbf{\Sigma}=(\boldsymbol{\mu} \times \nabla) \times \mathbf{B}=\nabla(\boldsymbol{\mu} \cdot \mathbf{B})-\mu(\nabla \cdot \mathbf{B}) .
$$

The external magnetic field $\mathbf{B}(\mathbf{r})$ satisfies $\nabla \cdot \mathbf{B}=0$ so the above equation can be written as

$$
\mathbf{F}=\nabla(\boldsymbol{\mu} \cdot \mathbf{B})=(\boldsymbol{\mu} ; \nabla) \mathbf{B}+\boldsymbol{\mu} \times(\nabla \times \mathbf{B}) .
$$

Compared with the expression for the force on a magnetic dipole, it has an additional term $\mu \times(\nabla \times B)$. Thus the two forces are different unless $\nabla \times \mathbf{B}=0$ in the loop case which would mean $J=\dot{D}=0$ in the region of the loop.

 Take the $z$-axis along the direction of the magnetic moment of the nucleus and apply a magnetic field $B=B(z) e_{z}$ in this direction. According to $F=(\mu \times \nabla) \times B$, the magnetic force is zero. But according to $F=$ $(\mu \cdot \nabla) \mathbf{B}$, the force is not zero. So whether the magnetic moment arises from magnetic charge or from electric current depends on whether or not the nucleus suffers a magnetic force.

\textbf{Topic} :Magnetostatic Field and Quasi-Stationary Electromagnetic Fields\\
\textbf{Book} :Problems and Solutions on Electromagnetism\\
\textbf{Final Answer} :(\boldsymbol{\mu} ; \nabla) \mathbf{B}+\boldsymbol{\mu} \times(\nabla \times \mathbf{B})\\


\textbf{Solution} :As

$$
\mathbf{F}=m \dot{\mathbf{v}}=q\left(\mathbf{E}+\frac{\mathbf{v} \times \mathbf{B}}{c}\right)
$$

we have

$$
(m \dot{\mathbf{v}}-q \vec{E})=q \frac{\mathbf{v} \times \mathbf{B}}{c} .
$$

It follows that

$$
\mathbf{v} \cdot(m \dot{v}-q \mathbf{E})=\mathbf{v} \cdot(\mathbf{v} \times \mathbf{B}) \frac{q}{c}=0
$$

Consider

$$
\begin{aligned}
\frac{d}{d t}\left[\frac{1}{2} m v^{2}+q \phi\right] &=m \mathbf{v} \cdot \dot{\mathbf{v}}+q \frac{d \phi}{d t}=m \mathbf{v} \cdot \dot{\mathbf{v}}+q \mathbf{v} \cdot \nabla \phi \\
&=\mathbf{v} \cdot(m \mathbf{v}+q \nabla \phi)=\mathbf{v} \cdot(m \dot{\mathbf{v}}-q \mathbf{E})=0
\end{aligned}
$$

where we have made use of

$$
\frac{d \phi}{d t}=\frac{\partial \phi}{\partial x} \frac{d x}{d t}+\frac{\partial \phi}{\partial y} \frac{d y}{d t}+\frac{\partial \phi}{\partial z} \frac{d z}{d t}=\mathbf{v} \cdot \nabla \phi .
$$

Hence

$$
\frac{1}{2} m v^{2}+q \phi=\text { Const. }
$$

 The magnetic force $F_{m}=q \frac{\sim \times B}{c}$ is perpendicular to $v$ so that if the particle moves in the $x$ direction the magnetic force will not affect the $x$-component of the motion. With $\mathbf{E}$ in the $x$ direction the particle's motion will be confined in that direction. Newton's second law gives

$$
m \ddot{x}=q E=q A e^{-t / \tau},
$$

i.e.,

$$
m d v=q A e^{-t / \tau} d t
$$

with

$$
v(0)=0, \quad m v=-q A \tau e^{-t / \tau}+q A \tau,
$$

or

$$
d x=q A \tau\left(1-e^{-t / \tau}\right) \frac{d t}{m} .
$$

With $x(0)=0$, this gives

$$
\begin{aligned}
x(t) &=q A \tau \frac{t}{m}+\frac{q A \tau^{2}}{m} e^{-t / \tau}-\frac{q A \tau^{2}}{m} \\
&=\frac{q A \tau}{m}\left[(t-\tau)+\tau e^{-t / \tau}\right]
\end{aligned}
$$
\textbf{Topic} :Magnetostatic Field and Quasi-Stationary Electromagnetic Fields\\
\textbf{Book} :Problems and Solutions on Electromagnetism\\
\textbf{Final Answer} :\frac{q A \tau}{m}\left[(t-\tau)+\tau e^{-t / \tau}\right]\\


\textbf{Solution} :The magnetic field produced at a point of radius vector $r$ from the center of the circle by the dipole of moment $\mathbf{M}_{0}$ is

$$
\mathrm{B}=\frac{\mu_{0}}{4 \pi}\left[\frac{3\left(\mathbf{M}_{0} \cdot \mathbf{r}\right) \mathbf{r}}{r^{5}}-\frac{\mathbf{M}_{0}}{r^{3}}\right] .
$$

This exerts a force on the particle moving in the circular orbit of

$$
\mathbf{r}=\left.\nabla(\mathbf{M} \cdot \mathbf{B})\right|_{r=\boldsymbol{R}} .
$$

Noting $M \cdot M_{0}=-M M_{0}, M_{0} \cdot \mathbf{r}=\mathbf{M} \cdot \mathbf{r}=0$, we have

$$
\Gamma=-\frac{3 \mu_{0} M M_{0}}{4 \pi R^{4}} e_{r} .
$$

This force acts towards the center and gives rise to the circular motion of the particle. Balancing the force with the centrifugal force,

$$
m \frac{v^{2}}{R}=\frac{3 \mu_{0} M M_{0}}{4 \pi R^{4}},
$$

gives the particle velocity as

$$
v=\sqrt{\frac{3 \mu_{0} M M_{0}}{4 \pi m R^{3}}} .
$$
\textbf{Topic} :Magnetostatic Field and Quasi-Stationary Electromagnetic Fields\\
\textbf{Book} :Problems and Solutions on Electromagnetism\\
\textbf{Final Answer} :\sqrt{\frac{3 \mu_{0} M M_{0}}{4 \pi m R^{3}}}\\


\textbf{Solution} :As $\mathbf{v}$ is very large, we can consider any deviation from a straight line path to be quite small in the emitting process. Let $v_{\perp}$ be the transverse velocity of the particle. We have

$$
m \frac{d \mathbf{v}_{\perp}}{d t}=q \mathbf{v} \times \mathbf{B} .
$$

As $\mathbf{v} \perp \mathrm{B}, d v_{\perp}=\frac{q}{m} B_{0} v d t=\frac{q}{m} B_{0} d r$ and

$$
v_{\perp}(b)=\int_{a}^{b} \frac{q}{m} B_{0} d r=\frac{q}{m} B_{0}(b-a) .
$$

At $r=b$ the angular momentum of the particle about the axis of the solenoid has magnitude

$$
\left|\mathbf{r} \times m \mathbf{v}_{\perp}(b)\right|_{r=b}=m b v_{\perp}(b)=q B_{0} b(b-a) .
$$

and direction $-\mathbf{e}_{z}$. Thus the angular momentum is

$$
\mathbf{J}_{p}=-q B_{0} b(b-a) \mathbf{e}_{z} .
$$

For $r>b, \mathbf{B}=0$ and $J_{p}$ is considered. So $J_{p}$ is the angular momentum of the particle about the axis for $r>b$.
\textbf{Topic} :Magnetostatic Field and Quasi-Stationary Electromagnetic Fields\\
\textbf{Book} :Problems and Solutions on Electromagnetism\\
\textbf{Final Answer} :-q B_{0} b(b-a) \mathbf{e}_{z}\\


\textbf{Solution} :As $\mathbf{v}$ is very large, we can consider any deviation from a straight line path to be quite small in the emitting process. Let $v_{\perp}$ be the transverse velocity of the particle. We have

$$
m \frac{d \mathbf{v}_{\perp}}{d t}=q \mathbf{v} \times \mathbf{B} .
$$

As $\mathbf{v} \perp \mathrm{B}, d v_{\perp}=\frac{q}{m} B_{0} v d t=\frac{q}{m} B_{0} d r$ and

$$
v_{\perp}(b)=\int_{a}^{b} \frac{q}{m} B_{0} d r=\frac{q}{m} B_{0}(b-a) .
$$

At $r=b$ the angular momentum of the particle about the axis of the solenoid has magnitude

$$
\left|\mathbf{r} \times m \mathbf{v}_{\perp}(b)\right|_{r=b}=m b v_{\perp}(b)=q B_{0} b(b-a) .
$$

and direction $-\mathbf{e}_{z}$. Thus the angular momentum is

$$
\mathbf{J}_{p}=-q B_{0} b(b-a) \mathbf{e}_{z} .
$$

For $r>b, \mathbf{B}=0$ and $J_{p}$ is considered. So $J_{p}$ is the angular momentum of the particle about the axis for $r>b$.

 After the particle has gone far away from the solenoid, the momentum density of the electromagnetic field at a point within the solenoid is

$$
g=\frac{\mathbf{E} \times \mathbf{H}}{c^{2}}=\varepsilon_{0} \mathbf{E} \times \mathbf{B}
$$

and the angular momentum density is

$$
\mathbf{j}=\mathbf{r} \times \mathbf{g}=\varepsilon_{0} \mathbf{r} \times(\mathbf{E} \times \mathbf{B})=\frac{B_{0} q}{2 \pi l} \mathbf{e}_{z},
$$

which is uniform. As there is no field outside the solenoid and inside the central rod the total angular momentum of the electromagnetic field is

$$
J_{\mathrm{EM}}=\pi\left(b^{2}-a^{2}\right) l j=\frac{q B_{0}\left(b^{2}-a^{2}\right)}{2} \mathbf{e}_{z} .
$$

Initially, $\mathbf{E}=0, \mathbf{v}_{\perp}=0$ and the solenoid is at rest, so the total angular momentum of the system is zero. The final angular momentum of the solenoid can be obtained from the conservation of the total angular momentum:

$$
J_{S}=-J_{E M}-J_{p}=\frac{q B_{0}}{2}(b-a)^{2} \mathbf{e}_{z} .
$$

This signifies that the solenoid in the final state rotates with a constant angular speed about its central axis, the sense of rotation being related to the direction $e_{z}$ by the right-hand screw rule.

\textbf{Topic} :Magnetostatic Field and Quasi-Stationary Electromagnetic Fields\\
\textbf{Book} :Problems and Solutions on Electromagnetism\\
\textbf{Final Answer} :\frac{q B_{0}}{2}(b-a)^{2} \mathbf{e}_{z}\\


\textbf{Solution} :Since there is no current between the magnet poles, $\nabla \times \mathbf{B}=0$, or

$$
\frac{\partial B_{x}}{\partial z}-\frac{\partial B_{z}}{\partial x}=0 .
$$

This implies that as $B_{x}$ depends on $z, B_{z} \neq 0$, i.e. there is a $z$-component outside the median plane.

The kinetic energy of a charged particle moving in a magnetic field is conserved. Hence the magnitude of its velocity is a constant. Let $\theta$ be the bending angle, then

$$
v_{y}=v \sin \theta, \quad v_{z}=v \cos \theta .
$$

The equation of the motion of the particle in the $y$ direction, since $v_{x}=0$, is

$$
m \frac{d v_{y}}{d t}=e B_{x} v_{z}
$$

or

$$
m v \frac{d}{d t}(\sin \theta)=e B_{x} v \cos \theta
$$

This gives

$$
d \theta=\frac{e B_{x}}{m} d t=\frac{e B_{x}}{m} \cdot \frac{d z}{v \cos \theta},
$$

or

$$
\cos \theta d \theta=\frac{e}{P} B_{x} d z .
$$

Suppose at $t=0$ the particle is at the origin and its velocity is along the $+z$ direction. Then $\left.\theta(z)\right|_{z=0}=0$ and we have

$$
\int_{0}^{\theta} \cos \theta d \theta=\frac{e}{P} \int_{0}^{z} B_{x} d z^{\prime},
$$

or

$$
\theta=\sin ^{-1}\left[\frac{e}{P} \int_{0}^{z} B_{x} d z^{\prime}\right] \text {. }
$$

The displacement $y$ is given by

$$
\begin{aligned}
y &=\int_{0}^{t} v_{y} d t^{\prime}=\int_{0}^{z} v \sin \theta \frac{d z^{\prime}}{v \cos \theta} \\
&=\int_{0}^{z} \tan \theta d z^{\prime}=\int_{0}^{z} \frac{\frac{e}{P} \int_{0}^{z^{\prime}} B_{x}\left(z^{\prime \prime}\right) d z^{\prime \prime}}{\left[1-\left(\frac{e}{P} \int_{0}^{z^{\prime}} B_{x}\left(z^{\prime \prime}\right) d z^{\prime \prime}\right)^{2}\right]^{1 / 2}} d z^{\prime} .
\end{aligned}
$$


\textbf{Topic} :Magnetostatic Field and Quasi-Stationary Electromagnetic Fields\\
\textbf{Book} :Problems and Solutions on Electromagnetism\\
\textbf{Final Answer} :\int_{0}^{z} \frac{\frac{e}{P} \int_{0}^{z^{\prime}} B_{x}\left(z^{\prime \prime}\right) d z^{\prime \prime}}{\left[1-\left(\frac{e}{P} \int_{0}^{z^{\prime}} B_{x}\left(z^{\prime \prime}\right) d z^{\prime \prime}\right)^{2}\right]^{1 / 2}} d z^{\prime}\\


\textbf{Solution} :With $\mu=\mu_{H} e_{x}$, the energy of such a hydrogen atom in the field

$$
W=\mu \cdot \mathbf{B}=-\frac{\mu_{0} \mu_{\mathrm{H}} i}{2 \pi y} .
$$

Thus the magnetic force on the atom is

$$
\mathbf{F}=\left.\nabla W\right|_{y=b}=\frac{\mu_{0} \mu_{H} i}{2 \pi b^{2}} \mathbf{e}_{y},
$$

and the torque on the atom is

$$
\mathbf{L}=\boldsymbol{\mu} \times \mathbf{B}=\mathbf{0} .
$$

 With $\mu=\mu_{\mathrm{H}} \mathbf{e}_{z}$, the energy is $W=\mu \cdot \mathbf{B}=0$. So the force exerted is $F=0$ and the torque is

$$
\mathrm{L}=\mu \times\left.\mathrm{B}\right|_{y=b}=-\frac{\mu_{0} \mu_{\mathrm{H}} i}{2 \pi b} \mathbf{e}_{y} .
$$

 In case (b), because the atom is exerted by a torque, Larmor precession will take place. The angular momentum of the atom, $M$, and its magnetic moment are related by

$$
\mu_{H}=g \frac{e}{2 m} M,
$$

where $g$ is the Landé factor. The rate of change of the angular momentum is equal to the torque acting on the atom,

$$
\frac{d M}{d t}=\mathrm{L}
$$

The magnitude of $M$ does not change, but $\mathrm{L}$ will give rise to a precession of $M$ about $B$, called the Larmor precession, of frequency $\omega$ given by

$$
\left|\frac{d \mathbf{M}}{d t}\right|=M \omega,
$$

or

$$
\omega=\frac{L}{M}=\frac{\mu_{0} \mu_{\mathrm{H}} i}{2 \pi b M}=\frac{e g \mu_{0} i}{4 \pi b m} .
$$

The precession is anti-clock wise if viewed from the side of positive $x$.

\textbf{Topic} :Magnetostatic Field and Quasi-Stationary Electromagnetic Fields\\
\textbf{Book} :Problems and Solutions on Electromagnetism\\
\textbf{Final Answer} :\frac{e g \mu_{0} i}{4 \pi b m}\\


\textbf{Solution} :As $\mathbf{B}=B_{0} \mathbf{e}_{x}$ and is uniform between the pole faces, the equation of the transverse motion of the particle is

$$
m \frac{d v_{y}}{d t}=-q v_{z} B_{0} \text {. }
$$

Since the speed $v$ does not change in a magnetic field, we have $v_{y}=$ $-v \sin \theta_{1}, v_{z}=v \cos \theta_{1}$, where $\theta_{1}$ is the deflecting angle in $y z$ plane. As $v_{y} \approx-v \theta_{1}, v_{z}=v$, and $P=m v=$ constant, the above becomes

$$
\frac{d \theta_{1}}{d t}=\frac{q v B_{0}}{P},
$$

or

$$
d \theta_{1}=\frac{q v B_{0}}{P} d t=\frac{q v B_{0}}{P v_{z}} d z,
$$

i.e.,

$$
\cos \theta_{1} d \theta_{1}=\frac{q B_{0}}{P} d z .
$$

Integrating

$$
\int_{0}^{\theta y} \cos \theta_{1} d \theta_{1}=\int_{0}^{L} \frac{q B_{0}}{P} d z
$$

we find

$$
\sin \theta_{y}=\frac{q B_{0} L}{P} .
$$

As $P \gg q B_{0} L$, we have approximately

$$
\theta_{y} \approx \frac{q B_{0} L}{P} .
$$
\textbf{Topic} :Magnetostatic Field and Quasi-Stationary Electromagnetic Fields\\
\textbf{Book} :Problems and Solutions on Electromagnetism\\
\textbf{Final Answer} :\frac{e g \mu_{0} i}{4 \pi b m}\\


\textbf{Solution} :In the Cartesian coordinates $(x, y, z)$ attached to a particle, the magnetic field can be expressed as

$$
\mathrm{B}=B \sin \theta \cos \varphi \mathbf{e}_{x}+B \sin \theta \sin \varphi \mathbf{e}_{y}+B \cos \theta \mathbf{e}_{z}
$$

As $\left|\chi_{x}\right|,\left|\chi_{y}\right|$ and $\left|\chi_{z}\right|$ are generally much smaller than 1, the magnetic field inside the particle (a small cylinder) may be taken to be $B$ also. The magnetization is given by

$$
\mathbf{M}=\chi \cdot \mathbf{H}=\chi \cdot \frac{\mathbf{B}}{\mu_{0}} .
$$

Let the volume of the small cylinder be $V$, then the energy of orientation is

$$
\begin{aligned}
E &=\mathbf{m} \cdot \mathbf{B}=\mathbf{B} \cdot\left(\chi \cdot \frac{\mathbf{B}}{\mu_{0}} V\right) \\
&=\frac{V}{\mu_{0}}\left(B_{x}, B_{y}, B_{z}\right)\left(\begin{array}{ccc}
\chi_{x} & \\
& \chi_{y} & \\
& \chi_{z}
\end{array}\right)\left(\begin{array}{l}
B_{x} \\
B_{y} \\
B_{z}
\end{array}\right) \\
&=\frac{V}{\mu_{0}}\left(\chi_{x} B_{x}^{2}+\chi_{y} B_{y}^{2}+\chi_{z} B_{z}^{2}\right) \\
&=\frac{V}{\mu_{0}}\left[\chi_{x}\left(B_{x}^{2}+B_{y}^{2}\right)+\chi_{z} B_{z}^{2}\right] \\
&=\frac{V}{\mu_{0}}\left[\chi_{x} B^{2} \sin ^{2} \theta+x_{z} B^{2} \cos ^{2} \theta\right]
\end{aligned}
$$
\textbf{Topic} :Magnetostatic Field and Quasi-Stationary Electromagnetic Fields\\
\textbf{Book} :Problems and Solutions on Electromagnetism\\
\textbf{Final Answer} :\frac{V}{\mu_{0}}\left[\chi_{x} B^{2} \sin ^{2} \theta+x_{z} B^{2} \cos ^{2} \theta\right]\\


\textbf{Solution} :In the Cartesian coordinates $(x, y, z)$ attached to a particle, the magnetic field can be expressed as

$$
\mathrm{B}=B \sin \theta \cos \varphi \mathbf{e}_{x}+B \sin \theta \sin \varphi \mathbf{e}_{y}+B \cos \theta \mathbf{e}_{z}
$$

As $\left|\chi_{x}\right|,\left|\chi_{y}\right|$ and $\left|\chi_{z}\right|$ are generally much smaller than 1, the magnetic field inside the particle (a small cylinder) may be taken to be $B$ also. The magnetization is given by

$$
\mathbf{M}=\chi \cdot \mathbf{H}=\chi \cdot \frac{\mathbf{B}}{\mu_{0}} .
$$

Let the volume of the small cylinder be $V$, then the energy of orientation is

$$
\begin{aligned}
E &=\mathbf{m} \cdot \mathbf{B}=\mathbf{B} \cdot\left(\chi \cdot \frac{\mathbf{B}}{\mu_{0}} V\right) \\
&=\frac{V}{\mu_{0}}\left(B_{x}, B_{y}, B_{z}\right)\left(\begin{array}{ccc}
\chi_{x} & \\
& \chi_{y} & \\
& \chi_{z}
\end{array}\right)\left(\begin{array}{l}
B_{x} \\
B_{y} \\
B_{z}
\end{array}\right) \\
&=\frac{V}{\mu_{0}}\left(\chi_{x} B_{x}^{2}+\chi_{y} B_{y}^{2}+\chi_{z} B_{z}^{2}\right) \\
&=\frac{V}{\mu_{0}}\left[\chi_{x}\left(B_{x}^{2}+B_{y}^{2}\right)+\chi_{z} B_{z}^{2}\right] \\
&=\frac{V}{\mu_{0}}\left[\chi_{x} B^{2} \sin ^{2} \theta+x_{z} B^{2} \cos ^{2} \theta\right]
\end{aligned}
$$

 The torque on the particle is

$$
\begin{aligned}
\tau &=-\frac{\partial E}{\partial \theta}=-\frac{V}{\mu_{0}}\left[\chi_{x} \cdot 2 \sin \theta \cos \theta+\chi_{z} \cdot 2 \cos \theta(-\sin \theta)\right] B^{2} \\
&=\frac{B^{2} V}{\mu_{0}}\left(\chi_{z}-\chi_{x}\right) \sin 2 \theta
\end{aligned}
$$
\textbf{Topic} :Magnetostatic Field and Quasi-Stationary Electromagnetic Fields\\
\textbf{Book} :Problems and Solutions on Electromagnetism\\
\textbf{Final Answer} :\frac{B^{2} V}{\mu_{0}}\left(\chi_{z}-\chi_{x}\right) \sin 2 \theta\\


\textbf{Solution} :The Lagrangian of a charge $q$ in an electromagnetic field is

$$
L=T-V=\frac{1}{2} m v^{2}-q\left(\varphi-\frac{\mathbf{v} \cdot \mathbf{A}}{c}\right)
$$

where $v$ is the velocity of the charge of mass $m, \varphi$ is the scalar potential, and $A$ is the vector potential. As

$$
\begin{aligned}
\mathbf{v} &=\dot{\rho} \mathbf{e}_{\rho}+\rho \dot{\theta} \mathbf{e}_{\theta}+\dot{z} \mathbf{e}_{z}, \\
\mathbf{A} &=\frac{I \pi a^{2} \rho}{\left(a^{2}+z^{2}\right)^{3 / 2}} \mathbf{e}_{\theta}, \quad \varphi=0,
\end{aligned}
$$

we have

$$
L=\frac{m}{2}\left(\dot{\rho}^{2}+\rho^{2} \dot{\theta}^{2}+\dot{z}^{2}\right)+\frac{I \pi a^{2} q \rho^{2}}{c\left(a^{2}+z^{2}\right)^{3 / 2}} \dot{\theta} .
$$

Hence the components of the canonical momentum:

$$
\begin{aligned}
P_{\rho} &=\frac{\partial L}{\partial \dot{\rho}}=m \dot{\rho}, \\
P_{\theta} &=\frac{\partial L}{\partial \dot{\theta}}=m \rho^{2} \dot{\theta}+\frac{I \pi a^{2} q p^{2}}{c\left(a^{2}+z^{2}\right)^{3 / 2}}, \\
P_{z} &=\frac{\partial L}{\partial \dot{z}}=m \dot{z}
\end{aligned}
$$

The Hamiltonian is then

$$
\begin{aligned}
H &=P_{\rho} \dot{\rho}+P_{\theta} \dot{\theta}+P_{z} \dot{z}-L \\
&=\frac{1}{2 m}\left[P_{\rho}^{2}+\frac{1}{\rho^{2}}\left(P_{\theta}-\frac{I \pi a^{2} q \rho^{2}}{c\left(a^{2}+z^{2}\right)^{3 / 2}}\right)^{2}+P_{z}^{2}\right] .
\end{aligned}
$$
i.e.,

 Using Hamilton's canonical equation $\dot{P}_{\theta}=-\frac{\partial H}{\partial \theta}$, we obtain $\dot{P}_{\theta}=0$,

$$
P_{\theta}=m \rho^{2} \dot{\theta}+\frac{I \pi a^{2} q \rho^{2}}{c\left(a^{2}+z^{2}\right)^{3 / 2}}=\text { const } .
$$

Initially when the particle was far away from the lens it was traveling along the axis of the circular loop $(\rho=0)$ with $v_{\theta}=0$. Since $P_{\theta}$ is a constant of the motion, we have $P_{\theta}=0$, giving

$$
\dot{\theta}=-\frac{I \pi a^{2} q}{m c\left(a^{2}+z^{2}\right)^{3 / 2}} .
$$

gives
\textbf{Topic} :Magnetostatic Field and Quasi-Stationary Electromagnetic Fields\\
\textbf{Book} :Problems and Solutions on Electromagnetism\\
\textbf{Final Answer} :-\frac{I \pi a^{2} q}{m c\left(a^{2}+z^{2}\right)^{3 / 2}}\\


\textbf{Solution} :The Lagrangian of a charge $q$ in an electromagnetic field is

$$
L=T-V=\frac{1}{2} m v^{2}-q\left(\varphi-\frac{\mathbf{v} \cdot \mathbf{A}}{c}\right)
$$

where $v$ is the velocity of the charge of mass $m, \varphi$ is the scalar potential, and $A$ is the vector potential. As

$$
\begin{aligned}
\mathbf{v} &=\dot{\rho} \mathbf{e}_{\rho}+\rho \dot{\theta} \mathbf{e}_{\theta}+\dot{z} \mathbf{e}_{z}, \\
\mathbf{A} &=\frac{I \pi a^{2} \rho}{\left(a^{2}+z^{2}\right)^{3 / 2}} \mathbf{e}_{\theta}, \quad \varphi=0,
\end{aligned}
$$

we have

$$
L=\frac{m}{2}\left(\dot{\rho}^{2}+\rho^{2} \dot{\theta}^{2}+\dot{z}^{2}\right)+\frac{I \pi a^{2} q \rho^{2}}{c\left(a^{2}+z^{2}\right)^{3 / 2}} \dot{\theta} .
$$

Hence the components of the canonical momentum:

$$
\begin{aligned}
P_{\rho} &=\frac{\partial L}{\partial \dot{\rho}}=m \dot{\rho}, \\
P_{\theta} &=\frac{\partial L}{\partial \dot{\theta}}=m \rho^{2} \dot{\theta}+\frac{I \pi a^{2} q p^{2}}{c\left(a^{2}+z^{2}\right)^{3 / 2}}, \\
P_{z} &=\frac{\partial L}{\partial \dot{z}}=m \dot{z}
\end{aligned}
$$

The Hamiltonian is then

$$
\begin{aligned}
H &=P_{\rho} \dot{\rho}+P_{\theta} \dot{\theta}+P_{z} \dot{z}-L \\
&=\frac{1}{2 m}\left[P_{\rho}^{2}+\frac{1}{\rho^{2}}\left(P_{\theta}-\frac{I \pi a^{2} q \rho^{2}}{c\left(a^{2}+z^{2}\right)^{3 / 2}}\right)^{2}+P_{z}^{2}\right] .
\end{aligned}
$$
i.e.,

 Using Hamilton's canonical equation $\dot{P}_{\theta}=-\frac{\partial H}{\partial \theta}$, we obtain $\dot{P}_{\theta}=0$,

$$
P_{\theta}=m \rho^{2} \dot{\theta}+\frac{I \pi a^{2} q \rho^{2}}{c\left(a^{2}+z^{2}\right)^{3 / 2}}=\text { const } .
$$

Initially when the particle was far away from the lens it was traveling along the axis of the circular loop $(\rho=0)$ with $v_{\theta}=0$. Since $P_{\theta}$ is a constant of the motion, we have $P_{\theta}=0$, giving

$$
\dot{\theta}=-\frac{I \pi a^{2} q}{m c\left(a^{2}+z^{2}\right)^{3 / 2}} .
$$

gives

 Another Hamilton's canonical equation $\dot{P}_{\rho}=-\frac{\partial H}{\partial \rho}$, with $P_{\theta}=0$,

$$
\dot{P}_{p}=-\frac{I^{2} \pi^{2} a^{4} q^{2} p}{m c^{2}\left(a^{2}+z^{2}\right)^{3}},
$$

or

$$
d P_{\rho}=-\frac{I^{2} \pi^{2} a^{4} q^{2} \rho}{m c^{2}\left(a^{2}+z^{2}\right)^{3} \dot{z}} d z .
$$

Since $\rho \simeq b$ and $\dot{z} \simeq u$ are nearly constant in the interaction region, the change in the radial momentum is

$$
\Delta P_{\rho} \approx-\frac{I^{2} \pi^{2} a^{4} q^{2} b}{m c^{2} u} \int_{-\infty}^{\infty} \frac{1}{\left(a^{2}+z^{2}\right)^{3}} d z=-\frac{3 \pi b}{8 m a u}\left(\frac{I q \pi}{c}\right)^{2} .
$$

Consider the orbit shown in Fig. 2.67. We have $\frac{\dot{\rho}_{0}}{u}=\frac{b}{\Gamma_{0}}$ at the object point and $-\frac{\dot{\rho}_{i}}{u}=\frac{b}{l_{i}}$ at the image point of the lens. Hence

$$
\frac{b}{l_{0}}+\frac{b}{l_{i}}=\frac{1}{u}\left(\dot{\rho}_{0}-\dot{\rho}_{i}\right)=-\frac{\Delta P_{\rho}}{m u}=\frac{3 \pi b}{8 a}\left(\frac{I q \pi}{m u c}\right)^{2}
$$

which can be written as

$$
\frac{1}{l_{0}}+\frac{1}{l_{i}}=\frac{1}{f}
$$

with

$$
f=\frac{8 a}{3 \pi}\left(\frac{m u c}{I q \pi}\right)^{2}
$$
\textbf{Topic} :Magnetostatic Field and Quasi-Stationary Electromagnetic Fields\\
\textbf{Book} :Problems and Solutions on Electromagnetism\\
\textbf{Final Answer} :\frac{8 a}{3 \pi}\left(\frac{m u c}{I q \pi}\right)^{2}\\


\textbf{Solution} :The radiation emitted per unit time by an accelerating nonrelativistic particle of charge $q$ with velocity $v \ll c$ is approximately

$$
P=\frac{2 q^{2}}{3 c^{3}} \dot{v}^{2} .
$$

in Gaussian units. The equation of the motion of the particle in the magnetic field $B$ is

$$
m_{0} \dot{\mathbf{v}}_{0}=\frac{q}{c}\left(\mathbf{v}_{0} \times \mathbf{B}\right),
$$

giving

$$
\dot{v}_{0}^{2}=\frac{q^{2}}{m_{0}^{2} c^{2}} v_{0}^{2} B^{2} \sin ^{2} \alpha,
$$

where $\alpha$ is the angle between $\mathbf{v}_{0}$ and $\mathbf{B}$. The rate of emission of radiation is then

$$
P=\frac{2 q^{4}}{3 m_{0}^{2} c^{5}} B^{2} v_{0}^{2} \sin ^{2} \alpha \mathrm{erg} / \mathrm{s} .
$$
\textbf{Topic} :Magnetostatic Field and Quasi-Stationary Electromagnetic Fields\\
\textbf{Book} :Problems and Solutions on Electromagnetism\\
\textbf{Final Answer} :\frac{2 q^{4}}{3 m_{0}^{2} c^{5}} B^{2} v_{0}^{2} \sin ^{2} \alpha \mathrm{erg} / \mathrm{s}\\


\textbf{Solution} :The radiation emitted per unit time by an accelerating nonrelativistic particle of charge $q$ with velocity $v \ll c$ is approximately

$$
P=\frac{2 q^{2}}{3 c^{3}} \dot{v}^{2} .
$$

in Gaussian units. The equation of the motion of the particle in the magnetic field $B$ is

$$
m_{0} \dot{\mathbf{v}}_{0}=\frac{q}{c}\left(\mathbf{v}_{0} \times \mathbf{B}\right),
$$

giving

$$
\dot{v}_{0}^{2}=\frac{q^{2}}{m_{0}^{2} c^{2}} v_{0}^{2} B^{2} \sin ^{2} \alpha,
$$

where $\alpha$ is the angle between $\mathbf{v}_{0}$ and $\mathbf{B}$. The rate of emission of radiation is then

$$
P=\frac{2 q^{4}}{3 m_{0}^{2} c^{5}} B^{2} v_{0}^{2} \sin ^{2} \alpha \mathrm{erg} / \mathrm{s} .
$$

 The radiation emitted by a charged particle moving in a magnetic field $\mathbf{B}$ is known as cyclotron radiation and has the form of the radiation of a Hertzian dipole. The particle executes Larmor precession perpendicular to $\mathrm{B}$ with angular frequency $\omega_{0}=\frac{Q B}{m_{\rho} c}$. Actually there are also weaker radiations of higher harmonic frequencies $2 \omega_{0}, 3 \omega_{0}, \ldots$. However, if $v_{0} \ll c$ is satisfied, the dipole radiation is the main component and the others may be neglected.
\textbf{Topic} :Magnetostatic Field and Quasi-Stationary Electromagnetic Fields\\
\textbf{Book} :Problems and Solutions on Electromagnetism\\
\textbf{Final Answer} :\frac{2 q^{4}}{3 m_{0}^{2} c^{5}} B^{2} v_{0}^{2} \sin ^{2} \alpha \mathrm{erg} / \mathrm{s}\\


\textbf{Solution} :The normal component of $B$, which is continuous across a boundary, is zero on the conducting surface: $B_{n}=0$.
\textbf{Topic} :Magnetostatic Field and Quasi-Stationary Electromagnetic Fields\\
\textbf{Book} :Problems and Solutions on Electromagnetism\\
\textbf{Final Answer} :\frac{2 q^{4}}{3 m_{0}^{2} c^{5}} B^{2} v_{0}^{2} \sin ^{2} \alpha \mathrm{erg} / \mathrm{s}\\


\textbf{Solution} :The normal component of $B$, which is continuous across a boundary, is zero on the conducting surface: $B_{n}=0$.

 As shown in Fig. 2.69, the image of the current is a current loop symmetric with respect to the surface of the conducting plane, but with an opposite direction of flow. The magnetic field above the planar conductor is the superposition of the magnetic fields produced by the two currents which satisfies the boundary condition $B_{n}=0$.

MATHPIX IMAGE

Fig. $2.69$

 Consider a current element Idl of the real current loop. As $x \ll r$, we may consider the image current as an infinite straight line current. Then the current element $I d l$ will experience an upward force of magnitude

$$
d F=I|d l \times \mathrm{B}|=I\left|d l \frac{\mu_{0}(-I)}{2 \pi(2 x)}\right|=\frac{\mu_{0} I^{2} d l}{4 \pi x} .
$$

The force on the entire current loop is

$$
F=\frac{\mu_{0} I^{2}}{4 \pi x} \cdot 2 \pi r=\frac{\mu_{0} I^{2}}{2} \frac{r}{x} .
$$



At the equilibrium height this force equals the downward gravity:

$$
\frac{\mu_{0} I^{2} r}{2 x}=m g
$$

giving

$$
x=\frac{\mu_{0} I^{2} r}{2 m g} .
$$

Suppose the loop is displaced a small distance $\delta$ vertically from the equilibrium height $x$, i.e., $x \rightarrow x+\delta, \delta \ll x$. The equation of the motion of the loop in the vertical direction is

$$
-m \ddot{\delta}=m g-\frac{\mu_{0} I^{2} r}{2(x+\delta)} \simeq m g-\frac{\mu_{0} I^{2} r}{2 x}\left(1-\frac{\delta}{x}\right) .
$$

Noting that $m g=\frac{\mu_{0} I^{2} r}{2 x}$, we get

$$
\ddot{\delta}+\frac{\mu_{0} I^{2} r}{2 m x^{2}} \delta=0 .
$$

This shows that the vertical motion is harmonic with angular frequency

$$
\omega_{0}=\sqrt{\frac{\mu_{0} I^{2} r}{2 m x^{2}}}=\frac{g}{I} \sqrt{\frac{2 m}{\mu_{0} r}} .
$$

\textbf{Topic} :Magnetostatic Field and Quasi-Stationary Electromagnetic Fields\\
\textbf{Book} :Problems and Solutions on Electromagnetism\\
\textbf{Final Answer} :\frac{g}{I} \sqrt{\frac{2 m}{\mu_{0} r}}\\


\textbf{Solution} :Let the time of the fall be $T$. In the process of falling, potential energy is converted into thermal energy $W_{t}$ and kinetic energy $W_{k}$ given by an order of magnitude analysis to be roughly (in Gaussian units)

$$
\begin{aligned}
W_{t} \sim I^{2} R T \approx\left(\frac{\phi}{c T R}\right)^{2} R T=\frac{B^{2}\left(\pi r_{1}^{2}\right)^{2}}{c^{2} R T} \\
=& \frac{m B^{2}\left(\pi r_{1}^{2}\right)^{2}}{c^{2} T \cdot\left(\frac{2 \pi r_{1}}{\sigma \pi r_{2}^{2}}\right) \cdot\left(\rho \cdot 2 \pi r_{1} \cdot \pi r_{2}^{2}\right)}=\frac{m r_{1}^{2}}{T} \cdot \frac{\sigma B^{2}}{4 \rho c^{2}}, \\
& W_{k} \sim \frac{1}{2} I \omega^{2} \approx \frac{1}{2} \cdot \frac{3}{2} m r_{1}^{2}\left(\frac{\pi}{2 T}\right)^{2}=\frac{3 \pi^{2} m r_{1}^{2}}{16 T^{2}},
\end{aligned}
$$

where

Putting

$$
\begin{aligned}
&r_{1}, r_{2}=\text { major and minor radii of the ring respectively, } \\
&\phi=\text { magnetic flux crossing the ring } \approx B \pi r_{1}^{2}, \\
&\rho=\text { density of gold, } \\
&\sigma=\text { conductivity of gold, } \\
&R=\text { resistance of the ring }=\frac{2 \pi r_{1}}{\sigma \pi r_{2}^{2}}, \\
&m=\text { mass of the ring }=\rho 2 \pi r_{1} \cdot \pi r_{2}^{2} \\
&c=\text { velocity of light in vacuum, } \\
&\omega=\text { angular velocity of fall } \approx \frac{\pi}{2 T} .
\end{aligned}
$$

$$
\begin{aligned}
T_{g} &=\sqrt{\frac{r_{1}}{g}}=\sqrt{\frac{1}{980}}=3.2 \times 10^{-2} \mathrm{~s}, \\
T_{B} &=\frac{4 \rho c^{2}}{\sigma B^{2}}=\frac{4 \times 19.3 \times 9 \times 10^{20}}{4 \times 10^{17} \times 10^{8}}=1.74 \times 10^{-3} \mathrm{~s}
\end{aligned}
$$

we can write the above as

$$
W_{t} \sim m g r_{1} \cdot \frac{T_{g}^{2}}{T \cdot T_{B}}, \quad W_{k} \sim m g r_{1} \cdot \frac{3 \pi^{2}}{16}\left(\frac{T_{g}}{T}\right)^{2} .
$$

The energy balance gives

$$
m g r_{1}=W_{t}+W_{k},
$$

or

$$
T^{2}-\left(\frac{T_{?}^{2}}{T_{B}}\right) T-\frac{3 \pi^{2}}{16} T_{?}^{2}=0 .
$$

Solving for $T$ we have

$$
T=\frac{T_{g}}{2}\left[\frac{T_{g}}{T_{B}}+\sqrt{\left(\frac{T_{g}}{T_{B}}\right)^{2}+\frac{3 \pi^{2}}{4}}\right]
$$

As $\left(\frac{T_{2}}{T_{B}}\right)^{2}>\frac{3 \pi^{2}}{4}, T \approx \frac{T_{2}^{2}}{T_{B}}$. Hence

$$
W_{t} \sim m g r_{1}, \quad W_{k} \sim m g r_{1} \cdot \frac{3 \pi^{2}}{16}\left(\frac{T_{B}}{T_{g}}\right) \ll m g r_{1} .
$$

It follows that the potential energy released by the fall goes mainly into raising the temperature of the ring.
 We neglect the kinetic energy of the ring. That is, we assume that the potential energy is changed entirely into thermal energy. Then the gravitational torque and magnetic torque on the ring approximately balance each other.

The magnetic flux crossing the ring is

$$
\phi(\theta)=B \pi r_{1}^{2} \sin \theta .
$$

The induced emf is

$$
\varepsilon=\frac{1}{c}\left|\frac{d \phi}{d t}\right|=\frac{1}{c} B \pi r_{1}^{2} \cos \theta \dot{\theta},
$$

giving the induced electric current as

$$
i=\frac{\varepsilon}{R}=B \pi r_{1}^{2} \cos \theta \cdot \frac{\dot{\theta}}{c R}
$$

and the magnetic moment of the ring as

$$
m=\frac{i \pi r_{1}^{2}}{c}=\frac{B\left(\pi r_{1}^{2}\right)^{2} \cos \theta \dot{\theta}}{c^{2} R} .
$$

Thus the magnetic torque on the ring is

$$
\tau_{m}=|\mathbf{m} \times \mathbf{B}|=\frac{\left(B \pi r_{1}^{2} \cos \theta\right)^{2} \dot{\theta}}{c^{2} R} .
$$

The gravitational torque on the ring is $\tau_{g}=m g r_{1} \sin \theta$. Therefore

$$
\tau_{m}=\tau_{g},
$$

or

$$
\frac{\left(B \pi r_{1}^{2} \cos \theta\right)^{2} \dot{\theta}}{c^{2} R}=m g r_{1} \sin \theta,
$$

giving

$$
d t=\frac{\sigma B^{2} r_{1} \cos ^{2} \theta d \theta}{4 \rho g c^{2} \sin \theta} .
$$

Integrating we find

$$
\begin{aligned}
T &=\frac{\sigma B^{2} r_{1}}{4 \rho g c^{2}} \int_{0.1}^{\frac{\pi}{2}} \frac{\cos ^{2} \theta d \theta}{\sin \theta}=\frac{\sigma B^{2} r_{1}}{4 \rho g c^{2}} \cdot 2 \\
&=2 \frac{T_{g}^{2}}{T_{B}}=2 \times \frac{\left(3.2 \times 10^{-2}\right)^{2}}{1.74 \times 10^{-3}}=1.2 \mathrm{~s} .
\end{aligned}
$$



\textbf{Topic} :Magnetostatic Field and Quasi-Stationary Electromagnetic Fields\\
\textbf{Book} :Problems and Solutions on Electromagnetism\\
\textbf{Final Answer} :12 \mathrm{~s}\\


\textbf{Solution} :Let the charge, mass and angular momentum of the particle be $q, m, L$ respectively. Use cylindrical coordinates $(R, \theta, z)$ with the $z$-axis along the axis of the circular orbit and the origin at its center. As we are interested in the steady component of the field, we can consider the charge orbiting the circle as a steady current loop. The vector potential at a point of radius vector $\mathbf{R}$ from the origin is

$$
\mathbf{A}(\mathbf{R})=\frac{\mu_{0}}{4 \pi} \int \frac{\mathbf{J}\left(\mathbf{r}^{\prime}\right)}{r} d V^{\prime}
$$

where $r=\left|\mathbf{R}-\mathbf{r}^{\prime}\right|$. For large distances take the approximation $r=R(1-$ $\frac{\mathbf{R} \cdot \mathbf{r}^{\prime}}{R^{2}}$ ) and write $\mathrm{J}\left(\mathbf{r}^{\prime}\right) d V^{\prime}=I d \mathbf{r}^{\prime}$. Then

$$
\begin{aligned}
\mathbf{A}(\mathbf{R}) & \approx \frac{\mu_{0} I}{4 \pi R} \oint\left(1+\frac{\mathbf{R} \cdot \mathbf{r}^{\prime}}{R^{2}}+\ldots\right) d \mathbf{r}^{\prime} \\
& \approx \frac{\mu_{0} I}{4 \pi R^{3}} \oint\left(\mathbf{R} \cdot \mathbf{r}^{\prime}\right) d \mathbf{r}^{\prime}
\end{aligned}
$$

integrating over the circular orbit. Write

$$
\begin{aligned}
\left(\mathbf{R} \cdot \mathbf{r}^{\prime}\right) d \mathbf{r}^{\prime}=& \frac{1}{2}\left[\left(\mathbf{R} \cdot \mathbf{r}^{\prime}\right) d \mathbf{r}^{\prime}-\left(\mathbf{R} \cdot d \mathbf{r}^{\prime}\right) \mathbf{r}^{\prime}\right] \\
&+\frac{1}{2}\left[\left(\mathbf{R} \cdot \mathbf{r}^{\prime}\right) d \mathbf{r}^{\prime}+\left(\mathbf{R} \cdot d \mathbf{r}^{\prime}\right) \mathbf{r}^{\prime}\right] .
\end{aligned}
$$

The symmetric part gives rise to an electric quadrupole field and will not be considered. The antisymmetric part can be written as

$$
\frac{1}{2}\left(\mathbf{r}^{\prime} \times d \mathbf{r}^{\prime}\right) \times \mathbf{R} .
$$

Hence, considering only the magnetic dipole field we have

$$
\begin{aligned}
\mathbf{A}(\mathbf{R}) &=\frac{\mu_{0}}{4 \pi}\left[\frac{I}{2} \oint r^{\prime} \times d r^{\prime}\right] \times \frac{\mathbf{R}}{R^{3}}, \\
&=\frac{\mu_{0}}{4 \pi} I \pi r^{2} e_{z} \times \frac{\mathbf{R}}{R^{3}}=\frac{\mu_{0}}{4 \pi} \mathbf{M} \times \frac{\mathbf{R}}{R^{3}},
\end{aligned}
$$

where $M=I \pi r^{2} e_{z}$ is the magnetic dipole moment of the loop. From $\mathbf{B}=\nabla \times \mathbf{A}$ we have

$$
\begin{aligned}
\mathrm{B} &=\frac{\mu_{0}}{4 \pi} \nabla \times\left(\mathrm{M} \times \frac{\mathrm{R}}{R^{3}}\right)=-\frac{\mu_{0}}{4 \pi}(M \cdot \nabla) \frac{\mathbf{R}}{R^{3}} \\
&=\frac{\mu_{0} q L}{8 \pi m}\left(\frac{3 \cos \theta}{R^{3}} \mathbf{e}_{R}-\frac{\mathbf{e}_{z}}{R^{3}}\right)=\frac{\mu_{0} q L}{8 \pi m R^{3}}\left(3 \cos \theta \mathrm{e}_{R}-\mathbf{e}_{z}\right),
\end{aligned}
$$

where we have used

$$
I=\frac{d q}{d t}=\frac{d q}{d l} \frac{d l}{d t}=\frac{q v}{2 \pi r}
$$

and

$$
M=I \pi r^{2}=\frac{q v r}{2}=\frac{q L}{2 m} .
$$
\textbf{Topic} :Magnetostatic Field and Quasi-Stationary Electromagnetic Fields\\
\textbf{Book} :Problems and Solutions on Electromagnetism\\
\textbf{Final Answer} :\frac{q L}{2 m}\\


\textbf{Solution} :when the angle between the plane of the loop and the magnetic field is $\alpha$, the magnetic flux passing through the loop is $\phi=B A \sin \alpha$. The induced emf and current are given by

$$
\varepsilon=-\frac{d \phi}{d t}=-B A \cos \alpha \dot{\alpha}, \quad i=\frac{\varepsilon}{R}=-\frac{B A \dot{\alpha} \cos \alpha}{R} .
$$

The magnetic moment of the loop is

$$
m=i A=-\frac{B A^{2} \cos \alpha}{R} \dot{\alpha} .
$$

Thus the magnetic torque on the loop is

$$
\tau_{m}=|\mathbf{m} \times \mathbf{B}|=-\frac{B^{2} A^{2} \cos ^{2} \alpha}{R} \dot{\alpha} .
$$

Besides, the torsion spring also provides a twisting torque $k \alpha$. Both torques will resist the rotation of the loop. Thus one has

$$
I \ddot{\alpha}+\frac{B^{2} A^{2} \cos ^{2} \alpha}{R} \dot{\alpha}+k \alpha=0 .
$$

As $\alpha \ll \theta$ and $\theta$ is itself small, we have $\cos ^{2} \alpha \approx 1$ and

$$
I \ddot{\alpha}+\frac{B^{2} A^{2}}{R} \dot{\alpha}+k a=0 .
$$

Let $\alpha=e^{C t}$ and obtain the characteristic equation

$$
I C^{2}+\frac{B^{2} A^{2}}{R} C+k=0 .
$$

The solution is

$$
C=\frac{-\frac{B^{2} A^{2}}{R} \pm \sqrt{\left(\frac{B^{2} A^{2}}{R}\right)^{2}-4 I k}}{2 I}=-\frac{B^{2} A^{2}}{2 I R} \pm j \sqrt{\frac{k}{I}-\left(\frac{B^{2} A^{2}}{2 I R}\right)^{2}} .
$$

Defining

$$
\beta=\frac{B^{2} A^{2}}{2 I R}, \quad \gamma=\sqrt{-\left(\frac{B^{2} A^{2}}{2 I R}\right)^{2}+\frac{k}{I}}=\sqrt{-\beta^{2}+\frac{k}{I}},
$$

we have two solutions

$$
C_{1}=-\beta+j \gamma, \quad C_{2}=-\beta-j \gamma .
$$

The general solution of the equation of motion is therefore

$$
\alpha=e^{-\beta t}\left[A_{1} \cos \gamma t+A_{2} \sin \gamma t\right] .
$$

Since $\left.\alpha\right|_{t=0}=\theta,\left.\dot{\alpha}\right|_{t=0}=0$, we find

$$
A_{1}=\theta, \quad A_{2}=\frac{\beta}{\gamma} A_{1}=\frac{\beta}{\gamma} \theta .
$$

Hence the rotational oscillation of the loop is described by

$$
\alpha(t)=\theta e^{-\beta t}\left[\cos \gamma t+\frac{\beta}{\gamma} \sin \gamma t\right] .
$$

Note that for the motion to be oscillatory, we require that $k>\beta^{2} I$, which was assumed to be the case.
\\
\textbf{Topic} :Magnetostatic Field and Quasi-Stationary Electromagnetic Fields\\
\textbf{Book} :Problems and Solutions on Electromagnetism\\
\textbf{Final Answer} :\theta e^{-\beta t}\left[\cos \gamma t+\frac{\beta}{\gamma} \sin \gamma t\right]\\


\textbf{Solution} :Suppose the long wires carrying currents $+I$ and $-I$ cross the $y$ axis at $+a$ and $-a$ respectively. Consider an arbitrary point $P$ and without loss of generality we can take the $y z$ plane to contain $P$. Let the distances of $P$ from the $y$ - and $z$-axes be $z$ and $y$ respectively, and its distances from the wires be $r_{1}$ and $r_{2}$ as shown in Fig. 2.77. Ampère's circuital law gives the magnetic inductions $\mathrm{B}_{1}$ and $\mathrm{B}_{2}$ at $P$ due to $+I$ and $-I$ respectively with magnitudes

$$
B_{1}=\frac{\mu_{0} I}{2 \pi r_{1}}, \quad B_{2}=\frac{\mu_{0} I}{2 \pi r_{2}},
$$

where $r_{1}=\left[z^{2}+(a-y)^{2}\right]^{\frac{1}{2}}$ and $r_{2}=\left[z^{2}+(a+y)^{2}\right]^{\frac{1}{2}}$, and directions as shown in Fig. 2.77. The total induction at $P, B=B_{1}+\mathbf{B}_{2}$, then has components

$$
\begin{aligned}
B_{x} &=0, \\
B_{y} &=-B_{1} \sin \theta_{1}+B_{2} \sin \theta_{2} \\
&=\frac{\mu_{0} I}{2 \pi}\left(-\frac{z}{r_{1}^{2}}+\frac{z}{r_{2}^{2}}\right)=-\frac{2 \mu_{0} I a y z}{\pi r_{1}^{2} r_{2}^{2}}, \\
B_{z} &=-B_{1} \cos \theta_{1}-B_{2} \cos \theta_{2} \\
&=-\frac{\mu_{0} I}{2 \pi}\left(\frac{a-y}{r_{1}^{2}}+\frac{a+y}{r_{2}^{2}}\right) .
\end{aligned}
$$



MATHPIX IMAGE

Fig. $2.77$

For a point in the $x z$ plane and distance $z$ from the $y$-axis, i.e., at coordinates $(0,0, z)$, the above reduces to

$$
\mathrm{B}=-\frac{\mu_{0} I}{\pi} \frac{a}{\left(z^{2}+a^{2}\right)} \mathbf{e}_{z} .
$$
\textbf{Topic} :Magnetostatic Field and Quasi-Stationary Electromagnetic Fields\\
\textbf{Book} :Problems and Solutions on Electromagnetism\\
\textbf{Final Answer} :-\frac{\mu_{0} I}{\pi} \frac{a}{\left(z^{2}+a^{2}\right)} \mathbf{e}_{z}\\


\textbf{Solution} :Suppose the long wires carrying currents $+I$ and $-I$ cross the $y$ axis at $+a$ and $-a$ respectively. Consider an arbitrary point $P$ and without loss of generality we can take the $y z$ plane to contain $P$. Let the distances of $P$ from the $y$ - and $z$-axes be $z$ and $y$ respectively, and its distances from the wires be $r_{1}$ and $r_{2}$ as shown in Fig. 2.77. Ampère's circuital law gives the magnetic inductions $\mathrm{B}_{1}$ and $\mathrm{B}_{2}$ at $P$ due to $+I$ and $-I$ respectively with magnitudes

$$
B_{1}=\frac{\mu_{0} I}{2 \pi r_{1}}, \quad B_{2}=\frac{\mu_{0} I}{2 \pi r_{2}},
$$

where $r_{1}=\left[z^{2}+(a-y)^{2}\right]^{\frac{1}{2}}$ and $r_{2}=\left[z^{2}+(a+y)^{2}\right]^{\frac{1}{2}}$, and directions as shown in Fig. 2.77. The total induction at $P, B=B_{1}+\mathbf{B}_{2}$, then has components

$$
\begin{aligned}
B_{x} &=0, \\
B_{y} &=-B_{1} \sin \theta_{1}+B_{2} \sin \theta_{2} \\
&=\frac{\mu_{0} I}{2 \pi}\left(-\frac{z}{r_{1}^{2}}+\frac{z}{r_{2}^{2}}\right)=-\frac{2 \mu_{0} I a y z}{\pi r_{1}^{2} r_{2}^{2}}, \\
B_{z} &=-B_{1} \cos \theta_{1}-B_{2} \cos \theta_{2} \\
&=-\frac{\mu_{0} I}{2 \pi}\left(\frac{a-y}{r_{1}^{2}}+\frac{a+y}{r_{2}^{2}}\right) .
\end{aligned}
$$



MATHPIX IMAGE

Fig. $2.77$

For a point in the $x z$ plane and distance $z$ from the $y$-axis, i.e., at coordinates $(0,0, z)$, the above reduces to

$$
\mathrm{B}=-\frac{\mu_{0} I}{\pi} \frac{a}{\left(z^{2}+a^{2}\right)} \mathbf{e}_{z} .
$$

 Equation (1) gives

$$
\frac{d B_{z}}{d z}=\frac{2 \mu_{0} I}{\pi} \frac{a z}{\left(z^{2}+a^{2}\right)^{2}} .
$$

Hence

$$
\frac{d B_{z}}{d z} / B_{z}=-\frac{2 z}{z^{2}+a^{2}}
$$
given by
\textbf{Topic} :Magnetostatic Field and Quasi-Stationary Electromagnetic Fields\\
\textbf{Book} :Problems and Solutions on Electromagnetism\\
\textbf{Final Answer} :-\frac{2 z}{z^{2}+a^{2}}\\


\textbf{Solution} :Suppose the long wires carrying currents $+I$ and $-I$ cross the $y$ axis at $+a$ and $-a$ respectively. Consider an arbitrary point $P$ and without loss of generality we can take the $y z$ plane to contain $P$. Let the distances of $P$ from the $y$ - and $z$-axes be $z$ and $y$ respectively, and its distances from the wires be $r_{1}$ and $r_{2}$ as shown in Fig. 2.77. Ampère's circuital law gives the magnetic inductions $\mathrm{B}_{1}$ and $\mathrm{B}_{2}$ at $P$ due to $+I$ and $-I$ respectively with magnitudes

$$
B_{1}=\frac{\mu_{0} I}{2 \pi r_{1}}, \quad B_{2}=\frac{\mu_{0} I}{2 \pi r_{2}},
$$

where $r_{1}=\left[z^{2}+(a-y)^{2}\right]^{\frac{1}{2}}$ and $r_{2}=\left[z^{2}+(a+y)^{2}\right]^{\frac{1}{2}}$, and directions as shown in Fig. 2.77. The total induction at $P, B=B_{1}+\mathbf{B}_{2}$, then has components

$$
\begin{aligned}
B_{x} &=0, \\
B_{y} &=-B_{1} \sin \theta_{1}+B_{2} \sin \theta_{2} \\
&=\frac{\mu_{0} I}{2 \pi}\left(-\frac{z}{r_{1}^{2}}+\frac{z}{r_{2}^{2}}\right)=-\frac{2 \mu_{0} I a y z}{\pi r_{1}^{2} r_{2}^{2}}, \\
B_{z} &=-B_{1} \cos \theta_{1}-B_{2} \cos \theta_{2} \\
&=-\frac{\mu_{0} I}{2 \pi}\left(\frac{a-y}{r_{1}^{2}}+\frac{a+y}{r_{2}^{2}}\right) .
\end{aligned}
$$



MATHPIX IMAGE

Fig. $2.77$

For a point in the $x z$ plane and distance $z$ from the $y$-axis, i.e., at coordinates $(0,0, z)$, the above reduces to

$$
\mathrm{B}=-\frac{\mu_{0} I}{\pi} \frac{a}{\left(z^{2}+a^{2}\right)} \mathbf{e}_{z} .
$$

 Equation (1) gives

$$
\frac{d B_{z}}{d z}=\frac{2 \mu_{0} I}{\pi} \frac{a z}{\left(z^{2}+a^{2}\right)^{2}} .
$$

Hence

$$
\frac{d B_{z}}{d z} / B_{z}=-\frac{2 z}{z^{2}+a^{2}}
$$
given by

 The magnetic lines of force are parallel to the $y z$ plane and are

$$
\frac{d y}{B_{y}}=\frac{d z}{B_{z}} .
$$

They have mirror symmetry with respect to the $x z$ plane, as shown by the dashed curves in Fig. 2.78. If we define the scalar magnetic potential $\phi_{m}$ by $\mathbf{H}=-\nabla \phi_{m}$, then the equipotentials are cylindrical surfaces everywhere perpendicular to the lines of force. Their intercepts in the $y z$ plane are shown as solid curves in the figure. Hence if the two wires carrying currents were replaced by a cylindrical permanent magnet piece with the two side surfaces $(+z$ and $-z)$ coinciding with the equipotential surfaces, the same magnetic lines of force would be obtained, since for an iron magnet of $\mu \rightarrow \infty$ the magnetic lines of force are approximately perpendicular to the surface of the magnetic poles.

MATHPIX IMAGE

Fig. $2.78$

Carrying the idea of magnetic charges further, we have

$$
\nabla \cdot \mathbf{H}=-\nabla^{2} \phi_{m}=\rho_{m},
$$

where $\rho_{m}$ is the magnetic charge density. Then applying the divergence theorem we have

$$
\oint_{S} \mathbf{H} \cdot d \mathbf{S}=\int_{V} \nabla \cdot \mathbf{H} d V=q_{m}
$$

where $q_{m}$ is the magnetic charge enclosed by $S$, showing that $\mathrm{H}$ is analogous to $\mathbf{D}$ in electrostatics. Applying the integral to a uniform cylinder, we have

$$
\oint_{C} \mathbf{H} \cdot d \mathrm{l}=\lambda_{m},
$$

where $C$ is the circumference of a cross section of the cylinder and $\lambda_{m}$ is the magnetic charge per unit length. Comparing with Ampère's circuital law, $\oint_{C} \mathbf{H} \cdot d l=I$, we have the equivalence

$$
I \leftrightarrow \lambda_{m} .
$$

Proceeding further the analogy between electric and magnetic fields we suppose that a metal pipe of the same cross section is used, instead of the cylindrical magnet piece, with charges $\pm \lambda$ per length on the side surfaces $\mp z$. Then an electrostatic field distribution is produced similar to the lines of force of $\mathrm{B}$ above. With the substitution $\mathrm{H} \rightarrow \mathrm{D}, I \rightarrow \lambda$, the relations in (a) and (b) are still valid.

 By analogy, Eq.
(1) gives

$$
E_{z}=-\frac{q}{4 \pi \varepsilon_{0}} \frac{a}{\left(z^{2}+a^{2}\right)} \text {. }
$$

With $a=z=5 \times 10^{-3} \mathrm{~m}, E_{z}=8 \times 10^{5} \mathrm{~V} / \mathrm{m}$, we have

$$
q=8.90 \times 10^{-7} \mathrm{C} \text {. }
$$

The potential difference between the two cylindrical sides carrying opposite charges $q$ per unit length is

$$
\begin{gathered}
\Delta V=\frac{q a}{4 \pi \varepsilon_{0}} \int_{z_{1}}^{z_{2}} \frac{d z}{z^{2}+a^{2}}=\left.\frac{q}{4 \pi \varepsilon_{0}} \arctan \left(\frac{z}{a}\right)\right|_{z_{1}} ^{z_{2}}=2.7 \times 10^{3} \mathrm{~V} \\
\text { with } z_{1}=4 \times 10^{-3} \mathrm{~m}, z_{2}=8 \times 10^{-3} \mathrm{~m}
\end{gathered}
$$

\textbf{Topic} :Magnetostatic Field and Quasi-Stationary Electromagnetic Fields\\
\textbf{Book} :Problems and Solutions on Electromagnetism\\
\textbf{Final Answer} :8 \times 10^{-3} \mathrm{~m}
\end{gathered}\\


\textbf{Solution} :A Thomson type apparatus is shown schematically in Fig. 2.79, where $V_{1}$ is the accelerating voltage and $V_{2}$ is the deflecting voltage.

MATHPIX IMAGE

Fig. $2.79$ With the addition of a magnetic field $B$ as shown, the electromagnetic field has the action of a velocity-filter. With given values of $V_{1}$ and $V_{2}$, we adjust the magnitude of $B$ so that the electrons strike the center $O$ of the screen. At this time the velocity of the electron is $v=E / B$ (since $e E=e v B$ ). Afterward the magnetic field $B$ is turned off and the displacement $y_{2}$ of the electrons on the screen is measured. The ratio $e / m$ is calculated as follows:

$$
\begin{aligned}
y_{1} &=\frac{1}{2} \cdot \frac{e E}{m}\left(\frac{L}{v}\right)^{2}, \\
y_{2}=\frac{D+\frac{L}{2}}{L / 2} y_{1} &=\frac{e E}{m v^{2}}\left(\frac{L^{2}}{2}+L D\right)=\frac{e}{m} \cdot \frac{d B^{2}}{V_{2}}\left(\frac{L^{2}}{2}+L D\right),
\end{aligned}
$$

giving

$$
e / m=\frac{V_{2} y_{2}}{d B^{2}\left(\frac{L^{2}}{2}+L D\right)} .
$$

 When the accelerating voltage is very large, relativistic effects must be considered. From energy conversation

$$
e V_{1}+m_{0} c^{2}=m c^{2},
$$

we find

$$
V_{1}=\left(\frac{m}{e}-\frac{m_{0}}{e}\right) c^{2} .
$$

As $\frac{e}{m}=\frac{1}{2} \frac{e}{m_{0}}$, the accelerating voltage is

$$
V_{1}=\frac{m_{0} c^{2}}{e}=\frac{9 \times 10^{16}}{1.8 \times 10^{11}}=5 \times 10^{5} \mathrm{~V}
$$

\textbf{Topic} :Magnetostatic Field and Quasi-Stationary Electromagnetic Fields\\
\textbf{Book} :Problems and Solutions on Electromagnetism\\
\textbf{Final Answer} :5 \times 10^{5} \mathrm{~V}\\


\textbf{Solution} :The constitutive equation for electric fields in a dielectric medium moving with velocity $v$ in a magnetic field $B$ is

$$
\mathbf{D}=k \varepsilon_{0} \mathbf{E}+\varepsilon_{0}(k-1) \mathbf{v} \times \mathbf{B},
$$

where $k$ is its relative dielectric constant. For a point distance $r$ from the axis of rotation, $\mathbf{v}=\boldsymbol{\omega} \times \mathbf{r}$ and $\mathbf{v} \times \mathbf{B}=(\boldsymbol{\omega} \cdot \mathbf{B}) \mathbf{r}-(\mathbf{r} \cdot \mathbf{B}) \boldsymbol{\omega}=\omega B \mathbf{r}$ as $\mathbf{r}$ is perpendicular to $B$. As there are no free charges, Gauss' flux theorem $\oint \mathbf{D} \cdot d \mathbf{S}=0$ gives $\mathbf{D}=0$. Then from $\mathbf{D}=\varepsilon_{0} \mathbf{L}+\mathbf{P}$ we get

$$
\mathbf{P}=-\varepsilon_{0} \mathbf{E}=\varepsilon_{0}\left(1-\frac{1}{k}\right) \omega B \mathbf{r} .
$$

Hence the volume bound charge density is

$$
\rho^{\prime}=-\nabla \cdot \mathbf{P}=-\frac{1}{r} \frac{\partial}{\partial r}\left(r P_{r}\right)=-2 \varepsilon_{0}\left(1-\frac{1}{k}\right) \omega B
$$

and the surface bound charge density is

$$
\sigma^{\prime}=P_{r}=\varepsilon_{0}\left(1-\frac{1}{k}\right) \omega B a,
$$

as $r=a$ for the cylinder's surface.
\textbf{Topic} :Magnetostatic Field and Quasi-Stationary Electromagnetic Fields\\
\textbf{Book} :Problems and Solutions on Electromagnetism\\
\textbf{Final Answer} :\varepsilon_{0}\left(1-\frac{1}{k}\right) \omega B a\\


\textbf{Solution} :The constitutive equation for electric fields in a dielectric medium moving with velocity $v$ in a magnetic field $B$ is

$$
\mathbf{D}=k \varepsilon_{0} \mathbf{E}+\varepsilon_{0}(k-1) \mathbf{v} \times \mathbf{B},
$$

where $k$ is its relative dielectric constant. For a point distance $r$ from the axis of rotation, $\mathbf{v}=\boldsymbol{\omega} \times \mathbf{r}$ and $\mathbf{v} \times \mathbf{B}=(\boldsymbol{\omega} \cdot \mathbf{B}) \mathbf{r}-(\mathbf{r} \cdot \mathbf{B}) \boldsymbol{\omega}=\omega B \mathbf{r}$ as $\mathbf{r}$ is perpendicular to $B$. As there are no free charges, Gauss' flux theorem $\oint \mathbf{D} \cdot d \mathbf{S}=0$ gives $\mathbf{D}=0$. Then from $\mathbf{D}=\varepsilon_{0} \mathbf{L}+\mathbf{P}$ we get

$$
\mathbf{P}=-\varepsilon_{0} \mathbf{E}=\varepsilon_{0}\left(1-\frac{1}{k}\right) \omega B \mathbf{r} .
$$

Hence the volume bound charge density is

$$
\rho^{\prime}=-\nabla \cdot \mathbf{P}=-\frac{1}{r} \frac{\partial}{\partial r}\left(r P_{r}\right)=-2 \varepsilon_{0}\left(1-\frac{1}{k}\right) \omega B
$$

and the surface bound charge density is

$$
\sigma^{\prime}=P_{r}=\varepsilon_{0}\left(1-\frac{1}{k}\right) \omega B a,
$$

as $r=a$ for the cylinder's surface.

 By symmetry and using Ampère's circuital law, we obtain the magnetic induction in a doughnut-shaped solenoid:

$$
B=\frac{\mu_{0} N I}{2 \pi r},
$$

where $r$ is the distance from the center of the doughnut. Consider a small section of length $d l$ of the solenoid. This section contains $\frac{N}{2 \pi R} d l$ turns of the winding, where $R$ is the radius of the doughnut. Take as current element a segment of this section which subtends an angle $d \theta$ at the axis of the solenoid:

$$
\Delta I=\frac{N I d l}{2 \pi R} \rho d \theta,
$$

where $\theta$ is the angle made by the radius from the axis to the segment and the line from the axis to center of the doughnut and $\rho$ is the radius of a loop of winding. The magnetic force on the current element is in the radial direction and has magnitude

$$
\begin{aligned}
d F &=\Delta I \cdot \frac{B}{2}=\frac{N I \rho}{4 \pi R} B d \theta d l \\
&=\frac{\mu_{0} N^{2} I^{2} \rho}{8 \pi^{2} R r} d \theta d l,
\end{aligned}
$$

where $B / 2$ is used, instead of $B$, because the magnetic field established by the current element itself has to be taken out from the total field. Note that $d F$ is perpendicular to the surface of the solenoid and only its component $d F \cdot \cos \theta$ along the line from the axis to the center of the doughnut is not canceled out with another element at $2 \pi-\theta$. As

$$
r=R+\rho \cos \theta,
$$

we have the total force on the doughnut

$$
\begin{aligned}
F &=\int \cos \theta d F \\
&=\frac{\mu_{0} N^{2} I^{2}}{8 \pi^{2} R} \int_{0}^{2 \pi R} d l \int_{0}^{2 \pi} \frac{\rho \cos \theta}{R+\rho \cos \theta} d \theta \\
&=\frac{\mu_{0} N^{2} I^{2}}{4 \pi} \int_{0}^{2 \pi}\left(1-\frac{R}{R+\rho \cos \theta}\right) d \theta \\
&=\frac{\mu_{0} N^{2} I^{2}}{4 \pi} \int_{0}^{2 \pi}\left[1-\left(1+\frac{\rho}{R} \cos \theta\right)^{-1}\right] d \theta \\
&=\frac{\mu_{0} N^{2} I^{2}}{2}\left\{1-\left[1-\left(\frac{\rho}{R}\right)^{2}\right]^{-\frac{1}{2}}\right\} \\
&=\frac{4 \pi \times 10^{-7} \times 1000^{2} \times 10^{2}}{2}\left[1-\frac{1}{\sqrt{1-0.05^{2}}}\right] \\
&=-0.079 \mathrm{~N}
\end{aligned}
$$

Hence, the force on one loop is

$$
\frac{F}{N}=-\frac{0.079}{1000}=-7.9 \times 10^{-5} \mathrm{~N}
$$

and points to the center of the doughnut.
\textbf{Topic} :Magnetostatic Field and Quasi-Stationary Electromagnetic Fields\\
\textbf{Book} :Problems and Solutions on Electromagnetism\\
\textbf{Final Answer} :-79 \times 10^{-5} \mathrm{~N}\\


\textbf{Solution} :The constitutive equation for electric fields in a dielectric medium moving with velocity $v$ in a magnetic field $B$ is

$$
\mathbf{D}=k \varepsilon_{0} \mathbf{E}+\varepsilon_{0}(k-1) \mathbf{v} \times \mathbf{B},
$$

where $k$ is its relative dielectric constant. For a point distance $r$ from the axis of rotation, $\mathbf{v}=\boldsymbol{\omega} \times \mathbf{r}$ and $\mathbf{v} \times \mathbf{B}=(\boldsymbol{\omega} \cdot \mathbf{B}) \mathbf{r}-(\mathbf{r} \cdot \mathbf{B}) \boldsymbol{\omega}=\omega B \mathbf{r}$ as $\mathbf{r}$ is perpendicular to $B$. As there are no free charges, Gauss' flux theorem $\oint \mathbf{D} \cdot d \mathbf{S}=0$ gives $\mathbf{D}=0$. Then from $\mathbf{D}=\varepsilon_{0} \mathbf{L}+\mathbf{P}$ we get

$$
\mathbf{P}=-\varepsilon_{0} \mathbf{E}=\varepsilon_{0}\left(1-\frac{1}{k}\right) \omega B \mathbf{r} .
$$

Hence the volume bound charge density is

$$
\rho^{\prime}=-\nabla \cdot \mathbf{P}=-\frac{1}{r} \frac{\partial}{\partial r}\left(r P_{r}\right)=-2 \varepsilon_{0}\left(1-\frac{1}{k}\right) \omega B
$$

and the surface bound charge density is

$$
\sigma^{\prime}=P_{r}=\varepsilon_{0}\left(1-\frac{1}{k}\right) \omega B a,
$$

as $r=a$ for the cylinder's surface.

 By symmetry and using Ampère's circuital law, we obtain the magnetic induction in a doughnut-shaped solenoid:

$$
B=\frac{\mu_{0} N I}{2 \pi r},
$$

where $r$ is the distance from the center of the doughnut. Consider a small section of length $d l$ of the solenoid. This section contains $\frac{N}{2 \pi R} d l$ turns of the winding, where $R$ is the radius of the doughnut. Take as current element a segment of this section which subtends an angle $d \theta$ at the axis of the solenoid:

$$
\Delta I=\frac{N I d l}{2 \pi R} \rho d \theta,
$$

where $\theta$ is the angle made by the radius from the axis to the segment and the line from the axis to center of the doughnut and $\rho$ is the radius of a loop of winding. The magnetic force on the current element is in the radial direction and has magnitude

$$
\begin{aligned}
d F &=\Delta I \cdot \frac{B}{2}=\frac{N I \rho}{4 \pi R} B d \theta d l \\
&=\frac{\mu_{0} N^{2} I^{2} \rho}{8 \pi^{2} R r} d \theta d l,
\end{aligned}
$$

where $B / 2$ is used, instead of $B$, because the magnetic field established by the current element itself has to be taken out from the total field. Note that $d F$ is perpendicular to the surface of the solenoid and only its component $d F \cdot \cos \theta$ along the line from the axis to the center of the doughnut is not canceled out with another element at $2 \pi-\theta$. As

$$
r=R+\rho \cos \theta,
$$

we have the total force on the doughnut

$$
\begin{aligned}
F &=\int \cos \theta d F \\
&=\frac{\mu_{0} N^{2} I^{2}}{8 \pi^{2} R} \int_{0}^{2 \pi R} d l \int_{0}^{2 \pi} \frac{\rho \cos \theta}{R+\rho \cos \theta} d \theta \\
&=\frac{\mu_{0} N^{2} I^{2}}{4 \pi} \int_{0}^{2 \pi}\left(1-\frac{R}{R+\rho \cos \theta}\right) d \theta \\
&=\frac{\mu_{0} N^{2} I^{2}}{4 \pi} \int_{0}^{2 \pi}\left[1-\left(1+\frac{\rho}{R} \cos \theta\right)^{-1}\right] d \theta \\
&=\frac{\mu_{0} N^{2} I^{2}}{2}\left\{1-\left[1-\left(\frac{\rho}{R}\right)^{2}\right]^{-\frac{1}{2}}\right\} \\
&=\frac{4 \pi \times 10^{-7} \times 1000^{2} \times 10^{2}}{2}\left[1-\frac{1}{\sqrt{1-0.05^{2}}}\right] \\
&=-0.079 \mathrm{~N}
\end{aligned}
$$

Hence, the force on one loop is

$$
\frac{F}{N}=-\frac{0.079}{1000}=-7.9 \times 10^{-5} \mathrm{~N}
$$

and points to the center of the doughnut.

 The electromagnetic field momentum incident on the mirror per unit time per unit area is $\frac{W}{4 \pi d^{2} c}$, where $W$ is the wattage of the bulb and $d$ is the distance of the mirror from the bulb. Suppose the mirror reflects totally. The change of momentum occurring on the mirror per unit time per unit area is the pressure

$$
p=\frac{2 W}{4 \pi d^{2} c}=\frac{2 \times 70}{4 \pi \times 1^{2} \times 3 \times 10^{8}}=3.7 \times 10^{-8} \mathrm{~N} / \mathrm{m}^{2} .
$$
\textbf{Topic} :Magnetostatic Field and Quasi-Stationary Electromagnetic Fields\\
\textbf{Book} :Problems and Solutions on Electromagnetism\\
\textbf{Final Answer} :37 \times 10^{-8} \mathrm{~N} / \mathrm{m}^{2}\\


\textbf{Solution} :The constitutive equation for electric fields in a dielectric medium moving with velocity $v$ in a magnetic field $B$ is

$$
\mathbf{D}=k \varepsilon_{0} \mathbf{E}+\varepsilon_{0}(k-1) \mathbf{v} \times \mathbf{B},
$$

where $k$ is its relative dielectric constant. For a point distance $r$ from the axis of rotation, $\mathbf{v}=\boldsymbol{\omega} \times \mathbf{r}$ and $\mathbf{v} \times \mathbf{B}=(\boldsymbol{\omega} \cdot \mathbf{B}) \mathbf{r}-(\mathbf{r} \cdot \mathbf{B}) \boldsymbol{\omega}=\omega B \mathbf{r}$ as $\mathbf{r}$ is perpendicular to $B$. As there are no free charges, Gauss' flux theorem $\oint \mathbf{D} \cdot d \mathbf{S}=0$ gives $\mathbf{D}=0$. Then from $\mathbf{D}=\varepsilon_{0} \mathbf{L}+\mathbf{P}$ we get

$$
\mathbf{P}=-\varepsilon_{0} \mathbf{E}=\varepsilon_{0}\left(1-\frac{1}{k}\right) \omega B \mathbf{r} .
$$

Hence the volume bound charge density is

$$
\rho^{\prime}=-\nabla \cdot \mathbf{P}=-\frac{1}{r} \frac{\partial}{\partial r}\left(r P_{r}\right)=-2 \varepsilon_{0}\left(1-\frac{1}{k}\right) \omega B
$$

and the surface bound charge density is

$$
\sigma^{\prime}=P_{r}=\varepsilon_{0}\left(1-\frac{1}{k}\right) \omega B a,
$$

as $r=a$ for the cylinder's surface.

 By symmetry and using Ampère's circuital law, we obtain the magnetic induction in a doughnut-shaped solenoid:

$$
B=\frac{\mu_{0} N I}{2 \pi r},
$$

where $r$ is the distance from the center of the doughnut. Consider a small section of length $d l$ of the solenoid. This section contains $\frac{N}{2 \pi R} d l$ turns of the winding, where $R$ is the radius of the doughnut. Take as current element a segment of this section which subtends an angle $d \theta$ at the axis of the solenoid:

$$
\Delta I=\frac{N I d l}{2 \pi R} \rho d \theta,
$$

where $\theta$ is the angle made by the radius from the axis to the segment and the line from the axis to center of the doughnut and $\rho$ is the radius of a loop of winding. The magnetic force on the current element is in the radial direction and has magnitude

$$
\begin{aligned}
d F &=\Delta I \cdot \frac{B}{2}=\frac{N I \rho}{4 \pi R} B d \theta d l \\
&=\frac{\mu_{0} N^{2} I^{2} \rho}{8 \pi^{2} R r} d \theta d l,
\end{aligned}
$$

where $B / 2$ is used, instead of $B$, because the magnetic field established by the current element itself has to be taken out from the total field. Note that $d F$ is perpendicular to the surface of the solenoid and only its component $d F \cdot \cos \theta$ along the line from the axis to the center of the doughnut is not canceled out with another element at $2 \pi-\theta$. As

$$
r=R+\rho \cos \theta,
$$

we have the total force on the doughnut

$$
\begin{aligned}
F &=\int \cos \theta d F \\
&=\frac{\mu_{0} N^{2} I^{2}}{8 \pi^{2} R} \int_{0}^{2 \pi R} d l \int_{0}^{2 \pi} \frac{\rho \cos \theta}{R+\rho \cos \theta} d \theta \\
&=\frac{\mu_{0} N^{2} I^{2}}{4 \pi} \int_{0}^{2 \pi}\left(1-\frac{R}{R+\rho \cos \theta}\right) d \theta \\
&=\frac{\mu_{0} N^{2} I^{2}}{4 \pi} \int_{0}^{2 \pi}\left[1-\left(1+\frac{\rho}{R} \cos \theta\right)^{-1}\right] d \theta \\
&=\frac{\mu_{0} N^{2} I^{2}}{2}\left\{1-\left[1-\left(\frac{\rho}{R}\right)^{2}\right]^{-\frac{1}{2}}\right\} \\
&=\frac{4 \pi \times 10^{-7} \times 1000^{2} \times 10^{2}}{2}\left[1-\frac{1}{\sqrt{1-0.05^{2}}}\right] \\
&=-0.079 \mathrm{~N}
\end{aligned}
$$

Hence, the force on one loop is

$$
\frac{F}{N}=-\frac{0.079}{1000}=-7.9 \times 10^{-5} \mathrm{~N}
$$

and points to the center of the doughnut.

 The electromagnetic field momentum incident on the mirror per unit time per unit area is $\frac{W}{4 \pi d^{2} c}$, where $W$ is the wattage of the bulb and $d$ is the distance of the mirror from the bulb. Suppose the mirror reflects totally. The change of momentum occurring on the mirror per unit time per unit area is the pressure

$$
p=\frac{2 W}{4 \pi d^{2} c}=\frac{2 \times 70}{4 \pi \times 1^{2} \times 3 \times 10^{8}}=3.7 \times 10^{-8} \mathrm{~N} / \mathrm{m}^{2} .
$$

 Let $\mathbf{E}_{0}$ and $\mathbf{B}_{0}$ be incoming electromagnetic field vectors and let $\mathbf{\Sigma}^{\prime}$ and $\boldsymbol{B}^{\prime}$ be the reflected fields. Applying the boundary relation $\mathbf{n} \times$ $\left(\mathbf{E}_{2}-\mathbf{E}_{1}\right)=0$ to the surface of the conductor we obtain $\mathbb{F}^{\prime}+\mathbf{E}_{0}=0$, or $\mathbb{F}^{\prime}=-\mathbb{E}_{0}$, since $\mathbf{E}_{0}$ and $\mathcal{\Sigma}^{\prime}$ are both tangential to the boundary. For a plane electromagnetic wave we have

$$
\mathbf{B}^{\prime}=\frac{1}{\omega} \mathbf{k}^{\prime} \times \mathbf{E}^{\prime}=\frac{1}{\omega}\left(-\mathbf{k}_{0}\right) \times\left(-\mathbf{E}_{0}\right)=\mathbf{B}_{0} .
$$

For the conductor the surface charge density is $\sigma=0$ and the surface current density is

$$
\begin{aligned}
i &=n \times\left(\mathbf{H}^{\prime}+\mathbf{H}_{0}\right)=2 \mathbf{n} \times \mathbf{H}_{0}=-2\left(k_{0} \times \mathbf{H}\right) / k_{0} \\
&=2 \varepsilon_{0} \omega_{0} \mathbf{E}_{0} / k_{0}=2 \varepsilon_{0} c \mathbf{\mathbf { E } _ { 0 }} .
\end{aligned}
$$

steps:
\textbf{Topic} :Magnetostatic Field and Quasi-Stationary Electromagnetic Fields\\
\textbf{Book} :Problems and Solutions on Electromagnetism\\
\textbf{Final Answer} :2 \varepsilon_{0} c \mathbf{\mathbf { E } _ { 0 }}\\


\textbf{Solution} :The constitutive equation for electric fields in a dielectric medium moving with velocity $v$ in a magnetic field $B$ is

$$
\mathbf{D}=k \varepsilon_{0} \mathbf{E}+\varepsilon_{0}(k-1) \mathbf{v} \times \mathbf{B},
$$

where $k$ is its relative dielectric constant. For a point distance $r$ from the axis of rotation, $\mathbf{v}=\boldsymbol{\omega} \times \mathbf{r}$ and $\mathbf{v} \times \mathbf{B}=(\boldsymbol{\omega} \cdot \mathbf{B}) \mathbf{r}-(\mathbf{r} \cdot \mathbf{B}) \boldsymbol{\omega}=\omega B \mathbf{r}$ as $\mathbf{r}$ is perpendicular to $B$. As there are no free charges, Gauss' flux theorem $\oint \mathbf{D} \cdot d \mathbf{S}=0$ gives $\mathbf{D}=0$. Then from $\mathbf{D}=\varepsilon_{0} \mathbf{L}+\mathbf{P}$ we get

$$
\mathbf{P}=-\varepsilon_{0} \mathbf{E}=\varepsilon_{0}\left(1-\frac{1}{k}\right) \omega B \mathbf{r} .
$$

Hence the volume bound charge density is

$$
\rho^{\prime}=-\nabla \cdot \mathbf{P}=-\frac{1}{r} \frac{\partial}{\partial r}\left(r P_{r}\right)=-2 \varepsilon_{0}\left(1-\frac{1}{k}\right) \omega B
$$

and the surface bound charge density is

$$
\sigma^{\prime}=P_{r}=\varepsilon_{0}\left(1-\frac{1}{k}\right) \omega B a,
$$

as $r=a$ for the cylinder's surface.

 By symmetry and using Ampère's circuital law, we obtain the magnetic induction in a doughnut-shaped solenoid:

$$
B=\frac{\mu_{0} N I}{2 \pi r},
$$

where $r$ is the distance from the center of the doughnut. Consider a small section of length $d l$ of the solenoid. This section contains $\frac{N}{2 \pi R} d l$ turns of the winding, where $R$ is the radius of the doughnut. Take as current element a segment of this section which subtends an angle $d \theta$ at the axis of the solenoid:

$$
\Delta I=\frac{N I d l}{2 \pi R} \rho d \theta,
$$

where $\theta$ is the angle made by the radius from the axis to the segment and the line from the axis to center of the doughnut and $\rho$ is the radius of a loop of winding. The magnetic force on the current element is in the radial direction and has magnitude

$$
\begin{aligned}
d F &=\Delta I \cdot \frac{B}{2}=\frac{N I \rho}{4 \pi R} B d \theta d l \\
&=\frac{\mu_{0} N^{2} I^{2} \rho}{8 \pi^{2} R r} d \theta d l,
\end{aligned}
$$

where $B / 2$ is used, instead of $B$, because the magnetic field established by the current element itself has to be taken out from the total field. Note that $d F$ is perpendicular to the surface of the solenoid and only its component $d F \cdot \cos \theta$ along the line from the axis to the center of the doughnut is not canceled out with another element at $2 \pi-\theta$. As

$$
r=R+\rho \cos \theta,
$$

we have the total force on the doughnut

$$
\begin{aligned}
F &=\int \cos \theta d F \\
&=\frac{\mu_{0} N^{2} I^{2}}{8 \pi^{2} R} \int_{0}^{2 \pi R} d l \int_{0}^{2 \pi} \frac{\rho \cos \theta}{R+\rho \cos \theta} d \theta \\
&=\frac{\mu_{0} N^{2} I^{2}}{4 \pi} \int_{0}^{2 \pi}\left(1-\frac{R}{R+\rho \cos \theta}\right) d \theta \\
&=\frac{\mu_{0} N^{2} I^{2}}{4 \pi} \int_{0}^{2 \pi}\left[1-\left(1+\frac{\rho}{R} \cos \theta\right)^{-1}\right] d \theta \\
&=\frac{\mu_{0} N^{2} I^{2}}{2}\left\{1-\left[1-\left(\frac{\rho}{R}\right)^{2}\right]^{-\frac{1}{2}}\right\} \\
&=\frac{4 \pi \times 10^{-7} \times 1000^{2} \times 10^{2}}{2}\left[1-\frac{1}{\sqrt{1-0.05^{2}}}\right] \\
&=-0.079 \mathrm{~N}
\end{aligned}
$$

Hence, the force on one loop is

$$
\frac{F}{N}=-\frac{0.079}{1000}=-7.9 \times 10^{-5} \mathrm{~N}
$$

and points to the center of the doughnut.

 The electromagnetic field momentum incident on the mirror per unit time per unit area is $\frac{W}{4 \pi d^{2} c}$, where $W$ is the wattage of the bulb and $d$ is the distance of the mirror from the bulb. Suppose the mirror reflects totally. The change of momentum occurring on the mirror per unit time per unit area is the pressure

$$
p=\frac{2 W}{4 \pi d^{2} c}=\frac{2 \times 70}{4 \pi \times 1^{2} \times 3 \times 10^{8}}=3.7 \times 10^{-8} \mathrm{~N} / \mathrm{m}^{2} .
$$

 Let $\mathbf{E}_{0}$ and $\mathbf{B}_{0}$ be incoming electromagnetic field vectors and let $\mathbf{\Sigma}^{\prime}$ and $\boldsymbol{B}^{\prime}$ be the reflected fields. Applying the boundary relation $\mathbf{n} \times$ $\left(\mathbf{E}_{2}-\mathbf{E}_{1}\right)=0$ to the surface of the conductor we obtain $\mathbb{F}^{\prime}+\mathbf{E}_{0}=0$, or $\mathbb{F}^{\prime}=-\mathbb{E}_{0}$, since $\mathbf{E}_{0}$ and $\mathcal{\Sigma}^{\prime}$ are both tangential to the boundary. For a plane electromagnetic wave we have

$$
\mathbf{B}^{\prime}=\frac{1}{\omega} \mathbf{k}^{\prime} \times \mathbf{E}^{\prime}=\frac{1}{\omega}\left(-\mathbf{k}_{0}\right) \times\left(-\mathbf{E}_{0}\right)=\mathbf{B}_{0} .
$$

For the conductor the surface charge density is $\sigma=0$ and the surface current density is

$$
\begin{aligned}
i &=n \times\left(\mathbf{H}^{\prime}+\mathbf{H}_{0}\right)=2 \mathbf{n} \times \mathbf{H}_{0}=-2\left(k_{0} \times \mathbf{H}\right) / k_{0} \\
&=2 \varepsilon_{0} \omega_{0} \mathbf{E}_{0} / k_{0}=2 \varepsilon_{0} c \mathbf{\mathbf { E } _ { 0 }} .
\end{aligned}
$$

steps:

 The work done by the external force can be considered in three

1. Point charge $q$ is brought from infinity to a distance $d$ from the conducting plane. When the distance between $q$ and the conducting plane is $z$, the (attractive) force on $q$ is given by the method of images to be

$$
F=-\frac{q^{2}}{4 \pi \varepsilon_{0}(2 z)^{2}} .
$$

In this step the external force does work

$$
W_{1}=-\int_{\infty}^{d} F d z=-\frac{q^{2}}{16 \pi \varepsilon_{0} d} .
$$

Note that the first minus sign applies because $F$ and $d z$ are in opposite directions.

2. Point charge $-q$ is brought from infinity to a distance $d$ from the conducting plane, but far away from charge $q$. The work done in this process by the external force is exactly the same as in step 1:

$$
W_{2}=W_{1}=-\frac{q^{2}}{16 \pi \varepsilon_{0} d} \text {. }
$$

3. The charge $-q$ is moved to a distance $r$ from $q$ keeping its distance from the conducting plane constant at $d$. When the charge $-q$ is at distance $x$ from $q$, the horizontal component of the (attractive) force on $-q$ is given by the method of images to be

$$
F=-\frac{q^{2}}{4 \pi \varepsilon_{0} x^{2}}+\frac{q^{2} x}{4 \pi \varepsilon_{0}\left(x^{2}+4 d^{2}\right)^{3 / 2}} .
$$

In this step the work done by the external force is

$$
\begin{aligned}
W_{3} &=-\int_{\infty}^{r} F d x=\int_{\infty}^{r} \frac{q^{2}}{4 \pi \varepsilon_{0} x^{2}} d x-\int_{\infty}^{r} \frac{q^{2} x}{4 \pi \varepsilon_{0}\left(x^{2}+4 d^{2}\right)^{3 / 2}} d x \\
&=-\frac{q}{4 \pi \varepsilon_{0} r}+\frac{q^{2}}{4 \pi \varepsilon_{0}\left(r^{2}+4 d^{2}\right)^{1 / 2}} .
\end{aligned}
$$

Hence the total work done by the external force is

$$
\begin{aligned}
W &=W_{1}+W_{2}+W_{3} \\
&=-\frac{q^{2}}{4 \pi \varepsilon_{0}}\left[\frac{1}{r}+\frac{1}{2 d}-\frac{1}{\left(r^{2}+4 d^{2}\right)^{1 / 2}}\right] .
\end{aligned}
$$

We can also solve the problem by considering the electrostatic energy of the system. The potential at the position of $q$ is

$$
\varphi_{1}=\frac{q}{4 \pi \varepsilon_{0}}\left(-\frac{1}{r}-\frac{1}{2 d}+\frac{1}{\sqrt{r^{2}+4 d^{2}}}\right)
$$

and that at $-q$ is

$$
\varphi_{2}=\frac{q}{4 \pi \varepsilon_{0}}\left(\frac{1}{r}+\frac{1}{2 d}-\frac{1}{\sqrt{r^{2}+4 d^{2}}}\right)
$$

again using the method of images. The electrostatic energy of the system is given by $W_{e}=\frac{1}{2} \Sigma q \varphi$. Taking the potential on the conducting surface to be zero, we find the work done by the external force to be

$$
W=W_{e}=-\frac{q^{2}}{4 \pi \varepsilon_{0}}\left(\frac{1}{r}+\frac{1}{2 d}-\frac{1}{\sqrt{r^{2}+4 d^{2}}}\right) .
$$

\textbf{Topic} :Magnetostatic Field and Quasi-Stationary Electromagnetic Fields\\
\textbf{Book} :Problems and Solutions on Electromagnetism\\
\textbf{Final Answer} :-\frac{q^{2}}{4 \pi \varepsilon_{0}}\left(\frac{1}{r}+\frac{1}{2 d}-\frac{1}{\sqrt{r^{2}+4 d^{2}}}\right)\\


\textbf{Solution} :If a plasma is stationary and its displacement current can be neglected, the electromagnetic field inside the plasma satisfies the following 

Maxwell's equations (on Gaussian units)

$$
\begin{array}{ll}
\nabla \cdot \mathbf{D}=4 \pi \rho_{f}, & \nabla \times \mathbf{E}=-\frac{1}{c} \frac{\partial \mathbf{B}}{\partial t}, \\
\nabla \cdot \mathbf{B}=0, & \nabla \times \mathbf{B}=\frac{4 \pi}{c} \mathbf{j}_{f},
\end{array}
$$

and if the plasma is ohmic we have also

$$
\mathbf{j}_{f}=\sigma \mathbf{E} .
$$

Thus

$$
\begin{aligned}
\nabla \times \mathbf{B} &=\frac{4 \pi}{c} \sigma \mathbf{E} \\
\nabla \times(\nabla \times \mathbf{B})=\nabla(\nabla \cdot \mathbf{B})-\nabla^{2} \mathbf{B} &=-\nabla^{2} \mathbf{B}=\frac{4 \pi}{c} \sigma \nabla \times \mathbf{E}=-\frac{4 \pi \sigma}{c^{2}} \frac{\partial \mathbf{B}}{\partial t}
\end{aligned}
$$

or

$$
\frac{\partial \mathbf{B}}{\partial t}=D \nabla^{2} \mathbf{B}
$$

with $D=\frac{c^{2}}{4 \pi \sigma}$. The equation (1) is a diffusion equation.

 If the velocity of the plasma is not zero, we have

$$
\mathbf{j}_{f}=\sigma\left(\mathbf{E}+\frac{1}{c} \mathbf{v} \times \mathbf{B}\right) .
$$

In the nonrelativistic approximation $v \ll c$, use of the above Maxwell's equation gives

$$
\nabla \times \mathbf{B}=\frac{4 \pi \sigma}{c}\left(\mathbf{E}+\frac{1}{c} \mathbf{v} \times \mathbf{B}\right)
$$

Taking curl of both sides gives

$$
\frac{\partial \mathbf{B}}{\partial t}=D \nabla^{2} \mathbf{B}+\nabla \times(\mathbf{v} \times \mathbf{B}) .
$$

 For a stationary plasma the magnetic field is determined by (1). From the initial condition we see that (1) can be reduced to the one-dimensional diffusion equation

$$
\frac{\partial B_{z}(x, t)}{\partial t}=D \frac{\partial^{2} B_{z}(x, t)}{\partial x^{2}}
$$

We separate the variables by letting $B_{z}(x, t)=X(x) T(t)$ and obtain

$$
\frac{1}{D T} \frac{d T}{d t}=\frac{1}{X} \frac{d^{2} X}{d x^{2}}=-\omega^{2}
$$

with solutions

$$
T(t)=A e^{-\omega^{2} D t}, \quad X(x)=C e^{i \omega x} .
$$

Hence

$$
B_{z}(x, t, \omega)=A(\omega) e^{-\omega^{2} D t} e^{i \omega x} .
$$

As $\omega$ is arbitrary the general solution is

$$
B_{z}(x, t)=\int_{-\infty}^{\infty} A(\omega) e^{-\omega^{2} D t} e^{i \omega x} d \omega .
$$

For $t=0$, the above reduces to

$$
B_{z}(x)=\int_{-\infty}^{\infty} A(\omega) e^{i \omega x} d \omega,
$$

and by Fourier transform we obtain

$$
A(\omega)=\frac{1}{2 \pi} \int_{-\infty}^{\infty} B_{z}(\xi) e^{-i \omega \xi} d \xi .
$$

Hence

$$
B_{z}(x, t)=\int_{-\infty}^{\infty} B_{z}(\xi)\left[\frac{1}{2 \pi} \int_{-\infty}^{\infty} e^{-\omega^{2} D t} e^{i \omega(x-\xi)} d \omega\right] d \xi,
$$

where the definite integral inside the brackets can be evaluated,

$$
\int_{-\infty}^{\infty} e^{-\omega^{2} D t} e^{i \omega(x-\xi)} d \omega=\sqrt{\frac{\pi}{D t}} e^{-(x-\xi)^{2} / 4 D t},
$$

and $B(\xi)$ is given by the initial condition

$$
B_{z}(\xi)=\left\{\begin{array}{lll}
B_{0}, & \text { for } & |\xi| \leq L \\
0, & \text { for } & |\xi|>L .
\end{array}\right.
$$

Therefore, the time evolution of the field is given by

$$
B_{z}(x, t)=\frac{B_{0}}{\sqrt{4 \pi D t}} \int_{-\frac{k}{2}}^{\frac{t}{2}} e^{-\frac{(z-\xi)^{2}}{4 D}} d \xi
$$
\textbf{Topic} :Magnetostatic Field and Quasi-Stationary Electromagnetic Fields\\
\textbf{Book} :Problems and Solutions on Electromagnetism\\
\textbf{Final Answer} :\frac{B_{0}}{\sqrt{4 \pi D t}} \int_{-\frac{k}{2}}^{\frac{t}{2}} e^{-\frac{(z-\xi)^{2}}{4 D}} d \xi\\


\textbf{Solution} :In the rest frame $\Sigma^{\prime}$ of the electron, the electromagnetic field at a point of radius vector $\mathbf{r}^{\prime}$ from it is, in Gaussian units,

$$
\mathbf{E}^{\prime}=\frac{e \mathbf{r}^{\prime}}{r^{\prime 3}}, \quad \mathbf{B}^{\prime}=0 .
$$

In the laboratory frame $\Sigma$, by Lorentz transformation (with $v \ll c$ ) the field is

$$
\begin{aligned}
&\mathbf{E}=\mathbf{E}^{\prime}-\frac{\mathbf{v}}{c} \times \mathbf{B}^{\prime}=\mathbf{E}^{\prime} \\
&\mathbf{B}=\mathbf{B}^{\prime}+\frac{\mathbf{v}}{c} \times \mathbf{E}^{\prime}=\frac{\mathbf{v}}{c} \times \mathbf{E}^{\prime} .
\end{aligned}
$$

The field point has coordinates $(r, \theta)$ in $\Sigma$ as shown in Fig. 2.83. As $v \ll c$, we have $r^{\prime} \approx r$ and

$$
\begin{aligned}
&\mathbf{E}=\mathbf{E}^{\prime}=\frac{e \mathbf{r}^{\prime}}{r^{\prime 3}} \simeq \frac{e \mathbf{r}}{r^{3}}, \\
&\mathbf{B}=\frac{\mathbf{v}}{c} \times \mathbf{E}^{\prime} \simeq \frac{e}{c} \cdot \frac{\mathbf{v} \times \mathbf{r}}{r^{3}},
\end{aligned}
$$

with magnitudes

$$
E=\frac{e}{r^{2}}, \quad B=\frac{e v \sin \theta}{c r^{2}} .
$$

 The momentum density of the field is

$$
g=\frac{N}{c^{2}}=\frac{1}{4 \pi c}(\mathbf{E} \times \mathbf{B}) .
$$

Substituting in $\mathbf{E}$ and $B$ yields

$$
g=\frac{e^{2}}{4 \pi c^{2}} \frac{\mathbf{r} \times(\mathbf{v} \times \mathbf{r})}{r^{6}}=\frac{e^{2}}{4 \pi c^{2} r^{5}}(\mathbf{v} r-v \mathbf{r} \cos \theta) .
$$

Hence the momentum of the electromagnetic field of the electron is

$$
\begin{aligned}
\mathbf{P}=\iiint_{\infty} g d V &=\mathbf{e}_{z} \int_{0}^{2 \pi} d \varphi \int_{0}^{\pi} d \theta \int_{a}^{\infty} d r \frac{e^{2} v\left(1-\cos ^{2} \theta\right)}{4 \pi c^{2} r^{4}} r^{2} \sin \theta \\
&=\frac{2 e^{2} v}{3 c^{2} a} \mathbf{e}_{z}=\frac{2 e^{2}}{3 c^{2} a} \mathbf{V}
\end{aligned}
$$

Note that in the integrand above the component of $g$ perpendicular to $\mathbf{V}$ will cancel out on integration; only the component parallel to $\mathbf{v}$ needs to be considered.

If the electromagnetic field momentum of the electron is equal to its mechanical momentum, $m \mathbf{v}$, i.e., $\frac{2 e^{2}}{3 c^{2} a} \mathbf{v}=m \mathbf{v}$, then

$$
a=\frac{2 e^{2}}{3 m c^{2}}=\frac{2}{3} \times 2.82 \times 10^{-5} \AA=1.88 \times 10^{-5} \AA
$$
\textbf{Topic} :Magnetostatic Field and Quasi-Stationary Electromagnetic Fields\\
\textbf{Book} :Problems and Solutions on Electromagnetism\\
\textbf{Final Answer} :188 \times 10^{-5} \AA\\


\textbf{Solution} :In the rest frame $\Sigma^{\prime}$ of the electron, the electromagnetic field at a point of radius vector $\mathbf{r}^{\prime}$ from it is, in Gaussian units,

$$
\mathbf{E}^{\prime}=\frac{e \mathbf{r}^{\prime}}{r^{\prime 3}}, \quad \mathbf{B}^{\prime}=0 .
$$

In the laboratory frame $\Sigma$, by Lorentz transformation (with $v \ll c$ ) the field is

$$
\begin{aligned}
&\mathbf{E}=\mathbf{E}^{\prime}-\frac{\mathbf{v}}{c} \times \mathbf{B}^{\prime}=\mathbf{E}^{\prime} \\
&\mathbf{B}=\mathbf{B}^{\prime}+\frac{\mathbf{v}}{c} \times \mathbf{E}^{\prime}=\frac{\mathbf{v}}{c} \times \mathbf{E}^{\prime} .
\end{aligned}
$$

The field point has coordinates $(r, \theta)$ in $\Sigma$ as shown in Fig. 2.83. As $v \ll c$, we have $r^{\prime} \approx r$ and

$$
\begin{aligned}
&\mathbf{E}=\mathbf{E}^{\prime}=\frac{e \mathbf{r}^{\prime}}{r^{\prime 3}} \simeq \frac{e \mathbf{r}}{r^{3}}, \\
&\mathbf{B}=\frac{\mathbf{v}}{c} \times \mathbf{E}^{\prime} \simeq \frac{e}{c} \cdot \frac{\mathbf{v} \times \mathbf{r}}{r^{3}},
\end{aligned}
$$

with magnitudes

$$
E=\frac{e}{r^{2}}, \quad B=\frac{e v \sin \theta}{c r^{2}} .
$$

 The momentum density of the field is

$$
g=\frac{N}{c^{2}}=\frac{1}{4 \pi c}(\mathbf{E} \times \mathbf{B}) .
$$

Substituting in $\mathbf{E}$ and $B$ yields

$$
g=\frac{e^{2}}{4 \pi c^{2}} \frac{\mathbf{r} \times(\mathbf{v} \times \mathbf{r})}{r^{6}}=\frac{e^{2}}{4 \pi c^{2} r^{5}}(\mathbf{v} r-v \mathbf{r} \cos \theta) .
$$

Hence the momentum of the electromagnetic field of the electron is

$$
\begin{aligned}
\mathbf{P}=\iiint_{\infty} g d V &=\mathbf{e}_{z} \int_{0}^{2 \pi} d \varphi \int_{0}^{\pi} d \theta \int_{a}^{\infty} d r \frac{e^{2} v\left(1-\cos ^{2} \theta\right)}{4 \pi c^{2} r^{4}} r^{2} \sin \theta \\
&=\frac{2 e^{2} v}{3 c^{2} a} \mathbf{e}_{z}=\frac{2 e^{2}}{3 c^{2} a} \mathbf{V}
\end{aligned}
$$

Note that in the integrand above the component of $g$ perpendicular to $\mathbf{V}$ will cancel out on integration; only the component parallel to $\mathbf{v}$ needs to be considered.

If the electromagnetic field momentum of the electron is equal to its mechanical momentum, $m \mathbf{v}$, i.e., $\frac{2 e^{2}}{3 c^{2} a} \mathbf{v}=m \mathbf{v}$, then

$$
a=\frac{2 e^{2}}{3 m c^{2}}=\frac{2}{3} \times 2.82 \times 10^{-5} \AA=1.88 \times 10^{-5} \AA
$$

 The energy of the field of the electron is

$$
\begin{aligned}
W &=\iiint_{\infty} \frac{1}{8 \pi}\left(E^{2}+B^{2}\right) d V=\iiint \frac{1}{8 \pi}\left[\frac{e^{2}}{r^{4}}+\frac{e^{2} v^{2} \sin ^{2} \theta}{c^{2} r^{4}}\right] d V \\
&=\frac{e^{2}}{8 \pi} \int_{0}^{2 \pi} d \varphi \int_{0}^{\pi} d \theta \int_{a}^{\infty} r^{2} \sin \theta\left[\frac{1+\frac{v^{2}}{c^{2}} \sin ^{2} \theta}{r^{4}}\right] d r \\
&=\frac{e^{2}}{2 a}\left(1+\frac{2}{3} \frac{v^{2}}{c^{2}}\right)=\frac{3 m c^{2}}{4}\left[1+\frac{2}{3} \frac{v^{2}}{c^{2}}\right]=\frac{3}{4} m c^{2}+\frac{1}{2} m v^{2} .
\end{aligned}
$$

It follows that for $v \ll c, \frac{1}{2} m v^{2} \ll W \lesssim m c^{2}$.

\textbf{Topic} :Magnetostatic Field and Quasi-Stationary Electromagnetic Fields\\
\textbf{Book} :Problems and Solutions on Electromagnetism\\
\textbf{Final Answer} :\frac{3}{4} m c^{2}+\frac{1}{2} m v^{2}\\


\textbf{Solution} :Integrating (3) we have

$$
\begin{aligned}
&v_{z}(t)=v_{z}(t=0)=0, \\
&z(t)=z(t=0)=\text { const. }
\end{aligned}
$$

Hence the coordinate and speed of the electron in the $z$ direction are the constants of the motion, in particular $v_{z}=0$.

 The work done by the electric field in moving an electron from cathode to anode is

$$
W=\int_{0}^{d}-e \cdot\left(-\frac{\partial \phi}{\partial x}\right) d x=e V_{0}
$$

since the magnetic field does no work. When the electron reaches the anode, the magnitude of its velocity $\mathbf{v}_{f}=v_{f x} \mathbf{i}+v_{f y} \mathbf{j}$ can be obtained by equating the kinetic energy of the electron to the work done by the electric field:

$$
\frac{1}{2} m v_{f}^{2}=e V_{0},
$$

giving

$$
v_{f}=\sqrt{\frac{2 e V_{0}}{m}} \text {. }
$$

If the electrons are not to reach the anode, we require that

$$
v_{f x}=0, \quad v_{f y}=\sqrt{\frac{2 e V_{0}}{m}} .
$$

Writing (2) as

$$
m \frac{d v_{y}}{d t}=e B_{0} \frac{d x}{d t}
$$

and integrating both sides, noting $v_{y}=0, x=0$, at $t=0$ we obtain

$$
m \sqrt{\frac{2 e V_{0}}{m}}=e B_{0} d
$$

giving

$$
B_{0}=\sqrt{\frac{2 m V_{0}}{e d^{2}}} .
$$

Therefore the induction of the magnetic field must be greater than $\sqrt{\frac{2 m V_{0}}{e d^{2}}}$ for the electrons to reflect back before reaching the anode. 

\textbf{Topic} :Magnetostatic Field and Quasi-Stationary Electromagnetic Fields\\
\textbf{Book} :Problems and Solutions on Electromagnetism\\
\textbf{Final Answer} :\sqrt{\frac{2 m V_{0}}{e d^{2}}}\\


\textbf{Solution} :As shown in Fig. 2.84, at a point $P$ just inside the surface of the sphere $(r \approx R)$, the magnetic field generated by the dipole $m$ is

$$
\begin{aligned}
B &=B_{r} e_{r}+B_{0} e_{\theta} \\
&=\frac{\mu_{0}}{4 \pi}\left(\frac{m \cos \theta}{R^{3}} e_{r}+\frac{m \sin \theta}{R^{3}} e_{0}\right) .
\end{aligned}
$$

So the electric field at the point $P$ is

$$
\begin{aligned}
\boldsymbol{\Sigma} &=-\boldsymbol{v} \times \mathbf{B}=-(\omega \times \mathbf{r}) \times\left.\mathbf{B}\right|_{\sigma \approx R} \\
&=\frac{\mu_{0}}{4 \pi}\left(\frac{m \omega \sin ^{2} \theta}{R^{2}} \mathbf{e}_{r}-\frac{2 m \omega \cos \theta \sin \theta}{R^{2}} \mathbf{e}_{0}\right),
\end{aligned}
$$

with the $\theta$-component

$$
E_{\theta}=-\frac{\mu_{0} m \omega \sin \theta \cos \theta}{2 \pi R^{2}}
$$



MATHPIX IMAGE

Fig. $2.84$

 Taking the potential on the equator as reference level the induced potential at a point of latitude $\alpha=\frac{\pi}{2}-\theta$ is

$$
\begin{aligned}
V &=-\int_{\frac{\pi}{2}}^{\frac{\pi}{2}-\alpha} E_{\theta} R d \theta \\
&=\int_{\frac{\pi}{2}}^{\frac{\pi}{2}-\alpha} \frac{\mu_{0} m \omega \sin \theta d(\sin \theta)}{2 \pi R}=\left.\frac{\mu_{0} m \omega \sin ^{2} \theta}{4 \pi R}\right|_{\frac{\pi}{2}} ^{\frac{\pi}{2}-\alpha} \\
&=\frac{\mu_{0} m \omega}{4 \pi R}\left(\cos ^{2} \alpha-1\right)=\frac{B_{P} \omega R^{2}}{2}\left(\cos ^{2} \alpha-1\right),
\end{aligned}
$$

where $B_{P}=\frac{\mu_{0} m}{2 \pi R^{3}}$ is the induction of the magnetic dipole at the north pole.
\textbf{Topic} :Magnetostatic Field and Quasi-Stationary Electromagnetic Fields\\
\textbf{Book} :Problems and Solutions on Electromagnetism\\
\textbf{Final Answer} :\frac{B_{P} \omega R^{2}}{2}\left(\cos ^{2} \alpha-1\right)\\


\textbf{Solution} :The circuit in Fig. $3.1$ can be redrawn as that in Fig. 3.2.

MATHPIX IMAGE

Fig. $3.2$

Let the currents flowing in the component circuits be as shown. By Kirchhoff's laws we have

$$
\begin{aligned}
V_{3} &=\left[2\left(i_{3}-i_{2}\right)+2\left(i_{3}-i_{4}\right)\right] R, \\
V_{2}-V_{3} &=\left[2\left(i_{2}-i_{1}\right)+\left(i_{2}-i_{4}\right)+2\left(i_{2}-i_{3}\right)\right] R,
\end{aligned}
$$



$$
\begin{aligned}
V_{1}-V_{2} &=\left[2 i_{1}+\left(i_{1}-i_{4}\right)+2\left(i_{1}-i_{2}\right)\right] R, \\
0 &=2\left(i_{4}-i_{3}\right)+\left(i_{4}-i_{2}\right)+\left(i_{4}-i_{1}\right)+2 i_{4} .
\end{aligned}
$$

After solving for $i_{4}$ we obtain

$$
V_{\text {out }}=2 i_{4} R=\frac{V_{1}}{3}+\frac{V_{2}}{6}+\frac{V_{3}}{12} \text {. }
$$

$V_{\text {out }}$ for various values of $V_{1}, V_{2}$, and $V_{3}$ are shown in the table below.

\begin{tabular}{|c|c|c|c||c|c|c|c|}
\hline$V_{1}$ & $V_{2}$ & $V_{3}$ & $V_{\text {out }}$ & $V_{1}$ & $V_{2}$ & $V_{3}$ & $V_{\text {out }}$ \\
\hline 0 & 0 & 0 & 0 & 1 & 0 & 0 & $\frac{1}{3}$ \\
\hline 0 & 0 & 1 & $\frac{1}{12}$ & 1 & 0 & 1 & $\frac{5}{12}$ \\
\hline 0 & 1 & 0 & $\frac{1}{6}$ & 1 & 1 & 0 & $\frac{1}{2}$ \\
\hline 0 & 1 & 1 & $\frac{1}{4}$ & 1 & 1 & 1 & $\frac{7}{12}$ \\
\hline
\end{tabular}

\textbf{Topic} :Circuit Analysis\\
\textbf{Book} :Problems and Solutions on Electromagnetism\\
\textbf{Final Answer} :\frac{V_{1}}{3}+\frac{V_{2}}{6}+\frac{V_{3}}{12}\\


\textbf{Solution} :According to Thévenin's theorem, the equivalent emf is the potential across $\mathrm{AB}$ when the output current is zero, i.e., the open-circuit voltage:

$$
\varepsilon_{0}=V_{\mathrm{AB}}=\frac{6}{24+6} \times 15=3 \mathrm{~V}
$$

The equivalent internal resistance is the resistance when the battery is shorted, i.e., the parallel combination of the resistances:

$$
r=\frac{6 \times 24}{6+24}=4.8 \Omega .
$$

Then the short-circuit current provided by the battery is

$$
I=\frac{\varepsilon_{0}}{r}=\frac{3 \mathrm{~V}}{4.8 \Omega}=0.625 \mathrm{~A}
$$

\textbf{Topic} :Circuit Analysis\\
\textbf{Book} :Problems and Solutions on Electromagnetism\\
\textbf{Final Answer} :0625 \mathrm{~A}\\


\textbf{Solution} :The capacitor has capacitance $C=\frac{\varepsilon S}{d}$. As $\varepsilon=\varepsilon_{0}$ for air, $C=$
\textbf{Topic} :Circuit Analysis\\
\textbf{Book} :Problems and Solutions on Electromagnetism\\
\textbf{Final Answer} :0625 \mathrm{~A}\\


\textbf{Solution} :Let the total resistance of the infinite network be $R$. After removing the resistances of the first section, the remaining circuit is still an infinite network which is equivalent to the original one. Its equivalent circuit is shown in Fig. $3.13$ and has total resistance

$$
R=R_{1}+\frac{R R_{2}}{R+R_{2}} .
$$

This gives a quadratic equation in $R$

$$
R^{2}-R_{1} R-R_{1} R_{2}=0 \text {. }
$$

The positive root

$$
R=\frac{R_{1}}{2}+\frac{\sqrt{R_{1}^{2}+4 R_{1} R_{2}}}{2}
$$

gives the equivalent resistance.
\\
\textbf{Topic} :Circuit Analysis\\
\textbf{Book} :Problems and Solutions on Electromagnetism\\
\textbf{Final Answer} :\frac{R_{1}}{2}+\frac{\sqrt{R_{1}^{2}+4 R_{1} R_{2}}}{2}\\


\textbf{Solution} :Let the total resistance of the infinite network be $R$. After removing the resistances of the first section, the remaining circuit is still an infinite network which is equivalent to the original one. Its equivalent circuit is shown in Fig. $3.13$ and has total resistance

$$
R=R_{1}+\frac{R R_{2}}{R+R_{2}} .
$$

This gives a quadratic equation in $R$

$$
R^{2}-R_{1} R-R_{1} R_{2}=0 \text {. }
$$

The positive root

$$
R=\frac{R_{1}}{2}+\frac{\sqrt{R_{1}^{2}+4 R_{1} R_{2}}}{2}
$$

gives the equivalent resistance.
 As $I_{0}=I_{1}+I_{2}$, the power dissipation is

$$
P=I_{1}^{2} R_{1}+I_{2}^{2} R_{2}=I_{1}^{2} R_{1}+\left(I_{0}-I_{1}\right)^{2} R_{2} \text {. }
$$

MATHPIX IMAGE

Fig. 3.13

To minimize, put $\frac{d P}{d I_{1}}=0$, which gives $2 I_{1} R_{1}-2\left(I_{0}-I_{1}\right) R_{2}=0$, or

This is the formula one usually uses.

$$
\frac{I_{1}}{I_{2}}=\frac{I_{1}}{I_{0}-I_{1}}=\frac{R_{2}}{R_{1}} .
$$

\textbf{Topic} :Circuit Analysis\\
\textbf{Book} :Problems and Solutions on Electromagnetism\\
\textbf{Final Answer} :\frac{R_{2}}{R_{1}}\\


\textbf{Solution} :The voltage across the two ends of the capacitors in series is

$$
V=\left|\frac{(1.5 \| 1)}{1.4+(1.5 \| 1)} \cdot 4-2\right|=0.8 \mathrm{~V}
$$



MATHPIX IMAGE

Fig. $3.17$

The voltage across the two ends of the $3 \mu \mathrm{F}$ capacitor is $\frac{6}{3+6} \times 0.8=$ $0.53 \mathrm{~V}$. So the energy stored in the $3 \mu \mathrm{F}$ capacitor is

$$
E=\frac{1}{2} \times 3 \times 10^{-6} \times 0.53^{2}=0.42 \times 10^{-6} \mathrm{~J} .
$$

\textbf{Topic} :Circuit Analysis\\
\textbf{Book} :Problems and Solutions on Electromagnetism\\
\textbf{Final Answer} :042 \times 10^{-6} \mathrm{~J}\\


\textbf{Solution} :If $q(0)=0$, then $d=-\frac{B}{A}$, and

$$
q=\frac{B}{A}\left(1-e^{-A t}\right)=\frac{\varepsilon R_{3}}{R_{1}+R_{2}}\left(1-\exp \left\{-\frac{R_{1}+R_{2}}{\left(R_{1} R_{2}+R_{2} R_{3}+R_{3} R_{1}\right) C} t\right\}\right)
$$
\textbf{Topic} :Circuit Analysis\\
\textbf{Book} :Problems and Solutions on Electromagnetism\\
\textbf{Final Answer} :\frac{\varepsilon R_{3}}{R_{1}+R_{2}}\left(1-\exp \left\{-\frac{R_{1}+R_{2}}{\left(R_{1} R_{2}+R_{2} R_{3}+R_{3} R_{1}\right) C} t\right\}\right)\\


\textbf{Solution} :Consider a resistance $R$ and an inductance $L$ in series with a battery of emf $\varepsilon$. We have

$$
\varepsilon-L \frac{d I}{d t}=I R
$$

or

$$
\frac{-R d I}{\varepsilon-I R}=-R \frac{d t}{L}
$$

Integrating we have

$$
\ln [\varepsilon-I(t) R]=-\frac{t}{\tau}+K,
$$

where $\tau=\frac{L}{R}$ and $K$ is a constant. Let $I=I(0)$ at $t=0$ and $I=I(\infty)$ for $t \rightarrow \infty$. Then

$$
K=\ln [\varepsilon-I(0) R], \quad I(\infty)=\frac{\varepsilon}{R},
$$

and the solution can be written as

$$
I(t)=I(\infty)+[I(0)-I(\infty)] e^{-\frac{1}{r}} .
$$

Now consider the circuit in Fig. 3.23.

(1) When the switch is just closed, we have

$$
I_{R_{2}}(0)=\frac{V}{R_{1}+R_{2}}=0.91 \mathrm{~A} .
$$

After it remains closed for a long time, we have

$$
I_{R_{2}}(\infty)=0,
$$

since in the steady state the entire current will pass through $L$ which has negligible resistance.

As the time constant of the circuit is

$$
\tau=\frac{L}{R_{1} \| R_{2}}=1.1 \mathrm{~s},
$$

we have

$$
\begin{aligned}
I_{R_{2}}(t) &=I_{R_{2}}(\infty)+\left[I_{R_{2}}(0)-I_{R_{2}}(\infty)\right] e^{-t / \tau} \\
&=0.91 e^{-0.91 t} \mathrm{~A},
\end{aligned}
$$



$$
\begin{aligned}
W_{R_{2}} &=\int_{0}^{\infty} I_{R_{2}}^{2}(t) R_{2} d t=\int_{0}^{\infty} 0.91^{2} e^{-1.82 t} \times 100 d t \\
&=45.5 \mathrm{~J} .
\end{aligned}
$$

(2) When the switch is just opened, we have

$$
I_{L}(0)=\frac{V}{R_{1}}=10 \mathrm{~A} .
$$

The energy stored in the inductance $L$ at this time will be totally dissipated in the resistance $R_{2}$. Thus the heat dissipated in $R_{2}$ is

$$
W_{R_{2}}=\frac{1}{2} L I_{L}^{2}(0)=\frac{1}{2} \times 10 \times 100=500 \mathrm{~J} \text {. }
$$

\textbf{Topic} :Circuit Analysis\\
\textbf{Book} :Problems and Solutions on Electromagnetism\\
\textbf{Final Answer} :500 \mathrm{~J}\\


\textbf{Solution} :$t=\infty$

Assume the inductor has negligible resistance. Then at $t=0$ and

$$
\begin{aligned}
&I_{L}(0)=0, \\
&I_{L}(\infty)=\frac{10}{200}=0.05 \mathrm{~A} .
\end{aligned}
$$

The equivalent resistance as seen from the ends of $L$ is

$$
R=200 \| 200=100 \Omega,
$$

giving the time constant as

$$
\tau=\frac{L}{R}=\frac{10^{-5}}{100}=10^{-7} \mathrm{~s} .
$$

At time $t$, the current passing through $L$ is

$$
\begin{aligned}
I_{L}(t) &=I_{L}(\infty)+\left(I_{L}(0)-I_{L}(\infty)\right) e^{-\frac{t}{r}} \\
&=0.05\left(1-e^{-10^{7} t}\right) \mathrm{A} .
\end{aligned}
$$

\textbf{Topic} :Circuit Analysis\\
\textbf{Book} :Problems and Solutions on Electromagnetism\\
\textbf{Final Answer} :005\left(1-e^{-10^{7} t}\right) \mathrm{A}\\


\textbf{Solution} :Because the current through an inductor cannot be changed suddenly, we still have

$$
i_{L}(0)=\frac{1}{1}=1 \mathrm{~A} .
$$
\textbf{Topic} :Circuit Analysis\\
\textbf{Book} :Problems and Solutions on Electromagnetism\\
\textbf{Final Answer} :1 \mathrm{~A}\\


\textbf{Solution} :Because the current through an inductor cannot be changed suddenly, we still have

$$
i_{L}(0)=\frac{1}{1}=1 \mathrm{~A} .
$$

 As $-L \frac{d i_{L}}{d t}=i_{L} R$,

$$
\left.\frac{d i_{L}}{d t}\right|_{t=0}=-i_{L}(0) \frac{R}{L}=1 \times \frac{10^{4}}{1}=-10^{4} \mathrm{~A} / \mathrm{s} .
$$
\textbf{Topic} :Circuit Analysis\\
\textbf{Book} :Problems and Solutions on Electromagnetism\\
\textbf{Final Answer} :-10^{4} \mathrm{~A} / \mathrm{s}\\


\textbf{Solution} :Because the current through an inductor cannot be changed suddenly, we still have

$$
i_{L}(0)=\frac{1}{1}=1 \mathrm{~A} .
$$

 As $-L \frac{d i_{L}}{d t}=i_{L} R$,

$$
\left.\frac{d i_{L}}{d t}\right|_{t=0}=-i_{L}(0) \frac{R}{L}=1 \times \frac{10^{4}}{1}=-10^{4} \mathrm{~A} / \mathrm{s} .
$$

 $v_{\mathrm{B}}(0)=-i_{L}(0) R=-1 \times 10^{4}=-10^{4} \mathrm{~V}$.
\textbf{Topic} :Circuit Analysis\\
\textbf{Book} :Problems and Solutions on Electromagnetism\\
\textbf{Final Answer} :-10^{4} \mathrm{~A} / \mathrm{s}\\


\textbf{Solution} :Because the current through an inductor cannot be changed suddenly, we still have

$$
i_{L}(0)=\frac{1}{1}=1 \mathrm{~A} .
$$

 As $-L \frac{d i_{L}}{d t}=i_{L} R$,

$$
\left.\frac{d i_{L}}{d t}\right|_{t=0}=-i_{L}(0) \frac{R}{L}=1 \times \frac{10^{4}}{1}=-10^{4} \mathrm{~A} / \mathrm{s} .
$$

 $v_{\mathrm{B}}(0)=-i_{L}(0) R=-1 \times 10^{4}=-10^{4} \mathrm{~V}$.

 As $v_{L}=v_{\mathrm{B}}=i_{L} R$,

$$
\begin{aligned}
\left.\frac{d v_{L}}{d t}\right|_{t=0} &=\left.\frac{d i_{L}}{d t}\right|_{t=0} R=-i_{L}(0) \frac{R^{2}}{L} \\
&=-1 \times \frac{\left(10^{4}\right)^{2}}{1}=-10^{8} \mathrm{~V} / \mathrm{s} .
\end{aligned}
$$
\textbf{Topic} :Circuit Analysis\\
\textbf{Book} :Problems and Solutions on Electromagnetism\\
\textbf{Final Answer} :-10^{8} \mathrm{~V} / \mathrm{s}\\


\textbf{Solution} :The impedance of an inductor is $j \omega L$ and the impedance of a capacitor is $\frac{1}{j \omega C}$. For $\omega$ very small, the currents passing through the capacitors may be neglected and the equivalent circuit is as shown in Fig. 3.32.

MATHPIX IMAGE

Fig. $3.32$

We thus have

$$
V_{\mathrm{BA}}=I Z=j \omega L_{1} I,
$$

where $I=i_{0} e^{j \omega t}$. As ac meters usually read the rms values, we have

$$
V_{\text {meter }}=\frac{1}{\sqrt{2}} i_{0} \omega L_{1} .
$$
\textbf{Topic} :Circuit Analysis\\
\textbf{Book} :Problems and Solutions on Electromagnetism\\
\textbf{Final Answer} :\frac{1}{\sqrt{2}} i_{0} \omega L_{1}\\


\textbf{Solution} :The impedance of an inductor is $j \omega L$ and the impedance of a capacitor is $\frac{1}{j \omega C}$. For $\omega$ very small, the currents passing through the capacitors may be neglected and the equivalent circuit is as shown in Fig. 3.32.

MATHPIX IMAGE

Fig. $3.32$

We thus have

$$
V_{\mathrm{BA}}=I Z=j \omega L_{1} I,
$$

where $I=i_{0} e^{j \omega t}$. As ac meters usually read the rms values, we have

$$
V_{\text {meter }}=\frac{1}{\sqrt{2}} i_{0} \omega L_{1} .
$$

 For $\omega$ very large, neglect the currents passing through the inductors and the equivalent circuit is as shown in Fig. 3.33. We have

$$
V_{\mathrm{BA}}=I Z=-\frac{j I}{\omega C_{1}}
$$

and

$$
\begin{aligned}
&V_{\text {meter }}=\frac{i_{0}}{\sqrt{2} \omega C_{1}} . \\
&\leftarrow \frac{A}{i_{0} \sin \omega t} \text { If } c_{1}
\end{aligned}
$$

Fig. $3.33$

 Let $\omega_{1}=\frac{1}{\sqrt{L_{1} C_{1}}}, \omega_{2}=\frac{1}{\sqrt{L_{2} C_{2}}}$. As $L_{2} C_{2}>L_{1} C_{1}, \omega_{1}>\omega_{2}$. The voltmeter reading versus $\omega$ is as shown in Fig. 3.34. The system is net inductive when $\omega$ is in the region $\left(0, \omega_{2}\right)$, and net capacitive when $\omega$ is in the region $\left(\omega_{1}, \infty\right)$. Resonance occurs at the characteristic angular frequencies $\omega_{1}$ and $\omega_{2}$

MATHPIX IMAGE

Fig. $3.34$

 The total impedance $L$ is the combination of two impedances $L_{1}$, $L_{2}$ in parallel:

$$
Z=\frac{Z_{1} Z_{2}}{Z_{1}+Z_{2}},
$$

where

$$
Z_{1}=\frac{-j L_{1}}{C_{1}\left(\omega L_{1}-\frac{1}{\omega C_{1}}\right)}, \quad Z_{2}=j\left(\omega L_{2}-\frac{1}{\omega C_{2}}\right) .
$$

Thus

$$
Z=\frac{j}{\frac{1}{\omega L_{1}}-\omega C_{1}+\frac{1}{\omega L_{2}-\frac{1}{\omega C_{2}}}} .
$$

Hence the voltmeter reading is

$$
V_{\mathrm{BA}}=\frac{i_{0}}{\sqrt{2}\left|\frac{1}{\omega L_{1}}-\omega C_{1}+\frac{1}{\omega L_{2}-\frac{1}{\omega C_{2}}}\right|} .
$$

Note that this reduces to the results in (a) and (b) for $\omega$ very small and very large. 

\textbf{Topic} :Circuit Analysis\\
\textbf{Book} :Problems and Solutions on Electromagnetism\\
\textbf{Final Answer} :\frac{i_{0}}{\sqrt{2}\left|\frac{1}{\omega L_{1}}-\omega C_{1}+\frac{1}{\omega L_{2}-\frac{1}{\omega C_{2}}}\right|}\\


\textbf{Solution} :Let the currents of the primary and secondary circuits be $I_{1}$ and $I_{2}$ respectively. We have

$$
\begin{gathered}
\dot{V}=\dot{I}_{1} R+L_{1} \ddot{I}_{1}+M \ddot{I}_{2}, \\
0=L_{2} \ddot{I}_{2}+M \ddot{I}_{1}+\frac{I_{2}}{C}
\end{gathered}
$$

Solving for $I_{1} \sim \exp (j \omega t)$, we have

$$
\begin{aligned}
I_{1} &=\frac{V}{R+j\left(\omega L_{1}+\frac{\omega^{3} M^{2}}{\frac{1}{c}-\omega^{2} L_{2}}\right)} \\
&=\frac{V_{0} e^{-j \varphi}}{\sqrt{R^{2}+\left(\omega L_{1}+\frac{\omega^{3} M^{2}}{\frac{1}{c}-\omega^{2} L_{2}}\right)^{2}}},
\end{aligned}
$$

where

$$
\varphi=\arctan \left(\frac{\omega L_{1}+\frac{\omega^{3} M^{2}}{\frac{1}{c}-\omega^{2} L_{2}}}{R}\right)
$$

is the phase angle of the input current with respect to the driving voltage. Applying the given conditions $L_{1}=L_{2}=M=L$, say, we have

$$
\begin{aligned}
I_{1} &=\frac{V_{0} e^{-j \varphi}}{\sqrt{R^{2}+\left(\frac{\omega L}{1-\omega^{2} L C}\right)^{2}}}, \\
\varphi &=\arctan \left(\frac{\omega L / R}{1-\omega^{2} L C}\right)
\end{aligned}
$$

or, taking the real part,

$$
i_{1}(t)=\frac{V_{0}}{Z} \cos (\omega t-\varphi)
$$

with

$$
Z=\sqrt{R^{2}+\left(\frac{\omega L}{1-\omega^{2} L C}\right)^{2}} .
$$
\textbf{Topic} :Circuit Analysis\\
\textbf{Book} :Problems and Solutions on Electromagnetism\\
\textbf{Final Answer} :\sqrt{R^{2}+\left(\frac{\omega L}{1-\omega^{2} L C}\right)^{2}}\\


\textbf{Solution} :Let the currents of the primary and secondary circuits be $I_{1}$ and $I_{2}$ respectively. We have

$$
\begin{gathered}
\dot{V}=\dot{I}_{1} R+L_{1} \ddot{I}_{1}+M \ddot{I}_{2}, \\
0=L_{2} \ddot{I}_{2}+M \ddot{I}_{1}+\frac{I_{2}}{C}
\end{gathered}
$$

Solving for $I_{1} \sim \exp (j \omega t)$, we have

$$
\begin{aligned}
I_{1} &=\frac{V}{R+j\left(\omega L_{1}+\frac{\omega^{3} M^{2}}{\frac{1}{c}-\omega^{2} L_{2}}\right)} \\
&=\frac{V_{0} e^{-j \varphi}}{\sqrt{R^{2}+\left(\omega L_{1}+\frac{\omega^{3} M^{2}}{\frac{1}{c}-\omega^{2} L_{2}}\right)^{2}}},
\end{aligned}
$$

where

$$
\varphi=\arctan \left(\frac{\omega L_{1}+\frac{\omega^{3} M^{2}}{\frac{1}{c}-\omega^{2} L_{2}}}{R}\right)
$$

is the phase angle of the input current with respect to the driving voltage. Applying the given conditions $L_{1}=L_{2}=M=L$, say, we have

$$
\begin{aligned}
I_{1} &=\frac{V_{0} e^{-j \varphi}}{\sqrt{R^{2}+\left(\frac{\omega L}{1-\omega^{2} L C}\right)^{2}}}, \\
\varphi &=\arctan \left(\frac{\omega L / R}{1-\omega^{2} L C}\right)
\end{aligned}
$$

or, taking the real part,

$$
i_{1}(t)=\frac{V_{0}}{Z} \cos (\omega t-\varphi)
$$

with

$$
Z=\sqrt{R^{2}+\left(\frac{\omega L}{1-\omega^{2} L C}\right)^{2}} .
$$



$$
p(t)=V(t) i_{1}(t)=\frac{V_{0}^{2}}{Z} \cos (\omega t-\varphi) \cos \omega t .
$$

Averaging over one cycle we have

$$
\begin{aligned}
P &=\bar{p}=\frac{V_{0}^{2}}{Z} \overline{\cos (\omega t-\varphi) \cos \omega t}=\frac{V_{0}^{2}}{2 Z} \cos \varphi \\
&=\frac{R}{2 Z^{2}} V_{0}^{2}=\frac{R V_{0}^{2} / 2}{R^{2}+\left(\frac{\omega L}{1-\omega^{2} L C}\right)^{2}} .
\end{aligned}
$$
\textbf{Topic} :Circuit Analysis\\
\textbf{Book} :Problems and Solutions on Electromagnetism\\
\textbf{Final Answer} :\frac{R V_{0}^{2} / 2}{R^{2}+\left(\frac{\omega L}{1-\omega^{2} L C}\right)^{2}}\\


\textbf{Solution} :Let the currents of the primary and secondary circuits be $I_{1}$ and $I_{2}$ respectively. We have

$$
\begin{gathered}
\dot{V}=\dot{I}_{1} R+L_{1} \ddot{I}_{1}+M \ddot{I}_{2}, \\
0=L_{2} \ddot{I}_{2}+M \ddot{I}_{1}+\frac{I_{2}}{C}
\end{gathered}
$$

Solving for $I_{1} \sim \exp (j \omega t)$, we have

$$
\begin{aligned}
I_{1} &=\frac{V}{R+j\left(\omega L_{1}+\frac{\omega^{3} M^{2}}{\frac{1}{c}-\omega^{2} L_{2}}\right)} \\
&=\frac{V_{0} e^{-j \varphi}}{\sqrt{R^{2}+\left(\omega L_{1}+\frac{\omega^{3} M^{2}}{\frac{1}{c}-\omega^{2} L_{2}}\right)^{2}}},
\end{aligned}
$$

where

$$
\varphi=\arctan \left(\frac{\omega L_{1}+\frac{\omega^{3} M^{2}}{\frac{1}{c}-\omega^{2} L_{2}}}{R}\right)
$$

is the phase angle of the input current with respect to the driving voltage. Applying the given conditions $L_{1}=L_{2}=M=L$, say, we have

$$
\begin{aligned}
I_{1} &=\frac{V_{0} e^{-j \varphi}}{\sqrt{R^{2}+\left(\frac{\omega L}{1-\omega^{2} L C}\right)^{2}}}, \\
\varphi &=\arctan \left(\frac{\omega L / R}{1-\omega^{2} L C}\right)
\end{aligned}
$$

or, taking the real part,

$$
i_{1}(t)=\frac{V_{0}}{Z} \cos (\omega t-\varphi)
$$

with

$$
Z=\sqrt{R^{2}+\left(\frac{\omega L}{1-\omega^{2} L C}\right)^{2}} .
$$



$$
p(t)=V(t) i_{1}(t)=\frac{V_{0}^{2}}{Z} \cos (\omega t-\varphi) \cos \omega t .
$$

Averaging over one cycle we have

$$
\begin{aligned}
P &=\bar{p}=\frac{V_{0}^{2}}{Z} \overline{\cos (\omega t-\varphi) \cos \omega t}=\frac{V_{0}^{2}}{2 Z} \cos \varphi \\
&=\frac{R}{2 Z^{2}} V_{0}^{2}=\frac{R V_{0}^{2} / 2}{R^{2}+\left(\frac{\omega L}{1-\omega^{2} L C}\right)^{2}} .
\end{aligned}
$$

 When $\omega=\frac{1}{\sqrt{L C}}, Z=+\infty$, and $i_{1}(t)=0$.
\textbf{Topic} :Circuit Analysis\\
\textbf{Book} :Problems and Solutions on Electromagnetism\\
\textbf{Final Answer} :\frac{R V_{0}^{2} / 2}{R^{2}+\left(\frac{\omega L}{1-\omega^{2} L C}\right)^{2}}\\


\textbf{Solution} :Let the currents of the primary and secondary circuits be $I_{1}$ and $I_{2}$ respectively. We have

$$
\begin{gathered}
\dot{V}=\dot{I}_{1} R+L_{1} \ddot{I}_{1}+M \ddot{I}_{2}, \\
0=L_{2} \ddot{I}_{2}+M \ddot{I}_{1}+\frac{I_{2}}{C}
\end{gathered}
$$

Solving for $I_{1} \sim \exp (j \omega t)$, we have

$$
\begin{aligned}
I_{1} &=\frac{V}{R+j\left(\omega L_{1}+\frac{\omega^{3} M^{2}}{\frac{1}{c}-\omega^{2} L_{2}}\right)} \\
&=\frac{V_{0} e^{-j \varphi}}{\sqrt{R^{2}+\left(\omega L_{1}+\frac{\omega^{3} M^{2}}{\frac{1}{c}-\omega^{2} L_{2}}\right)^{2}}},
\end{aligned}
$$

where

$$
\varphi=\arctan \left(\frac{\omega L_{1}+\frac{\omega^{3} M^{2}}{\frac{1}{c}-\omega^{2} L_{2}}}{R}\right)
$$

is the phase angle of the input current with respect to the driving voltage. Applying the given conditions $L_{1}=L_{2}=M=L$, say, we have

$$
\begin{aligned}
I_{1} &=\frac{V_{0} e^{-j \varphi}}{\sqrt{R^{2}+\left(\frac{\omega L}{1-\omega^{2} L C}\right)^{2}}}, \\
\varphi &=\arctan \left(\frac{\omega L / R}{1-\omega^{2} L C}\right)
\end{aligned}
$$

or, taking the real part,

$$
i_{1}(t)=\frac{V_{0}}{Z} \cos (\omega t-\varphi)
$$

with

$$
Z=\sqrt{R^{2}+\left(\frac{\omega L}{1-\omega^{2} L C}\right)^{2}} .
$$



$$
p(t)=V(t) i_{1}(t)=\frac{V_{0}^{2}}{Z} \cos (\omega t-\varphi) \cos \omega t .
$$

Averaging over one cycle we have

$$
\begin{aligned}
P &=\bar{p}=\frac{V_{0}^{2}}{Z} \overline{\cos (\omega t-\varphi) \cos \omega t}=\frac{V_{0}^{2}}{2 Z} \cos \varphi \\
&=\frac{R}{2 Z^{2}} V_{0}^{2}=\frac{R V_{0}^{2} / 2}{R^{2}+\left(\frac{\omega L}{1-\omega^{2} L C}\right)^{2}} .
\end{aligned}
$$

 When $\omega=\frac{1}{\sqrt{L C}}, Z=+\infty$, and $i_{1}(t)=0$.

 When $\omega \rightarrow \frac{1}{\sqrt{L C}}, \tan \varphi=\infty$, and $\varphi=\frac{\pi}{2}$.

\textbf{Topic} :Circuit Analysis\\
\textbf{Book} :Problems and Solutions on Electromagnetism\\
\textbf{Final Answer} :\frac{R V_{0}^{2} / 2}{R^{2}+\left(\frac{\omega L}{1-\omega^{2} L C}\right)^{2}}\\


\textbf{Solution} :Assuming that the primary and secondary currents are directed as in Fig. 3.36, we have the circuit equations

$$
\begin{aligned}
\dot{V}_{0} &=R_{1} \dot{I}_{1}+\dot{I}_{1}\left[\frac{1}{j \omega C}+j \omega L\right]+j \omega M \dot{I}_{2}, \\
0 &=\dot{I}_{2} R_{2}+j \omega L \dot{I}_{2}+j \omega M \dot{I}_{1} .
\end{aligned}
$$

The above simultaneous equations have solution

$$
I_{2}=\frac{j \omega M C V_{0}}{C\left[\omega^{2}\left(L^{2}-M^{2}\right)-R_{1} R_{2}\right]-L+j\left[\frac{R_{2}}{\omega}-\omega L C\left(R_{1}+R_{2}\right)\right]} .
$$

As $P_{2}=\frac{1}{2}\left|I_{2}\right|^{2} R_{2}$, when $\left|I_{2}\right|$ is maximized $P_{2}$ is maximized also. We have

$$
\left|I_{2}\right|=\frac{\omega V_{0}}{\left\{\left[\frac{1}{M}\left(\omega^{2} L^{2}-R_{1} R_{2}-L / C\right)-\omega^{2} M\right]^{2}+\left[\frac{1}{M C}\left(\frac{R_{2}}{\omega}-\omega L C\left(R_{1}+R_{2}\right)\right)\right]^{2}\right\}^{\frac{1}{2}}}
$$

As the numerator is fixed and the denominator is the square root of the sum of two squared terms, when the two squared terms are minimum at the same time $\left|I_{2}\right|$ will achieve its maximum. The minimum of the second squared term is zero, for which we require

$$
C=\frac{R_{2}}{\omega^{2} L\left(R_{1}+R_{2}\right)},
$$

giving

$$
\left|I_{2}\right|=\frac{\omega V_{0}}{\omega^{2} M+\frac{R_{1} R_{2}+R_{1} \omega^{2} L^{2} / R_{2}}{M}} .
$$

Minimizing the above denominator, we require

$$
M=\sqrt{\frac{R_{1} R_{2}}{\omega^{2}}+\frac{R_{1}}{R_{2}} L^{2}}
$$

Hence, for

$$
M=\sqrt{\frac{R_{1} R_{2}}{\omega^{2}}+\frac{R_{1}}{R_{2}} L^{2}}, \quad C=\frac{R_{2}}{\omega^{2} L\left(R_{1}+R_{2}\right)}
$$

$P_{2}$ is maximum, having the value

$$
P_{2}=\frac{1}{2}\left|I_{2}\right|^{2} R_{2}=\frac{V_{0}^{2}}{8 R_{1}\left(1+\frac{\omega^{2} L^{2}}{R_{2}^{2}}\right)} .
$$

Supposing $\omega L / R_{2}>10$, we obtain

$$
P_{2}=\frac{V_{0}^{2} R_{2}^{2}}{8 \omega^{2} L^{2} R_{1}}
$$

as the maximum power dissipated in $R_{2}$.

\textbf{Topic} :Circuit Analysis\\
\textbf{Book} :Problems and Solutions on Electromagnetism\\
\textbf{Final Answer} :\frac{V_{0}^{2} R_{2}^{2}}{8 \omega^{2} L^{2} R_{1}}\\


\textbf{Solution} :The equivalent resistance in the primary circuit due to the resistance $R_{s}$ in the secondary circuit is

$$
R_{s}^{\prime}=\left(\frac{N_{p}}{N_{s}}\right)^{2} R_{s}
$$

Hence the time constant of the primary circuit is

$$
\tau=\left(R_{p}+R_{s}^{\prime}\right) C,
$$

and the voltage drop across $C$ is

$$
V_{C}=V e^{-t / \tau}
$$

The primary current is then

$$
i_{p}=-C \frac{d V_{C}}{d t}=-C V e^{-t / \tau}\left(-\frac{1}{\tau}\right)=\frac{V}{R_{p}+R_{s}^{\prime}} e^{-t / \tau}
$$

Initially, the primary current is

$$
i_{p}(0)=\frac{V}{R_{p}+R_{s}^{\prime}}=\frac{V}{R_{p}+\left(\frac{N_{p}}{N_{s}}\right)^{2} R_{s}} .
$$
\textbf{Topic} :Circuit Analysis\\
\textbf{Book} :Problems and Solutions on Electromagnetism\\
\textbf{Final Answer} :\frac{V}{R_{p}+\left(\frac{N_{p}}{N_{s}}\right)^{2} R_{s}}\\


\textbf{Solution} :The equivalent resistance in the primary circuit due to the resistance $R_{s}$ in the secondary circuit is

$$
R_{s}^{\prime}=\left(\frac{N_{p}}{N_{s}}\right)^{2} R_{s}
$$

Hence the time constant of the primary circuit is

$$
\tau=\left(R_{p}+R_{s}^{\prime}\right) C,
$$

and the voltage drop across $C$ is

$$
V_{C}=V e^{-t / \tau}
$$

The primary current is then

$$
i_{p}=-C \frac{d V_{C}}{d t}=-C V e^{-t / \tau}\left(-\frac{1}{\tau}\right)=\frac{V}{R_{p}+R_{s}^{\prime}} e^{-t / \tau}
$$

Initially, the primary current is

$$
i_{p}(0)=\frac{V}{R_{p}+R_{s}^{\prime}}=\frac{V}{R_{p}+\left(\frac{N_{p}}{N_{s}}\right)^{2} R_{s}} .
$$



$$
i_{s}(0)=i_{p}(0) \frac{N_{p}}{N_{s}}=\frac{N_{p} N_{s} V}{N_{s}^{2} R_{p}+N_{p}^{2} R_{s}}
$$
\textbf{Topic} :Circuit Analysis\\
\textbf{Book} :Problems and Solutions on Electromagnetism\\
\textbf{Final Answer} :\frac{N_{p} N_{s} V}{N_{s}^{2} R_{p}+N_{p}^{2} R_{s}}\\


\textbf{Solution} :The equivalent resistance in the primary circuit due to the resistance $R_{s}$ in the secondary circuit is

$$
R_{s}^{\prime}=\left(\frac{N_{p}}{N_{s}}\right)^{2} R_{s}
$$

Hence the time constant of the primary circuit is

$$
\tau=\left(R_{p}+R_{s}^{\prime}\right) C,
$$

and the voltage drop across $C$ is

$$
V_{C}=V e^{-t / \tau}
$$

The primary current is then

$$
i_{p}=-C \frac{d V_{C}}{d t}=-C V e^{-t / \tau}\left(-\frac{1}{\tau}\right)=\frac{V}{R_{p}+R_{s}^{\prime}} e^{-t / \tau}
$$

Initially, the primary current is

$$
i_{p}(0)=\frac{V}{R_{p}+R_{s}^{\prime}}=\frac{V}{R_{p}+\left(\frac{N_{p}}{N_{s}}\right)^{2} R_{s}} .
$$



$$
i_{s}(0)=i_{p}(0) \frac{N_{p}}{N_{s}}=\frac{N_{p} N_{s} V}{N_{s}^{2} R_{p}+N_{p}^{2} R_{s}}
$$

 For $V_{C}$ to fall to $V_{C}=e^{-1} V$, the time required is

$$
t=\tau=\left[R_{p}+\left(\frac{N_{p}}{N_{s}}\right)^{2} R_{s}\right] C
$$
\textbf{Topic} :Circuit Analysis\\
\textbf{Book} :Problems and Solutions on Electromagnetism\\
\textbf{Final Answer} :\left[R_{p}+\left(\frac{N_{p}}{N_{s}}\right)^{2} R_{s}\right] C\\


\textbf{Solution} :The equivalent resistance in the primary circuit due to the resistance $R_{s}$ in the secondary circuit is

$$
R_{s}^{\prime}=\left(\frac{N_{p}}{N_{s}}\right)^{2} R_{s}
$$

Hence the time constant of the primary circuit is

$$
\tau=\left(R_{p}+R_{s}^{\prime}\right) C,
$$

and the voltage drop across $C$ is

$$
V_{C}=V e^{-t / \tau}
$$

The primary current is then

$$
i_{p}=-C \frac{d V_{C}}{d t}=-C V e^{-t / \tau}\left(-\frac{1}{\tau}\right)=\frac{V}{R_{p}+R_{s}^{\prime}} e^{-t / \tau}
$$

Initially, the primary current is

$$
i_{p}(0)=\frac{V}{R_{p}+R_{s}^{\prime}}=\frac{V}{R_{p}+\left(\frac{N_{p}}{N_{s}}\right)^{2} R_{s}} .
$$



$$
i_{s}(0)=i_{p}(0) \frac{N_{p}}{N_{s}}=\frac{N_{p} N_{s} V}{N_{s}^{2} R_{p}+N_{p}^{2} R_{s}}
$$

 For $V_{C}$ to fall to $V_{C}=e^{-1} V$, the time required is

$$
t=\tau=\left[R_{p}+\left(\frac{N_{p}}{N_{s}}\right)^{2} R_{s}\right] C
$$

 As

$$
i_{s}=i_{p} \frac{N_{p}}{N_{s}}=\frac{N_{p} N_{s} V}{N_{s}^{2} R_{p}+N_{p}^{2} R_{s}} e^{-t / \tau} \text {, }
$$

the energy dissipated in $R_{3}$ is

$$
\begin{aligned}
W_{R S} &=\int_{0}^{\infty} i_{s}^{2} R_{s} d t=\left(\frac{N_{p} N_{s} V}{N_{s}^{2} R_{p}+N_{p}^{2} R_{s}}\right)^{2} R_{s} \int_{0}^{\infty} e^{-2 t / \tau} d t \\
&=\left(\frac{N_{p} N_{s} V}{N_{s}^{2} R_{p}+N_{p}^{2} R_{s}}\right)^{2} R_{s} \cdot \frac{1}{2}\left[R_{p}+\left(\frac{N_{p}}{N_{s}}\right)^{2} R_{s}\right] C \\
&=\frac{1}{2} \frac{N_{p}^{2} V^{2}}{N_{s}^{2} R_{p}+N_{p}^{2} R_{s}} R_{s} C .
\end{aligned}
$$

\textbf{Topic} :Circuit Analysis\\
\textbf{Book} :Problems and Solutions on Electromagnetism\\
\textbf{Final Answer} :\frac{1}{2} \frac{N_{p}^{2} V^{2}}{N_{s}^{2} R_{p}+N_{p}^{2} R_{s}} R_{s} C\\


\textbf{Solution} :For the circuit in Fig. 3.38, we have

$$
\begin{aligned}
\dot{V}_{0} &=\dot{I} \cdot\left[\left(R+\frac{1}{j \omega C}\right) \| R\right] \cdot \frac{R}{R+\frac{1}{j \omega C}} \\
&=\dot{I} \cdot \frac{R^{2}}{2 R+\frac{1}{j \omega C}}=\dot{I} \frac{R^{2}}{\sqrt{4 R^{2}+\frac{1}{\omega^{2} C^{2}}}}<\arctan (1 / 2 \omega R C) .
\end{aligned}
$$

For the circuit in Fig. 3.39, we have

$$
\dot{V}_{0}=j \omega M \dot{I}=\dot{I} M \omega \angle \pi / 2 \text {. }
$$

For the former to "fake" the latter, we require

$$
\left\{\begin{array}{l}
\frac{R^{2}}{\sqrt{4 R^{2}+\frac{1}{\omega^{2} C^{2}}}}=M \omega, \\
\frac{\pi}{2}-\arctan \left(\frac{1}{2 R \omega C}\right)=\theta .
\end{array}\right.
$$

Equation (2) gives $\omega R C=\frac{1}{2} \tan \theta$. With $\theta=0.01, \omega R C=0.005$. Equation (1) then gives

$$
R=M \omega \sqrt{4+\frac{1}{(\omega R C)^{2}}}=10^{-3} \times 2 \pi \times 200 \times \sqrt{4+\frac{1}{0.005^{2}}} \approx 251 \Omega
$$

and hence

$$
C=\frac{0.005}{\omega R}=\frac{0.005}{2 \pi \times 200 \times 251} \approx 1.6 \times 10^{-8} \mathrm{~F}=0.016 \mu \mathrm{F}
$$

\textbf{Topic} :Circuit Analysis\\
\textbf{Book} :Problems and Solutions on Electromagnetism\\
\textbf{Final Answer} :0016 \mu \mathrm{F}\\


\textbf{Solution} :When a dc voltage is connected to the box a finite current flows. Since both $C$ and $L$ are lossless, this shows that $R$ must be in parallel with $C$ or with both $L, C$. At resonance a large current of $100 \mathrm{~A}$ is observed for an ac rms voltage of $1 \mathrm{~V}$. This large resonance is not possible if $L$ and $C$ are in parallel, whatever the connection of $R$. The only possible circuit is then the one shown in Fig. $3.40$ with $L, C$ in series. Since a dc voltage of $1.5 \mathrm{~V}$ gives rise to a current of $1.5 \mathrm{~mA}$, we have

$$
R=\frac{V}{I}=\frac{1.5}{1.5 \times 10^{-3}}=10^{3} \Omega .
$$

The impedance for the circuit in Fig. $3.40$ is

$$
Z=\frac{1}{\frac{1}{R}+\frac{1}{j\left(\omega L-\frac{1}{\omega C}\right)}}=\frac{1}{\frac{1}{R}+\frac{1}{j \omega L\left(1-\frac{\omega_{0}^{2}}{\omega^{2}}\right)}},
$$

giving

$$
L=\left[\frac{1}{\frac{1}{Z^{2}}-\frac{1}{R^{2}}} \cdot \frac{1}{\omega^{2}\left(1-\frac{\omega_{0}^{2}}{\omega^{2}}\right)^{2}}\right]^{-1 / 2}
$$

where $\omega_{0}^{2}=\frac{1}{L C}$

MATHPIX IMAGE

Fig. $3.40$

The resonance occurs at $\omega_{0}=2000 \pi \mathrm{rad} / \mathrm{s}$. At $\omega=120 \pi \mathrm{rad} / \mathrm{s}$, $V_{\text {rms }}=1 \mathrm{~V}$ gives $I_{\text {rms }}=\frac{1}{100} \mathrm{~A}$, corresponding to

$$
|Z|=\frac{V_{\mathrm{rms}}}{I_{\mathrm{rms}}}=100 \Omega=\frac{R}{10},
$$

at

Hence

$$
\frac{\omega_{0}}{\omega}=\frac{50}{3} .
$$

$$
\begin{aligned}
&L \approx \frac{\omega}{\omega_{0}^{2}}|Z|=\frac{60 \times 2 \pi}{(1000 \times 2 \pi)^{2}} \times 100=0.95 \mathrm{mH}, \\
&C=\frac{1}{L \omega_{0}^{2}}=\frac{1}{\omega|Z|}=\frac{1}{60 \times 2 \pi \times 100} \approx 27 \mu \mathrm{F} .
\end{aligned}
$$

\textbf{Topic} :Circuit Analysis\\
\textbf{Book} :Problems and Solutions on Electromagnetism\\
\textbf{Final Answer} :\frac{1}{60 \times 2 \pi \times 100} \approx 27 \mu \mathrm{F}\\


\textbf{Solution} :Neglecting edge effects, the magnetic field in the solenoid is uniform everywhere. From Ampère's circuital law $\oint \mathbf{B} \cdot d l=\mu_{0} I$, we find the magnetic field induction inside the solenoid as $B=\mu_{0} n I$, where $n=\frac{N}{l}$ is the turn density of the solenoid. The total magnetic flux crossing the coil is $\psi=N B A$. The inductance of the coil is given by the definition

$$
L=\frac{\psi}{I}=\frac{N \mu_{0} N I A}{l I}=\frac{N^{2} \mu_{0} A}{l} .
$$

With $A=\pi \times 10^{-4} \mathrm{~m}^{2}$, we have

$$
L=\frac{100^{2} \times 4 \pi \times 10^{-7} \times \pi \times 10^{-4}}{0.1}=3.95 \times 10^{-5} \mathrm{H} .
$$

\textbf{Topic} :Circuit Analysis\\
\textbf{Book} :Problems and Solutions on Electromagnetism\\
\textbf{Final Answer} :395 \times 10^{-5} \mathrm{H}\\


\textbf{Solution} :The equivalent circuit is shown in Fig. 3.45, for which

$$
V=I R+L \frac{d I}{d t},
$$

with

$$
\left.V\right|_{t<0}=V_{0}=I_{0} R,\left.\quad V\right|_{t \geq 0}=0 .
$$

Thus for $t \geq 0$

$$
-\frac{d t}{\tau}=\frac{d I}{I}
$$

where $\tau=\frac{L}{R}$. Hence

$$
I=I_{0} e^{-t / \tau}=\frac{V_{0}}{R} e^{-t / \tau} .
$$

The self-inductance of the torus is

$$
\begin{aligned}
L &=\mu \mu_{0} \frac{N^{2} A}{2 \pi R} \\
&=\frac{10^{4 \times 2} \times 4 \pi \times 10^{-7} \times 10^{3} \times 5 \times 10^{-4}}{2 \pi \times 20 \times 10^{-2}}=50 \mathrm{H} .
\end{aligned}
$$

For $I=I_{0} e^{-1}, t=\tau=\frac{L}{R}=\frac{50}{10}=5 \mathrm{~s}$

\textbf{Topic} :Circuit Analysis\\
\textbf{Book} :Problems and Solutions on Electromagnetism\\
\textbf{Final Answer} :50 \mathrm{H}\\


\textbf{Solution} :When the magnetic flux crossing the circular loop changes an emf $\varepsilon$ will be induced producing an induced current $i$. Besides, a self-inductance emf $L \frac{d i}{d t}$ is produced as well. Thus we have

$$
\varepsilon+i R+L \frac{d i}{d t}=0
$$

with

$$
\varepsilon=-\frac{d \phi}{d t}, \quad i=\frac{d q}{d t}, \quad i(\infty)=0, \quad i(0)=0 .
$$

The circuit equations can be written as

$$
-d \phi+R d q+L d i=0 .
$$

Integrating over $t$ from 0 to $\infty$ then gives

$$
-\Delta \phi+R q=0
$$

as $\Delta i=0$. Hence

$$
q=\frac{\Delta \phi}{R}=\frac{B \pi a^{2}}{R} .
$$

This shows that $L$ has no effect on the value of $q$. It only leads to a slower decay of $i$.


\textbf{Topic} :Circuit Analysis\\
\textbf{Book} :Problems and Solutions on Electromagnetism\\
\textbf{Final Answer} :\frac{B \pi a^{2}}{R}\\


\textbf{Solution} :Let the current in the winding of the solenoid be $i$. The magnetic induction inside the solenoid is then $B=\mu_{0} n i$ with direction along the axis, $n$ being the number of turns per unit length of the winding.

The total magnetic flux linkage is

$$
\psi=N \phi=N B S=N^{2} \mu_{0} S i / l .
$$

Hence the self-inductance is

$$
\begin{aligned}
L &=\frac{\psi}{i}=N^{2} \mu_{0} S / l \\
& \approx \frac{1000^{2} \times 4 \pi \times 10^{-7} \times 10^{-4}}{1 / 2}=2.513 \times 10^{-4} \mathrm{H} .
\end{aligned}
$$

The total magnetic flux linkage in the secondary winding produced by the currrent $i$ is $\psi^{\prime}=N^{\prime} \phi$, giving the mutual inductance as

$$
M=\frac{\psi^{\prime}}{i}=\frac{N N^{\prime} \mu_{o} S}{l}=2.513 \times 10^{-5} \mathrm{H} .
$$

Because of the magnetic flux linkage $\psi^{\prime}=M I, I$ being the current in the secondary, an emf will be induced in the solenoid when the constant current $I$ in the secondary is suddenly stopped. Kirchhoff's law gives for the induced current $i$ in the solenoid

$$
-\frac{d \psi^{\prime}}{d t}=R i+L \frac{d i}{d t},
$$

or

$$
-d \psi^{\prime}=R i d t+L d i=R d q+L d i
$$

Integrating over $t$ from $t=0$ to $t=\infty$ gives $-\Delta \psi^{\prime}=R q$, since $i(0)=$ $i(\infty)=0$. Thus the total charge passing through the resistance is

$$
q=\frac{-\Delta \psi^{\prime}}{R}=\frac{M I}{R}=\frac{2.513 \times 10^{-5} \times 1}{10^{3}}=2.76 \times 10^{-7} \mathrm{C}
$$

\textbf{Topic} :Circuit Analysis\\
\textbf{Book} :Problems and Solutions on Electromagnetism\\
\textbf{Final Answer} :276 \times 10^{-7} \mathrm{C}\\


\textbf{Solution} :The direction of $\mathbf{B}_{2}$ is illustrated in Fig. 3.47. 

MATHPIX IMAGE

Fig. $3.47$

Let the self-inductance of the coil $L$ be $L_{1}$, then

$$
\varepsilon_{1}=M \frac{d i_{2}}{d t}+L_{1} \frac{d i_{1}}{d t},
$$

or

$$
\frac{d q}{d t}=i_{1}=\frac{\varepsilon_{1}}{R_{1}}=\frac{M}{R_{1}} \frac{d i_{2}}{d t}+\frac{L_{1}}{R_{1}} \frac{d i_{1}}{d t}
$$

with

$$
i_{2}(0)=0, \quad i_{2}(\infty)=1 \mathrm{~A}, \quad i_{1}(0)=i_{1}(\infty)=0 .
$$

Integrating the circuit equation we obtain

$$
q_{1}=\int_{0}^{\infty} \frac{M}{R_{1}} d i_{2}+\int_{0}^{\infty} \frac{L_{1}}{R_{1}} d i_{1}=\frac{M}{R_{1}} i_{2}(\infty) .
$$

When the coil is moved into the magnetic field $\mathbf{B}_{2}$, its induced emf is

$$
\varepsilon_{2}=-\frac{d \psi}{d t}
$$

with

$$
\psi_{1}(\infty)=-N B_{2} S, \quad \psi_{1}(0)=0
$$

Thus

$$
\frac{d q_{2}}{d t}=i_{2}=\frac{\varepsilon_{2}}{R_{1}}=\frac{-1}{R_{1}} \frac{d \psi_{1}}{d t}
$$

giving

$$
q_{2}=\frac{N B_{2} \pi a^{2}}{R_{1}} .
$$

As $q \propto \theta$, we have

$$
\frac{\theta_{1}}{\theta_{2}}=\frac{q_{1}}{q_{2}}=\frac{M i_{2}(\infty)}{N B_{2} \pi a^{2}},
$$

or

$$
B_{2}=\frac{\theta_{2} M i_{2}(\infty)}{\theta_{1} N \pi a^{2}}=\frac{1 \times 1 \times 1}{0.5 \times 100 \times \pi \times 10^{-4}}=63.4 \mathrm{~T} .
$$

\textbf{Topic} :Circuit Analysis\\
\textbf{Book} :Problems and Solutions on Electromagnetism\\
\textbf{Final Answer} :634 \mathrm{~T}\\


\textbf{Solution} :Let the upper plate carry charge $+Q$ and the lower plate carry charge $-Q$ at time $t$. Due to the continuity of the tangential component of electric intensity across an interface, we have

$$
\mathbf{E}=\frac{Q}{\pi a^{2} \varepsilon_{0}} \mathbf{e}_{z} \text { at } z<h .
$$

For $r \leq b, \mathbf{b}=\sigma \mathbf{E}$, where $\sigma=\frac{1}{p}$, giving

$$
I=j \pi b^{2}=\sigma \mathbf{E} \pi b^{2} .
$$

Thus

$$
\sigma \frac{Q}{\pi a^{2} \varepsilon_{0}} \pi b^{2}=-\frac{d Q}{d t},
$$

or

$$
Q=Q_{0} e^{-t / \tau}
$$

where $\tau=\frac{a^{2} \varepsilon_{0} \rho}{b^{2}}$. Hence

$$
\mathbf{E}=E_{0} e^{-t / \tau} \mathbf{e}_{z}=\frac{V_{0}}{h} e^{-t / \tau} \mathbf{e}_{z} .
$$
\textbf{Topic} :Circuit Analysis\\
\textbf{Book} :Problems and Solutions on Electromagnetism\\
\textbf{Final Answer} :\frac{V_{0}}{h} e^{-t / \tau} \mathbf{e}_{z}\\


\textbf{Solution} :Let the upper plate carry charge $+Q$ and the lower plate carry charge $-Q$ at time $t$. Due to the continuity of the tangential component of electric intensity across an interface, we have

$$
\mathbf{E}=\frac{Q}{\pi a^{2} \varepsilon_{0}} \mathbf{e}_{z} \text { at } z<h .
$$

For $r \leq b, \mathbf{b}=\sigma \mathbf{E}$, where $\sigma=\frac{1}{p}$, giving

$$
I=j \pi b^{2}=\sigma \mathbf{E} \pi b^{2} .
$$

Thus

$$
\sigma \frac{Q}{\pi a^{2} \varepsilon_{0}} \pi b^{2}=-\frac{d Q}{d t},
$$

or

$$
Q=Q_{0} e^{-t / \tau}
$$

where $\tau=\frac{a^{2} \varepsilon_{0} \rho}{b^{2}}$. Hence

$$
\mathbf{E}=E_{0} e^{-t / \tau} \mathbf{e}_{z}=\frac{V_{0}}{h} e^{-t / \tau} \mathbf{e}_{z} .
$$

 Applying Ampère's circuital law $\oint \mathbf{B} \cdot d \mathbf{l}=\mu_{0} I$ to a coaxial circle of radius $r<b$ on a cross section of the solid cylinder:

$$
\oint \mathbf{B} \cdot d \mathbf{r}=\mu_{0} \iint \mathbf{j} \cdot d \mathbf{S}
$$

one has

$$
B \cdot 2 \pi r=\mu_{0} j \pi r^{2},
$$

or

$$
\mathbf{B}=\frac{\mu_{0} j r}{2} \mathbf{e}_{\theta}=\frac{\mu_{0} r V_{0}}{2 \rho h} e^{-t / \tau} \mathbf{e}_{\theta}, \quad(r<b)
$$

where $e_{\theta}$ is a unit vector tangential to the circle. For $b<r<a$, the circuital law

$$
\oint \mathbf{B} \cdot d \mathbf{r}=j \pi b^{2} \mu_{0}
$$

gives

$$
\mathrm{B}=\frac{\mu_{0} b^{2} V_{0}}{2 r \rho h} e^{-t / \tau} \mathbf{e}_{\theta}
$$
\textbf{Topic} :Circuit Analysis\\
\textbf{Book} :Problems and Solutions on Electromagnetism\\
\textbf{Final Answer} :\frac{\mu_{0} b^{2} V_{0}}{2 r \rho h} e^{-t / \tau} \mathbf{e}_{\theta}\\


\textbf{Solution} :Let the upper plate carry charge $+Q$ and the lower plate carry charge $-Q$ at time $t$. Due to the continuity of the tangential component of electric intensity across an interface, we have

$$
\mathbf{E}=\frac{Q}{\pi a^{2} \varepsilon_{0}} \mathbf{e}_{z} \text { at } z<h .
$$

For $r \leq b, \mathbf{b}=\sigma \mathbf{E}$, where $\sigma=\frac{1}{p}$, giving

$$
I=j \pi b^{2}=\sigma \mathbf{E} \pi b^{2} .
$$

Thus

$$
\sigma \frac{Q}{\pi a^{2} \varepsilon_{0}} \pi b^{2}=-\frac{d Q}{d t},
$$

or

$$
Q=Q_{0} e^{-t / \tau}
$$

where $\tau=\frac{a^{2} \varepsilon_{0} \rho}{b^{2}}$. Hence

$$
\mathbf{E}=E_{0} e^{-t / \tau} \mathbf{e}_{z}=\frac{V_{0}}{h} e^{-t / \tau} \mathbf{e}_{z} .
$$

 Applying Ampère's circuital law $\oint \mathbf{B} \cdot d \mathbf{l}=\mu_{0} I$ to a coaxial circle of radius $r<b$ on a cross section of the solid cylinder:

$$
\oint \mathbf{B} \cdot d \mathbf{r}=\mu_{0} \iint \mathbf{j} \cdot d \mathbf{S}
$$

one has

$$
B \cdot 2 \pi r=\mu_{0} j \pi r^{2},
$$

or

$$
\mathbf{B}=\frac{\mu_{0} j r}{2} \mathbf{e}_{\theta}=\frac{\mu_{0} r V_{0}}{2 \rho h} e^{-t / \tau} \mathbf{e}_{\theta}, \quad(r<b)
$$

where $e_{\theta}$ is a unit vector tangential to the circle. For $b<r<a$, the circuital law

$$
\oint \mathbf{B} \cdot d \mathbf{r}=j \pi b^{2} \mu_{0}
$$

gives

$$
\mathrm{B}=\frac{\mu_{0} b^{2} V_{0}}{2 r \rho h} e^{-t / \tau} \mathbf{e}_{\theta}
$$

 For $0<r<b$ and between the conducting plates, the Poynting vector is $\mathbf{S}=\mathbf{E} \times \mathbf{H}=\frac{\mathbb{E} \times \mathbf{B}}{\mu_{0}}=\frac{V_{0}}{h} e^{-t / \tau} \cdot \frac{r V_{0}}{2 \rho h} e^{-t / \tau}\left(\mathbf{e}_{z} \times \mathbf{e}_{\theta}\right)=-\frac{r}{2 \rho}\left(\frac{V_{0}}{h}\right)^{2} e^{-2 t / \tau} \mathbf{e}_{r}$. For $b<r<a$, we have

$$
\mathbf{S}=\frac{V_{0}}{h} e^{-t / \tau} \cdot \frac{b^{2} V_{0}}{2 r \rho h} e^{-t / \tau} \mathbf{e}_{r}=-\frac{1}{2 r \rho}\left(\frac{V_{0} b}{h}\right)^{2} e^{-2 t / \tau} \mathbf{e}_{r}
$$

The directions of $S$ at $r=a$ and $r=b$ are both given by $-e_{r}$, i.e., the electromagnetic energy flows radially inwards into the solid cylinder between the plates (in the ideal case). This energy provides for the loss of energy due to Joule heating in the solid cylinder where a current flows. This can be seen as follows. For $b<r<a$ the inward energy flow per unit time is $\frac{1}{2 r \rho}\left(\frac{V_{0} b}{h}\right)^{2} \cdot 2 \pi r h e^{-2 t / \tau}=\frac{\pi}{\rho h}\left(V_{0} b\right)^{2} e^{-2 t / \tau}$, independent of $r$. But for $r<b$, the power in-flow is $\frac{r}{2 \rho}\left(\frac{V_{0}}{h}\right)^{2} \cdot 2 \pi r h e^{-2 t / \tau}=\frac{\pi}{\rho h}\left(V_{0} r\right)^{2} e^{-2 t / \tau}$, decreasing as $r$ decreases.
\textbf{Topic} :Circuit Analysis\\
\textbf{Book} :Problems and Solutions on Electromagnetism\\
\textbf{Final Answer} :-\frac{1}{2 r \rho}\left(\frac{V_{0} b}{h}\right)^{2} e^{-2 t / \tau} \mathbf{e}_{r}\\


\textbf{Solution} :Suppose the solenoid carries a current $I$. The magnetic induction

$$
B=\mu_{0} n I=\mu_{0} N I / l,
$$

and the magnetic flux linkage is

$$
\psi=N B S=N \frac{\mu_{0} N I}{l} \cdot \pi r^{2}=\frac{I \mu_{0} N^{2} \pi r^{2}}{l} .
$$

Hence the self-inductance is

$$
\begin{aligned}
L=\frac{\psi}{I} &=\frac{\mu_{0} N^{2} \pi r^{2}}{l}=\frac{4 \pi \times 10^{-7} \times 1000^{2} \times \pi \times 0.1^{2}}{2} \\
&=1.97 \times 10^{-2} \mathrm{H} .
\end{aligned}
$$
\textbf{Topic} :Circuit Analysis\\
\textbf{Book} :Problems and Solutions on Electromagnetism\\
\textbf{Final Answer} :197 \times 10^{-2} \mathrm{H}\\


\textbf{Solution} :Suppose the solenoid carries a current $I$. The magnetic induction

$$
B=\mu_{0} n I=\mu_{0} N I / l,
$$

and the magnetic flux linkage is

$$
\psi=N B S=N \frac{\mu_{0} N I}{l} \cdot \pi r^{2}=\frac{I \mu_{0} N^{2} \pi r^{2}}{l} .
$$

Hence the self-inductance is

$$
\begin{aligned}
L=\frac{\psi}{I} &=\frac{\mu_{0} N^{2} \pi r^{2}}{l}=\frac{4 \pi \times 10^{-7} \times 1000^{2} \times \pi \times 0.1^{2}}{2} \\
&=1.97 \times 10^{-2} \mathrm{H} .
\end{aligned}
$$



$$
B=\frac{\mu_{0} N I}{l}=\frac{4 \pi \times 10^{-7} \times 1000 \times 2000}{2}=1.26 \mathrm{~Wb} / \mathrm{m}^{2} .
$$
\textbf{Topic} :Circuit Analysis\\
\textbf{Book} :Problems and Solutions on Electromagnetism\\
\textbf{Final Answer} :126 \mathrm{~Wb} / \mathrm{m}^{2}\\


\textbf{Solution} :Suppose the solenoid carries a current $I$. The magnetic induction

$$
B=\mu_{0} n I=\mu_{0} N I / l,
$$

and the magnetic flux linkage is

$$
\psi=N B S=N \frac{\mu_{0} N I}{l} \cdot \pi r^{2}=\frac{I \mu_{0} N^{2} \pi r^{2}}{l} .
$$

Hence the self-inductance is

$$
\begin{aligned}
L=\frac{\psi}{I} &=\frac{\mu_{0} N^{2} \pi r^{2}}{l}=\frac{4 \pi \times 10^{-7} \times 1000^{2} \times \pi \times 0.1^{2}}{2} \\
&=1.97 \times 10^{-2} \mathrm{H} .
\end{aligned}
$$



$$
B=\frac{\mu_{0} N I}{l}=\frac{4 \pi \times 10^{-7} \times 1000 \times 2000}{2}=1.26 \mathrm{~Wb} / \mathrm{m}^{2} .
$$



$$
W_{m}=\frac{L}{2} I^{2}=\frac{1.97 \times 10^{-2} \times 2000^{2}}{2}=3.94 \times 10^{4} \mathrm{~J} .
$$
\textbf{Topic} :Circuit Analysis\\
\textbf{Book} :Problems and Solutions on Electromagnetism\\
\textbf{Final Answer} :394 \times 10^{4} \mathrm{~J}\\


\textbf{Solution} :Suppose the solenoid carries a current $I$. The magnetic induction

$$
B=\mu_{0} n I=\mu_{0} N I / l,
$$

and the magnetic flux linkage is

$$
\psi=N B S=N \frac{\mu_{0} N I}{l} \cdot \pi r^{2}=\frac{I \mu_{0} N^{2} \pi r^{2}}{l} .
$$

Hence the self-inductance is

$$
\begin{aligned}
L=\frac{\psi}{I} &=\frac{\mu_{0} N^{2} \pi r^{2}}{l}=\frac{4 \pi \times 10^{-7} \times 1000^{2} \times \pi \times 0.1^{2}}{2} \\
&=1.97 \times 10^{-2} \mathrm{H} .
\end{aligned}
$$



$$
B=\frac{\mu_{0} N I}{l}=\frac{4 \pi \times 10^{-7} \times 1000 \times 2000}{2}=1.26 \mathrm{~Wb} / \mathrm{m}^{2} .
$$



$$
W_{m}=\frac{L}{2} I^{2}=\frac{1.97 \times 10^{-2} \times 2000^{2}}{2}=3.94 \times 10^{4} \mathrm{~J} .
$$

 The circuit equation is

$$
\varepsilon=i R+L \frac{d i}{d t},
$$

giving

$$
i=\frac{\varepsilon}{R}\left(1-e^{-t / \tau}\right)=i(\infty)\left(1-e^{-t / \tau}\right),
$$

where $\tau=\frac{L}{R}=0.197 \mathrm{~s}$ is the time constant of the circuit. As

$$
\varepsilon=20 \mathrm{~V}, \quad R=0.1 \Omega, \quad L=1.97 \times 10^{-2} \mathrm{H},
$$

we have

$$
i(t)=200\left(1-e^{-s t}\right) \mathrm{A} .
$$



\textbf{Topic} :Circuit Analysis\\
\textbf{Book} :Problems and Solutions on Electromagnetism\\
\textbf{Final Answer} :200\left(1-e^{-s t}\right) \mathrm{A}\\


\textbf{Solution} :Since $\frac{Q}{C}=i R=-\frac{d Q}{d t} R$, we have $Q=Q_{0} e^{-t / \tau}$ with $\tau=R C$. As $E=\frac{\sigma}{\varepsilon_{0}}$, we have $E=\frac{Q}{\pi r^{2} \varepsilon_{0}} e^{-t / \tau}$. Comparing this with $E=E_{0} e^{-t / \tau}$, we find

$$
E_{0}=\frac{Q}{\pi r^{2} \varepsilon_{0}}, \quad \tau=R C=\frac{R \varepsilon_{0} \pi r_{0}^{2}}{d} .
$$
\textbf{Topic} :Circuit Analysis\\
\textbf{Book} :Problems and Solutions on Electromagnetism\\
\textbf{Final Answer} :\frac{R \varepsilon_{0} \pi r_{0}^{2}}{d}\\


\textbf{Solution} :Since $\frac{Q}{C}=i R=-\frac{d Q}{d t} R$, we have $Q=Q_{0} e^{-t / \tau}$ with $\tau=R C$. As $E=\frac{\sigma}{\varepsilon_{0}}$, we have $E=\frac{Q}{\pi r^{2} \varepsilon_{0}} e^{-t / \tau}$. Comparing this with $E=E_{0} e^{-t / \tau}$, we find

$$
E_{0}=\frac{Q}{\pi r^{2} \varepsilon_{0}}, \quad \tau=R C=\frac{R \varepsilon_{0} \pi r_{0}^{2}}{d} .
$$

 To find $\mathbf{E}$ for case (a), we have assumed that the charge $Q$ is uniformly distributed over the plates at any time and the edge effects may be neglected. These approximations are good if $d \ll r_{0}$.

 By symmetry and Maxwell's integral equation

$$
\oint \mathbf{H} \cdot d \mathbf{l}=\iint_{S} \frac{\partial \mathbf{D}}{\partial t} \cdot d \mathbf{S},
$$

where

$$
\mathbf{D}=\varepsilon_{0} \mathbf{E},
$$

we find

$$
\mathbf{H}=-\frac{\varepsilon_{0} r E}{2 \tau} \mathbf{e}_{\theta}, \quad \mathrm{B}=-\frac{\mu_{0} \varepsilon_{0} r E}{2 \tau} \mathbf{e}_{\theta},
$$

taking approximations similar to those stated in (b).
\textbf{Topic} :Circuit Analysis\\
\textbf{Book} :Problems and Solutions on Electromagnetism\\
\textbf{Final Answer} :-\frac{\mu_{0} \varepsilon_{0} r E}{2 \tau} \mathbf{e}_{\theta}\\


\textbf{Solution} :Since $\frac{Q}{C}=i R=-\frac{d Q}{d t} R$, we have $Q=Q_{0} e^{-t / \tau}$ with $\tau=R C$. As $E=\frac{\sigma}{\varepsilon_{0}}$, we have $E=\frac{Q}{\pi r^{2} \varepsilon_{0}} e^{-t / \tau}$. Comparing this with $E=E_{0} e^{-t / \tau}$, we find

$$
E_{0}=\frac{Q}{\pi r^{2} \varepsilon_{0}}, \quad \tau=R C=\frac{R \varepsilon_{0} \pi r_{0}^{2}}{d} .
$$

 To find $\mathbf{E}$ for case (a), we have assumed that the charge $Q$ is uniformly distributed over the plates at any time and the edge effects may be neglected. These approximations are good if $d \ll r_{0}$.

 By symmetry and Maxwell's integral equation

$$
\oint \mathbf{H} \cdot d \mathbf{l}=\iint_{S} \frac{\partial \mathbf{D}}{\partial t} \cdot d \mathbf{S},
$$

where

$$
\mathbf{D}=\varepsilon_{0} \mathbf{E},
$$

we find

$$
\mathbf{H}=-\frac{\varepsilon_{0} r E}{2 \tau} \mathbf{e}_{\theta}, \quad \mathrm{B}=-\frac{\mu_{0} \varepsilon_{0} r E}{2 \tau} \mathbf{e}_{\theta},
$$

taking approximations similar to those stated in (b).



$$
U=\frac{\varepsilon_{0}}{2} E^{2}+\frac{1}{2 \mu_{0}} B^{2}=\frac{1}{2} \varepsilon_{0} E^{2}\left[1+\frac{\mu_{0} \varepsilon_{0} r^{2}}{4 \tau^{2}}\right] .
$$
\textbf{Topic} :Circuit Analysis\\
\textbf{Book} :Problems and Solutions on Electromagnetism\\
\textbf{Final Answer} :\frac{1}{2} \varepsilon_{0} E^{2}\left[1+\frac{\mu_{0} \varepsilon_{0} r^{2}}{4 \tau^{2}}\right]\\


\textbf{Solution} :For each set the voltage is $k V$ and internal resistance is $k R_{\mathrm{i}}$. After the $n / k$ sets are connected in parallel, the total voltage is still $k V$, but the total internal resistance becomes $\frac{k R_{i}}{n / k}=\frac{k^{2} R_{i}}{n}$. The power in $R$ will be maximum when the load-resistance $R$ matches the internal resistance, i.e., $R=\frac{k^{2} R_{1}}{n}$. Hence $k=\sqrt{\frac{n R}{R_{\mathbf{i}}}}$ for maximum power, which has the value

$$
P_{\max }=\left(\frac{k V}{2 R}\right)^{2} R=\frac{k^{2} V^{2}}{4 R}=\frac{n V^{2}}{4 R_{\mathrm{i}}} .
$$

\textbf{Topic} :Circuit Analysis\\
\textbf{Book} :Problems and Solutions on Electromagnetism\\
\textbf{Final Answer} :\frac{n V^{2}}{4 R_{\mathrm{i}}}\\


\textbf{Solution} :As $A_{0}$ is large and the input resistance is much greater than $R_{1}$ and $R_{2}$, while the output resistance is much less than $R_{1}$ and $R_{2}$, we can consider the circuit with feedback as an ideal amplifier.

Taking $i_{1}=-i_{2}$, then

$$
V_{i}-V=-\frac{R_{1}}{R_{2}}\left(V_{0}-V\right),
$$

or

$$
\frac{V_{i}}{V_{0}}-\frac{V}{V_{0}}=-\frac{R_{1}}{R_{2}}\left(1-\frac{V}{V_{0}}\right) .
$$



Putting

$$
A_{\mathrm{F}}=V_{0} / V_{i}, \quad A_{0}=V_{0} / V,
$$

the above becomes

$$
\frac{1}{A_{\mathrm{F}}}=-\frac{R_{1}}{R_{2}}+\left(1+\frac{R_{1}}{R_{2}}\right) \frac{1}{A_{0}}
$$

giving

$$
A_{\mathrm{F}}=\frac{1}{-\frac{R_{1}}{R_{2}}+\left(1+\frac{R_{1}}{R_{2}}\right) \frac{1}{A_{0}}} .
$$

As $A_{0}$ is large, $A_{\mathrm{F}} \approx-\frac{R_{2}}{R_{1}}$. It follows that $A_{\mathrm{F}}$ is independent of $A_{0}$ but is determined by $R_{1} / R_{2}$. Hence the amplification is stable.

\textbf{Topic} :Circuit Analysis\\
\textbf{Book} :Problems and Solutions on Electromagnetism\\
\textbf{Final Answer} :\frac{1}{-\frac{R_{1}}{R_{2}}+\left(1+\frac{R_{1}}{R_{2}}\right) \frac{1}{A_{0}}}\\


\textbf{Solution} :The amplifier may be considered as an ideal operational amplifier with "virtually grounded" inverting input. Then

$$
\frac{V_{\text {in }}-0}{R_{1}+\frac{1}{j \omega C}}=\frac{0-V_{\text {out }}}{R_{2}},
$$

or

$$
\frac{V_{\text {out }}}{V_{\text {in }}}=-\frac{R_{2}}{R_{1}+\frac{1}{j \omega C}}
$$

The phase difference between the input and output voltages is

$$
\phi=\pi-\arctan \left(\frac{-\frac{1}{\omega C}}{R_{1}}\right)=\pi+\arctan \left(\frac{1}{\omega C R_{1}}\right)
$$

\textbf{Topic} :Circuit Analysis\\
\textbf{Book} :Problems and Solutions on Electromagnetism\\
\textbf{Final Answer} :\pi+\arctan \left(\frac{1}{\omega C R_{1}}\right)\\


\textbf{Solution} :The circuit is that of an opposite-phase integrator made up of an ideal amplifier. We have

$$
V_{\text {out }}=-\frac{1}{C_{f}} \int_{0}^{t} \frac{V_{\text {in }}}{R} d t+V_{0}(0), \quad\left(\left|V_{\text {out }}\right|<10 \mathrm{~V}\right)
$$

If $V_{0}(0)=0$ at the initial time, then

$$
V_{\text {out }}=-\frac{1}{C_{f}} \int_{0}^{t} \frac{V_{\mathrm{in}}}{R} d t
$$

 If the input voltage at terminals $\left(J_{1}, J_{2}\right)$ is sinusoidal with frequency $\omega$, the input impedance across the terminals is $R+\frac{1}{j \omega C_{s}}$.
\textbf{Topic} :Circuit Analysis\\
\textbf{Book} :Problems and Solutions on Electromagnetism\\
\textbf{Final Answer} :-\frac{1}{C_{f}} \int_{0}^{t} \frac{V_{\mathrm{in}}}{R} d t\\


\textbf{Solution} :This is a relaxation oscillator having positive feedback shunted by $R_{1}$ and $R_{2}$ and discharged through an $R C$ circuit. When stability is reached, the output is a rectangular wave with amplitude equal to the saturated voltage. Let $V_{\mathrm{C}}=+10 \mathrm{~V}$. The potential at point $\mathrm{B}$ is $V_{\mathrm{B}}=$ $\frac{R_{2}}{R_{1}+R_{2}} \times V_{\mathrm{C}}=2 \mathrm{~V}$. The capacitor $\mathrm{C}$ is charged through $R$, and $V_{\mathrm{A}}$ will increase from $-2 \mathrm{~V}$ to $+2 \mathrm{~V}$. When $V_{\mathrm{A}}$ is higher than $V_{\mathrm{B}}, V_{\mathrm{C}}$ will decrease to $-10 \mathrm{~V}$ and $\mathrm{C}$ will discharge through $R$. When $V_{\mathrm{A}}$ is lower than the potential at $\mathrm{B}$, which is now $\frac{R_{2}}{R_{1}+R_{2}} \times(-10)=-2 \mathrm{~V}, V_{\mathrm{C}}$ will again increase to $+10 \mathrm{~V}$. So following each charging the circuit relaxes back to the starting point, i.e., relaxation oscillation occurs. The charging of the capacitor follows

$$
V=V_{0}[1-\exp (-t / R C)] \text {, or } t=R C \ln \left(\frac{V_{0}}{V_{0}-V}\right) \text {. }
$$

The charging time $T$ from $V_{1}$ to $V_{2}$ is

$$
T=R C \ln \frac{V_{0}-V_{1}}{V_{0}-V_{2}} .
$$

The charging time is given by $V_{0}=10 \mathrm{~V}, V_{1}=-2 \mathrm{~V}, V_{2}=2 \mathrm{~V}$, i.e., $T_{1}=R C \ln \frac{10+2}{10-2}=8.1 \mathrm{~ms}$; the discharging time is given by $V_{0}=-10 \mathrm{~V}$, $V_{1}=2 \mathrm{~V}, V_{2}=-2 \mathrm{~V}$, i.e., $T_{2}=R C \ln \frac{-10-2}{-10+2}=8.1 \mathrm{~ms}$. Hence the oscillation frequency is $\frac{1}{T_{1}+T_{2}}=61.6 \mathrm{~Hz}$.
\textbf{Topic} :Circuit Analysis\\
\textbf{Book} :Problems and Solutions on Electromagnetism\\
\textbf{Final Answer} :R C \ln \frac{V_{0}-V_{1}}{V_{0}-V_{2}}\\


\textbf{Solution} :Let the output voltage be $V_{0}$. As the operational amplifiers can be considered ideal, we have the following equations:

$$
\text { at point } 1: \quad \frac{v_{1}}{R}=-C \frac{d v_{2}}{d t} \text {. }
$$



$$
\text { at point 2: } \frac{v_{2}}{R}=-C \frac{d v_{0}}{d t} \text {, }
$$

$$
\text { at point 3: } \quad \frac{v_{0}}{\frac{R}{2}}+\frac{v_{2}}{3 R}=-\frac{v_{1}}{R} \text {. }
$$

(1) and (2) give $\frac{v_{1}}{R}=R C^{2} \frac{d^{2} v_{0}}{d t^{2}}$,

(2) and (3) give $\frac{2 v_{0}}{R}-\frac{C}{3} \frac{d v_{0}}{d t}=-\frac{v_{1}}{R}$.

Then (4) and (5) give

$$
\frac{d^{2} v_{0}}{d t^{2}}-\frac{1}{3} \frac{d v_{0}}{d t}+2 v_{0}=0
$$

taking $R C=1$. This is the differential equation that can be solved by the analog computer.

MATHPIX IMAGE

Fig. 3.63

Initially when $S_{1}$ and $S_{2}$ are just opened we have

$$
\begin{aligned}
&v_{0}(0)=-3 \mathrm{~V}, \\
&v_{1}(0)=v_{2}(0)=0, \\
&\left.C \frac{d v_{0}}{d t}\right|_{t=0}=0,
\end{aligned}
$$

so the initial conditions are

$$
\left\{\begin{array}{l}
v_{0}=-3 \\
\left.\frac{d v_{0}}{d t}\right|_{t=0}=0 .
\end{array}\right.
$$



\textbf{Topic} :Circuit Analysis\\
\textbf{Book} :Problems and Solutions on Electromagnetism\\
\textbf{Final Answer} :0 
\end{array}\right\\


\textbf{Solution} :The equation can be written as

$$
\frac{d^{2} v}{d t^{2}}=-10 \frac{d v}{d t}+\frac{1}{3} v+6 \sin \omega t
$$

The block diagram of the design is shown in Fig. $3.64$ and the circuit diagram in Fig. 3.65.

MATHPIX IMAGE

$$
\begin{aligned}
&1 \text { - addometer } \\
&2,3 \text { - integrator } \\
&4,5 \text { - ratiometer }
\end{aligned}
$$

Fig. 3.64

MATHPIX IMAGE

Fig. 3.65 

Note when switches $S_{1}$ and $S_{2}$ are closed at the initial time, the voltage of the source is $V_{8}=\sin \omega t$.


\textbf{Topic} :Circuit Analysis\\
\textbf{Book} :Problems and Solutions on Electromagnetism\\
\textbf{Final Answer} :0 
\end{array}\right\\


\textbf{Solution} :In the circuit for $Q_{2}, \beta=\frac{I_{c}}{I_{b}}=\frac{100 \mathrm{~K}}{5 \mathrm{~K}}=20$. Since in a practical circuit $\beta$ is always much larger than $20, Q_{2}$ is saturated.
 As $Q_{2}$ is saturated, $V_{c}\left(Q_{2}\right)=-0.3 \mathrm{~V}$. Hence

$$
V_{b}\left(Q_{1}\right)=6-\frac{6+0.3}{25+50} \times 25=3.9 \mathrm{~V}
$$

Thus the base-emitter voltage of $Q_{1}$ is $3.9 \mathrm{~V}$.
\textbf{Topic} :Circuit Analysis\\
\textbf{Book} :Problems and Solutions on Electromagnetism\\
\textbf{Final Answer} :39 \mathrm{~V}\\


\textbf{Solution} :We first use the Laplace transform to find the transmission function $H(s)$ of the network in the frequency domain. The Laplace transform of the equation $e_{\mathrm{i}}(t)=A \cdot U(t)$ is $E_{\mathrm{i}}(s)=A / S$. Similarly, the Laplace transform of the output $e_{0}(t)$ is

$$
E_{0}(s)=\frac{1}{2} A\left[\frac{1}{S}-\frac{1}{S+1 / \tau}\right] \text {. }
$$

Hence the transmission function is

$$
H(s)=\frac{E_{0}(s)}{E_{\mathrm{i}}(s)}=\frac{\frac{1}{2 \tau}}{s+\frac{1}{\tau}} .
$$

The Laplace transform of the new input $e_{\mathrm{i}}(t)=4 \cos (\omega t)$ is

$$
E_{\mathrm{i}}(s)=\frac{4 s}{\omega^{2}+s^{2}},
$$

giving the output as

$$
\begin{aligned}
E_{0}(s) &=E_{\mathrm{i}}(s) \cdot H_{\mathrm{i}}(s)=\frac{4 s}{\omega^{2}+s^{2}} \times \frac{\frac{1}{2 \tau}}{s+\frac{1}{\tau}} \\
&=\frac{2}{\tau}\left[\frac{1}{s+\frac{1}{\tau}} \cdot \frac{-\frac{1}{\tau}}{\omega^{2}+\left(\frac{1}{\tau}\right)^{2}}+\frac{\frac{1}{2}}{s-i \omega} \times \frac{1}{i \omega+\frac{1}{\tau}}+\frac{\frac{1}{2}}{s+i \omega} \times \frac{1}{i \omega+\frac{1}{\tau}}\right]
\end{aligned}
$$

where $\omega \tau=2 \pi \times 1500 \times 1.2 \times 10^{-4} \approx 1$. The reverse transformation of $E_{0}(s)$ gives the open-circuit output voltage

$$
\begin{aligned}
V_{0}(t) &=\frac{2}{\tau}\left[\frac{-\frac{1}{\tau}}{\omega^{2}+\left(\frac{1}{\tau}\right)^{2}} e^{-\frac{1}{\tau}}+\frac{\frac{1}{\tau} \cos (\omega t)+\omega \sin (\omega t)}{\omega^{2}+\left(\frac{1}{\tau}\right)^{2}}\right] \\
& \simeq-e^{-t / \tau}+\cos (\omega t)+\sin (\omega t) .
\end{aligned}
$$

\textbf{Topic} :Circuit Analysis\\
\textbf{Book} :Problems and Solutions on Electromagnetism\\
\textbf{Final Answer} :\frac{2}{\tau}\left[\frac{-\frac{1}{\tau}}{\omega^{2}+\left(\frac{1}{\tau}\right)^{2}} e^{-\frac{1}{\tau}}+\frac{\frac{1}{\tau} \cos (\omega t)+\omega \sin (\omega t)}{\omega^{2}+\left(\frac{1}{\tau}\right)^{2}}\right] \\
& \simeq-e^{-t / \tau}+\cos (\omega t)+\sin (\omega t)\\


\textbf{Solution} :From Fig. 3.72b, we have

$$
\left\{\begin{array}{l}
V(x, t)=V(x+d x, t)+R I(t, x) d x+L d x \frac{\partial I(t, x)}{\partial t} \\
I(x, t)=\frac{\partial V(x+d x, t)}{\partial t} C d x+I(x+d x, t)
\end{array}\right.
$$

or

$$
\left\{\begin{array}{l}
-\frac{\partial V}{\partial x}=I R+L \frac{\partial I}{\partial t}, \\
-\frac{\partial I}{\partial x}=C \frac{\partial V}{\partial t} .
\end{array}\right.
$$

Eliminating $V$ we have

$$
\begin{aligned}
\frac{\partial^{2} I}{\partial x^{2}} &=-C \frac{\partial}{\partial x}\left(\frac{\partial V}{\partial t}\right)=-C \frac{\partial}{\partial t}\left(\frac{\partial V}{\partial x}\right) \\
&=+C \frac{\partial}{\partial t}\left(I R+L \frac{\partial I}{\partial t}\right) \\
&=R C \frac{\partial I}{\partial t}+L C \frac{\partial^{2} I}{\partial t^{2}} .
\end{aligned}
$$

 Similarly, eliminating $I$ we have

$$
\begin{aligned}
\frac{\partial^{2} V}{\partial x^{2}} &=-R \frac{\partial I}{\partial x}-L \frac{\partial}{\partial x}\left(\frac{\partial I}{\partial t}\right) \\
&=R C \frac{\partial V}{\partial t}-L \frac{\partial}{\partial t}\left(\frac{\partial I}{\partial x}\right) \\
&=R C \frac{\partial V}{\partial t}+L C \frac{\partial^{2} V}{\partial t^{2}}
\end{aligned}
$$

As

$$
\rho d x=C d x \cdot V, \quad V=\rho / C,
$$

the above then gives

$$
\frac{\partial^{2} \rho}{\partial x^{2}}=R C \frac{\partial \rho}{\partial t}+L C \frac{\partial^{2} \rho}{\partial t^{2}} .
$$
spectively

 The energy and rate of energy dissipation per unit length are re-

$$
W=\frac{1}{2} L I^{2}+\frac{1}{2} C V^{2}, \quad P=I^{2} R .
$$

The energy flux is

$$
\mathbf{S}=I V \mathbf{e}_{x} .
$$
\textbf{Topic} :Circuit Analysis\\
\textbf{Book} :Problems and Solutions on Electromagnetism\\
\textbf{Final Answer} :I V \mathbf{e}_{x}\\


\textbf{Solution} :From Fig. 3.72b, we have

$$
\left\{\begin{array}{l}
V(x, t)=V(x+d x, t)+R I(t, x) d x+L d x \frac{\partial I(t, x)}{\partial t} \\
I(x, t)=\frac{\partial V(x+d x, t)}{\partial t} C d x+I(x+d x, t)
\end{array}\right.
$$

or

$$
\left\{\begin{array}{l}
-\frac{\partial V}{\partial x}=I R+L \frac{\partial I}{\partial t}, \\
-\frac{\partial I}{\partial x}=C \frac{\partial V}{\partial t} .
\end{array}\right.
$$

Eliminating $V$ we have

$$
\begin{aligned}
\frac{\partial^{2} I}{\partial x^{2}} &=-C \frac{\partial}{\partial x}\left(\frac{\partial V}{\partial t}\right)=-C \frac{\partial}{\partial t}\left(\frac{\partial V}{\partial x}\right) \\
&=+C \frac{\partial}{\partial t}\left(I R+L \frac{\partial I}{\partial t}\right) \\
&=R C \frac{\partial I}{\partial t}+L C \frac{\partial^{2} I}{\partial t^{2}} .
\end{aligned}
$$

 Similarly, eliminating $I$ we have

$$
\begin{aligned}
\frac{\partial^{2} V}{\partial x^{2}} &=-R \frac{\partial I}{\partial x}-L \frac{\partial}{\partial x}\left(\frac{\partial I}{\partial t}\right) \\
&=R C \frac{\partial V}{\partial t}-L \frac{\partial}{\partial t}\left(\frac{\partial I}{\partial x}\right) \\
&=R C \frac{\partial V}{\partial t}+L C \frac{\partial^{2} V}{\partial t^{2}}
\end{aligned}
$$

As

$$
\rho d x=C d x \cdot V, \quad V=\rho / C,
$$

the above then gives

$$
\frac{\partial^{2} \rho}{\partial x^{2}}=R C \frac{\partial \rho}{\partial t}+L C \frac{\partial^{2} \rho}{\partial t^{2}} .
$$
spectively

 The energy and rate of energy dissipation per unit length are re-

$$
W=\frac{1}{2} L I^{2}+\frac{1}{2} C V^{2}, \quad P=I^{2} R .
$$

The energy flux is

$$
\mathbf{S}=I V \mathbf{e}_{x} .
$$

 As the wave is sinusoidal, let

$$
V=V_{0} \exp [i(k x-\omega t)] .
$$

Substitution in the differential equation for $V$ gives

$$
k^{2}=L C \omega^{2}+i R C \omega .
$$

Since $k$ is complex, putting $k=K+i \lambda$ and equating the real and imaginary parts separately we have

$$
\begin{aligned}
K^{2}-\lambda^{2} &=L C \omega^{2}, \\
2 K \lambda &=R C \omega .
\end{aligned}
$$

Solving these we obtain

$$
\begin{aligned}
K^{2} &=\frac{1}{2}\left(\sqrt{L^{2} C^{2} \omega^{4}+R^{2} C^{2} \omega^{2}}+L C \omega^{2}\right), \\
\lambda^{2} &=\frac{1}{2}\left(\sqrt{L^{2} C^{2} \omega^{4}+R^{2} C^{2} \omega^{2}}-L C \omega^{2}\right) .
\end{aligned}
$$

As $V$ is sinusoidal, so is $I$. Hence the equation $-\frac{\partial I}{\partial x}=C \frac{\partial V}{\partial t}$ gives

$$
I=\frac{\omega C}{k} V=I_{0} e^{-\lambda x} \exp \left[i\left(K x-\omega t+\varphi_{0}\right)\right]
$$

where

$$
I_{0}=\frac{C \omega V_{0}}{\sqrt{K^{2}+\lambda^{2}}}, \quad \varphi_{0}=\arctan \left(\frac{K}{\lambda}\right) \text {. }
$$

Actually $I(x, t)=\operatorname{Re} I=I_{0} e^{-\lambda x} \cos \left(K x-\omega t+\varphi_{0}\right)$. In the limit $R / L \omega \ll 1$

$$
K^{2}=L C \omega^{2}, \quad \lambda^{2}=\frac{R^{2} C}{4 L} .
$$

So the attenuation length is

$$
\frac{1}{\lambda}=\frac{2 \sqrt{L / C}}{R},
$$

and the propagation speed is

$$
v=\frac{\omega}{K}=\frac{1}{\sqrt{L} \bar{C}} .
$$

\textbf{Topic} :Circuit Analysis\\
\textbf{Book} :Problems and Solutions on Electromagnetism\\
\textbf{Final Answer} :\frac{1}{\sqrt{L} \bar{C}}\\


\textbf{Solution} :MATHPIX IMAGE
\textbf{Topic} :Circuit Analysis\\
\textbf{Book} :Problems and Solutions on Electromagnetism\\
\textbf{Final Answer} :\frac{1}{\sqrt{L} \bar{C}}\\


\textbf{Solution} :As

$$
V_{0}=-G V_{i}, \quad C\left(V_{i}-V_{0}\right)=Q,
$$

we have

since $G \gg 1$.

$$
V_{0}=-\frac{G}{1+G} \cdot \frac{Q}{C} \approx-\frac{Q}{C},
$$

 Let $Z_{0}$ and $Z_{l}$ be the characteristic and load impedances respectively. The reflection coefficient is

$$
\rho=\frac{Z_{l}-Z_{0}}{Z_{l}+Z_{0}} .
$$

Thus reflection normally takes place at the end of the delay line unless $\rho=0$, i.e., $Z_{0}=Z_{l}$, and the line is said to be matched. In order that the signal is not disturbed by the reflection, the ends of the line must be matched.

 When a positive pulse is applied to the input end $V_{\text {in }}$, the thyratron conducts and the potential at point $A$ will be the same as at point $B$ so that a potential drop of $2000 \mathrm{~V}$ is produced, generating a negative high-voltage pulse at the output end $V_{0}$. The width of the pulse is determined by the upper delay line in the open circuit to be

$$
t_{w}=2 \tau=\frac{2 \times 10 \times 30.48}{3 \times 10^{10}} \approx 20 \mu \mathrm{s} \text {. }
$$

The amplitude of the output pulse is given by the voltage drop across the matching resistance of the lower delay line to be

$$
\frac{2000 \times 50}{50+50}=1000 \mathrm{~V} \text {. }
$$

\textbf{Topic} :Circuit Analysis\\
\textbf{Book} :Problems and Solutions on Electromagnetism\\
\textbf{Final Answer} :1000 \mathrm{~V}\\


\textbf{Solution} :The magnetic induction at a point between the cylinders distance $r$ from the axis is in the $\mathbf{e}_{\theta}$ direction and has magnitude

$$
B=\frac{\mu_{0} I}{2 \pi r},
$$

$I$ being the current in the inner conductor. The magnetic flux crossing a longitudinal cross section of a unit length of the horn is

$$
\phi=2 \int_{0.05}^{0.4} B d r \approx \frac{\mu_{0} I}{\pi} \int_{0.05}^{0.4} \frac{1}{r} d r=8.3 \times 10^{-7} \cdot I .
$$

Hence the inductance is approximately

$$
L=\frac{\phi}{I} \approx 8.3 \times 10^{-7} \mathrm{H} \text {. }
$$

 Let the current of the RCL loop be $i(t)$. We have

$$
u_{C}+u_{L}+u_{R}=0
$$

with

$$
I=C \frac{d u_{C}}{d t}, \quad u_{R}=R C \frac{d u_{C}}{d t}, \quad u_{L}=L \frac{d I}{d t}=L C \frac{d^{2} u_{C}}{d t^{2}}
$$

i.e.,

$$
L C \frac{d^{2} u_{C}}{d t^{2}}+R C \frac{d u_{C}}{d t}+u_{C}=0,
$$

and the initial condition

$$
u_{C}(0)=V_{0} \text {. }
$$

To solve the equation for $u_{C}$, let $u_{C}=u_{0} e^{-i \omega t}$. Substituting, we have

$$
\omega=-i \alpha \pm \omega_{d},
$$

where $\omega_{d}=\sqrt{\omega_{0}^{2}-\alpha^{2}}$ with

$$
\begin{aligned}
\omega_{0} &=\frac{1}{\sqrt{L C}}, \\
\alpha &=\frac{R}{2 L}
\end{aligned}
$$

Thus

$$
u_{C}=u_{0} e^{-\alpha t \pm i \omega_{d} t},
$$

or

$$
I=C u_{0}\left(-\alpha \pm i \omega_{d}\right) e^{-\alpha t \pm i \omega_{d} t} .
$$

With the data given, we have

$$
\begin{aligned}
\alpha &=\frac{R}{2 L}=1.118 \times 10^{3} \mathrm{~s}^{-1}, \\
\omega_{0} &=\frac{1}{\sqrt{L C}}=1.047 \times 10^{5} \mathrm{~s}^{-1},
\end{aligned}
$$

so that $\omega_{0} \gg \alpha$ and $\omega_{d} \approx \omega_{0}$. Hence the current in the loop is

$$
\begin{aligned}
I(t) & \approx \operatorname{Re}\left[\mp i C \omega_{0} V_{0} e^{-\alpha t} e^{\pm i \omega_{0} t}\right] \\
&=-C \omega_{0} V_{0} e^{-\alpha t} \sin \left(\omega_{0} t\right) \\
&=-3.52 \times 10^{6} e^{-1118 t} \sin \left(1.047 \times 10^{5} t\right) .
\end{aligned}
$$

For maximum $I(t)$,

$$
\frac{d I(t)}{d t}=0
$$

i.e.,

$$
\tan \left(\omega_{0} t\right) \approx \omega_{0} t=\frac{\omega_{0}}{\alpha},
$$

giving

$$
t=\frac{1}{\alpha} .
$$

Therefore the current is maximum at $t=8.94 \times 10^{-4} \mathrm{~s}$.



$$
\begin{aligned}
I_{\max } &=3.52 \times 10^{6} e^{-1118 \times 8.94 \times 10^{-4}} \sin \left(1.047 \times 10^{5} \times 8.94 \times 10^{-4}\right) \\
&=1.29 \times 10^{6} \mathrm{~A} .
\end{aligned}
$$
\textbf{Topic} :Circuit Analysis\\
\textbf{Book} :Problems and Solutions on Electromagnetism\\
\textbf{Final Answer} :129 \times 10^{6} \mathrm{~A}\\


\textbf{Solution} :The magnetic induction at a point between the cylinders distance $r$ from the axis is in the $\mathbf{e}_{\theta}$ direction and has magnitude

$$
B=\frac{\mu_{0} I}{2 \pi r},
$$

$I$ being the current in the inner conductor. The magnetic flux crossing a longitudinal cross section of a unit length of the horn is

$$
\phi=2 \int_{0.05}^{0.4} B d r \approx \frac{\mu_{0} I}{\pi} \int_{0.05}^{0.4} \frac{1}{r} d r=8.3 \times 10^{-7} \cdot I .
$$

Hence the inductance is approximately

$$
L=\frac{\phi}{I} \approx 8.3 \times 10^{-7} \mathrm{H} \text {. }
$$

 Let the current of the RCL loop be $i(t)$. We have

$$
u_{C}+u_{L}+u_{R}=0
$$

with

$$
I=C \frac{d u_{C}}{d t}, \quad u_{R}=R C \frac{d u_{C}}{d t}, \quad u_{L}=L \frac{d I}{d t}=L C \frac{d^{2} u_{C}}{d t^{2}}
$$

i.e.,

$$
L C \frac{d^{2} u_{C}}{d t^{2}}+R C \frac{d u_{C}}{d t}+u_{C}=0,
$$

and the initial condition

$$
u_{C}(0)=V_{0} \text {. }
$$

To solve the equation for $u_{C}$, let $u_{C}=u_{0} e^{-i \omega t}$. Substituting, we have

$$
\omega=-i \alpha \pm \omega_{d},
$$

where $\omega_{d}=\sqrt{\omega_{0}^{2}-\alpha^{2}}$ with

$$
\begin{aligned}
\omega_{0} &=\frac{1}{\sqrt{L C}}, \\
\alpha &=\frac{R}{2 L}
\end{aligned}
$$

Thus

$$
u_{C}=u_{0} e^{-\alpha t \pm i \omega_{d} t},
$$

or

$$
I=C u_{0}\left(-\alpha \pm i \omega_{d}\right) e^{-\alpha t \pm i \omega_{d} t} .
$$

With the data given, we have

$$
\begin{aligned}
\alpha &=\frac{R}{2 L}=1.118 \times 10^{3} \mathrm{~s}^{-1}, \\
\omega_{0} &=\frac{1}{\sqrt{L C}}=1.047 \times 10^{5} \mathrm{~s}^{-1},
\end{aligned}
$$

so that $\omega_{0} \gg \alpha$ and $\omega_{d} \approx \omega_{0}$. Hence the current in the loop is

$$
\begin{aligned}
I(t) & \approx \operatorname{Re}\left[\mp i C \omega_{0} V_{0} e^{-\alpha t} e^{\pm i \omega_{0} t}\right] \\
&=-C \omega_{0} V_{0} e^{-\alpha t} \sin \left(\omega_{0} t\right) \\
&=-3.52 \times 10^{6} e^{-1118 t} \sin \left(1.047 \times 10^{5} t\right) .
\end{aligned}
$$

For maximum $I(t)$,

$$
\frac{d I(t)}{d t}=0
$$

i.e.,

$$
\tan \left(\omega_{0} t\right) \approx \omega_{0} t=\frac{\omega_{0}}{\alpha},
$$

giving

$$
t=\frac{1}{\alpha} .
$$

Therefore the current is maximum at $t=8.94 \times 10^{-4} \mathrm{~s}$.



$$
\begin{aligned}
I_{\max } &=3.52 \times 10^{6} e^{-1118 \times 8.94 \times 10^{-4}} \sin \left(1.047 \times 10^{5} \times 8.94 \times 10^{-4}\right) \\
&=1.29 \times 10^{6} \mathrm{~A} .
\end{aligned}
$$

 At $r=15 \mathrm{~cm}$, we have

$$
B=\frac{\mu_{0} I}{2 \pi r}=\frac{4 \pi \times 10^{-7} \times 1.29 \times 10^{6}}{2 \pi \times 0.15}=1.72 \mathrm{wbm}^{-2} .
$$
\textbf{Topic} :Circuit Analysis\\
\textbf{Book} :Problems and Solutions on Electromagnetism\\
\textbf{Final Answer} :172 \mathrm{wbm}^{-2}\\


\textbf{Solution} :Let the complex voltage be

$$
\tilde{V}=V_{0} e^{-i \omega t}
$$

Kirchhoff's equations for loops 1 and 2 are respectively

$$
\begin{aligned}
&\tilde{V}=\tilde{I}_{1} R_{1}+\frac{1}{i \omega C}\left(\tilde{I}_{1}+\tilde{I}_{2}\right), \\
&O=\tilde{I}_{2} R_{2}+\frac{1}{i \omega C}\left(\tilde{I}_{1}+\tilde{I}_{2}\right) .
\end{aligned}
$$

Eq.
(2) gives

$$
\left(R_{2}-\frac{i}{\omega C}\right) \tilde{I}_{2}=\frac{i}{\omega C} \tilde{I}_{1} .
$$

Its substitution in (1) gives

$$
\tilde{I}_{1}=\frac{\left(R_{2}-\frac{i}{\omega C}\right)}{R_{1} R_{2}-\frac{i}{\omega C}\left(R_{1}+R_{2}\right)} \tilde{V}
$$

The voltage drop through resistance $R_{1}$ is

$$
\tilde{V}_{1}=\tilde{I}_{1} R_{1}=\frac{R_{1}\left(R_{2}-\frac{i}{\omega C}\right)}{R_{1} R_{2}-\frac{i}{\omega C}\left(R_{1}+R_{2}\right)} \tilde{V}
$$

so the real voltage drop through $R_{1}$ is

$$
V_{1}=\sqrt{\frac{1+\left(\omega R_{2} C\right)^{2}}{\left(\omega R_{1} R_{2} C\right)^{2}+\left(R_{1}+R_{2}\right)^{2}}} R_{1} V_{0} \cos (\omega t+\varphi)
$$

where

$$
\varphi=\arctan \left[\frac{\omega C R_{2}^{2}}{R_{1} R_{2}{ }^{2} \omega^{2} C^{2}+\left(R_{1}+R_{2}\right)}\right]
$$
\textbf{Topic} :Circuit Analysis\\
\textbf{Book} :Problems and Solutions on Electromagnetism\\
\textbf{Final Answer} :\arctan \left[\frac{\omega C R_{2}^{2}}{R_{1} R_{2}{ }^{2} \omega^{2} C^{2}+\left(R_{1}+R_{2}\right)}\right]\\


\textbf{Solution} :Let the complex voltage be

$$
\tilde{V}=V_{0} e^{-i \omega t}
$$

Kirchhoff's equations for loops 1 and 2 are respectively

$$
\begin{aligned}
&\tilde{V}=\tilde{I}_{1} R_{1}+\frac{1}{i \omega C}\left(\tilde{I}_{1}+\tilde{I}_{2}\right), \\
&O=\tilde{I}_{2} R_{2}+\frac{1}{i \omega C}\left(\tilde{I}_{1}+\tilde{I}_{2}\right) .
\end{aligned}
$$

Eq.
(2) gives

$$
\left(R_{2}-\frac{i}{\omega C}\right) \tilde{I}_{2}=\frac{i}{\omega C} \tilde{I}_{1} .
$$

Its substitution in (1) gives

$$
\tilde{I}_{1}=\frac{\left(R_{2}-\frac{i}{\omega C}\right)}{R_{1} R_{2}-\frac{i}{\omega C}\left(R_{1}+R_{2}\right)} \tilde{V}
$$

The voltage drop through resistance $R_{1}$ is

$$
\tilde{V}_{1}=\tilde{I}_{1} R_{1}=\frac{R_{1}\left(R_{2}-\frac{i}{\omega C}\right)}{R_{1} R_{2}-\frac{i}{\omega C}\left(R_{1}+R_{2}\right)} \tilde{V}
$$

so the real voltage drop through $R_{1}$ is

$$
V_{1}=\sqrt{\frac{1+\left(\omega R_{2} C\right)^{2}}{\left(\omega R_{1} R_{2} C\right)^{2}+\left(R_{1}+R_{2}\right)^{2}}} R_{1} V_{0} \cos (\omega t+\varphi)
$$

where

$$
\varphi=\arctan \left[\frac{\omega C R_{2}^{2}}{R_{1} R_{2}{ }^{2} \omega^{2} C^{2}+\left(R_{1}+R_{2}\right)}\right]
$$

 When $V(t)=A \delta(t)$, we use the relation

$$
\delta(t)=\frac{1}{2 \pi} \int_{-\infty}^{\infty} e^{i \omega t} d \omega
$$

and write the voltage drop through $R_{1}$ as

$$
\begin{aligned}
V_{1} &=\frac{A}{2 \pi} \int_{-\infty}^{\infty} \frac{R_{1}\left(R_{2}-\frac{i}{\omega C}\right)}{R_{1} R_{2}-\frac{i}{\omega C}\left(R_{1}+R_{2}\right)} e^{i \omega t} d \omega \\
&=\frac{A}{2 \pi} \int_{-\infty}^{\infty} \frac{\left(\omega-\frac{i}{R_{2} C}\right)}{\left(\omega-\omega_{1}\right)} e^{i \omega t} d \omega,
\end{aligned}
$$

where $\omega_{1}=i \frac{R_{1}+R_{2}}{C R_{1} R_{2}}$. The integrand has a singular point at $\omega=\omega_{1}$. Using the residue theorem we find the solution $V_{1} \propto \exp \left(i \omega_{1} t\right)=\exp \left(-\frac{R_{1}+R_{2}}{C R_{1} R_{2}} t\right)$. Hence $V_{1}$ is zero for $t<0$ and

$$
V_{1} \propto \exp \left(-\frac{R_{1}+R_{2}}{C R_{1} R_{2}} t\right)
$$

for $t>0$.

\textbf{Topic} :Circuit Analysis\\
\textbf{Book} :Problems and Solutions on Electromagnetism\\
\textbf{Final Answer} :\arctan \left[\frac{\omega C R_{2}^{2}}{R_{1} R_{2}{ }^{2} \omega^{2} C^{2}+\left(R_{1}+R_{2}\right)}\right]\\


\textbf{Solution} :As the applied voltage is sinusoidal, the complex voltage and current are respectively

$$
\tilde{V}=V_{0} e^{i \omega t}, \quad \tilde{I}=I_{0} e^{i \omega t} .
$$

The average power in a period is

$$
\bar{P}=\frac{1}{2} \operatorname{Re}\left(\tilde{V}^{*} \tilde{I}\right)=\frac{1}{2} \operatorname{Re}\left(\frac{\tilde{V}^{*} \tilde{V}}{Z}\right)=\frac{V_{0}^{2}}{2} \operatorname{Re}\left(\frac{1}{Z}\right),
$$

where the star * denotes the complex conjugate and $Z$ is the impedance of the circuit, $Z=\frac{\tilde{V}}{\tilde{I}}$. Let $Z_{1}=\frac{1}{i \omega C}, Z_{2}=i \omega L$, and assume any mutual inductance to be negligible. If $L$ is the total impedance of the network, consider the equivalent circuit shown in Fig. $3.85$ whose total impedance is still $Z$. Thus

$$
Z=Z_{1}+\frac{1}{\frac{1}{Z_{2}}+\frac{1}{Z}}=Z_{1}+\frac{Z Z_{2}}{Z+Z_{2}}
$$

or

$$
Z^{2}-Z_{1} Z-Z_{1} Z_{2}=0
$$

MATHPIX IMAGE

Fig. $3.85$

As $Z>0$, this equation has only one solution

$$
Z=\frac{Z_{1}}{2}+\frac{\sqrt{Z_{1}^{2}+4 Z_{1} Z_{2}}}{2}
$$

With $\frac{1}{2 \sqrt{L C}}=\omega_{0}$ the solution becomes

$$
Z=\frac{1}{2 i \omega C}\left(1+\sqrt{1-\frac{\omega^{2}}{\omega_{0}^{2}}}\right)
$$

For $\omega<\omega_{0}, \sqrt{1-\frac{\omega^{2}}{\omega_{0}^{2}}}$ is a real number so that $\operatorname{Re}\left(\frac{1}{Z}\right)=0$, i.e., $\bar{P}=0$.

For $\omega>\omega_{0}, \operatorname{Re}\left(\frac{1}{Z}\right)=\frac{1}{2 \omega L} \sqrt{\frac{\omega^{2}}{\omega_{0}^{2}}-1}$ and

$$
\bar{P}=\frac{V_{0}^{2}}{4 \omega L} \sqrt{\frac{\omega^{2}}{\omega_{0}^{2}}-1}
$$



\textbf{Topic} :Circuit Analysis\\
\textbf{Book} :Problems and Solutions on Electromagnetism\\
\textbf{Final Answer} :\frac{V_{0}^{2}}{4 \omega L} \sqrt{\frac{\omega^{2}}{\omega_{0}^{2}}-1}\\


\textbf{Solution} :When $S$ is opened, we have

$$
I_{2}=0, \quad I_{1}=\frac{V_{0}}{\sqrt{R_{1}^{2}+\omega^{2} L_{1}^{2}}} .
$$
\textbf{Topic} :Circuit Analysis\\
\textbf{Book} :Problems and Solutions on Electromagnetism\\
\textbf{Final Answer} :\frac{V_{0}}{\sqrt{R_{1}^{2}+\omega^{2} L_{1}^{2}}}\\


\textbf{Solution} :When $S$ is opened, we have

$$
I_{2}=0, \quad I_{1}=\frac{V_{0}}{\sqrt{R_{1}^{2}+\omega^{2} L_{1}^{2}}} .
$$

 With $S$ closed we have the circuit equations

$$
\begin{aligned}
&V=I_{1} R_{1}+L_{1} \frac{\partial I_{1}}{\partial t}+M \frac{\partial I_{2}}{\partial t} \\
&0=I_{2}\left(R_{2}+R\right)+L_{2} \frac{\partial I_{2}}{\partial t}+M \frac{\partial I_{1}}{\partial t} .
\end{aligned}
$$

As $V=V_{0} \sin \omega t$, let

$$
V=V_{0} e^{-i \omega t}, \quad I_{1}=I_{10} e^{-i \omega t}, \quad I_{2}=I_{20} e^{-i \omega t} .
$$

The circuit equations become

$$
\begin{aligned}
&V=I_{1}\left(R_{1}-i \omega L_{1}\right)-i \omega M I_{2} \\
&0=-i \omega M I_{1}+I_{2}\left[\left(R_{2}+R\right)-i \omega L_{2}\right]
\end{aligned}
$$

Defining

$$
\begin{aligned}
\Delta &=\left|\begin{array}{cc}
R_{1}-i \omega L_{1} & -i \omega M \\
-i \omega M & \left(R_{2}+R\right)-i \omega L_{2}
\end{array}\right| \\
&=R_{1}\left(R_{2}+R\right)+\omega^{2}\left(M^{2}-L_{1} L_{2}\right)-i \omega\left[L_{1}\left(R_{2}+R\right)+L_{2} R_{1}\right],
\end{aligned}
$$

we have

$$
\begin{aligned}
I_{2} &=\frac{1}{\Delta}\left|\begin{array}{cc}
R_{1}-i \omega L_{1} & V \\
-i \omega M & 0
\end{array}\right| \\
&=\frac{i \omega M V}{\Delta},
\end{aligned}
$$

and

$$
I_{20}=\frac{\omega M V_{0}}{\sqrt{\left[\omega L_{1}\left(R+R_{2}\right)+\omega L_{2} R_{1}\right]^{2}+\left[\omega^{2}\left(M^{2}-L_{1} L_{2}\right)+R_{1}\left(R_{2}+R\right)\right]^{2}}} .
$$
\textbf{Topic} :Circuit Analysis\\
\textbf{Book} :Problems and Solutions on Electromagnetism\\
\textbf{Final Answer} :\frac{\omega M V_{0}}{\sqrt{\left[\omega L_{1}\left(R+R_{2}\right)+\omega L_{2} R_{1}\right]^{2}+\left[\omega^{2}\left(M^{2}-L_{1} L_{2}\right)+R_{1}\left(R_{2}+R\right)\right]^{2}}}\\


\textbf{Solution} :The impedance is given by

$$
\begin{aligned}
Z(\omega) &=j \omega L+R+\frac{1}{j \omega C}+\frac{\frac{1}{j \omega C_{1}} \cdot j \omega L_{1}}{\frac{1}{j \omega C_{1}}+j \omega L_{1}} \\
&=R+j \omega L+\frac{1}{j \omega C}+\frac{j \omega L_{1}}{1-\omega^{2} L_{1} C_{1}} \\
&=R+j\left(\omega L-\frac{1}{\omega C}+\frac{\omega L_{1}}{1-\omega^{2} L_{1} C_{1}}\right) .
\end{aligned}
$$
\textbf{Topic} :Circuit Analysis\\
\textbf{Book} :Problems and Solutions on Electromagnetism\\
\textbf{Final Answer} :R+j\left(\omega L-\frac{1}{\omega C}+\frac{\omega L_{1}}{1-\omega^{2} L_{1} C_{1}}\right)\\


\textbf{Solution} :The impedance is given by

$$
\begin{aligned}
Z(\omega) &=j \omega L+R+\frac{1}{j \omega C}+\frac{\frac{1}{j \omega C_{1}} \cdot j \omega L_{1}}{\frac{1}{j \omega C_{1}}+j \omega L_{1}} \\
&=R+j \omega L+\frac{1}{j \omega C}+\frac{j \omega L_{1}}{1-\omega^{2} L_{1} C_{1}} \\
&=R+j\left(\omega L-\frac{1}{\omega C}+\frac{\omega L_{1}}{1-\omega^{2} L_{1} C_{1}}\right) .
\end{aligned}
$$

 The complex current is

$$
I=\frac{V}{Z}=\frac{V}{R+j\left(\omega L-\frac{1}{\omega C}+\frac{\omega L_{1}}{1-\omega^{2} L_{1} C_{1}}\right)} .
$$

So its amplitude is

$$
I_{0}=\frac{V_{0}}{\left[R^{2}+\left(\omega L-\frac{1}{\omega C}+\frac{\omega L_{1}}{1-\omega^{2} L_{1} C_{1}}\right)^{2}\right]^{1 / 2}},
$$

where $V_{0}$ is the amplitude of the input voltage. Inspection shows that

$$
\left(I_{0}\right)_{\max }=\frac{V_{0}}{R}, \quad\left(I_{0}\right)_{\min }=0 .
$$

When $I_{0}$ is minimum, i.e., $I_{0}=0$,

$$
\omega L-\frac{1}{\omega C}+\frac{\omega L_{1}}{1-\omega^{2} L_{1} C_{1}}=\infty .
$$

The solutions of this equation are $\omega=0, \omega=\infty$, and $\omega=\frac{1}{\sqrt{L_{1} C_{1}}}$. Discarding the first two solutions, we have $\omega=\frac{1}{\sqrt{L_{1} C_{1}}}$ for the observation of the minimum current.

\textbf{Topic} :Circuit Analysis\\
\textbf{Book} :Problems and Solutions on Electromagnetism\\
\textbf{Final Answer} :\infty\\


\textbf{Solution} :Take the origin at the starting point of the wire and its direction as the $x$ direction and suppose the voltage amplitude at the starting point is $V_{0}$. Consider a segment $x$ to $x+d x$. By Kirchhoff's law we have

$$
\begin{aligned}
u(t, x) &=u(t, x+d x)+l d x \frac{\partial i(t, x)}{\partial t}+r i(t, x) d x \\
i(t, x) &=i(t, x+d x)+C d x \frac{\partial u(t, x)}{\partial t}
\end{aligned}
$$

i.e.

$$
\frac{-\partial u}{\partial x}=l \frac{\partial i}{\partial t}+r i, \quad \frac{-\partial i}{\partial x}=C \frac{\partial u}{\partial t} .
$$

Assuming solution of the form $e^{-j(\omega t-K x)}$, then

$$
\frac{\partial}{\partial t} \sim-j \omega, \quad \frac{\partial}{\partial x} \sim j K,
$$

and the above equations become

$$
\begin{aligned}
&i(r-j \omega l)+j K u=0, \\
&i(j K)-j \omega C u=0 .
\end{aligned}
$$

The condition that this system of equations has non-zero solutions is

$$
\mid \begin{gathered}
r-j \omega l \\
j K
\end{gathered}
$$

$$
\begin{gathered}
j K \\
-j \omega C
\end{gathered} \mid=-j \omega C(r-j \omega l)+K^{2}=0
$$

giving

$$
K=\sqrt{\omega^{2} l C+j \omega C r} .
$$

Let $K=\alpha+j \beta$, then

$$
\begin{aligned}
\alpha^{2}-\beta^{2} &=\omega^{2} l C \\
2 \alpha \beta &=\omega C r
\end{aligned}
$$

and we have

$$
\begin{aligned}
&u=V_{0} e^{-\beta x} e^{j(\alpha x-\omega t)} \\
&i=\frac{\omega C}{K} u=\frac{\omega C V_{0}}{\sqrt{\alpha^{2}+\beta^{2}}} e^{-\beta x} e^{j(\alpha x-\omega t+\varphi)}
\end{aligned}
$$

where we have made use of the fact that $u=V_{0}$ when $x=t=0$, and $\varphi$ is given by

$$
\tan \varphi=\frac{\beta}{\alpha} .
$$

The expressions can be simplified if

$$
r \ll \omega l,
$$

for we then have

$$
K=\omega \sqrt{l C}\left(1+j \frac{r}{\omega l}\right)^{\frac{1}{2}} \approx \omega \sqrt{l C}+j \frac{r}{2} \sqrt{\frac{C}{l}} .
$$



Accordingly,

$$
\begin{gathered}
u=V_{0} \exp [j \omega(\sqrt{l C} x-t)] \exp \left(\frac{-r}{2} \sqrt{\frac{C}{l} x}\right), \\
i=\frac{\omega C}{K} u \approx \sqrt{\frac{C}{l}} V_{0} \exp [-j \omega(\sqrt{l C} x-t)] \exp \left(-\frac{r}{2} \sqrt{\frac{C}{l}} x\right) .
\end{gathered}
$$

\textbf{Topic} :Circuit Analysis\\
\textbf{Book} :Problems and Solutions on Electromagnetism\\
\textbf{Final Answer} :\frac{\omega C}{K} u \approx \sqrt{\frac{C}{l}} V_{0} \exp [-j \omega(\sqrt{l C} x-t)] \exp \left(-\frac{r}{2} \sqrt{\frac{C}{l}} x\right) 
\end{gathered}\\


\textbf{Solution} :As $l \gg \rho$, the magnetic induction inside the solenoid has

$$
B=\mu_{0} \frac{N I}{l}
$$

and is along the axis of the solenoid. The magnetic flux linkage for the solenoid is

$$
\Psi=N B S=N \cdot \mu_{0} \frac{N I}{l} \cdot \pi \rho^{2}=\frac{\mu_{0} N^{2} I \pi \rho^{2}}{l},
$$

so the self-inductance is

$$
L=\frac{\Psi}{I}=\frac{\pi \mu_{0} N^{2} \rho^{2}}{l} .
$$

As $d \gg l$, the magnetic field produced by one solenoid at the location of the other can be approximated by that of a magnetic dipole. As the two solenoids are coaxial, this field may be expressed as $B_{M}=\frac{\mu_{0}}{4 \pi} \cdot \frac{2 m}{d^{3}}$ with $m=N I \pi \rho^{2}$, i.e.,

$$
B_{M}=\frac{\mu_{0} N I \rho^{2}}{2 d^{3}} .
$$

Hence

$$
\Psi_{M}=N B_{M} S=N \frac{\mu_{0} N I \rho^{2}}{2 d^{3}} \pi \rho^{2}=\frac{\mu_{0} N^{2} \rho^{2} \pi \rho^{2} I}{2 d^{3}},
$$



giving the mutual inductance as

$$
M=\frac{\Psi_{M}}{I}=\frac{\pi \mu_{0} N^{2} \rho^{4}}{2 d^{3}} .
$$

The units of $L$ and $M$ are $\mathrm{H}=\mathrm{A} \cdot \mathrm{s} / \mathrm{V}$.
\textbf{Topic} :Circuit Analysis\\
\textbf{Book} :Problems and Solutions on Electromagnetism\\
\textbf{Final Answer} :\frac{\pi \mu_{0} N^{2} \rho^{4}}{2 d^{3}}\\


\textbf{Solution} :As $l \gg \rho$, the magnetic induction inside the solenoid has

$$
B=\mu_{0} \frac{N I}{l}
$$

and is along the axis of the solenoid. The magnetic flux linkage for the solenoid is

$$
\Psi=N B S=N \cdot \mu_{0} \frac{N I}{l} \cdot \pi \rho^{2}=\frac{\mu_{0} N^{2} I \pi \rho^{2}}{l},
$$

so the self-inductance is

$$
L=\frac{\Psi}{I}=\frac{\pi \mu_{0} N^{2} \rho^{2}}{l} .
$$

As $d \gg l$, the magnetic field produced by one solenoid at the location of the other can be approximated by that of a magnetic dipole. As the two solenoids are coaxial, this field may be expressed as $B_{M}=\frac{\mu_{0}}{4 \pi} \cdot \frac{2 m}{d^{3}}$ with $m=N I \pi \rho^{2}$, i.e.,

$$
B_{M}=\frac{\mu_{0} N I \rho^{2}}{2 d^{3}} .
$$

Hence

$$
\Psi_{M}=N B_{M} S=N \frac{\mu_{0} N I \rho^{2}}{2 d^{3}} \pi \rho^{2}=\frac{\mu_{0} N^{2} \rho^{2} \pi \rho^{2} I}{2 d^{3}},
$$



giving the mutual inductance as

$$
M=\frac{\Psi_{M}}{I}=\frac{\pi \mu_{0} N^{2} \rho^{4}}{2 d^{3}} .
$$

The units of $L$ and $M$ are $\mathrm{H}=\mathrm{A} \cdot \mathrm{s} / \mathrm{V}$.

 Let the emf in the first circuit be $\varepsilon=V \cos \omega t=\operatorname{Re}\left(V e^{j \omega t}\right)$. Then we have for the two circuits

$$
\begin{gathered}
L \frac{d I_{1}}{d t}+M \frac{d I_{2}}{d t}+I_{1} R=V e^{j \omega t}, \\
L \frac{d I_{2}}{d t}+M \frac{d I_{1}}{d t}+I_{2} R=0 .
\end{gathered}
$$

As $I_{1}, I_{2} \sim e^{j \omega t}$, we have $\frac{d}{d t} \rightarrow j \omega$ and the above equations become

$$
\begin{aligned}
&j \omega L I_{1}+j \omega M I_{2}+I_{1} R=V e^{j \omega t}, \\
&j \omega L I_{2}+j \omega M I_{1}+I_{2} R=0 .
\end{aligned}
$$

(1) $\pm(2)$ give

$$
\begin{aligned}
&j \omega L\left(I_{1}+I_{2}\right)+j \omega M\left(I_{1}+I_{2}\right)+R\left(I_{1}+I_{2}\right)=V e^{j \omega t} \\
&j \omega L\left(I_{1}-I_{2}\right)-j \omega M\left(I_{1}-I_{2}\right)+R\left(I_{1}-I_{2}\right)=V e^{j \omega t}
\end{aligned}
$$

Hence

$$
I_{1}+I_{2}=\frac{V e^{j \omega t}}{j \omega(L+M)+R}, \quad I_{1}-I_{2}=\frac{V e^{j \omega t}}{j \omega(L-M)+R},
$$

whence

$$
\begin{aligned}
I_{2} &=\frac{1}{2}\left[\frac{V}{j \omega(L+M)+R}-\frac{V}{j \omega(L-M)+R}\right] e^{j \omega t} \\
&=\frac{-j \omega M V e^{j \omega t}}{[j \omega(L+M)+R][j \omega(L-M)+R]} \\
&=\frac{-j \omega M V e^{j \omega t}}{R^{2}-\omega^{2}\left(L^{2}-M^{2}\right)+2 j \omega L R} .
\end{aligned}
$$

Writing $\operatorname{Re} I_{2}=I_{20} \cos \left(\omega t+\varphi_{0}\right)$, we have

$$
\begin{aligned}
I_{20} &=\frac{\omega M V}{\sqrt{\left[R^{2}-\omega^{2}\left(L^{2}-M^{2}\right)\right]^{2}+4 \omega^{2} L^{2} R^{2}}}, \\
\varphi_{0} &=\pi-\arctan \frac{2 \omega L R}{R^{2}-\omega^{2}\left(L^{2}-M^{2}\right)}
\end{aligned}
$$

Using the given data and noting that $L \gg M$, we get

$$
\begin{aligned}
I_{20} \approx & \frac{\omega M V}{R^{2}+2 \omega^{2} L^{2}}=\frac{\mu_{0} \pi N^{2} \rho^{4} \omega V l^{2}}{2 d^{3}\left[R^{2} l^{2}+2 \omega^{2} \mu_{0}^{2} \pi^{2} N^{4} \rho^{4}\right]}, \\
\varphi_{0} & \approx \pi-\arctan \frac{2 \omega L R}{R^{2}-\omega^{2} L^{2}} \\
&=\pi-\arctan \frac{2 \mu_{0} \pi \omega R N^{2} \rho^{2} l}{R^{2} l^{2}-\omega^{2} \mu_{0}^{2} \pi^{2} N^{4} \rho^{4}}
\end{aligned}
$$
\textbf{Topic} :Circuit Analysis\\
\textbf{Book} :Problems and Solutions on Electromagnetism\\
\textbf{Final Answer} :\pi-\arctan \frac{2 \mu_{0} \pi \omega R N^{2} \rho^{2} l}{R^{2} l^{2}-\omega^{2} \mu_{0}^{2} \pi^{2} N^{4} \rho^{4}}\\


\textbf{Solution} :As $l \gg \rho$, the magnetic induction inside the solenoid has

$$
B=\mu_{0} \frac{N I}{l}
$$

and is along the axis of the solenoid. The magnetic flux linkage for the solenoid is

$$
\Psi=N B S=N \cdot \mu_{0} \frac{N I}{l} \cdot \pi \rho^{2}=\frac{\mu_{0} N^{2} I \pi \rho^{2}}{l},
$$

so the self-inductance is

$$
L=\frac{\Psi}{I}=\frac{\pi \mu_{0} N^{2} \rho^{2}}{l} .
$$

As $d \gg l$, the magnetic field produced by one solenoid at the location of the other can be approximated by that of a magnetic dipole. As the two solenoids are coaxial, this field may be expressed as $B_{M}=\frac{\mu_{0}}{4 \pi} \cdot \frac{2 m}{d^{3}}$ with $m=N I \pi \rho^{2}$, i.e.,

$$
B_{M}=\frac{\mu_{0} N I \rho^{2}}{2 d^{3}} .
$$

Hence

$$
\Psi_{M}=N B_{M} S=N \frac{\mu_{0} N I \rho^{2}}{2 d^{3}} \pi \rho^{2}=\frac{\mu_{0} N^{2} \rho^{2} \pi \rho^{2} I}{2 d^{3}},
$$



giving the mutual inductance as

$$
M=\frac{\Psi_{M}}{I}=\frac{\pi \mu_{0} N^{2} \rho^{4}}{2 d^{3}} .
$$

The units of $L$ and $M$ are $\mathrm{H}=\mathrm{A} \cdot \mathrm{s} / \mathrm{V}$.

 Let the emf in the first circuit be $\varepsilon=V \cos \omega t=\operatorname{Re}\left(V e^{j \omega t}\right)$. Then we have for the two circuits

$$
\begin{gathered}
L \frac{d I_{1}}{d t}+M \frac{d I_{2}}{d t}+I_{1} R=V e^{j \omega t}, \\
L \frac{d I_{2}}{d t}+M \frac{d I_{1}}{d t}+I_{2} R=0 .
\end{gathered}
$$

As $I_{1}, I_{2} \sim e^{j \omega t}$, we have $\frac{d}{d t} \rightarrow j \omega$ and the above equations become

$$
\begin{aligned}
&j \omega L I_{1}+j \omega M I_{2}+I_{1} R=V e^{j \omega t}, \\
&j \omega L I_{2}+j \omega M I_{1}+I_{2} R=0 .
\end{aligned}
$$

(1) $\pm(2)$ give

$$
\begin{aligned}
&j \omega L\left(I_{1}+I_{2}\right)+j \omega M\left(I_{1}+I_{2}\right)+R\left(I_{1}+I_{2}\right)=V e^{j \omega t} \\
&j \omega L\left(I_{1}-I_{2}\right)-j \omega M\left(I_{1}-I_{2}\right)+R\left(I_{1}-I_{2}\right)=V e^{j \omega t}
\end{aligned}
$$

Hence

$$
I_{1}+I_{2}=\frac{V e^{j \omega t}}{j \omega(L+M)+R}, \quad I_{1}-I_{2}=\frac{V e^{j \omega t}}{j \omega(L-M)+R},
$$

whence

$$
\begin{aligned}
I_{2} &=\frac{1}{2}\left[\frac{V}{j \omega(L+M)+R}-\frac{V}{j \omega(L-M)+R}\right] e^{j \omega t} \\
&=\frac{-j \omega M V e^{j \omega t}}{[j \omega(L+M)+R][j \omega(L-M)+R]} \\
&=\frac{-j \omega M V e^{j \omega t}}{R^{2}-\omega^{2}\left(L^{2}-M^{2}\right)+2 j \omega L R} .
\end{aligned}
$$

Writing $\operatorname{Re} I_{2}=I_{20} \cos \left(\omega t+\varphi_{0}\right)$, we have

$$
\begin{aligned}
I_{20} &=\frac{\omega M V}{\sqrt{\left[R^{2}-\omega^{2}\left(L^{2}-M^{2}\right)\right]^{2}+4 \omega^{2} L^{2} R^{2}}}, \\
\varphi_{0} &=\pi-\arctan \frac{2 \omega L R}{R^{2}-\omega^{2}\left(L^{2}-M^{2}\right)}
\end{aligned}
$$

Using the given data and noting that $L \gg M$, we get

$$
\begin{aligned}
I_{20} \approx & \frac{\omega M V}{R^{2}+2 \omega^{2} L^{2}}=\frac{\mu_{0} \pi N^{2} \rho^{4} \omega V l^{2}}{2 d^{3}\left[R^{2} l^{2}+2 \omega^{2} \mu_{0}^{2} \pi^{2} N^{4} \rho^{4}\right]}, \\
\varphi_{0} & \approx \pi-\arctan \frac{2 \omega L R}{R^{2}-\omega^{2} L^{2}} \\
&=\pi-\arctan \frac{2 \mu_{0} \pi \omega R N^{2} \rho^{2} l}{R^{2} l^{2}-\omega^{2} \mu_{0}^{2} \pi^{2} N^{4} \rho^{4}}
\end{aligned}
$$

 The calculation in (b) is valid only under quasistationary conditions. This requires

$$
d \ll \lambda=\frac{2 \pi c}{\omega},
$$

or

$$
\omega \ll \frac{2 \pi d}{c} .
$$



MATHPIX IMAGE







MATHPIX IMAGE




\textbf{Topic} :Circuit Analysis\\
\textbf{Book} :Problems and Solutions on Electromagnetism\\
\textbf{Final Answer} :\pi-\arctan \frac{2 \mu_{0} \pi \omega R N^{2} \rho^{2} l}{R^{2} l^{2}-\omega^{2} \mu_{0}^{2} \pi^{2} N^{4} \rho^{4}}\\


\textbf{Solution} :$f=\frac{\omega}{2 \pi}=10^{8} \mathrm{~Hz}$.
\\
\textbf{Topic} :Electromagnetic Waves\\
\textbf{Book} :Problems and Solutions on Electromagnetism\\
\textbf{Final Answer} :\pi-\arctan \frac{2 \mu_{0} \pi \omega R N^{2} \rho^{2} l}{R^{2} l^{2}-\omega^{2} \mu_{0}^{2} \pi^{2} N^{4} \rho^{4}}\\


\textbf{Solution} :$f=\frac{\omega}{2 \pi}=10^{8} \mathrm{~Hz}$.
 $\lambda=\frac{2 \pi}{k}=3 \mathrm{~m}$.
\\
\textbf{Topic} :Electromagnetic Waves\\
\textbf{Book} :Problems and Solutions on Electromagnetism\\
\textbf{Final Answer} :\pi-\arctan \frac{2 \mu_{0} \pi \omega R N^{2} \rho^{2} l}{R^{2} l^{2}-\omega^{2} \mu_{0}^{2} \pi^{2} N^{4} \rho^{4}}\\


\textbf{Solution} :$f=\frac{\omega}{2 \pi}=10^{8} \mathrm{~Hz}$.
 $\lambda=\frac{2 \pi}{k}=3 \mathrm{~m}$.
 The wave is propagating along the positive $x$ direction.
\\
\textbf{Topic} :Electromagnetic Waves\\
\textbf{Book} :Problems and Solutions on Electromagnetism\\
\textbf{Final Answer} :\pi-\arctan \frac{2 \mu_{0} \pi \omega R N^{2} \rho^{2} l}{R^{2} l^{2}-\omega^{2} \mu_{0}^{2} \pi^{2} N^{4} \rho^{4}}\\


\textbf{Solution} :$f=\frac{\omega}{2 \pi}=10^{8} \mathrm{~Hz}$.
 $\lambda=\frac{2 \pi}{k}=3 \mathrm{~m}$.
 The wave is propagating along the positive $x$ direction.
 As $\mathbf{E}, \mathbf{B}$, and $k$ form a right-hand set, $B$ is parallel to $k \times \mathbf{E}$. As $\mathbf{k}$ and $\mathbf{E}$ are respectively in the $x$ and $y$ directions the magnetic field is in the $z$ direction.

\textbf{Topic} :Electromagnetic Waves\\
\textbf{Book} :Problems and Solutions on Electromagnetism\\
\textbf{Final Answer} :\pi-\arctan \frac{2 \mu_{0} \pi \omega R N^{2} \rho^{2} l}{R^{2} l^{2}-\omega^{2} \mu_{0}^{2} \pi^{2} N^{4} \rho^{4}}\\


\textbf{Solution} :If we express the current density $J(r, t)$ as the Fourier integral

$$
J(\mathbf{r}, t)=\int_{-\infty}^{\infty} J_{\omega}(\mathbf{r}) \mathrm{e}^{-i \omega t} d \omega,
$$

the retarded vector potential can be rewritten:

$$
\begin{aligned}
\mathbf{A}(\mathbf{r}, t) &=\frac{\mu_{0}}{4 \pi} \int \frac{J\left(\mathbf{r}^{\prime}, t-\frac{\left|\mathbf{r}-\mathbf{r}^{\prime}\right|}{c}\right)}{\left|\mathbf{r}-\mathbf{r}^{\prime}\right|} d^{3} r^{\prime} \\
&=\frac{\mu_{0}}{4 \pi} \int \frac{1}{\left|\mathbf{r}-\mathbf{r}^{\prime}\right|} d^{3} r^{\prime} \int_{-\infty}^{\infty} \mathbf{J}_{\omega}\left(\mathbf{r}^{\prime}\right) \exp \left[-i \omega\left(t-\frac{\left|\mathbf{r}-\mathbf{r}^{\prime}\right|}{c}\right)\right] d \omega \\
&=\frac{\mu_{0}}{4 \pi} \int_{-\infty}^{\infty} e^{-i \omega t} d \omega \int \frac{\mathbf{J}_{\omega}\left(\mathbf{r}^{\prime}\right) e^{i K\left|\mathbf{r}-\mathbf{r}^{\prime}\right|}}{\left|\mathbf{r}-\mathbf{r}^{\prime}\right|} d^{3} r^{\prime}
\end{aligned}
$$

where the volume integral is over all space. Thus the Fourier transform of the vector potential is

$$
\mathbf{A}_{\omega}(\mathbf{r})=\frac{\mu_{0}}{4 \pi} \int \frac{J_{\omega}\left(\mathbf{r}^{\prime}\right) \exp \left[i K\left|\mathbf{r}-\mathbf{r}^{\prime}\right|\right]}{\left|\mathbf{r}-\mathbf{r}^{\prime}\right|} d^{3} r^{\prime}
$$

Similarly the Fourier transform of the scalar potential $\phi(\mathbf{r}, t)$ is

$$
\phi_{\omega}(\mathbf{r})=\frac{1}{4 \pi \varepsilon_{0}} \int \rho_{\omega}\left(\mathbf{r}^{\prime}\right) \frac{\exp \left[i K\left|\mathbf{r}-\mathbf{r}^{\prime}\right|\right]}{\left|\mathbf{r}-\mathbf{r}^{\prime}\right|} d^{3} r^{\prime}
$$

The continuity equation that expresses charge-current conservation, $\frac{\partial \rho}{\partial t}+$ $\nabla \cdot \mathbf{J}=0$, is written in terms of Fourier integrals as

$$
\frac{\partial}{\partial t} \int_{-\infty}^{\infty} \rho_{\omega}(\mathbf{r}) e^{-i \omega t} d \omega+\nabla \cdot \int_{-\infty}^{\infty} \mathbf{J}_{\omega}(\mathbf{r}) e^{-i \omega t} d \omega=0
$$

or

$$
\int_{-\infty}^{\infty}\left[-i \omega \rho_{\omega}(\mathbf{r})+\nabla \cdot \mathbf{J}_{\omega}(\mathbf{r})\right] e^{-i \omega t} d \omega=0
$$

Hence

$$
\nabla \cdot J_{\omega}-i \omega \rho_{\omega}=0 .
$$

In the far zone $r \rightarrow \infty$, we make the approximation

$$
\left|\mathbf{r}-\mathbf{r}^{\prime}\right| \approx r-\frac{\mathbf{r} \cdot \mathbf{r}^{\prime}}{r} .
$$

Then

$$
\begin{aligned}
J_{\omega} \frac{e^{i K\left|\mathbf{r}-\mathbf{r}^{\prime}\right|}}{\left|\mathbf{r}-\mathbf{r}^{\prime}\right|} & \approx J_{\omega} \frac{e^{i K r}}{r}\left(1-i K \frac{\mathbf{r} \cdot \mathbf{r}^{\prime}}{r}-\cdots\right)\left(1+\frac{\mathbf{r} \cdot \mathbf{r}^{\prime}}{\mathbf{r}^{2}}+\cdots\right) \\
& \approx J_{\omega} \frac{e^{i K r}}{r}\left(1-i K \frac{\mathbf{r} \cdot \mathbf{r}^{\prime}}{r}\right)
\end{aligned}
$$

Consider

$$
\nabla^{\prime} \cdot\left(x^{\prime} \mathbf{J}_{\omega}\right)=J_{\omega x^{\prime}}+x^{\prime} \nabla \cdot \mathbf{J}_{\omega} .
$$

As

$$
\int \nabla^{\prime} \cdot\left(x^{\prime} \mathbf{J}_{\omega}\right) d^{3} r^{\prime}=\oint x^{\prime} \mathbf{J}_{\omega} \cdot d S^{\prime}=0
$$

for a finite current distribution,

$$
\begin{aligned}
\int \mathbf{J}_{\omega} d^{3} r^{\prime} &=-\int \mathbf{r}^{\prime} \nabla^{\prime} \cdot \mathbf{J}_{\omega} d^{3} r^{\prime} \\
&=-i \omega \int \mathbf{r}^{\prime} \rho_{\omega}\left(\mathbf{r}^{\prime}\right) d^{3} r^{\prime}
\end{aligned}
$$

Also

$$
\begin{aligned}
\mathbf{J}_{\omega}\left(\mathbf{r} \cdot \mathbf{r}^{\prime}\right) &=\frac{1}{2}\left[\mathbf{J}_{\omega}\left(\mathbf{r} \cdot \mathbf{r}^{\prime}\right)-\left(\mathbf{r} \cdot \mathbf{J}_{\omega}\right) \mathbf{r}^{\prime}\right]+\frac{1}{2}\left[\mathbf{J}_{\omega}\left(\mathbf{r} \cdot \mathbf{r}^{\prime}\right)+\left(\mathbf{r} \cdot \mathbf{J}_{\omega}\right) \mathbf{r}^{\prime}\right] \\
&=\frac{1}{2}\left(\mathbf{r}^{\prime} \times \mathbf{J}_{\omega}\right) \times \mathbf{r}+\frac{1}{2}\left[\mathbf{J}_{\omega}\left(\mathbf{r} \cdot \mathbf{r}^{\prime}\right)+\left(\mathbf{r} \cdot \mathbf{J}_{\omega}\right) \mathbf{r}^{\prime}\right] .
\end{aligned}
$$

The second term on the right-hand side would give rise to an electric quadrupole field. It is neglected as we are interested only in the lowest multipole field. Hence

$$
\mathbf{A}_{\omega}(\mathbf{r}) \underset{r \rightarrow \infty}{\longrightarrow} \frac{\mu_{0}}{4 \pi}\left(-i \omega \mathbf{p}_{\omega} \frac{e^{i K r}}{r}-i K \mathbf{m}_{\omega} \times \frac{\mathbf{r}}{r} e^{i K r}\right)
$$

where

$$
\mathbf{p}_{\omega}=\int \mathbf{r}^{\prime} \rho_{\omega}\left(\mathbf{r}^{\prime}\right) d^{3} r^{\prime}, \quad \mathbf{m}_{\omega}=\frac{1}{2} \int \mathbf{r}^{\prime} \times \mathbf{J}_{\omega}\left(\mathbf{r}^{\prime}\right) d^{3} r^{\prime} .
$$

are the electric and magnetic dipole moments of the sources. To find $\mathbf{E}_{\omega}(\mathbf{r})$ in the far zone, we use Maxwell's equation

$$
\nabla \times \mathbf{B}=\mu_{0} \epsilon_{0} \frac{\partial \mathbf{E}}{\partial t}+\mu_{0} \mathbf{J}=\frac{1}{c^{2}} \frac{\partial \mathbf{E}}{\partial t}
$$

assuming the source to be finite. In terms of Fourier transforms, the above becomes

$$
\nabla \times \int \mathbf{B}_{\omega}(\mathbf{r}) e^{-i \omega t} d \omega=\frac{1}{c^{2}} \frac{\partial}{\partial t} \int \mathbf{E}_{\omega}(\mathbf{r}) e^{-i \omega t} d \omega
$$

or

$$
\int \nabla \times \mathbf{B}_{\omega}(\mathbf{r}) e^{-i \omega t} d \omega=-\int \frac{i \omega}{c^{2}} \mathbf{E}_{\omega}(\mathbf{r}) e^{-i \omega t} d \omega,
$$

giving

$$
\mathbf{E}_{\omega}(\mathbf{r})=\frac{i c^{2}}{\omega} \nabla \times \mathbf{B}_{\omega} .
$$

Similarly, from $B=\nabla \times \mathbf{A}$ we have

$$
\mathbf{B}_{\omega}(\mathbf{r})=\nabla \times \mathbf{A}_{\omega}(\mathbf{r}) .
$$

For a current density $\mathbf{J}_{\omega}(\mathbf{r})=\mathbf{r} f(r)$, we have

$$
\mathbf{m}_{\omega}=\frac{1}{2} \int \mathbf{r}^{\prime} \times \mathbf{r}^{\prime} f\left(r^{\prime}\right) d^{3} r^{\prime}=0 .
$$

Also,

$$
\mathbf{p}_{\omega}=\int \mathbf{r}^{\prime} \rho_{\omega}\left(\mathbf{r}^{\prime}\right) d^{3} r^{\prime}=\frac{i}{\omega} \int \mathbf{J}_{\omega}\left(\mathbf{r}^{\prime}\right) d^{3} r^{\prime}=\frac{i}{\omega} \int \mathbf{r}^{\prime} f\left(r^{\prime}\right) d^{3} r^{\prime},
$$

using (1) and assuming the current distribution to be finite.

Hence, using (2),

$$
\mathbf{A}_{\omega}(\mathbf{r}) \approx \frac{\mu_{0} e^{i K r}}{4 \pi r} \int \mathbf{r}^{\prime} f\left(r^{\prime}\right) d^{3} r^{\prime}
$$

Then

$$
\begin{aligned}
\mathbf{B}_{\omega}(\mathbf{r}) &=\nabla \times \mathbf{A}_{\omega}(\mathbf{r}) \approx \frac{i \mu_{0} K e^{i K r}}{4 \pi r} \hat{\mathbf{r}} \times \int \mathbf{r}^{\prime} f\left(r^{\prime}\right) d^{3} r^{\prime} \\
\mathbf{E}_{\omega}(\mathbf{r}) &=\frac{i c^{2}}{\omega} \nabla \times \mathbf{B}_{\omega}(\mathbf{r}) \\
& \approx-i c \frac{\mu_{0} K e^{i K r}}{4 \pi r} \hat{\mathbf{r}} \times\left[\hat{\mathbf{r}} \times \int \mathbf{r}^{\prime} f\left(r^{\prime}\right) d^{3} r^{\prime}\right]
\end{aligned}
$$

where $\hat{r}=\frac{r}{r}$, terms of higher orders in $\frac{1}{r}$ having been neglected.



\textbf{Topic} :Electromagnetic Waves\\
\textbf{Book} :Problems and Solutions on Electromagnetism\\
\textbf{Final Answer} :\frac{i c^{2}}{\omega} \nabla \times \mathbf{B}_{\omega}(\mathbf{r}) \\
& \approx-i c \frac{\mu_{0} K e^{i K r}}{4 \pi r} \hat{\mathbf{r}} \times\left[\hat{\mathbf{r}} \times \int \mathbf{r}^{\prime} f\left(r^{\prime}\right) d^{3} r^{\prime}\right]\\


\textbf{Solution} :Maxwell's equations in a source-free, homogeneous non-conducting medium are

$$
\left\{\begin{array}{l}
\nabla \times \mathbf{E}=-\frac{\partial \mathbf{B}}{\partial t}, \\
\nabla \times \mathbf{H}=\frac{\partial \mathbf{D}}{\partial t}, \\
\nabla \cdot \mathbf{D}=0, \\
\nabla \cdot \mathbf{B}=0,
\end{array}\right.
$$

where $\mathbf{D}=\varepsilon \mathbf{E}, \mathbf{B}=\mu \mathbf{H}, \varepsilon, \mu$ being constants. As

$$
\nabla \times(\nabla \times \mathbf{E})=\nabla(\nabla \cdot \mathbf{E})-\nabla^{2} \mathbf{E}=-\nabla^{2} \mathbf{E}
$$

and Eq.
(2) can be written as

$$
\nabla \times\left(\frac{\partial \mathbf{B}}{\partial t}\right)=\mu \varepsilon \frac{\partial^{2} \mathbf{E}}{\partial t^{2}},
$$

Eq.
(1) gives

$$
\nabla^{2} \mathbf{E}-\mu \varepsilon \frac{\partial^{2} \mathbf{E}}{\partial t^{2}}=0 .
$$

Similarly, one finds

$$
\nabla^{2} \mathbf{B}-\mu \varepsilon \frac{\partial^{2} \mathbf{B}}{\partial t^{2}}=0 .
$$

Thus each of the field vectors $\mathbf{E}$ and $\mathbf{B}$ satisfies the wave equation. A comparison with the standard wave equation $\nabla^{2} \mathbf{E}-\frac{1}{v^{2}} \frac{\partial^{2} \mathbf{E}}{\partial t^{2}}=0$ shows that the wave velocity is

$$
v=\frac{1}{\sqrt{\varepsilon \mu}} .
$$

Solutions corresponding to plane electromagnetic waves of angular frequency $\omega$ are

$$
\begin{aligned}
&\mathbf{E}(\mathbf{r}, t)=\mathbf{E}_{0} e^{i(\mathbf{k} \cdot \mathbf{r}-\omega t)}, \\
&\mathbf{B}(\mathbf{r}, t)=\mathbf{B}_{0} e^{i(\mathbf{k} \cdot \mathbf{r}-\omega t)},
\end{aligned}
$$

where the wave vector $\mathbf{k}$ and the amplitudes $\mathbf{E}_{0}$ and $\mathbf{B}_{0}$ form an orthogonal right-hand set. Furthermore $v=\frac{\omega}{k}$, and $\mathbf{E}, \mathbf{B}$ are related by

$$
\mathbf{B}=\sqrt{\mu \varepsilon} \frac{\mathbf{k}}{k} \times \mathbf{E} .
$$
\textbf{Topic} :Electromagnetic Waves\\
\textbf{Book} :Problems and Solutions on Electromagnetism\\
\textbf{Final Answer} :\sqrt{\mu \varepsilon} \frac{\mathbf{k}}{k} \times \mathbf{E}\\


\textbf{Solution} :A plane electromagnetic wave can be decomposed into two polarized components with mutually perpendicular planes of polarization. In the interface we take an arbitrary direction as the $x$ direction, and the direction perpendicular to it as the $y$ direction, and decompose the incident wave into two polarized components with $\mathbf{E}$ parallel to these two directions. We also decompose the reflected and refracted waves in a similar manner. As $\mathbf{E}, \mathbf{H}$ and $k$ form a right-hand set, we have for the two polarizations:

$$
\begin{array}{cc}
\frac{x \text {-polarization }}{E_{x}, H_{y}} & \frac{y \text {-polarization }}{E_{y},-H_{x}} \\
E_{x}^{\prime},-H_{y}^{\prime} & E_{y}^{\prime}, H_{x}^{\prime} \\
E_{x}^{\prime \prime}, H_{y}^{\prime \prime} & E_{y}^{\prime \prime},-H_{x}^{\prime \prime}
\end{array}
$$

The boundary condition that $E_{t}$ and $H_{t}$ are continuous across the interface gives for the $x$-polarization

$$
E_{x}+E_{x}^{\prime}=E_{x}^{\prime \prime}
$$



$$
H_{y}-H_{y}^{\prime}=H_{y}^{\prime \prime} .
$$

For a plane wave we have $\sqrt{\mu}|H|=\sqrt{\varepsilon}|E|$. With $\mu \approx \mu_{0}$ and $\sqrt{\frac{\varepsilon}{\varepsilon_{0}}}=n$,
(2) becomes

$$
E_{x}-E_{x}^{\prime}=n E_{x}^{\prime \prime}
$$

(1) and (3) give

$$
E_{x}^{\prime}=\left(\frac{1-n}{1+n}\right) E_{x} .
$$

Since for normal incidence, the plane of incidence is arbitrary, the same result holds for $y$-polarization. Hence for normal incidence, we have

$$
E^{\prime}=\left(\frac{1-n}{1+n}\right) E .
$$

The intensity of a wave is given by the magnitude of the Poynting vector $\mathbf{N}$ over one period. We have

$$
\mathbf{N}=\operatorname{Re} \mathbf{E} \times \operatorname{Re} \mathbf{H}=\frac{1}{4}\left(\mathbf{E} \times \mathbf{H}+\mathbf{E}^{*} \times \mathbf{H}^{*}+\mathbf{E} \times \mathbf{H}^{*}+\mathbf{E}^{*} \times \mathbf{H}\right) .
$$

As the first two terms in the last expression contain the time factor $e^{\pm 2 i \omega t}$, they vanish on taking average over one period. Hence the intensity is

$$
I=\langle N\rangle=\frac{1}{2} \operatorname{Re}\left|\mathbf{E} \times \mathbf{H}^{*}\right|=\frac{1}{2} E H^{*}=\frac{1}{2} \sqrt{\frac{\varepsilon}{\mu}} E_{0}^{2},
$$

$\mathbf{E}_{0}$ being the amplitude of the $\mathbf{E}$ field.

Therefore the coefficient of reflection is

$$
R=\frac{E_{0}^{\prime 2}}{E_{0}^{2}}=\left(\frac{1-n}{1+n}\right)^{2} .
$$

 The average momentum density of a wave is given by $G=\frac{(N)}{v^{2}}=$ $\frac{I}{v^{2}}$. So the average momentum impinging normally on a unit area per unit time is $G v$. The radiation pressure exerted on the glass plate is therefore

$$
\begin{aligned}
P &=G c-\left(-G^{\prime} c\right)-G^{\prime \prime} v \\
&=G c\left(1+\frac{G^{\prime}}{G}-\frac{G^{\prime \prime}}{G} \frac{v}{c}\right) \\
&=\frac{I}{c}\left(1+\frac{I^{\prime}}{I}-\frac{c}{v} \frac{I^{\prime \prime}}{I}\right) .
\end{aligned}
$$

$(1)+(3)$ gives

$$
\frac{E_{0}^{\prime \prime}}{E_{0}}=\frac{2}{1+n},
$$

or

$$
\frac{I^{\prime \prime}}{I}=\sqrt{\frac{\varepsilon}{\varepsilon_{0}}} \frac{E_{0}^{\prime \prime 2}}{E_{0}^{2}} \frac{4 n}{(1+n)^{2}} .
$$

With $\frac{I^{\prime}}{I}=\left(\frac{1-n}{1+n}\right)^{2}$ also, we have

$$
\begin{aligned}
P &=\frac{I}{c}\left[1+\left(\frac{1-n}{1+n}\right)^{2}-\frac{4 n^{2}}{(1+n)^{2}}\right] \\
&=2 \frac{I}{c}\left(\frac{1-n}{1+n}\right) .
\end{aligned}
$$

\textbf{Topic} :Electromagnetic Waves\\
\textbf{Book} :Problems and Solutions on Electromagnetism\\
\textbf{Final Answer} :2 \frac{I}{c}\left(\frac{1-n}{1+n}\right)\\


\textbf{Solution} :For normal incidence the plane of incidence is arbitrary. So we can take the electric vector as in the plane of the diagram in Fig. 4.6.

MATHPIX IMAGE

Fig. $4.6$

The incident wave is represented by

$$
\begin{gathered}
\mathbf{E}_{1}=\mathbf{E}_{10} e^{i\left(k_{1} z-\omega t\right)}, \\
\mathbf{B}_{1}=\mathbf{B}_{10} e^{i\left(k_{1} z-\omega t\right)}, \\
B_{10}=\frac{n}{c} E_{10}, \quad k_{1}=\frac{\omega}{c} n ;
\end{gathered}
$$

the reflected wave by

$$
\begin{gathered}
\mathbf{E}_{2}=\mathbf{E}_{20} e^{-i\left(k_{2} z+\omega t\right)}, \\
\mathbf{B}_{2}=\mathbf{B}_{20} e^{-i\left(k_{2} z+\omega t\right)}, \\
B_{20}=\frac{n E_{20}}{c}, \quad k_{2}=\frac{\omega}{c} n ;
\end{gathered}
$$

and the transmitted wave by

$$
\begin{gathered}
\mathbf{E}_{3}=\mathbf{E}_{30} e^{i\left(k_{3} z-\omega t\right)}, \\
\mathbf{B}_{3}=\mathbf{B}_{30} e^{i\left(k_{3} z-\omega t\right)}, \\
B_{30}=\frac{n_{2} E_{30}}{c}, \quad k_{3}=\frac{\omega}{c} n_{2} .
\end{gathered}
$$

The boundary condition at the interface is that $E_{t}$ and $H_{t}$ are continuous. Thus

$$
E_{10}-E_{20}=E_{30},
$$



$$
B_{10}+B_{20}=\frac{\mu}{\mu_{2}} B_{30} \approx B_{30},
$$

assuming the media to be non-ferromagnetic so that $\mu \approx \mu_{2} \approx \mu_{0}$. Equation (2) can be written as

$$
E_{10}+E_{20}=\frac{n_{2}}{n} E_{30} .
$$

(1) and (3) give

$$
E_{20}=\frac{n_{2}-n}{n_{2}+n} E_{10}=\frac{i n \rho}{2 n+i n \rho} E_{10}=\frac{\rho}{\sqrt{\rho^{2}+4}} e^{i \varphi} E_{10},
$$

with

$$
\tan \varphi=\frac{2}{\rho} .
$$

The phase shift of the electric vector of the reflected wave with respect to that of the incident wave is therefore

$$
\varphi=\arctan \left(\frac{2}{\rho}\right) .
$$

\textbf{Topic} :Electromagnetic Waves\\
\textbf{Book} :Problems and Solutions on Electromagnetism\\
\textbf{Final Answer} :\arctan \left(\frac{2}{\rho}\right)\\


\textbf{Solution} :Using Maxwell's equation

$$
\nabla \times \mathbf{B}=\frac{1}{c} \frac{\partial \mathrm{E}}{\partial t}
$$

and the definition $k=\frac{\omega}{c}$ for empty space, we obtain

$$
\mathbf{E}=\frac{i c}{\omega} \nabla \times \mathbf{B}=\frac{i}{k}\left|\begin{array}{ccc}
\mathbf{e}_{x} & \mathbf{e}_{y} & \mathbf{e}_{z} \\
\frac{\partial}{\partial x} & \frac{\partial}{\partial y} & 0 \\
0 & 0 & B_{z}
\end{array}\right|
$$

where $\frac{\partial}{\partial z}=0$ as $\mathbf{B}$ does not depend on $z$.

Hence

$$
\begin{gathered}
E_{x}=\operatorname{Im}\left(\frac{i}{k} B_{0} e^{a x} i k e^{i w}\right)=-B_{0} e^{a x} \sin w \\
E_{y}=\operatorname{Im}\left(-\frac{i}{k} B_{0} a e^{a x} e^{i w}\right)=-\frac{a c}{\omega} B_{0} e^{a x} \cos w \\
E_{z}=0
\end{gathered}
$$
\textbf{Topic} :Electromagnetic Waves\\
\textbf{Book} :Problems and Solutions on Electromagnetism\\
\textbf{Final Answer} :0
\end{gathered}\\


\textbf{Solution} :Using Maxwell's equation

$$
\nabla \times \mathbf{B}=\frac{1}{c} \frac{\partial \mathrm{E}}{\partial t}
$$

and the definition $k=\frac{\omega}{c}$ for empty space, we obtain

$$
\mathbf{E}=\frac{i c}{\omega} \nabla \times \mathbf{B}=\frac{i}{k}\left|\begin{array}{ccc}
\mathbf{e}_{x} & \mathbf{e}_{y} & \mathbf{e}_{z} \\
\frac{\partial}{\partial x} & \frac{\partial}{\partial y} & 0 \\
0 & 0 & B_{z}
\end{array}\right|
$$

where $\frac{\partial}{\partial z}=0$ as $\mathbf{B}$ does not depend on $z$.

Hence

$$
\begin{gathered}
E_{x}=\operatorname{Im}\left(\frac{i}{k} B_{0} e^{a x} i k e^{i w}\right)=-B_{0} e^{a x} \sin w \\
E_{y}=\operatorname{Im}\left(-\frac{i}{k} B_{0} a e^{a x} e^{i w}\right)=-\frac{a c}{\omega} B_{0} e^{a x} \cos w \\
E_{z}=0
\end{gathered}
$$

 If the wave form remains unchanged during propagation, we have

$$
d w=k d y-w d t=0,
$$

or $\frac{d y}{d t}=\frac{w}{k}=c$. Hence the wave propagates along the $y$ direction with a speed $v=c$.
\textbf{Topic} :Electromagnetic Waves\\
\textbf{Book} :Problems and Solutions on Electromagnetism\\
\textbf{Final Answer} :0\\


\textbf{Solution} :With the coordinates shown in Fig. $4.7$ and writing $z$ for $d$, the electric field of the incident wave is $\mathbf{E}_{0} \cos (k z-\omega t)$. Because the electric field changes phase by $\pi$ on reflection from the interface, the amplitude $\mathbf{E}_{0}^{\prime}$ of the reflected wave is opposite in direction to $\mathbf{E}_{0}$. A fraction $\rho$ of the energy is reflected. As energy is proportional to $E_{0}^{2}$, we have

$$
E_{0}^{\prime 2}=\rho E_{0}^{2} .
$$

Thus the electric field of the reflected wave is

$$
\mathbf{E}^{\prime}=-\sqrt{\rho} \mathbf{E}_{0} \cos (-k z-\omega t) .
$$

Hence the total electric field in the first medium is

$$
\mathbf{E}=\mathbf{E}_{0} \cos (k z-\omega t)-\sqrt{\rho} \mathbf{E}_{0} \cos (k z+\omega t),
$$

giving

$$
E^{2}=E_{0}^{2} \cos ^{2}(k z-\omega t)+\rho E_{0}^{2} \cos ^{2}(k z+\omega t)-\sqrt{\rho} E_{0}^{2}[\cos (2 k z)+\cos (2 \omega t)] \text {. }
$$

Taking average over a period $T=\frac{2 \pi}{\omega}$ we have

$$
\left\langle E^{2}\right\rangle=\frac{1}{T} \int_{0}^{T} E^{2} d t=\frac{(1+\rho) E_{0}^{2}}{2}-\sqrt{\rho} E_{0}^{2} \cos (2 k z) .
$$

When $k z=m \pi$, or $z=\frac{m c}{2 \nu n_{1}}$, where $m$ is an integer $0,1,2, \ldots,\left\langle E^{2}\right\rangle$ will be minimum with the value

$$
\left\langle E^{2}\right\rangle_{\min }=\frac{(1-\sqrt{\rho})^{2}}{2} E_{0}^{2} .
$$

When $k z=\frac{(2 m+1) \pi}{2}$, or $z=\frac{(2 m+1) c}{4 \nu n_{1}}$, where $m=0,1,2, \ldots,\left\langle E^{2}\right\rangle$ is maximum with the value

$$
\left\langle E^{2}\right\rangle_{\max }=\frac{(1+\sqrt{\rho})^{2}}{2} E_{0}^{2} .
$$
\textbf{Topic} :Electromagnetic Waves\\
\textbf{Book} :Problems and Solutions on Electromagnetism\\
\textbf{Final Answer} :\frac{(1+\sqrt{\rho})^{2}}{2} E_{0}^{2}\\


\textbf{Solution} :With the coordinates shown in Fig. $4.7$ and writing $z$ for $d$, the electric field of the incident wave is $\mathbf{E}_{0} \cos (k z-\omega t)$. Because the electric field changes phase by $\pi$ on reflection from the interface, the amplitude $\mathbf{E}_{0}^{\prime}$ of the reflected wave is opposite in direction to $\mathbf{E}_{0}$. A fraction $\rho$ of the energy is reflected. As energy is proportional to $E_{0}^{2}$, we have

$$
E_{0}^{\prime 2}=\rho E_{0}^{2} .
$$

Thus the electric field of the reflected wave is

$$
\mathbf{E}^{\prime}=-\sqrt{\rho} \mathbf{E}_{0} \cos (-k z-\omega t) .
$$

Hence the total electric field in the first medium is

$$
\mathbf{E}=\mathbf{E}_{0} \cos (k z-\omega t)-\sqrt{\rho} \mathbf{E}_{0} \cos (k z+\omega t),
$$

giving

$$
E^{2}=E_{0}^{2} \cos ^{2}(k z-\omega t)+\rho E_{0}^{2} \cos ^{2}(k z+\omega t)-\sqrt{\rho} E_{0}^{2}[\cos (2 k z)+\cos (2 \omega t)] \text {. }
$$

Taking average over a period $T=\frac{2 \pi}{\omega}$ we have

$$
\left\langle E^{2}\right\rangle=\frac{1}{T} \int_{0}^{T} E^{2} d t=\frac{(1+\rho) E_{0}^{2}}{2}-\sqrt{\rho} E_{0}^{2} \cos (2 k z) .
$$

When $k z=m \pi$, or $z=\frac{m c}{2 \nu n_{1}}$, where $m$ is an integer $0,1,2, \ldots,\left\langle E^{2}\right\rangle$ will be minimum with the value

$$
\left\langle E^{2}\right\rangle_{\min }=\frac{(1-\sqrt{\rho})^{2}}{2} E_{0}^{2} .
$$

When $k z=\frac{(2 m+1) \pi}{2}$, or $z=\frac{(2 m+1) c}{4 \nu n_{1}}$, where $m=0,1,2, \ldots,\left\langle E^{2}\right\rangle$ is maximum with the value

$$
\left\langle E^{2}\right\rangle_{\max }=\frac{(1+\sqrt{\rho})^{2}}{2} E_{0}^{2} .
$$

 As $\mathbf{E}, \mathbf{B}$, and $\mathbf{k}$ form an orthogonal right-hand set, we see from Fig. $4.7$ that the amplitude $B_{0}$ of the magnetic field of the reflected wave is in the same direction as that of the incident wave $\mathbf{B}_{0}$, hence no phase change occurs. The amplitudes of the magnetic fields are

$$
B_{0}=n_{1} E_{0}, \quad B_{0}^{\prime}=n_{1} E_{0}^{\prime}=\sqrt{\rho} n_{1} E_{0}=\sqrt{\rho} B_{0} .
$$

MATHPIX IMAGE

Fig. 4.7 The total magnetic field in the first medium is

$$
\mathbf{B}(z, t)=\mathbf{B}_{0} \cos (k z-\omega t)+\sqrt{\rho} \mathbf{B}_{0} \cos (k z+\omega t)
$$

giving

$$
\begin{aligned}
B^{2}=& n_{1}^{2} E_{0}^{2} \cos ^{2}(k z-\omega t)+\rho n_{1}^{2} E_{0}^{2} \cos ^{2}(k z+\omega t) \\
&+\sqrt{\rho} n_{1}^{2} E_{0}^{2}[(\cos (2 k z)+\cos (2 \omega t)]
\end{aligned}
$$

with the average value

$$
\left\langle B^{2}\right\rangle=\frac{(1+\rho) n_{1}^{2} E_{0}^{2}}{2}+\sqrt{\rho} n_{1}^{2} \cos (2 k z)
$$

Hence $\left\langle B^{2}\right\rangle$ will be maximum for $k z=m \pi$ and minimum for $k z=\frac{2 m+1}{2} \pi$.
\textbf{Topic} :Electromagnetic Waves\\
\textbf{Book} :Problems and Solutions on Electromagnetism\\
\textbf{Final Answer} :\frac{(1+\rho) n_{1}^{2} E_{0}^{2}}{2}+\sqrt{\rho} n_{1}^{2} \cos (2 k z)\\


\textbf{Solution} :Consider a cylindrical surface of diameter $D$ in the dielectric medium. Suppose that the electric field inside the cylinder is $E$ and that outside is zero. As the index of refraction of the medium is $n=n_{0}+n_{2} E^{2}$, the index outside is $n_{0}$. Consider a beam of radiation propagating along the axis of the cylinder. A ray making an angle $\theta$ with the axis will be totally reflected at the cylindrical surface if

$$
n \sin \left(\frac{\pi}{2}-\theta\right) \geq n_{0},
$$

i.e.,

$$
n \geq \frac{n_{0}}{\cos \theta} .
$$

The diffraction spread $\theta_{\mathrm{d}}=1.22 \lambda_{n} / D$ will be counterbalanced by the total internal reflection if $n=n_{0}+n_{2} E^{2} \geq \frac{n_{0}}{\cos \theta_{d}}$. Hence we require an electric intensity greater than a critical value

$$
E_{c}=\sqrt{\frac{n_{0}}{n_{2}}\left[\left(\cos \theta_{\mathrm{d}}\right)^{-1}-1\right]} .
$$

Assume the radiation to be plane electromagnetic waves we have

$$
\sqrt{\varepsilon} E=\sqrt{\mu} H .
$$

Waves with the critical electric intensity have average Poynting vector

$$
\langle N\rangle=\frac{1}{2} E H^{*}=\frac{1}{2} \sqrt{\frac{\varepsilon}{\mu}} E_{c}^{2} .
$$

Hence the threshold radiation power is

$$
\langle P\rangle=\langle N\rangle \frac{\pi D^{2}}{4}=\frac{\pi D^{2}}{8} \sqrt{\frac{\varepsilon}{\mu}} \frac{n_{0}}{n_{2}}\left(\frac{1}{\cos \theta_{\mathrm{d}}}-1\right) .
$$

As

$$
\begin{gathered}
n=\frac{n_{0}}{\cos \theta_{\mathrm{d}}} \approx \sqrt{\frac{\varepsilon}{\varepsilon_{0}}}, \quad \mu \approx \mu_{0}, \\
\langle P\rangle=\frac{\pi c \varepsilon_{0} D^{2}}{8} \frac{n_{0}^{2}}{n_{2}} \cdot \frac{1-\cos \theta_{\mathrm{d}}}{\cos ^{2} \theta_{\mathrm{d}}} .
\end{gathered}
$$

With $\theta_{\mathrm{d}}=1.22 \lambda_{n} / D \ll 1$, we have

$$
\langle P\rangle=\frac{\pi c \varepsilon_{0} D^{2}}{16} \frac{n_{0}^{2}}{n_{2}} \theta_{d}^{2}=\frac{\pi c \varepsilon_{0}}{16} \frac{n_{0}^{2}}{n_{2}}\left(1.22 \lambda_{n}\right)^{2} .
$$

Since $n_{0} \lambda_{n}=\lambda$ is the wavelength in vacuum, we obtain

$$
\langle P\rangle=\frac{\pi c \varepsilon_{0}}{16 n_{2}}(1.22 \lambda)^{2} .
$$

\textbf{Topic} :Electromagnetic Waves\\
\textbf{Book} :Problems and Solutions on Electromagnetism\\
\textbf{Final Answer} :\frac{\pi c \varepsilon_{0}}{16 n_{2}}(122 \lambda)^{2}\\


\textbf{Solution} :The electric vector of right circularly polarized light can be represented by the real part of

$$
\mathbf{E}_{\mathbf{R}}(z, t)=\left(E_{0} \mathbf{e}_{x}+E_{0} e^{-i \frac{\pi}{2}} \mathbf{e}_{y}\right) e^{-i \omega t+i k^{z} z}
$$

and that of the left circularly polarized light by the real part of

$$
\mathbf{E}_{L}(z, t)=\left(E_{0} \mathbf{e}_{x}+E_{0} e^{i \frac{\pi}{2}} \mathbf{e}_{y}\right) e^{-i \omega t+i k_{-} z},
$$

where

$$
k_{+}=\frac{\omega}{c} n_{+}, \quad k_{-}=\frac{\omega}{c} n_{-} .
$$

 A linearly polarized light can be decomposed into right and left circularly polarized waves:

$$
\begin{aligned}
\mathbf{E}(z, t) &=\mathbf{E}_{\mathrm{R}}(z, t)+\mathbf{E}_{\mathrm{L}}(z, t) \\
&=\left(E_{0} \mathbf{e}_{x}+E_{0} e^{-i \frac{\pi}{2}} \mathbf{e}_{y}\right) e^{-i \omega t+i k_{+} z}+\left(E_{0} \mathbf{e}_{x}+E_{0} e^{i \frac{\pi}{2}} \mathbf{e}_{y}\right) e^{-i \omega t+i k_{-} z} .
\end{aligned}
$$

At the point of incidence $z=0, \mathbf{E}(0, t)=2 E_{0} e^{-i \omega t} \mathbf{e}_{x}$, which represents a wave with E polarized in the $e_{x}$ direction. At a distance $z$ into the medium, we have

$$
\mathbf{E}(z, t)=E_{0}\left[\left(e^{i\left(k_{+}-k_{-}\right) z}+1\right) \mathbf{e}_{x}+\left(e^{i\left(k_{+}-k_{-}\right) z-i \frac{\pi}{2}}+e^{i \frac{\pi}{2}}\right) \mathbf{e}_{y}\right] e^{-i \omega t+i k_{-} z}
$$

and thus

$$
\frac{E_{y}}{E_{x}}=\frac{\cos \left[\left(k_{+}-k_{-}\right) z-\frac{\pi}{2}\right]}{1+\cos \left[\left(k_{+}-k_{-}\right) z\right]}=\frac{\sin \left[\left(k_{+}-k_{-}\right) z\right]}{1+\cos \left[\left(k_{+}-k_{-}\right) z\right]}=\tan \left[\frac{\left(k_{+}-k_{-}\right) z}{2}\right] .
$$

Hence on traversing a distance $z$ in the medium the plane of the electric vector has rotated by an angle

$$
\varphi=\frac{k_{+}-k_{-}}{2} z=\frac{1}{2} \cdot \frac{\omega}{c}\left(n_{+}-n_{-}\right) z .
$$

As $n_{+}>n_{-}$(assuming $\beta>0$ ), $\varphi>0$. That is, the rotation of the plane of polarization is anti-clockwise looking against the direction of propagation.
\textbf{Topic} :Electromagnetic Waves\\
\textbf{Book} :Problems and Solutions on Electromagnetism\\
\textbf{Final Answer} :\frac{1}{2} \cdot \frac{\omega}{c}\left(n_{+}-n_{-}\right) z\\


\textbf{Solution} :The electric vector of right circularly polarized light can be represented by the real part of

$$
\mathbf{E}_{\mathbf{R}}(z, t)=\left(E_{0} \mathbf{e}_{x}+E_{0} e^{-i \frac{\pi}{2}} \mathbf{e}_{y}\right) e^{-i \omega t+i k^{z} z}
$$

and that of the left circularly polarized light by the real part of

$$
\mathbf{E}_{L}(z, t)=\left(E_{0} \mathbf{e}_{x}+E_{0} e^{i \frac{\pi}{2}} \mathbf{e}_{y}\right) e^{-i \omega t+i k_{-} z},
$$

where

$$
k_{+}=\frac{\omega}{c} n_{+}, \quad k_{-}=\frac{\omega}{c} n_{-} .
$$

 A linearly polarized light can be decomposed into right and left circularly polarized waves:

$$
\begin{aligned}
\mathbf{E}(z, t) &=\mathbf{E}_{\mathrm{R}}(z, t)+\mathbf{E}_{\mathrm{L}}(z, t) \\
&=\left(E_{0} \mathbf{e}_{x}+E_{0} e^{-i \frac{\pi}{2}} \mathbf{e}_{y}\right) e^{-i \omega t+i k_{+} z}+\left(E_{0} \mathbf{e}_{x}+E_{0} e^{i \frac{\pi}{2}} \mathbf{e}_{y}\right) e^{-i \omega t+i k_{-} z} .
\end{aligned}
$$

At the point of incidence $z=0, \mathbf{E}(0, t)=2 E_{0} e^{-i \omega t} \mathbf{e}_{x}$, which represents a wave with E polarized in the $e_{x}$ direction. At a distance $z$ into the medium, we have

$$
\mathbf{E}(z, t)=E_{0}\left[\left(e^{i\left(k_{+}-k_{-}\right) z}+1\right) \mathbf{e}_{x}+\left(e^{i\left(k_{+}-k_{-}\right) z-i \frac{\pi}{2}}+e^{i \frac{\pi}{2}}\right) \mathbf{e}_{y}\right] e^{-i \omega t+i k_{-} z}
$$

and thus

$$
\frac{E_{y}}{E_{x}}=\frac{\cos \left[\left(k_{+}-k_{-}\right) z-\frac{\pi}{2}\right]}{1+\cos \left[\left(k_{+}-k_{-}\right) z\right]}=\frac{\sin \left[\left(k_{+}-k_{-}\right) z\right]}{1+\cos \left[\left(k_{+}-k_{-}\right) z\right]}=\tan \left[\frac{\left(k_{+}-k_{-}\right) z}{2}\right] .
$$

Hence on traversing a distance $z$ in the medium the plane of the electric vector has rotated by an angle

$$
\varphi=\frac{k_{+}-k_{-}}{2} z=\frac{1}{2} \cdot \frac{\omega}{c}\left(n_{+}-n_{-}\right) z .
$$

As $n_{+}>n_{-}$(assuming $\beta>0$ ), $\varphi>0$. That is, the rotation of the plane of polarization is anti-clockwise looking against the direction of propagation.

 The Lorentz force on an electron in the electromagnetic field of a plane electromagnetic wave is $-e(\mathbf{E}+\mathbf{v} \times \mathbf{B})$, where $\mathbf{v}$ is the velocity of the electron. As $\sqrt{\varepsilon_{0}}|\mathbf{E}|=\sqrt{\mu_{0}}|\mathbf{H}|$, or $|\mathbf{B}|=|\mathbf{E}| / c$, we have

$$
\frac{|\mathbf{v} \times \mathbf{B}|}{|\mathbf{E}|} \approx \frac{v}{c} \ll 1 \text {. }
$$

Hence the magnetic force exerted by the wave on the electron may be neglected. The equation of the motion of an electron in $B_{0}$ and the electromagnetic field of the wave, neglecting collisions, is

$$
\boldsymbol{m} \ddot{\mathbf{r}}=-e \mathbf{E}-e \mathbf{v} \times \mathbf{B}_{0},
$$

where $\mathbf{E}$ is the sum of $\mathbf{E}_{R}$ and $\mathbf{E}_{L}$ in (a). Consider an electron at an arbitrary point $z$. Then the solution of the equation of motion has the form

$$
r=r_{0} e^{-i \omega t}
$$

Substitution gives

$$
-m \omega^{2} \mathbf{r}=-e \mathbf{E}-e(-i \omega) \mathbf{r} \times \mathbf{B}_{0} .
$$

The electron, oscillating in the field of the wave, acts as an oscillating dipole, the dipole moment per unit volume being $P=-n_{0}$ er. The above equation then gives

$$
m \omega^{2} \mathrm{P}=-n_{0} e^{2} \mathbf{E}-i \omega e \mathbf{P} \times \mathbf{B}_{0}
$$

or, using $\mathbf{P}=\chi \varepsilon_{0} \mathbf{E}$

$$
m \omega^{2} \chi \varepsilon_{0} \mathbf{E}+n_{0} e^{2} \mathbf{E}=-i \omega e \chi \varepsilon_{0} \mathbf{E} \times \mathbf{B}_{0}
$$

Defining

$$
\omega_{P}^{2}=\frac{n_{0} e^{2}}{m \varepsilon_{0}}, \quad \omega_{B}=\frac{n_{0} e}{\varepsilon_{0} B_{0}},
$$

and with $\mathbf{B}_{0}=B_{0} \mathbf{e}_{z}$, the above becomes

$$
\chi \frac{\omega^{2}}{\omega_{P}^{2}} \mathbf{E}+\mathbf{E}=-i \chi \frac{\omega}{\omega_{B}} \mathbf{E} \times \mathbf{e}_{z},
$$

or

$$
\begin{aligned}
&\left(1+\chi \frac{\omega^{2}}{\omega_{P}^{2}}\right) E_{x}+i \chi \frac{\omega}{\omega_{B}} E_{y}=0 \\
&\left(1+\chi \frac{\omega^{2}}{\omega_{P}^{2}}\right) E_{y}-i \chi \frac{\omega}{\omega_{B}} E_{x}=0
\end{aligned}
$$

$(1) \pm i \times(2)$ gives

$$
\left(1+\chi \frac{\omega^{2}}{\omega_{P}^{2}}\right)\left(E_{x} \pm i E_{y}\right) \pm \frac{\chi \omega}{\omega_{B}}\left(E_{x} \pm i E_{y}\right)=0 .
$$

Note that $E_{x}-i E_{y}=0$ and $E_{x}+i E_{y}=0$ represent the right and left circularly polarized waves respectively. Hence for the right circularly polarized component, whose polarizability is denoted by $\chi_{+}, E_{x}+i E_{y} \neq 0$ so that

$$
\left(1+\chi+\frac{\omega^{2}}{\omega_{P}^{2}}\right)+\chi+\frac{\omega}{\omega_{B}}=0 \text {, }
$$

or

$$
x+=-\frac{1}{\frac{\omega^{2}}{\omega_{P}^{2}}+\frac{\omega}{\omega_{B}}} .
$$

Similarly for the left polarization we have

$$
x-=-\frac{1}{\frac{\omega^{2}}{\omega_{P}^{2}}-\frac{\omega}{\omega_{B}}}
$$

The permittivity of a medium is given by $\varepsilon=(1+\chi) \varepsilon_{0}$ so that the refractive index is

$$
n=\sqrt{\frac{\varepsilon}{\varepsilon_{0}}}=\sqrt{1+x}
$$

Hence for the two polarizations we have the refractive indices

$$
n_{\pm}=\sqrt{1-\frac{1}{\frac{\omega^{2}}{\omega_{P}^{2}} \pm \frac{\omega}{\omega_{B}}}}=\sqrt{1-\frac{\omega_{P}^{2}}{\omega^{2} \pm \frac{\omega_{P}^{2} \omega}{\omega_{B}}}}=\sqrt{1-\frac{\omega_{P}^{2}}{\omega\left(\omega \pm \omega_{B}^{\prime}\right)}},
$$

where

$$
\omega_{B}^{\prime}=\frac{\omega_{P}^{2}}{\omega_{B}}=\frac{n_{0} e^{2}}{m \varepsilon_{0}} \cdot \frac{\varepsilon_{0} B_{0}}{n_{0} e}=\frac{B_{0} e}{m} .
$$

For frequencies sufficiently high so that $\omega \gg \omega_{P}, \omega \gg \omega_{B}^{\prime}$, we obtain approximately

$$
\begin{aligned}
n_{\pm} &=\left[1-\frac{\omega_{P}^{2}}{\omega^{2}}\left(1 \pm \frac{\omega_{B}^{\prime}}{\omega}\right)^{-1}\right]^{\frac{2}{2}} \\
& \approx 1-\frac{1}{2} \frac{\omega_{P}^{2}}{\omega^{2}}\left(1 \mp \frac{\omega_{B}^{\prime}}{\omega}\right) \\
&=n \pm \beta
\end{aligned}
$$

with $n \approx 1-\frac{\omega_{p}^{2}}{2 \omega^{2}}, \beta \approx \frac{\omega_{P}^{2}}{2 \omega^{2}} \frac{\omega_{B}^{\prime}}{\omega}$.


\textbf{Topic} :Electromagnetic Waves\\
\textbf{Book} :Problems and Solutions on Electromagnetism\\
\textbf{Final Answer} :n \pm \beta\\


\textbf{Solution} :Using Maxwell's equations $\oint \mathbf{E} \cdot d \mathbf{r}=-\int \frac{\partial \mathbf{B}}{\partial t} \cdot d \mathbf{S}$ and $\oint \mathbf{H} \cdot d \mathbf{r}=$ $\int\left(\frac{\partial \mathrm{D}}{\partial i}+\mathbf{J}\right) \cdot d \mathbf{S}$, we find that at the boundary of two dielectric media the tangential components of $\mathbf{E}$ and $\mathbf{H}$ are each continuous. Then as $\mathbf{E}, \mathbf{H}$ and the direction of propagation of a plane electromagnetic wave form an orthogonal right-hand set, we have for normal incidence

$$
E+E^{\prime \prime}=E^{\prime}, \quad H-H^{\prime \prime}=H^{\prime},
$$

where the prime and double prime indicate the reflected and refracted components respectively. Also the following relation holds for plane waves,

$$
H=\sqrt{\frac{\varepsilon}{\mu}} E=\sqrt{\frac{\varepsilon}{\varepsilon_{0}}} \sqrt{\frac{\varepsilon_{0}}{\mu}} E \approx n \sqrt{\frac{\varepsilon_{0}}{\mu_{0}}} E .
$$

Hence the $H$ equation can be written as

$$
E-E^{\prime \prime}=n E^{\prime},
$$

taking the first medium as air $(n=1)$.

Eliminating $E^{\prime}$, we get

$$
E^{\prime \prime}=\frac{1-n}{1+n} E .
$$

For normal incidence, the plane of incidence is arbitrary and this relation holds irrespective of the polarization state. Hence

$$
E_{\mathrm{L}}^{\prime \prime}=\frac{1-n_{\mathrm{L}}}{1+n_{\mathrm{L}}} E_{\mathrm{L}}, \quad E_{\mathrm{R}}^{\prime \prime}=\frac{1-n_{\mathrm{R}}}{1+n_{\mathrm{R}}} E_{\mathrm{R}} .
$$

The incident light can be decomposed into left-hand and right-hand circularly polarized components:

$$
E=\left(\begin{array}{c}
E_{0} \\
0
\end{array}\right)=E_{0}\left(\begin{array}{l}
1 \\
0
\end{array}\right)=\frac{1}{2} E_{0}\left(\begin{array}{c}
1 \\
i
\end{array}\right)+\frac{1}{2} E_{0}\left(\begin{array}{c}
1 \\
-i
\end{array}\right),
$$

where $\left(\begin{array}{l}1 \\ i\end{array}\right)$ represents the left-hand circularly polarized light and $\left(\begin{array}{c}1 \\ -i\end{array}\right)$ the right-hand one. Hence the reflected amplitude is

$$
\begin{aligned}
E^{\prime \prime}=& \frac{1}{2} E_{0} \frac{1-n_{R}}{1+n_{R}}\left(\begin{array}{l}
1 \\
i
\end{array}\right)+\frac{1}{2} E_{0} \frac{1-n_{L}}{1+n_{L}}\left(\begin{array}{c}
1 \\
-i
\end{array}\right) \\
&=\frac{1}{2} E_{0}\left(\begin{array}{c}
\frac{1-n_{R}}{1+n_{R}}+\frac{1-n_{L}}{1+n_{L}} \\
i\left(\frac{1-n_{R}}{1+n_{R}}-\frac{1-n_{L}}{1+n_{L}}\right)
\end{array}\right) .
\end{aligned}
$$

This shows that the reflected light is elliptically polarized and the ratio of intensities is

$$
\begin{aligned}
\frac{I^{\prime \prime}}{I} &=\frac{1}{4}\left[\left(\frac{1-n_{\mathrm{R}}}{1+n_{\mathrm{R}}}+\frac{1-n_{\mathrm{L}}}{1+n_{\mathrm{L}}}\right)^{2}+\left(\frac{1-n_{\mathrm{R}}}{1+n_{\mathrm{R}}}-\frac{1-n_{\mathrm{L}}}{1+n_{\mathrm{L}}}\right)^{2}\right] \\
&=\frac{1}{4}\left[2\left(\frac{1-n_{\mathrm{R}}}{1+n_{\mathrm{R}}}\right)^{2}+2\left(\frac{1-n_{\mathrm{L}}}{1+n_{\mathrm{L}}}\right)^{2}\right]=\frac{1}{2}\left[\left(\frac{1-n_{\mathrm{R}}}{1+n_{\mathrm{R}}}\right)^{2}+\left(\frac{1-n_{\mathrm{L}}}{1+n_{\mathrm{L}}}\right)^{2}\right] .
\end{aligned}
$$


\textbf{Topic} :Electromagnetic Waves\\
\textbf{Book} :Problems and Solutions on Electromagnetism\\
\textbf{Final Answer} :\frac{1}{2}\left[\left(\frac{1-n_{\mathrm{R}}}{1+n_{\mathrm{R}}}\right)^{2}+\left(\frac{1-n_{\mathrm{L}}}{1+n_{\mathrm{L}}}\right)^{2}\right]\\


\textbf{Solution} :The equation of motion of an electron in the field of an electromagnetic wave is

$$
m_{\mathrm{e}} \frac{d \mathbf{v}}{d t}=-e \mathbf{E},
$$

where we have neglected the action of the magnetic field, which is of magnitude $v E / c$, as $v \ll c$. For a wave of angular frequency $\omega, \frac{\partial}{\partial t} \rightarrow-i \omega$ and the above gives

$$
\mathbf{v}=-i \frac{e}{m_{\mathrm{e}} \omega} \mathbf{E}
$$

Thus the current density is

$$
\mathbf{j}=-n_{\mathrm{e}} e \mathbf{v}=i \frac{n_{\mathrm{e}} e^{2} \mathbf{E}}{m_{\mathrm{e}} \omega} .
$$
\textbf{Topic} :Electromagnetic Waves\\
\textbf{Book} :Problems and Solutions on Electromagnetism\\
\textbf{Final Answer} :i \frac{n_{\mathrm{e}} e^{2} \mathbf{E}}{m_{\mathrm{e}} \omega}\\


\textbf{Solution} :The equation of motion of an electron in the field of an electromagnetic wave is

$$
m_{\mathrm{e}} \frac{d \mathbf{v}}{d t}=-e \mathbf{E},
$$

where we have neglected the action of the magnetic field, which is of magnitude $v E / c$, as $v \ll c$. For a wave of angular frequency $\omega, \frac{\partial}{\partial t} \rightarrow-i \omega$ and the above gives

$$
\mathbf{v}=-i \frac{e}{m_{\mathrm{e}} \omega} \mathbf{E}
$$

Thus the current density is

$$
\mathbf{j}=-n_{\mathrm{e}} e \mathbf{v}=i \frac{n_{\mathrm{e}} e^{2} \mathbf{E}}{m_{\mathrm{e}} \omega} .
$$

 Maxwell's equations are

$$
\begin{aligned}
&\nabla \cdot \mathbf{D}=\rho, \\
&\nabla \times \mathbf{E}=-\frac{\partial \mathbf{B}}{\partial t}, \\
&\nabla \cdot \mathbf{B}=0, \\
&\nabla \times \mathbf{B}=\mu_{0} \mathbf{j}+\mu_{0} \varepsilon_{0} \frac{\partial \mathbf{E}}{\partial t} .
\end{aligned}
$$

Equations (2) and (4) give

$$
\nabla \times(\nabla \times \mathbf{E})=\nabla(\nabla \cdot \mathbf{E})-\nabla^{2} \mathbf{E}=-\frac{\partial}{\partial t}(\nabla \times \mathbf{B})=-\frac{\partial}{\partial t}\left(\mu_{0} \mathbf{j}+\frac{1}{c^{2}} \frac{\partial \mathbf{E}}{\partial t}\right)
$$

as $c=\left(\mu_{0} \varepsilon_{0}\right)^{-\frac{1}{2}}$

We can take the medium to be charge free apart from the free electrons.

Thus (1) gives $\nabla \cdot \mathbf{E}=\frac{\rho}{\varepsilon_{0}}=0$. We can also write

$$
\mu_{0} \frac{\partial \mathbf{j}}{\partial t}=-\frac{\mu_{0}}{i \omega} \frac{\partial^{2} \mathbf{j}}{\partial t^{2}}=-\frac{\mu_{0} n_{\mathrm{e}} e^{2}}{m_{\mathrm{e}} \omega^{2}} \frac{\partial^{2} \mathrm{E}}{\partial t^{2}} .
$$

Hence

$$
\nabla^{2} \mathbf{E}-\frac{1}{c^{2}}\left(1-\frac{\omega_{P}^{2}}{\omega^{2}}\right) \frac{\partial^{2} \mathbf{E}}{\partial t^{2}}=0
$$

with

$$
\omega_{P}^{2}=\frac{n_{\mathrm{e}} e^{2}}{m_{\mathrm{e}} \varepsilon_{0}}
$$

Similarly, we obtain

$$
\nabla^{2} \mathrm{~B}-\frac{1}{c^{2}}\left(1-\frac{\omega_{P}^{2}}{\omega^{2}}\right) \frac{\partial^{2} \mathrm{~B}}{\partial t^{2}}=0 .
$$

The wave equations can be written in the form

$$
\nabla^{2} \mathbf{E}_{0}+\frac{\omega^{2}}{c^{2}}\left(1-\frac{\omega_{P}^{2}}{\omega^{2}}\right) \mathbf{E}_{0}=0
$$

by putting $\mathbf{E}(\mathbf{r}, t)=\mathbf{E}_{0}(\mathbf{r}) \exp (-i \omega t)$, giving the spatial dependence. giving

 The solution of the last equation is of the form $\mathbf{E}_{0}(\mathbf{r}) \sim \exp (i K \cdot r)$,

$$
K^{2} c^{2}=\omega^{2}-\omega_{P}^{2} .
$$

The necessary and sufficient condition that the electromagnetic waves propagate in this medium indefinitely is that $K$ is real, i.e. $\omega^{2}>\omega_{P}^{2}$, or

$$
n_{\mathrm{e}}<\frac{\varepsilon_{0} m_{\mathrm{e}} \omega^{2}}{e^{2}} .
$$

\textbf{Topic} :Electromagnetic Waves\\
\textbf{Book} :Problems and Solutions on Electromagnetism\\
\textbf{Final Answer} :i \frac{n_{\mathrm{e}} e^{2} \mathbf{E}}{m_{\mathrm{e}} \omega}\\


\textbf{Solution} :For a ohmic conducting medium of permittivity $\varepsilon$, permeability $\mu$ and conductivity $\sigma$, the general wave equation to be used is

$$
\nabla^{2} \mathbf{E}-\mu \varepsilon \ddot{\mathbf{E}}-\mu \sigma \dot{\mathbf{E}}=0 .
$$

For plane electromagnetic waves of angular frequency $\omega, \mathbf{E}(\mathbf{r}, t)=$ $\mathbf{E}_{0}(\mathbf{r}) e^{-i \omega t}$, the above becomes

$$
\nabla^{2} \mathbf{E}_{0}+\mu \varepsilon \omega^{2}\left(1+\frac{i \sigma}{\omega \varepsilon}\right) \mathbf{E}_{0}=0 .
$$

Comparing this with the wave equation for a dielectric, we see that for the conductor we have to replace

$$
\mu \varepsilon \rightarrow \mu \varepsilon\left(1+\frac{i \sigma}{\omega \varepsilon}\right)
$$

if we wish to use the results for a dielectric.

Consider the plane wave as incident on the conductor along the inward normal, whose direction is taken to be the $z$-axis. Then in the conductor the electromagnetic wave can be represented as

$$
\mathbf{E}=\mathbf{E}_{0} e^{i(k z-\omega t)} .
$$

The wave vector has magnitude

$$
k=\frac{\omega}{v}=\omega \sqrt{\mu \varepsilon}\left(1+\frac{i \sigma}{\omega \varepsilon}\right)^{\frac{1}{2}} .
$$

Let $k=\beta+i \alpha$. We have

$$
\beta^{2}-\alpha^{2}=\omega^{2} \mu \varepsilon, \quad \alpha \beta=\frac{1}{2} \omega \mu \sigma .
$$

For a good conductor, i.e. for $\frac{\sigma}{\epsilon \omega} \gg 1$, we have the solution

$$
\alpha=\beta=\pm \sqrt{\frac{\omega \varepsilon \sigma}{2}} .
$$

In the conductor we then have

$$
\mathbf{E}=\mathbf{E}_{0} e^{-\alpha z} e^{i(\beta z-\omega t)}
$$

By the definition of the wave vector, $\beta$ has to take the positive sign. As the wave cannot amplify in the conductor, $\alpha$ has also to take the positive sign. The attenuation length $\delta$ is the distance the wave travels for its amplitude to reduce to $e^{-1}$ of its initial value. Thus

$$
\delta=\frac{1}{\alpha}=\sqrt{\frac{2}{\omega \mu \sigma}} .
$$


\textbf{Topic} :Electromagnetic Waves\\
\textbf{Book} :Problems and Solutions on Electromagnetism\\
\textbf{Final Answer} :\sqrt{\frac{2}{\omega \mu \sigma}}\\


\textbf{Solution} :The equation of the motion of an electron in the field of the X-rays

$$
m \ddot{\mathbf{x}}=-e \mathbf{E}=-e \mathbf{E}_{0} e^{-i \omega t} \text {. }
$$

Its solution has the form $x=x_{0} e^{-i \omega t}$. Substitution gives

$$
m \omega^{2} \mathbf{x}=e \mathbf{E} .
$$

Each electron acts as a Hertzian dipole, so the polarization vector of the metal is

$$
\mathbf{P}=-n e \mathbf{x}=\chi \varepsilon_{0} \mathbf{E}
$$

giving the polarizability as

$$
\chi=-\frac{n e^{2}}{m \varepsilon_{0} \omega^{2}} .
$$

Let $\omega_{P}^{2}=\frac{n e^{2}}{m \varepsilon_{0}}$, then the index of refraction of the metal is

$$
n=\sqrt{1+\chi}=\left(1-\frac{\omega_{P}^{2}}{\omega^{2}}\right)^{1 / 2}
$$

and the critical angle is

$$
\theta_{0}=\arcsin \left(1-\frac{\omega_{P}^{2}}{\omega^{2}}\right)^{1 / 2}
$$
\textbf{Topic} :Electromagnetic Waves\\
\textbf{Book} :Problems and Solutions on Electromagnetism\\
\textbf{Final Answer} :\arcsin \left(1-\frac{\omega_{P}^{2}}{\omega^{2}}\right)^{1 / 2}\\


\textbf{Solution} :The equation of the motion of an electron in the field of the X-rays

$$
m \ddot{\mathbf{x}}=-e \mathbf{E}=-e \mathbf{E}_{0} e^{-i \omega t} \text {. }
$$

Its solution has the form $x=x_{0} e^{-i \omega t}$. Substitution gives

$$
m \omega^{2} \mathbf{x}=e \mathbf{E} .
$$

Each electron acts as a Hertzian dipole, so the polarization vector of the metal is

$$
\mathbf{P}=-n e \mathbf{x}=\chi \varepsilon_{0} \mathbf{E}
$$

giving the polarizability as

$$
\chi=-\frac{n e^{2}}{m \varepsilon_{0} \omega^{2}} .
$$

Let $\omega_{P}^{2}=\frac{n e^{2}}{m \varepsilon_{0}}$, then the index of refraction of the metal is

$$
n=\sqrt{1+\chi}=\left(1-\frac{\omega_{P}^{2}}{\omega^{2}}\right)^{1 / 2}
$$

and the critical angle is

$$
\theta_{0}=\arcsin \left(1-\frac{\omega_{P}^{2}}{\omega^{2}}\right)^{1 / 2}
$$

 As the X-rays are assumed to be polarized with $\mathbf{E}$ perpendicular to the plane of incidence, $\mathbf{E}$ is tangential to the metal surface. Letting a prime and a double-prime indicate the reflected and refracted rays respectively, we have

$$
\begin{gathered}
E+E^{\prime}=E^{\prime \prime}, \\
H \cos \theta-H^{\prime} \cos \theta^{\prime}=H^{\prime \prime} \cos \theta^{\prime \prime} .
\end{gathered}
$$

Note that $\mathbf{E}, \mathbf{H}$ and the direction of propagation form a right-hand set. As

$$
\theta=\theta^{\prime}, \quad \sqrt{\varepsilon} E=\sqrt{\mu} H, \quad \mu \approx \mu_{0},
$$

(2) can be written as

$$
E-E^{\prime}=\sqrt{\frac{\varepsilon}{\varepsilon_{0}}} \frac{\cos \theta^{\prime \prime}}{\cos \theta} E^{\prime \prime}=\frac{n \cos \theta^{\prime \prime}}{\cos \theta} E^{\prime \prime}
$$

(1) and (3) together give

$$
\frac{E^{\prime}}{E}=\frac{\cos \theta-n \cos \theta^{\prime \prime}}{\cos \theta+n \cos \theta^{\prime \prime}}
$$

As $\theta=\theta^{\prime}$ and the intensity is $\frac{1}{2} \sqrt{\frac{\varepsilon}{\mu}} E_{0}^{2}$, the reflection coefficient is

$$
R=\left(\frac{E^{\prime}}{E}\right)^{2}=\left(\frac{\cos \theta-n \cos \theta^{\prime \prime}}{\cos \theta+n \cos \theta}\right)^{2} .
$$



\textbf{Topic} :Electromagnetic Waves\\
\textbf{Book} :Problems and Solutions on Electromagnetism\\
\textbf{Final Answer} :\left(\frac{\cos \theta-n \cos \theta^{\prime \prime}}{\cos \theta+n \cos \theta}\right)^{2}\\


\textbf{Solution} :For this cavity resonator the wavelength of the stationary wave mode $(m, n, p)$ is given by

$$
\begin{aligned}
\lambda_{m, n, p} &=\frac{2}{\sqrt{\left(\frac{m}{a}\right)^{2}+\left(\frac{n}{b}\right)^{2}+\left(\frac{p}{h}\right)^{2}}} \\
&=\frac{2}{\sqrt{\frac{m^{2}}{4}+\frac{n^{2}}{9}+p^{2}}} \mathrm{~cm} .
\end{aligned}
$$

For $\frac{4}{\sqrt{5}} \leq \lambda \leq \frac{8}{\sqrt{13}}, \frac{13}{16} \leq \frac{m^{2}}{4}+\frac{n^{2}}{9}+p^{2} \leq \frac{5}{4}$

As the integers $m, n, p$ must be either 0 or positive with $m n+n p+p m \neq 0$, we have

$$
\begin{array}{lll}
m=1, n=3 ; & m=2, n=1 ; & \text { for } p=0 ; \\
m=1, n=0 ; & m=0, n=1 ; & \text { for } p=1 .
\end{array}
$$

However, each set of $m, n, p$ corresponds to a TE and a TM mode. Hence in the wavelength range $\frac{4}{\sqrt{5}} \leq \lambda \leq \frac{8}{\sqrt{13}} \mathrm{~cm}$ there are eight resonant modes: 2 for each $(1,3,0),(2,1,0),(1,0,1)$ and $(0,1,1)$.

 The wavelengths of the four double modes are respectively $\frac{4}{\sqrt{5}}, \frac{6}{\sqrt{10}}$, $\frac{4}{\sqrt{5}}, \frac{6}{\sqrt{10}} \mathrm{~cm}$. However there are only two different resonant wavelengths.
\textbf{Topic} :Electromagnetic Waves\\
\textbf{Book} :Problems and Solutions on Electromagnetism\\
\textbf{Final Answer} :1 
\end{array}\\


\textbf{Solution} :The boundary conditions are that the tangential component of $\mathbf{E}$ and the normal component of $\mathbf{B}$ are zero on the surface of a perfect conductor. In this case

$$
\begin{array}{ll}
B_{y}=0, & E_{x}=E_{z}=0, \text { for } y=0,2 \mathrm{~cm} ; \\
B_{z}=0, & E_{x}=E_{y}=0, \text { for } z=0,1 \mathrm{~cm} .
\end{array}
$$

It follows from $\nabla \cdot \mathbf{E}=0$ that $\frac{\partial E_{y}}{\partial y}=0$ for $y=0,2 \mathrm{~cm}$ and $\frac{\partial E_{x}}{\partial z}=0$ for $z=0,1 \mathrm{~cm}$ also.
 For sinusoidal waves of angular frequency $\omega$, the wave equation reduces to Helmholtz's equation

$$
\nabla^{2} \mathrm{E}+k^{2} \mathrm{E}=0
$$

with

$$
k^{2}=\frac{\omega^{2}}{c^{2}},
$$

and Maxwell's equation

$$
\nabla \times \mathbf{E}=-\frac{\partial \mathbf{B}}{\partial t}
$$

reduces to

$$
\mathbf{B}=-\frac{i}{\omega} \nabla \times \mathbf{E} .
$$

For the lowest mode, $E_{x}=E_{y}=0, E=E_{z}$. Thus it is a TE wave, given by the wave equation $\nabla^{2} E_{z}+k^{2} E_{z}=0$. The magnetic vector is then given by

$$
B_{x}=\frac{-i}{\omega} \frac{\partial E_{z}}{\partial y}, \quad B_{y}=\frac{i}{\omega} \frac{\partial E_{z}}{\partial x}, \quad B_{z}=0 .
$$

 For the lowest mode, the wave can be represented by

$$
E_{z}=Y(y) Z(z) e^{i\left(k^{\prime} x-\omega t\right)} .
$$

Helmholtz's equation can then be separated into

$$
\frac{d^{2} Y}{d y^{2}}+k_{1}^{2} Y=0, \quad \frac{d^{2} Z}{d z^{2}}+k_{2}^{2} Z=0,
$$

with $k_{1}^{2}+k_{2}^{2}=k^{2}-k^{2}$. The solutions are

$$
\begin{aligned}
&Y=A_{1} \cos \left(k_{1} y\right)+A_{2} \sin \left(k_{1} y\right) \\
&Z=B_{1} \cos \left(k_{2} z\right)+B_{2} \sin \left(k_{2} z\right)
\end{aligned}
$$

The boundary conditions that

$$
\begin{gathered}
E_{z}=0 \text { for } y=0,2, \\
\frac{\partial E_{z}}{\partial z}=0 \text { for } z=0,1
\end{gathered}
$$

give $A_{1}=B_{2}=0, k_{1}=\frac{m}{2} \pi, k_{2}=n \pi, m, n$ being 0 or positive integers. Hence

$$
\begin{gathered}
k^{\prime 2}=\frac{\omega^{2}}{c^{2}}-\left[\left(\frac{m}{2}\right)^{2}+n^{2}\right] \pi^{2} \\
E_{z}=C \sin \left(\frac{m}{2} \pi y\right) \cos (n \pi z) e^{i\left(k^{\prime} x-\omega t\right)} .
\end{gathered}
$$

Let the phase velocity in the waveguide be $v$. Then $k^{\prime}=\frac{\omega}{v}$, or

$$
\omega=v\left\{\frac{\omega^{2}}{c^{2}}-\left[\left(\frac{m}{2}\right)^{2}+n^{2}\right] \pi^{2}\right\}^{\frac{1}{2}}
$$

$n$ can be allowed to have zero value without $E_{z}$ vanishing identically. Hence the lowest mode is $\mathrm{TE}_{10}$, whose phase velocity is

$$
v=\frac{\omega}{\sqrt{\left(\frac{\omega^{2}}{c^{2}}-\frac{\pi^{2}}{4}\right)}}>c .
$$

The group velocity is

$$
v_{g}=\frac{d \omega}{d k^{\prime}}=\left(\frac{d k^{\prime}}{d \omega}\right)^{-1}=\frac{c^{2}}{\omega} k^{\prime}=\frac{c^{2}}{\omega^{2}} \sqrt{\frac{\omega^{2}}{c^{2}}-\frac{\pi^{2}}{4}}=\frac{c^{2}}{v} .
$$
\textbf{Topic} :Electromagnetic Waves\\
\textbf{Book} :Problems and Solutions on Electromagnetism\\
\textbf{Final Answer} :\frac{c^{2}}{v}\\


\textbf{Solution} :If the inside of the waveguide is vacuum, we have

$$
k^{2}=\mu_{0} \varepsilon_{0} \omega^{2},
$$

or

$$
k_{z}^{2}=\mu_{0} \varepsilon_{0} \omega^{2}-\left[\left(\frac{m \pi}{b}\right)^{2}+\left(\frac{n \pi}{a}\right)^{2}\right] .
$$

If $k_{z}^{2}<0, k_{z}$ is purely imaginary and the traveling wave $\sim e^{i k_{z} z}$ becomes exponentially attenuating, i.e., no propagation. Hence the cutoff frequency is given by

$$
\omega_{m n}=\frac{\pi}{\sqrt{\varepsilon_{0} \mu}} \sqrt{\left(\frac{m}{b}\right)^{2}+\left(\frac{n}{a}\right)^{2}} .
$$
\textbf{Topic} :Electromagnetic Waves\\
\textbf{Book} :Problems and Solutions on Electromagnetism\\
\textbf{Final Answer} :\frac{\pi}{\sqrt{\varepsilon_{0} \mu}} \sqrt{\left(\frac{m}{b}\right)^{2}+\left(\frac{n}{a}\right)^{2}}\\


\textbf{Solution} :We first consider a square waveguide whose cross section has sides of length $a$. The electric vector of the electromagnetic wave propagating along $+z$ direction is given by

$$
\begin{aligned}
&E_{x}=A_{1} \cos \left(k_{1} x\right) \sin \left(k_{2} y\right) e^{i\left(k_{3} z-\omega t\right)}, \\
&E_{y}=A_{2} \sin \left(k_{1} x\right) \cos \left(k_{2} y\right) e^{i\left(k_{3} z-\omega t\right)} \\
&E_{z}=A_{3} \sin \left(k_{1} x\right) \sin \left(k_{2} y\right) e^{i\left(k_{3} z-\omega t\right)}
\end{aligned}
$$

with

$$
\begin{gathered}
k_{1}^{2}+k_{2}^{2}+k_{3}^{2}=k^{2}=\mu_{0} \varepsilon_{0} \omega^{2}=\frac{\omega^{2}}{c^{2}}, \\
k_{1} A_{1}+k_{2} A_{2}-i k_{3} A_{3}=0, \\
k_{1}=\frac{m \pi}{a}, \\
k_{2}=\frac{n \pi}{a} .
\end{gathered}
$$

The boundary conditions being satisfied are

$$
E_{x}=E_{z}=0 \text { for } y=0 \text { and } E_{y}=E_{z}=0 \text { for } x=a \text {. }
$$

For the waveguide with triangular cross section, we have to choose from the above those that satisfy additional boundary conditions on the $y=x$ plane: $E_{z}=0, E_{x} \cos \frac{\pi}{4}+E_{y} \sin \frac{\pi}{4}=0$ for $y=x$. The former condition gives $A_{3}=0$, while the latter gives $A_{1}=A_{2}$ and $\tan \left(k_{1} x\right)=-\tan \left(k_{2} x\right)$, or $A_{1}=-A_{2}$ and $\tan \left(k_{1} x\right)=\tan \left(k_{2} x\right)$, i.e., either $k_{1}=-k_{2}, A_{1}=A_{2}$, or $k_{1}=k_{2}, A_{1}=-A_{2}$. Thus for the waveguide under consideration we have

$$
\begin{aligned}
&E_{x}=-A \cos \left(k_{1} x\right) \sin \left(k_{1} y\right) e^{i\left(k_{3} z-\omega t\right)}, \\
&E_{y}=A \sin \left(k_{1} x\right) \cos \left(k_{1} y\right) e^{i\left(k_{3} z-\omega t\right)}, \\
&E_{z}=0
\end{aligned}
$$

with

$$
k_{1}=\frac{n \pi}{a}, \quad k_{3}=\sqrt{\frac{\omega^{2}}{c^{2}}-2 \frac{n^{2} \pi^{2}}{a^{2}}} .
$$

The associated magnetic field can be found using $\nabla \times \mathbf{E}=-\frac{\partial \mathbf{B}}{\partial t}$, or $\mathbf{k} \times \mathbf{E}=$ $\omega \mathbf{B}$ as

$$
\begin{aligned}
B_{x}=&-\frac{k_{3}}{\omega} E_{y}=-\frac{k_{3}}{\omega} A \sin \left(k_{1} x\right) \cos \left(k_{1} y\right) e^{i\left(k_{s} z-\omega t\right)}, \\
B_{y}=& \frac{k_{3}}{\omega} E_{x}=-\frac{k_{3}}{\omega} A \cos \left(k_{1} x\right) \sin \left(k_{1} y\right) e^{i\left(k_{3} z-\omega t\right)}, \\
B_{z}=& \frac{1}{\omega}\left(k_{1} E_{y}-k_{2} E_{x}\right)=\frac{k_{1}}{\omega} A\left[\sin \left(k_{1} x\right) \cos \left(k_{1} y\right)\right.\\
&\left.+\cos \left(k_{1} x\right) \sin \left(k_{1} y\right)\right] e^{i\left(k_{3} z-\omega t\right)} \\
=& \frac{k_{1}}{\omega} A \sin \left[k_{1}(x+y)\right] e^{i\left(k_{3} z-\omega t\right)}
\end{aligned}
$$

Thus the allowed modes are $\mathrm{TE}_{n,-n}$ or $\mathrm{TE}_{n, n}$, but not TM. The cutoff frequencies are

$$
\omega_{n}=\sqrt{2} \frac{n \pi c}{a} .
$$

\textbf{Topic} :Electromagnetic Waves\\
\textbf{Book} :Problems and Solutions on Electromagnetism\\
\textbf{Final Answer} :\sqrt{2} \frac{n \pi c}{a}\\


\textbf{Solution} :Consider first the region $z>0$. Assume $\mu=\mu_{0}$. For sinusoidal waves $\frac{\partial}{\partial t} \rightarrow-i \omega$, and the wave equation becomes

$$
\left(\nabla^{2}+\varepsilon \frac{\omega^{2}}{c^{2}}\right)\left\{\frac{\mathbf{E}^{\prime}}{\mathbf{B}^{\prime}}\right\}=0
$$

where $\varepsilon$ is the relative dielectric constant of the medium, i.e. permittivity $=\varepsilon \varepsilon_{0}$. Because of cylindrical symmetry, special solutions of the above equation are

$$
\mathbf{E}^{\prime}(\mathbf{r}, t)=\mathbf{E}^{\prime}(x, y) e^{i\left(k^{\prime} z-\omega t\right)}
$$



$$
\mathrm{B}^{\prime}(\mathbf{r}, t)=\mathrm{B}^{\prime}(x, y) e^{i\left(k^{\prime} z-\omega t\right)},
$$

with

$$
k^{2}=\varepsilon \frac{\omega^{2}}{c^{2}} .
$$

Let

$$
\nabla^{2}=\nabla_{t}^{2}+\frac{\partial^{2}}{\partial z^{2}},
$$

$\nabla_{t}^{2}$ being the transverse part of the Laplace operator $\nabla^{2}$. Decompose the electromagnetic field into transverse and longitudinal components:

$$
\mathbf{E}^{\prime}=\mathbf{E}_{t}^{\prime}+E_{z} \mathbf{e}_{z}, \quad \mathbf{B}^{\prime}=\mathbf{B}_{t}^{\prime}+B_{z} \mathbf{e}_{z} .
$$

For TEM waves $B_{z}^{\prime}=E_{z}^{\prime}=0$. Then Maxwell's equation for a charge-free medium $\nabla \cdot \mathbf{E}^{\prime}=0$ reduces to

$$
\nabla_{t} \cdot \mathbf{E}_{t}^{\prime}=0 .
$$

Also from Maxwell's equation $\nabla \times \mathbf{E}^{\prime}=-\frac{\partial \mathbf{B}^{\prime}}{\partial t}=i \omega \mathbf{B}^{\prime}$ we have

$$
\nabla_{t} \times \mathbf{E}_{t}^{\prime}=0 \text {. }
$$

These equations allow us to introduce a scalar function $\phi$ such that

$$
\mathbf{E}_{t}^{\prime}=-\nabla \phi, \quad \nabla^{2} \phi=0 .
$$

Furthermore, symmetry requires that $\phi$ is a function of $r$ only and the last equation reduces to

$$
\frac{1}{r} \frac{\partial}{\partial r}\left(r \frac{\partial \phi}{\partial r}\right)=0,
$$

whose solution is

$$
\phi=C \ln r+C^{\prime}
$$

$C, C^{\prime}$ being constants.

Then the electric field is

$$
\mathbf{E}_{t}^{\prime}(\mathbf{r}, t)=\frac{C}{r} e^{i\left(k^{\prime} z-\omega t\right)} \mathbf{e}_{r},
$$

and the associated magnetic field is given by $\nabla \times \mathbf{E}^{\prime}=-\frac{\partial \mathrm{B}}{\partial t}$ with $\nabla \rightarrow i k^{\prime}$, $\frac{\partial}{\partial t} \rightarrow-i \omega$ as

$$
\mathrm{B}_{t}^{\prime}(\mathbf{r}, t)=\frac{\sqrt{\varepsilon}}{c} \mathbf{e}_{z} \times \mathbf{E}_{t}^{\prime}=\frac{C \sqrt{\varepsilon}}{r c} e^{i\left(k^{\prime} z-\omega t\right)} \mathbf{e}_{\theta} .
$$

Therefore, in the $z>0$ region which is filled with a medium of relative dielectric constant $\varepsilon$, the TEM waves can be represented as

$$
\begin{aligned}
\mathbf{E}^{\prime}(x, t) &=\frac{C}{r} e^{i\left(k^{\prime} z-\omega t\right)} \mathbf{e}_{r}, \\
\mathbf{B}^{\prime}(x, t) &=\frac{C \sqrt{\varepsilon}}{r c} e^{i\left(k^{\prime} z-\omega t\right)} \mathbf{e}_{\theta} .
\end{aligned}
$$

Similarly, for the $z<0$ region, $\varepsilon=1$ and the TEM waves are given by

$$
\begin{aligned}
&\mathbf{E}(x, t)=\frac{A}{r} e^{i(k z-\omega t)} \mathbf{e}_{r}, \\
&\mathbf{B}(x, t)=\frac{A}{r c} e^{i(k z-\omega t)} \mathbf{e}_{\theta},
\end{aligned}
$$

where $A$ and $C$ are constants, and $k=\frac{\omega}{c}$.

 Consider a TEM wave incident normally on the interface $z=0$ from the vacuum side. Assuming that the transmitted and reflected waves both remain in the TEM mode, the incident and transmitted waves are given by $\mathbf{E}, \mathbf{B}$ and $\mathbf{E}^{\prime}, \mathbf{B}^{\prime}$ respectively. Let the reflected wave be represented as

$$
\begin{aligned}
\mathbf{E}^{\prime \prime}(\mathbf{r}, t) &=\frac{D}{r} e^{-i(k z+\omega t)} \mathbf{e}_{r}, \\
\mathbf{B}^{\prime \prime}(\mathbf{r}, t) &=-\frac{D}{r c} e^{-i(k z+\omega t)} \mathbf{e}_{\theta} .
\end{aligned}
$$

Note that the negative sign for $\mathbf{B}^{\prime \prime}$ is introduced so that $\mathbf{E}^{\prime \prime}, \mathbf{B}^{\prime \prime}$ and $\mathbf{k}^{\prime \prime}=$ $-k$ form a right-hand set.

The boundary conditions that $\mathrm{E}_{t}$ and $\mathrm{H}_{t}$ are continuous across the interface give

$$
\begin{aligned}
&\left.\left(E_{r}+E_{r}^{\prime \prime}-E_{r}^{\prime}\right)\right|_{z=0}=0, \\
&\left.\left(B_{\theta}+B_{\theta}^{\prime \prime}-B_{\theta}^{\prime}\right)\right|_{z}=0=0,
\end{aligned}
$$

and hence

$$
C=\frac{2 A}{1+\sqrt{\varepsilon}}, \quad D=\frac{1-\sqrt{\varepsilon}}{1+\sqrt{\varepsilon}} A .
$$
tively
\textbf{Topic} :Electromagnetic Waves\\
\textbf{Book} :Problems and Solutions on Electromagnetism\\
\textbf{Final Answer} :\frac{1-\sqrt{\varepsilon}}{1+\sqrt{\varepsilon}} A\\


\textbf{Solution} :Take the $y$-axis along the current and the $z$-axis perpendicular to the current sheet as shown in Fig. 4.19. Let the current per unit width be $\alpha=\alpha e^{-i \omega t} e_{y}$. Consider a unit square area with sides parallel to the $x$ and $y$ axes. At large distances from the current sheet, the current in the area may be considered as a Hertzian dipole of dipole moment $p$ given by

$$
\dot{\mathbf{p}}=\alpha e^{-i \omega t} e_{y} \text {. }
$$

MATHPIX IMAGE

Fig. $4.19$

Hence the power radiated, averaged over one period, from unit area of the sheet is

$$
P=\frac{1}{4 \pi \varepsilon_{0}} \frac{\omega^{2}|\dot{\mathrm{p}}|^{2}}{3 c^{3}}=\frac{\alpha^{2} \omega^{2}}{12 \pi \varepsilon_{0} c^{3}} .
$$

As the thickness $\delta$ is very small, the radiation is emitted mainly from the top and bottom surfaces of the area so that the power radiated per unit area from each side of the thin sheet is

$$
\frac{P}{2}=\frac{\alpha^{2} \omega^{2}}{24 \pi \varepsilon_{0} c^{3}} .
$$
\textbf{Topic} :Electromagnetic Waves\\
\textbf{Book} :Problems and Solutions on Electromagnetism\\
\textbf{Final Answer} :\frac{\alpha^{2} \omega^{2}}{24 \pi \varepsilon_{0} c^{3}}\\


\textbf{Solution} :Take the $y$-axis along the current and the $z$-axis perpendicular to the current sheet as shown in Fig. 4.19. Let the current per unit width be $\alpha=\alpha e^{-i \omega t} e_{y}$. Consider a unit square area with sides parallel to the $x$ and $y$ axes. At large distances from the current sheet, the current in the area may be considered as a Hertzian dipole of dipole moment $p$ given by

$$
\dot{\mathbf{p}}=\alpha e^{-i \omega t} e_{y} \text {. }
$$

MATHPIX IMAGE

Fig. $4.19$

Hence the power radiated, averaged over one period, from unit area of the sheet is

$$
P=\frac{1}{4 \pi \varepsilon_{0}} \frac{\omega^{2}|\dot{\mathrm{p}}|^{2}}{3 c^{3}}=\frac{\alpha^{2} \omega^{2}}{12 \pi \varepsilon_{0} c^{3}} .
$$

As the thickness $\delta$ is very small, the radiation is emitted mainly from the top and bottom surfaces of the area so that the power radiated per unit area from each side of the thin sheet is

$$
\frac{P}{2}=\frac{\alpha^{2} \omega^{2}}{24 \pi \varepsilon_{0} c^{3}} .
$$

 The average power is related to the amplitude of the ac, $I$, by

$$
P=\frac{1}{2} I^{2} R
$$

where $R$ is the resistance. Hence the effective radiation resistance per unit area

$$
R=\frac{2 P}{\alpha^{2}}=\frac{\omega^{2}}{6 \pi \varepsilon_{0} c^{3}} .
$$

 Electromagnetic radiation of energy density $U$ carries a momentum $\frac{U}{c}$. Hence the loss of momentum per unit time per unit area of one surface of the sheet is $\frac{P}{2 c}$. Momentum conservation requires a pressure exerting on the sheet of the same amount:

$$
F=\frac{P}{2 c}=\frac{\alpha^{2} \omega^{2}}{24 \pi \varepsilon_{0} c^{4}} .
$$

Taking the frequency of the alternating current as $f=50 \mathrm{~Hz}$ and with $\alpha=1000 \mathrm{~A}, \varepsilon_{0}=8.85 \times 10^{-12} \mathrm{~F} / \mathrm{m}$, we have

$$
F \approx 1.83 \times 10^{-14} \mathrm{~N} .
$$


\textbf{Topic} :Electromagnetic Waves\\
\textbf{Book} :Problems and Solutions on Electromagnetism\\
\textbf{Final Answer} :\frac{\alpha^{2} \omega^{2}}{24 \pi \varepsilon_{0} c^{3}}\\


\textbf{Solution} :The moment of the dipole is $p=2 Q a \cos \left(\omega_{0} t\right) e_{z}$, or the real part of $2 Q a e^{-i \omega_{0} t} e_{z}$. The average power per unit solid angle at $\mathbf{R}$, as shown in Fig. 4.29, is

$$
\frac{d \bar{P}}{d \Omega}=\frac{\langle S\rangle}{R^{-2}}=\frac{|\ddot{\mathbf{p}}|^{2}}{32 \pi^{2} \varepsilon_{0} c^{3}} \sin ^{2} \theta=\frac{Q^{2} a^{2} \omega_{0}^{4} \sin ^{2} \theta}{8 \pi^{2} \varepsilon_{0} c^{3}} .
$$



MATHPIX IMAGE

Fig. $4.29$
\\
\textbf{Topic} :Electromagnetic Waves\\
\textbf{Book} :Problems and Solutions on Electromagnetism\\
\textbf{Final Answer} :\frac{Q^{2} a^{2} \omega_{0}^{4} \sin ^{2} \theta}{8 \pi^{2} \varepsilon_{0} c^{3}}\\


\textbf{Solution} :The moment of the dipole is $p=2 Q a \cos \left(\omega_{0} t\right) e_{z}$, or the real part of $2 Q a e^{-i \omega_{0} t} e_{z}$. The average power per unit solid angle at $\mathbf{R}$, as shown in Fig. 4.29, is

$$
\frac{d \bar{P}}{d \Omega}=\frac{\langle S\rangle}{R^{-2}}=\frac{|\ddot{\mathbf{p}}|^{2}}{32 \pi^{2} \varepsilon_{0} c^{3}} \sin ^{2} \theta=\frac{Q^{2} a^{2} \omega_{0}^{4} \sin ^{2} \theta}{8 \pi^{2} \varepsilon_{0} c^{3}} .
$$



MATHPIX IMAGE

Fig. $4.29$
 The dipole approximation is valid if $R \gg \lambda \gg a$.
 A current flows through the wire connecting the two small metallic balls of density

$$
\mathrm{j}(x, t)=\mathbf{e}_{z} \frac{d}{d t} \int \rho d z=-i \omega_{0} Q \delta(x) \delta(y) e^{-i \omega_{0} t} \mathbf{e}_{z}, \quad|z| \leq a
$$

and produces a vector potential

$$
\mathbf{A}(\mathbf{x}, t)=\frac{\mu_{0}}{4 \pi} \int_{V} \frac{\mathrm{j}\left(\mathrm{x}^{\prime}, t^{\prime}\right)}{r} d V^{\prime},
$$

where $t^{\prime}=t-\frac{r}{c}, \mathbf{r}=\mathrm{x}-\mathrm{x}^{\prime}$, and $V$ is the region occupied by the current distribution. Thus

$$
\begin{aligned}
A(x, t) &=\frac{-\mu_{0}}{4 \pi} \int_{V} \frac{i \omega_{0} Q \delta\left(x^{\prime}\right) \delta\left(y^{\prime}\right) e^{-i \omega_{0}\left(t-\frac{r}{c}\right)}}{r} d x^{\prime} d y^{\prime} d z^{\prime} \mathbf{e}_{z} \\
&=-i \frac{\mu_{0} \omega_{0} Q e^{-i \omega_{0} t}}{4 \pi} \int_{-a}^{a} \frac{e^{i k_{0} r}}{r} d z^{\prime} \mathbf{e}_{z}
\end{aligned}
$$

where $k_{0}=\frac{\omega_{0}}{c}$. Let $R=|x|$, then $r^{2}=R^{2}-2 R z^{\prime} \cos \theta+z^{\prime 2}$, as shown in Fig. 4.29. Hence

$$
A(x, t)=-i \frac{\mu_{0} \omega_{0} Q e^{-i \omega_{0} t}}{4 \pi} \int_{-a}^{a} \frac{e^{i k_{0} \sqrt{R^{2}+z^{\prime 2}-2 R z^{\prime} \cos \theta}}}{\sqrt{R^{2}+z^{\prime 2}-2 R z^{\prime} \cos \theta}} d z^{\prime} \mathrm{e}_{z}
$$

This is the exact solution. To find the integral analytically, we assume $R \gg$ $a$ and use the approximation $\frac{1}{r} \approx \frac{1}{R}, \sqrt{R^{2}+z^{\prime 2}-2 R z^{\prime} \cos \theta} \approx R-z^{\prime} \cos \theta$. Then

$$
\begin{aligned}
\mathbf{A}(x, t) &=\frac{-i \mu_{0} \omega_{0} Q e^{i\left(k_{0} R-\omega_{0} t\right)}}{4 \pi R} \int_{-a}^{a} e^{-i k_{0} z^{\prime} \cos \theta} d z^{\prime} \mathbf{e}_{z} \\
&=-\frac{i Q e^{i\left(k_{0} R-\omega_{0} t\right)}}{2 \pi \varepsilon_{0} c R} \frac{\sin \left(k_{0} a \cos \theta\right)}{\cos \theta} \mathbf{e}_{z}
\end{aligned}
$$

In spherical coordinates, $\mathbf{e}_{z}=\cos \theta \mathbf{e}_{R}-\sin \theta \mathbf{e}_{\mathrm{e}}$. We can then write $\mathrm{A}=$ $A_{R} \mathbf{e}_{R}+A_{\theta} \mathbf{e}_{\theta}$ with $A_{R}, A_{\theta}$ independent of the angle $\varphi$.

The magnetic field is given by

$$
\mathbf{B}=\nabla \times \mathbf{A}=\frac{1}{R}\left[\frac{\partial}{\partial R}\left(R A_{\theta}\right)-\frac{\partial}{\partial \theta} A_{R}\right] \mathbf{e}_{\varphi}
$$

As we are only interested in the radiation field which varies as $\frac{1}{R}$, we can neglect the second differential on the right-hand side. Hence

$$
B_{\phi} \approx \frac{1}{R} \frac{\partial\left(R A_{\theta}\right)}{\partial R}=-\frac{k_{0} Q e^{i\left(k_{0} R-\omega t\right)}}{2 \pi \varepsilon_{0} c R} \cdot \frac{\sin (k a \cos \theta) \sin \theta}{\cos \theta},
$$

so that

$$
\overline{\mathrm{S}}=\frac{c}{2 \mu_{0}}|\mathrm{~B}|^{2} \mathrm{e}_{R}=\frac{\omega_{0}^{2} Q^{2}}{8 \pi \varepsilon_{0} c R^{2}} \frac{\sin ^{2}(k a \cos \theta) \sin ^{2} \theta}{\cos ^{2} \theta} \mathbf{e}_{R},
$$

and finally

$$
\frac{d \bar{P}}{d \Omega}=\frac{\bar{S}}{R^{-2}}=\frac{\omega_{0}^{2} Q^{2}}{8 \pi^{2} \varepsilon_{0} c} \frac{\sin ^{2} \theta \sin ^{2}(k a \cos \theta)}{\cos ^{2} \theta} .
$$

If the condition $\lambda \gg a$ is also satisfied, then $\sin (k a \cos \theta) \approx k a \cos \theta$ and the above expression reduces to that for the dipole approximation.

\textbf{Topic} :Electromagnetic Waves\\
\textbf{Book} :Problems and Solutions on Electromagnetism\\
\textbf{Final Answer} :\frac{\omega_{0}^{2} Q^{2}}{8 \pi^{2} \varepsilon_{0} c} \frac{\sin ^{2} \theta \sin ^{2}(k a \cos \theta)}{\cos ^{2} \theta}\\


\textbf{Solution} :As $|r| \gg \lambda \gg z_{0}$, multipole expansion may be used to calculate the electromagnetic field. For the radiation field we need to consider only components which vary as $\frac{1}{r}$. The electric dipole moment of the system is

$$
\mathbf{P}=\left(q z_{1}+q z_{2}\right) \mathbf{e}_{z}=0 .
$$

Hence the dipole field is zero.

MATHPIX IMAGE

Fig. $4.30$

The vector potential of the electric quadrupole radiation field is given by

$$
\mathbf{A}(\mathbf{r}, t)=-\frac{\mu_{0}}{4 \pi} \frac{\omega}{2 r} e^{-i \omega t^{\prime}} \int\left(\mathbf{k} \cdot \mathbf{r}^{\prime}\right) \mathbf{r}^{\prime} \rho d V^{\prime},
$$

where

$$
t^{\prime}=t-\frac{r}{c}, \quad \mathbf{k}=\frac{\omega}{c} \frac{\mathbf{r}}{r} .
$$



Hence the magnetic induction is

$$
\begin{aligned}
\mathbf{B} &=\nabla \times \mathbf{A}=i \mathbf{k} \times \mathbf{A}=-i \frac{\mu_{0}}{4 \pi} \frac{\omega k^{2}}{2 r^{3}} e^{i(k r-\omega t)} \int \mathbf{r} \times \mathbf{r}^{\prime}\left(\mathbf{r} \cdot \mathbf{r}^{\prime}\right) \rho d V^{\prime} \\
&=-i \frac{\mu_{0}}{4 \pi} \frac{\omega^{3}}{2 r^{3} c^{2}} e^{i(k r-\omega t)} \sum_{n} \mathbf{r} \times \mathbf{r}^{\prime}\left(\mathbf{r} \cdot \mathbf{r}^{\prime}\right) q_{n}
\end{aligned}
$$

As

$$
\mathbf{r}=r \mathbf{e}_{r}, \quad \mathbf{r}^{\prime}=\pm z_{0}\left(\mathbf{e}_{r} \cos \theta-\mathbf{e}_{\theta} \sin \theta\right)
$$

in spherical coordinates, we have

$$
\begin{aligned}
\mathbf{B} &=i \frac{\mu_{0}}{4 \pi} \frac{\omega^{3}}{2 r^{3} c^{2}} e^{i(k r-\omega t)} 2 r^{2} z_{0}^{2} q \sin \theta \cos \theta \mathbf{e}_{\varphi} \\
&=i \frac{\mu_{0}}{4 \pi} \frac{\omega^{3} z_{0}^{2} q}{r c^{2}} \sin \theta \cos \theta e^{i(k r-\omega t)} \mathbf{e}_{\varphi} .
\end{aligned}
$$

Then using Maxwell's equations $\nabla \times \mathbf{H}=\dot{\mathbf{D}}$ or

$$
\mathbf{E}=c \mathbf{B} \times \mathbf{e}_{r},
$$

we find

$$
\mathbf{E}=\frac{i \mu_{0}}{4 \pi} \frac{\omega^{3} z_{0}^{2} q}{r c} \sin \theta \cos \theta e^{i(k r-\omega t)} \mathbf{e}_{\theta} .
$$

Actually $\mathbf{E}$ and $\mathbf{B}$ are given by the real parts of the above expressions.
\textbf{Topic} :Electromagnetic Waves\\
\textbf{Book} :Problems and Solutions on Electromagnetism\\
\textbf{Final Answer} :\frac{i \mu_{0}}{4 \pi} \frac{\omega^{3} z_{0}^{2} q}{r c} \sin \theta \cos \theta e^{i(k r-\omega t)} \mathbf{e}_{\theta}\\


\textbf{Solution} :As $|r| \gg \lambda \gg z_{0}$, multipole expansion may be used to calculate the electromagnetic field. For the radiation field we need to consider only components which vary as $\frac{1}{r}$. The electric dipole moment of the system is

$$
\mathbf{P}=\left(q z_{1}+q z_{2}\right) \mathbf{e}_{z}=0 .
$$

Hence the dipole field is zero.

MATHPIX IMAGE

Fig. $4.30$

The vector potential of the electric quadrupole radiation field is given by

$$
\mathbf{A}(\mathbf{r}, t)=-\frac{\mu_{0}}{4 \pi} \frac{\omega}{2 r} e^{-i \omega t^{\prime}} \int\left(\mathbf{k} \cdot \mathbf{r}^{\prime}\right) \mathbf{r}^{\prime} \rho d V^{\prime},
$$

where

$$
t^{\prime}=t-\frac{r}{c}, \quad \mathbf{k}=\frac{\omega}{c} \frac{\mathbf{r}}{r} .
$$



Hence the magnetic induction is

$$
\begin{aligned}
\mathbf{B} &=\nabla \times \mathbf{A}=i \mathbf{k} \times \mathbf{A}=-i \frac{\mu_{0}}{4 \pi} \frac{\omega k^{2}}{2 r^{3}} e^{i(k r-\omega t)} \int \mathbf{r} \times \mathbf{r}^{\prime}\left(\mathbf{r} \cdot \mathbf{r}^{\prime}\right) \rho d V^{\prime} \\
&=-i \frac{\mu_{0}}{4 \pi} \frac{\omega^{3}}{2 r^{3} c^{2}} e^{i(k r-\omega t)} \sum_{n} \mathbf{r} \times \mathbf{r}^{\prime}\left(\mathbf{r} \cdot \mathbf{r}^{\prime}\right) q_{n}
\end{aligned}
$$

As

$$
\mathbf{r}=r \mathbf{e}_{r}, \quad \mathbf{r}^{\prime}=\pm z_{0}\left(\mathbf{e}_{r} \cos \theta-\mathbf{e}_{\theta} \sin \theta\right)
$$

in spherical coordinates, we have

$$
\begin{aligned}
\mathbf{B} &=i \frac{\mu_{0}}{4 \pi} \frac{\omega^{3}}{2 r^{3} c^{2}} e^{i(k r-\omega t)} 2 r^{2} z_{0}^{2} q \sin \theta \cos \theta \mathbf{e}_{\varphi} \\
&=i \frac{\mu_{0}}{4 \pi} \frac{\omega^{3} z_{0}^{2} q}{r c^{2}} \sin \theta \cos \theta e^{i(k r-\omega t)} \mathbf{e}_{\varphi} .
\end{aligned}
$$

Then using Maxwell's equations $\nabla \times \mathbf{H}=\dot{\mathbf{D}}$ or

$$
\mathbf{E}=c \mathbf{B} \times \mathbf{e}_{r},
$$

we find

$$
\mathbf{E}=\frac{i \mu_{0}}{4 \pi} \frac{\omega^{3} z_{0}^{2} q}{r c} \sin \theta \cos \theta e^{i(k r-\omega t)} \mathbf{e}_{\theta} .
$$

Actually $\mathbf{E}$ and $\mathbf{B}$ are given by the real parts of the above expressions.

 The average Poynting vector is

$$
\begin{aligned}
\langle\mathrm{N}\rangle &=\frac{1}{2 \mu_{0}} \operatorname{Re}\left(\mathbf{E} \times \mathrm{B}^{*}\right) \\
&=\frac{\mu_{0}}{32 \pi^{2}} \frac{\omega^{6} z_{0}^{4} q^{2}}{r^{2} c^{3}} \sin ^{2} \theta \cos ^{2} \theta \mathbf{e}_{r},
\end{aligned}
$$

so the average power radiated per unit solid angle is

$$
\frac{d P}{d \Omega}=\frac{\langle\mathrm{N}\rangle}{r^{-2}}=\frac{\mu_{0}}{32 \pi^{2}} \frac{\omega^{6} z_{0}^{4} q^{2}}{c^{3}} \sin ^{2} \theta \cos ^{2} \theta .
$$
\textbf{Topic} :Electromagnetic Waves\\
\textbf{Book} :Problems and Solutions on Electromagnetism\\
\textbf{Final Answer} :\frac{\mu_{0}}{32 \pi^{2}} \frac{\omega^{6} z_{0}^{4} q^{2}}{c^{3}} \sin ^{2} \theta \cos ^{2} \theta\\


\textbf{Solution} :As $|r| \gg \lambda \gg z_{0}$, multipole expansion may be used to calculate the electromagnetic field. For the radiation field we need to consider only components which vary as $\frac{1}{r}$. The electric dipole moment of the system is

$$
\mathbf{P}=\left(q z_{1}+q z_{2}\right) \mathbf{e}_{z}=0 .
$$

Hence the dipole field is zero.

MATHPIX IMAGE

Fig. $4.30$

The vector potential of the electric quadrupole radiation field is given by

$$
\mathbf{A}(\mathbf{r}, t)=-\frac{\mu_{0}}{4 \pi} \frac{\omega}{2 r} e^{-i \omega t^{\prime}} \int\left(\mathbf{k} \cdot \mathbf{r}^{\prime}\right) \mathbf{r}^{\prime} \rho d V^{\prime},
$$

where

$$
t^{\prime}=t-\frac{r}{c}, \quad \mathbf{k}=\frac{\omega}{c} \frac{\mathbf{r}}{r} .
$$



Hence the magnetic induction is

$$
\begin{aligned}
\mathbf{B} &=\nabla \times \mathbf{A}=i \mathbf{k} \times \mathbf{A}=-i \frac{\mu_{0}}{4 \pi} \frac{\omega k^{2}}{2 r^{3}} e^{i(k r-\omega t)} \int \mathbf{r} \times \mathbf{r}^{\prime}\left(\mathbf{r} \cdot \mathbf{r}^{\prime}\right) \rho d V^{\prime} \\
&=-i \frac{\mu_{0}}{4 \pi} \frac{\omega^{3}}{2 r^{3} c^{2}} e^{i(k r-\omega t)} \sum_{n} \mathbf{r} \times \mathbf{r}^{\prime}\left(\mathbf{r} \cdot \mathbf{r}^{\prime}\right) q_{n}
\end{aligned}
$$

As

$$
\mathbf{r}=r \mathbf{e}_{r}, \quad \mathbf{r}^{\prime}=\pm z_{0}\left(\mathbf{e}_{r} \cos \theta-\mathbf{e}_{\theta} \sin \theta\right)
$$

in spherical coordinates, we have

$$
\begin{aligned}
\mathbf{B} &=i \frac{\mu_{0}}{4 \pi} \frac{\omega^{3}}{2 r^{3} c^{2}} e^{i(k r-\omega t)} 2 r^{2} z_{0}^{2} q \sin \theta \cos \theta \mathbf{e}_{\varphi} \\
&=i \frac{\mu_{0}}{4 \pi} \frac{\omega^{3} z_{0}^{2} q}{r c^{2}} \sin \theta \cos \theta e^{i(k r-\omega t)} \mathbf{e}_{\varphi} .
\end{aligned}
$$

Then using Maxwell's equations $\nabla \times \mathbf{H}=\dot{\mathbf{D}}$ or

$$
\mathbf{E}=c \mathbf{B} \times \mathbf{e}_{r},
$$

we find

$$
\mathbf{E}=\frac{i \mu_{0}}{4 \pi} \frac{\omega^{3} z_{0}^{2} q}{r c} \sin \theta \cos \theta e^{i(k r-\omega t)} \mathbf{e}_{\theta} .
$$

Actually $\mathbf{E}$ and $\mathbf{B}$ are given by the real parts of the above expressions.

 The average Poynting vector is

$$
\begin{aligned}
\langle\mathrm{N}\rangle &=\frac{1}{2 \mu_{0}} \operatorname{Re}\left(\mathbf{E} \times \mathrm{B}^{*}\right) \\
&=\frac{\mu_{0}}{32 \pi^{2}} \frac{\omega^{6} z_{0}^{4} q^{2}}{r^{2} c^{3}} \sin ^{2} \theta \cos ^{2} \theta \mathbf{e}_{r},
\end{aligned}
$$

so the average power radiated per unit solid angle is

$$
\frac{d P}{d \Omega}=\frac{\langle\mathrm{N}\rangle}{r^{-2}}=\frac{\mu_{0}}{32 \pi^{2}} \frac{\omega^{6} z_{0}^{4} q^{2}}{c^{3}} \sin ^{2} \theta \cos ^{2} \theta .
$$

 The total radiated power is

$$
\begin{aligned}
P &=\int \frac{d P}{d \Omega} d \Omega=\frac{\mu_{0}}{32 \pi^{2}} \frac{\omega^{6} z_{0}^{4} q^{2}}{c^{3}} \int_{0}^{\pi} 2 \pi \sin ^{3} \theta \cos ^{2} \theta d \theta \\
&=\frac{\mu_{0}}{60 \pi} \frac{\omega^{6} z_{0}^{4} q^{2}}{c^{3}} .
\end{aligned}
$$

The total radiated power varies as $\omega^{6}$ for electric quadrupole radiation, and as $\omega^{4}$ for electric dipole radiation.

\textbf{Topic} :Electromagnetic Waves\\
\textbf{Book} :Problems and Solutions on Electromagnetism\\
\textbf{Final Answer} :\frac{\mu_{0}}{60 \pi} \frac{\omega^{6} z_{0}^{4} q^{2}}{c^{3}}\\


\textbf{Solution} :Use Cartesian coordinates as shown in Fig. 4.34. The action of the conducting plane on the $x>0$ space is equivalent to that of an image dipole at $\left(-\frac{a}{2}, 0,0\right)$ of moment

$$
\mathbf{P}^{\prime}=-\mathbf{P}=-P_{0} e^{-i \omega t} \mathbf{e}_{z}
$$

MATHPIX IMAGE

Fig. $4.34$ The vector potential at a point $r$ is

$$
\begin{aligned}
\mathbf{A} &=\frac{\mu_{0}}{4 \pi}\left(\frac{\dot{\mathbf{P}}}{r_{1}}+\frac{\dot{\mathbf{P}}}{r_{2}}\right) \\
&=-i \frac{\mu_{0}}{4 \pi} \omega P_{0}\left(\frac{e^{i k r_{1}}}{r_{1}}-\frac{e^{i k r_{2}}}{r_{2}}\right) e^{-i \omega t} \mathbf{e}_{z}
\end{aligned}
$$

As we are only interested in the radiation field which dominates at $r \gg a$, we use the approximation

$$
r_{1} \approx r-\frac{a}{2} \mathbf{e}_{x} \cdot \mathbf{e}_{r}, \quad r_{2} \approx r+\frac{a}{2} \mathbf{e}_{x} \cdot \mathbf{e}_{r}, \quad \frac{1}{r_{1}} \approx \frac{1}{r_{2}} \approx \frac{1}{r},
$$

$\mathbf{e}_{r}, \mathbf{e}_{\theta}, \mathbf{e}_{\varphi}$ being the unit vectors in spherical coordinates. As $\boldsymbol{\epsilon}_{x}=$ $e_{r} \sin \theta \cos \varphi+e_{\theta} \cos \theta \cos \varphi-e_{\varphi} \sin \varphi$, we have

$$
\begin{aligned}
&r_{1} \approx r-\frac{a}{2} \sin \theta \cos \varphi, \\
&r_{2} \approx r+\frac{a}{2} \sin \theta \cos \varphi
\end{aligned}
$$

and

$$
\begin{aligned}
& \mathbf{A} \approx i \frac{\mu_{0}}{4 \pi} \frac{\omega P_{0}}{r}\left(e^{i \frac{k a}{2} \sin \theta \cos \varphi}-e^{-i \frac{k a}{2}} \sin \theta \sin \varphi\right) e^{i(k r-\omega t)} \mathbf{e}_{z} \\
& =-\frac{\mu_{0}}{2 \pi} \frac{\omega P_{0}}{r} e^{i(k r-\omega t)} \sin \left(\frac{k a}{2} \sin \theta \cos \varphi\right) \mathbf{e}_{z} .
\end{aligned}
$$

In spherical coordinates

$$
\mathbf{e}_{z}=\mathrm{e}_{r} \cos \theta-\mathbf{e}_{\theta} \sin \theta .
$$

To obtain $\mathbf{B}=\nabla \times \mathbf{A}$, we neglect terms of orders higher than $\frac{1}{r}$ and obtain

$$
\begin{aligned}
\mathbf{B}(\mathbf{r}, t) & \approx \frac{\mathbf{e}_{\varphi}}{r} \frac{\partial}{\partial r}\left(r A_{\theta}\right) \\
&=-\frac{\mathbf{e}_{\varphi}}{r} \frac{\partial}{\partial r}(r \sin \theta A) \\
&=\frac{i \omega^{2} P_{0} e^{i(k r-\omega t)}}{2 \pi \varepsilon_{0} c^{3} r} \sin \theta \sin \left(\frac{k}{2} a \sin \theta \cos \varphi\right) \mathbf{e}_{\varphi}
\end{aligned}
$$

The associated electric field intensity is

$$
\begin{aligned}
\mathbf{E}(\mathbf{r}, t) &=c \mathbf{B} \times \mathbf{e}_{r} \\
& \approx \frac{i \omega^{2} P_{0} e^{i(k r-\omega t)}}{2 \pi \varepsilon_{0} c^{2} r} \sin \theta \sin \left(\frac{k}{2} a \sin \theta \cos \varphi\right) \mathbf{e}_{\theta} .
\end{aligned}
$$

The average Poynting vector is

$$
\bar{S}=\frac{\varepsilon_{0} c}{2}|E|^{2} \mathbf{e}_{r}=\frac{\omega^{4} P_{0}^{2} \sin ^{2} \theta}{8 \pi^{2} \varepsilon_{0} c^{3} r^{2}} \sin ^{2}\left(\frac{k}{2} a \sin \theta \cos \varphi\right) \mathbf{e}_{r}
$$

The angular distribution of the radiation is therefore given by

$$
\frac{d \bar{P}}{d \Omega}=\frac{\bar{S}}{r^{-2}}=\frac{\omega^{4} P_{0}^{2} \sin ^{2} \theta}{8 \pi^{2} \varepsilon_{0} c^{3}} \sin ^{2}\left(\frac{k}{2} a \sin \theta \cos \varphi\right) .
$$

If $\lambda \gg a$, then $\sin \left(\frac{k}{2} a \sin \theta \cos \varphi\right) \approx \frac{k}{2} a \sin \theta \cos \varphi$ and we have the approximate expression

$$
\frac{d \bar{P}}{d \Omega} \approx \frac{\omega^{6} P_{0}^{2} a^{2} \sin ^{4} \theta \cos ^{2} \varphi}{32 \pi^{2} \varepsilon_{0} c^{5}} .
$$

\textbf{Topic} :Electromagnetic Waves\\
\textbf{Book} :Problems and Solutions on Electromagnetism\\
\textbf{Final Answer} :\frac{\mu_{0}}{60 \pi} \frac{\omega^{6} z_{0}^{4} q^{2}}{c^{3}}\\


\textbf{Solution} :Suppose the car is moving towards the radar with velocity $v$. Let the radar frequency be $\nu_{0}$ and the frequency of the signal as received by the car be $\nu_{1}$. The situation is the same as if the car were stationary and the radar moved toward it with velocity $v$. Hence the relativistic Doppler effect gives

$$
\nu_{1}=\nu_{0} \sqrt{\frac{1+v / c}{1-v / c}} \approx \nu_{0}\left(1+\frac{v}{c}\right) \text {, }
$$

correct to the first power of $v / c$. Now the car acts like a source of frequency $\nu_{1}$, so the frequency of the reflected signal as received by the radar (also correct to the first power of $v / c)$ is

$$
\nu_{2}=\nu_{1} \sqrt{\frac{1+v / c}{1-v / c}} \approx \nu_{1}\left(1+\frac{v}{c}\right) \approx \nu_{0}\left(1+\frac{v}{c}\right)^{2} \approx \nu_{0}\left(1+\frac{2 v}{c}\right) .
$$

Thus the beat frequency is

$$
\nu_{2}-\nu_{0}=\nu_{0} \cdot \frac{2 v}{c}=10^{9} \times \frac{2 \times 30}{3 \times 10^{8}}=200 \mathrm{~Hz} \text {. }
$$

The result is the same if we had assumed the car to be moving away from the stationary radar. For then we would have to replace $v$ by $-v$ in the above and obtain $\nu_{0}-\nu_{2} \approx \nu_{0} \frac{2 v}{c}$.

\textbf{Topic} :Relativity, Particle-Field Interactions\\
\textbf{Book} :Problems and Solutions on Electromagnetism\\
\textbf{Final Answer} :200 \mathrm{~Hz}\\


\textbf{Solution} :Consider two inertial frames $\Sigma$ and $\Sigma^{\prime}$ with $\Sigma^{\prime}$ moving with a constant velocity $v$ along the $x$ direction relative to $\Sigma$. Let the velocity and acceleration of an object moving in the $x$ direction be $u, a=\frac{d u}{d t}$, and $u^{\prime}, a^{\prime}=\frac{d u^{\prime}}{d t^{\prime}}$ in the two frames respectively. Lorentz transformation gives

$$
x=\gamma\left(x^{\prime}+\beta c t^{\prime}\right), \quad c t=\gamma\left(c t^{\prime}+\beta x^{\prime}\right),
$$

where $\beta=\frac{v}{c}, \gamma=\left(1-\beta^{2}\right)^{-\frac{1}{2}}$. Then the velocity of the object is transformed according to

$$
u=\frac{u^{\prime}+v}{1+\frac{v u^{\prime}}{c^{2}}} .
$$

Differentiating the above, we have

$$
\begin{gathered}
d t=\gamma\left(d t^{\prime}+\frac{v}{c^{2}} d x^{\prime}\right)=\gamma d t^{\prime}\left(1+\frac{v}{c^{2}} u^{\prime}\right), \\
d u=\frac{d u^{\prime}}{\gamma^{2}\left(1+\frac{v u^{\prime}}{c^{2}}\right)^{2}},
\end{gathered}
$$

whose ratio gives the transformation of acceleration:

$$
a=\frac{a^{\prime}}{\gamma^{3}\left(1+\frac{v u^{\prime}}{c^{2}}\right)^{3}} \text {. }
$$

Now assume that $\Sigma$ is the inertial frame attached to the fixed stars and $\Sigma^{\prime}$ is the inertial frame in which the spaceship is momentarily at rest. Then in $\Sigma^{\prime}$

$$
u^{\prime}=0, \quad a^{\prime}=g,
$$

and Eqs.
(1) and (2) give

$$
u=v, \quad a=\frac{g}{\gamma^{3}}
$$

with $\gamma=\left(1-\frac{u^{2}}{c^{2}}\right)^{-\frac{1}{2}}$. As the velocity of the spaceship is increased from 0 to $v$ in $\Sigma$, the distance traveled is

$$
x=\int u d t=\int_{0}^{v} \frac{u d u}{a}=\frac{1}{g} \int_{0}^{v} \frac{u d u}{\left(1-\frac{u^{2}}{c^{2}}\right)^{3 / 2}}=\frac{c^{2}}{g}\left\{\frac{1}{\sqrt{1-\frac{v^{2}}{c^{2}}}}-1\right\} .
$$



\textbf{Topic} :Relativity, Particle-Field Interactions\\
\textbf{Book} :Problems and Solutions on Electromagnetism\\
\textbf{Final Answer} :\frac{c^{2}}{g}\left\{\frac{1}{\sqrt{1-\frac{v^{2}}{c^{2}}}}-1\right\}\\


\textbf{Solution} :In the inertial frame $S$, the $y$-component of the velocity of the package should be $c / 2$ in order that the package will have the same $y$ coordinate as ship $(2)$ as the package passes through the distance $\Delta x=d$. The velocity of the package in $S$ can be expressed in the form

$$
\mathbf{u}=u_{x} \mathbf{e}_{x}+u_{y} \mathbf{e}_{y}
$$

with $u_{y}=c / 2$. As $u=|\mathbf{u}|=\frac{3}{4} c$,

$$
u_{x}=\sqrt{u^{2}-u_{y}^{2}}=\frac{\sqrt{5}}{4} c .
$$

Let $S^{\prime}$ be the inertial frame fixed on ship (1). In $S^{\prime}$ the velocity of the package is

$$
\mathbf{u}^{\prime}=u_{x}^{\prime} \mathbf{e}_{x}^{\prime}+u_{y}^{\prime} \mathbf{e}_{y}^{\prime} \text {. }
$$

$S^{\prime}$ moves with speed $c / 2$ relative to $S$ along the $-y$ direction, i.e. the velocity of $S^{\prime}$ relative to $S$ is $\mathbf{v}=-c e_{y} / 2$. Velocity transformation then gives

$$
\begin{gathered}
u_{y}^{\prime}=\frac{u_{y}-v}{1-\frac{v u_{y}}{c^{2}}}=\frac{\frac{c}{2}+\frac{c}{2}}{1+\frac{1}{4}}=\frac{4}{5} c, \\
u_{x}^{\prime}=\frac{u_{x} \sqrt{1-\frac{v^{2}}{c^{2}}}}{1-\frac{v u_{y}}{c^{2}}}=\frac{\frac{\sqrt{5}}{4} c \sqrt{1-\frac{1}{4}}}{1+\frac{1}{4}}=\sqrt{\frac{3}{20}} c .
\end{gathered}
$$



MATHPIX IMAGE

Fig. $5.1$

MATHPIX IMAGE

Fig. 5.2

Let $\alpha^{\prime}$ be the angle between the velocity $\mathbf{u}^{\prime}$ of the package in $S^{\prime}$ and the $x^{\prime}$ axis as shown in Fig. 5.2. Then

$$
\tan \alpha^{\prime}=\frac{u_{y}^{\prime}}{u_{x}^{\prime}}=\frac{8}{\sqrt{15}}, \quad \text { or } \quad \alpha^{\prime}=\arctan \left(\frac{8}{\sqrt{15}}\right) \text {. }
$$

 In $S^{\prime}$ the speed of the package is

$$
u^{\prime}=\left|u^{\prime}\right|=\sqrt{u_{x}^{\prime 2}+u_{y}^{\prime 2}}=\frac{\sqrt{79}}{10} c .
$$

\textbf{Topic} :Relativity, Particle-Field Interactions\\
\textbf{Book} :Problems and Solutions on Electromagnetism\\
\textbf{Final Answer} :\frac{\sqrt{79}}{10} c\\


\textbf{Solution} :Conservation of momentum is expressed by the equations

$$
\begin{gathered}
\frac{h \nu}{c}=\frac{h \nu^{\prime}}{c} \cos \theta+\gamma m v \cos \varphi, \\
\frac{h \nu^{\prime}}{c} \sin \theta=\gamma m v \sin \varphi,
\end{gathered}
$$

where $\theta$ is the angle between the directions of motion of the incident and scattered photons, $\varphi$ is that between the incident photon and the recoiling electron, as shown in Fig. $5.3, m$ is the rest mass of an electron, $\beta=\frac{v}{c}, v$ being the speed of the recoil electron, and $\gamma=\left(1-\beta^{2}\right)^{-\frac{1}{2}}$.

MATHPIX IMAGE

Fig. $5.3$

Conservation of energy is expressed by the equation

$$
h \nu+m c^{2}=h \nu^{\prime}+\gamma m c^{2} .
$$
to

 For back scattering, $\theta=180^{\circ}, \varphi=0^{\circ}$. The above equations reduce

$$
\begin{gathered}
h \nu+h \nu^{\prime}=\gamma \beta m c^{2}, \\
h \nu-h \nu^{\prime}=(\gamma-1) m c^{2} .
\end{gathered}
$$

Squaring both sides of Eq.
(1) we have

$$
\left(h \nu+h \nu^{\prime}\right)^{2}=\gamma^{2} \beta^{2} m^{2} c^{4}=\left(\gamma^{2}-1\right) m^{2} c^{4} .
$$

Combining Eqs.
(2) and (3), we have

$$
4 h^{2} \nu \nu^{\prime}=2 m c^{2} h\left(\nu-\nu^{\prime}\right),
$$

or

$$
h \nu^{\prime}=\frac{h \nu}{\frac{2 h \nu}{m c^{2}}+1},
$$

which is the energy of the scattered photon.


\textbf{Topic} :Relativity, Particle-Field Interactions\\
\textbf{Book} :Problems and Solutions on Electromagnetism\\
\textbf{Final Answer} :\frac{h \nu}{\frac{2 h \nu}{m c^{2}}+1}\\


\textbf{Solution} :The induced charges in the metal sheet will move on the surface of the sheet along the general direction of motion of the charged particle. The acceleration of the induced charges moving on the undulating surface will lead to emission of bremsstrahlung (braking radiation). The radiation detected by a distant observer located along the $\theta$ direction is that resulting from the constructive interference in that direction. Hence, the wavelength of the radiation satisfies the condition

$$
\frac{L}{v}-\frac{L \cos \theta}{c}=m \frac{\lambda}{c},
$$

where $m$ is an integer, or

$$
\lambda_{m}=\frac{L}{m}\left(\frac{c}{v}-\cos \theta\right)
$$

For $m=1, \lambda_{1}=L\left(\frac{c}{v}-\cos \theta\right)$.

We can also approach the problem by regarding the effect of the metal sheet as that of an image charge, which together with the real charge forms an oscillating dipole of velocity $v=v e_{x}$ and frequency of vibration $f_{0}=\frac{v}{L}$. From the formula of Doppler shift the frequency detected by the observer is

$$
f(\theta)=\frac{1+\beta \cos \theta}{\sqrt{1-\beta^{2}}} f_{0} \approx\left(1+\frac{v}{c} \cos \theta\right) f,
$$

and the corresponding wavelength is

$$
\lambda(\theta)=\frac{c}{f}=\frac{c L}{v}\left(1-\frac{v}{c} \cos \theta\right) .
$$

This result is the same as the foregoing $\lambda_{1}$. 

\textbf{Topic} :Relativity, Particle-Field Interactions\\
\textbf{Book} :Problems and Solutions on Electromagnetism\\
\textbf{Final Answer} :\frac{c L}{v}\left(1-\frac{v}{c} \cos \theta\right)\\


\textbf{Solution} :Let the electromagnetic fields be $\mathbf{E}^{\prime}, \mathrm{B}^{\prime}$ in frame $S^{\prime}\left(0 x^{\prime} y^{\prime} z^{\prime}\right)$ where the plates are at rest; and be $\mathbf{E}, \mathbf{B}$ in the laboratory frame $S(0 x y z)$. The field vectors transform according to

$$
\begin{gathered}
E_{x}=E_{x}^{\prime}, \quad B_{x}=B_{x}^{\prime}, \\
E_{y}=\gamma\left(E_{y}^{\prime}+\beta c B_{z}^{\prime}\right), \quad B_{y}=\gamma\left(B_{y}^{\prime}-\frac{\beta}{c} E_{z}^{\prime}\right), \\
E_{z}=\gamma\left(E_{z}^{\prime}-\beta c B_{y}^{\prime}\right), \quad B_{z}=\gamma\left(B_{z}^{\prime}+\frac{\beta}{c} E_{y}^{\prime}\right),
\end{gathered}
$$

where $\beta=\frac{v}{c}, \gamma=\left(1-\beta^{2}\right)^{-\frac{1}{2}}$.

In the rest frame $S^{\prime}$,

$$
\begin{gathered}
B_{x}^{\prime}=B_{y}^{\prime}=B_{z}^{\prime}=0, \\
E_{x}^{\prime}=E_{y}^{\prime}=0, \quad E_{z}^{\prime}=-\frac{\sigma}{\varepsilon_{0}},
\end{gathered}
$$

so that

$$
\begin{gathered}
E_{x}=0, \quad B_{x}=0, \\
E_{y}=0, \quad B_{y}=\frac{\gamma \beta}{\varepsilon_{0} c} \sigma, \\
E_{z}=-\frac{\gamma \sigma}{\varepsilon_{0}}, \quad B_{z}=0 .
\end{gathered}
$$

Hence in the laboratory frame, the electric intensity is in the $-z$ direction and has magnitude $\frac{\gamma \sigma}{\varepsilon_{0}}$, while the magnetic induction is in the $+y$ direction and has magnitude $\frac{\gamma v}{\varepsilon_{0} c^{2}} \sigma$, where $\gamma=\left(1-\frac{v^{2}}{c^{2}}\right)^{-\frac{1}{2}}$.

\textbf{Topic} :Relativity, Particle-Field Interactions\\
\textbf{Book} :Problems and Solutions on Electromagnetism\\
\textbf{Final Answer} :0 
\end{gathered}\\


\textbf{Solution} :Let $\Sigma$ and $\Sigma^{\prime}$ be the rest frames of the observers $A$ and $B$ respectively, the common $x$-axis being along the axis of the conducting wire, which is fixed in $\Sigma$, as shown in Fig. 5.6. In $\Sigma, \rho=0, j=\frac{i}{\pi r_{0}^{2}} \mathbf{e}_{x}$, so the electric and magnetic fields in $\Sigma$ are respectively

$$
\begin{gathered}
\mathbf{E}=0, \\
\mathbf{B}(r)= \begin{cases}\frac{\mu_{0} i r}{2 \pi r_{0}^{2}} \mathbf{e}_{\varphi}, & \left(r<r_{0}\right) \\
\frac{\mu_{0} i}{2 \pi r} \mathbf{e}_{\varphi}, & \left(r>r_{0}\right)\end{cases}
\end{gathered}
$$

where $\mathbf{e}_{x}, \mathbf{e}_{r}$, and $\mathbf{e}_{\varphi}$ form an orthogonal system. Lorentz transformation gives the electromagnetic field as seen in $\Sigma^{\prime}$ as

$$
\begin{gathered}
E_{\|}^{\prime}=E_{\|}=0, \quad B_{\|}^{\prime}=B_{\|}=0, \\
\mathbf{E}^{\prime}=\mathbf{E}_{\perp}^{\prime}=\gamma\left(\mathbf{E}_{\perp}+\mathbf{v} \times \mathbf{B}_{\perp}\right)=-\gamma v B \mathbf{e}_{r}= \begin{cases}-\frac{\mu_{0} \gamma i v r}{2 \pi r_{0}^{2}} \mathbf{e}_{r}, & \left(r<r_{0}\right) \\
-\frac{\mu_{0} \gamma i v}{2 \pi r} \mathbf{e}_{r}, & \left(r>r_{0}\right)\end{cases} \\
\mathbf{B}^{\prime}=\mathbf{B}_{\perp}^{\prime}=\gamma\left(\mathbf{B}_{\perp}-\frac{\mathbf{v} \times \mathbf{E}_{\perp}}{c^{2}}\right)=\gamma B \mathbf{e}_{\varphi}= \begin{cases}\frac{\mu_{0} i \gamma r}{2 \pi r_{0}^{2}} \mathbf{e}_{\varphi}, & \left(r<r_{0}\right) \\
\frac{\mu_{0} i \gamma}{2 \pi r} \mathbf{e}_{\varphi}, & \left(r>r_{0}\right)\end{cases}
\end{gathered}
$$

where $\gamma=\frac{1}{\sqrt{1-v^{2} / c^{2}}}$, and the lengths $r$ and $r_{0}$ are not changed by the transformation.

MATHPIX IMAGE

Fig. $5.6$
\textbf{Topic} :Relativity, Particle-Field Interactions\\
\textbf{Book} :Problems and Solutions on Electromagnetism\\
\textbf{Final Answer} :\begin{cases}\frac{\mu_{0} i \gamma r}{2 \pi r_{0}^{2}} \mathbf{e}_{\varphi} & \left(r<r_{0}\right) \\
\frac{\mu_{0} i \gamma}{2 \pi r} \mathbf{e}_{\varphi} & \left(r>r_{0}\right)\end{cases}
\end{gathered}\\


\textbf{Solution} :Let $\Sigma$ and $\Sigma^{\prime}$ be the rest frames of the observers $A$ and $B$ respectively, the common $x$-axis being along the axis of the conducting wire, which is fixed in $\Sigma$, as shown in Fig. 5.6. In $\Sigma, \rho=0, j=\frac{i}{\pi r_{0}^{2}} \mathbf{e}_{x}$, so the electric and magnetic fields in $\Sigma$ are respectively

$$
\begin{gathered}
\mathbf{E}=0, \\
\mathbf{B}(r)= \begin{cases}\frac{\mu_{0} i r}{2 \pi r_{0}^{2}} \mathbf{e}_{\varphi}, & \left(r<r_{0}\right) \\
\frac{\mu_{0} i}{2 \pi r} \mathbf{e}_{\varphi}, & \left(r>r_{0}\right)\end{cases}
\end{gathered}
$$

where $\mathbf{e}_{x}, \mathbf{e}_{r}$, and $\mathbf{e}_{\varphi}$ form an orthogonal system. Lorentz transformation gives the electromagnetic field as seen in $\Sigma^{\prime}$ as

$$
\begin{gathered}
E_{\|}^{\prime}=E_{\|}=0, \quad B_{\|}^{\prime}=B_{\|}=0, \\
\mathbf{E}^{\prime}=\mathbf{E}_{\perp}^{\prime}=\gamma\left(\mathbf{E}_{\perp}+\mathbf{v} \times \mathbf{B}_{\perp}\right)=-\gamma v B \mathbf{e}_{r}= \begin{cases}-\frac{\mu_{0} \gamma i v r}{2 \pi r_{0}^{2}} \mathbf{e}_{r}, & \left(r<r_{0}\right) \\
-\frac{\mu_{0} \gamma i v}{2 \pi r} \mathbf{e}_{r}, & \left(r>r_{0}\right)\end{cases} \\
\mathbf{B}^{\prime}=\mathbf{B}_{\perp}^{\prime}=\gamma\left(\mathbf{B}_{\perp}-\frac{\mathbf{v} \times \mathbf{E}_{\perp}}{c^{2}}\right)=\gamma B \mathbf{e}_{\varphi}= \begin{cases}\frac{\mu_{0} i \gamma r}{2 \pi r_{0}^{2}} \mathbf{e}_{\varphi}, & \left(r<r_{0}\right) \\
\frac{\mu_{0} i \gamma}{2 \pi r} \mathbf{e}_{\varphi}, & \left(r>r_{0}\right)\end{cases}
\end{gathered}
$$

where $\gamma=\frac{1}{\sqrt{1-v^{2} / c^{2}}}$, and the lengths $r$ and $r_{0}$ are not changed by the transformation.

MATHPIX IMAGE

Fig. $5.6$

 Let the charge density of the wire in $\Sigma^{\prime}$ be $\rho^{\prime}$, then the electric field produced by $\rho^{\prime}$ for $r<r_{0}$ is given by Gauss' law

$$
2 \pi r E_{r}^{\prime}=\rho^{\prime} \pi r^{2} / \varepsilon_{0}
$$

to be

$$
\mathbf{E}^{\prime}=\frac{\rho^{\prime} r}{2 \varepsilon_{0}} \mathbf{e}_{r} . \quad\left(r<r_{0}\right)
$$

Comparing this with the expression for $\mathbb{}^{\prime}$ above we have

$$
\rho^{\prime}=-\frac{v i \gamma}{\pi r_{0}^{2} c^{2}},
$$

where we have used $\mu_{0} \varepsilon_{0}=\frac{1}{c^{2}}$.
\\
\textbf{Topic} :Relativity, Particle-Field Interactions\\
\textbf{Book} :Problems and Solutions on Electromagnetism\\
\textbf{Final Answer} :-\frac{v i \gamma}{\pi r_{0}^{2} c^{2}}\\


\textbf{Solution} :Use cylindrical coordinates with the $z$-axis along the axis of the ion beam such that the flow of the ions is in the $+z$ direction. Let $\Sigma^{\prime}$ and $\Sigma$ be the rest frame of the ions and the laboratory frame respectively, the former moving with velocity $v$ relative to the latter in the $+z$ direction. The charge per unit length in $\Sigma$ is $q$. In $\Sigma^{\prime}$ it is given by $q=\gamma\left(\frac{q^{\prime}+\beta j^{\prime}}{c}\right)=\gamma q^{\prime}$, or $q^{\prime}=q / \gamma$, where $\gamma=\left(1-\beta^{2}\right)^{-\frac{1}{2}}, \beta=\frac{v}{c}$. In $\Sigma^{\prime}$ the electronic field is given by Gauss' law $2 \pi E_{r}^{\prime}=\frac{r^{2}}{R^{2}} \frac{q^{\prime}}{\varepsilon_{0}}$ to be

$$
\mathbf{E}^{\prime}=\frac{r q^{\prime}}{2 \pi \varepsilon_{0} R^{2}} \mathbf{e}_{r} \cdot \quad(r<R)
$$

As the ions are stationary,

$$
B^{\prime}=0 \text {. }
$$

Transforming to $\Sigma$ we have $\mathbf{E}_{\perp}=\gamma\left(\mathbf{E}_{\perp}^{\prime}-\mathbf{v} \times \mathbf{B}_{\perp}^{\prime}\right)=\gamma \mathbf{E}_{\perp}^{\prime}, E_{\|}=E_{\|}^{\prime}=0$, or

$$
\mathbf{E}=\gamma \mathbf{E}^{\prime}=\frac{r \gamma q^{\prime}}{2 \pi \varepsilon_{0} R^{2}} \mathbf{e}_{r}=\frac{r q}{2 \pi \varepsilon_{0} R^{2}} \mathbf{e}_{r}
$$

and $\mathbf{B}_{\perp}=\gamma\left(\mathbf{B}_{\perp}^{\prime}+\frac{v \times E_{\perp}^{\prime}}{c^{2}}\right)=\gamma \frac{v \times E_{\perp}^{\prime}}{c^{2}}, \mathbf{B}_{\|}=\mathbf{B}_{\|}^{\prime}=0$, or

$$
\mathbf{B}=\gamma \frac{v}{c^{2}} \cdot \frac{r q^{\prime}}{2 \pi \varepsilon_{0} R^{2}} \mathbf{e}_{\theta}=\frac{v}{c} \cdot \frac{r q}{2 \pi \varepsilon_{0} c R^{2}} \mathbf{e}_{\theta} .
$$

Note that, as $r$ is transverse to $\mathbf{v}, r^{\prime}=r$. Hence the total force acting on an ion of charge $Q$ at distance $r<R$ from the axis in the laboratory frame is

$$
\begin{aligned}
\mathbf{F} &=Q \mathbf{E}+Q \mathbf{v} \times \mathbf{B} \\
&=\left(Q \cdot \frac{q r}{2 \pi \varepsilon_{0} R^{2}}-Q \frac{v^{2}}{c^{2}} \frac{q r}{2 \pi \varepsilon_{0} R^{2}}\right) \mathbf{e}_{r} \\
&=\frac{Q q r}{2 \pi \varepsilon_{0} R^{2}}\left(1-\frac{v^{2}}{c^{2}}\right) \mathbf{e}_{r}=\frac{Q q r}{2 \pi \varepsilon_{0} R^{2} \gamma^{2}} \mathbf{e}_{r} .
\end{aligned}
$$

If $v \ll c$, then $F=\frac{Q g r}{2 \pi \varepsilon_{0} R^{2}} \mathbf{e}_{r}$, which is what one would obtain if both the charge and the ion beam were stationary.

\textbf{Topic} :Relativity, Particle-Field Interactions\\
\textbf{Book} :Problems and Solutions on Electromagnetism\\
\textbf{Final Answer} :\frac{Q q r}{2 \pi \varepsilon_{0} R^{2} \gamma^{2}} \mathbf{e}_{r}\\


\textbf{Solution} :The Liénard-Wiechert potentials at $P$ due to the charge are given

$$
\phi=\frac{e}{4 \pi \varepsilon_{0}\left[r-\frac{\mathbf{v}}{c} \cdot \mathbf{r}\right]}, \quad \mathbf{A}=\frac{e \mathbf{v}}{4 \pi \varepsilon_{0} c^{2}\left[r-\frac{\mathbf{v}}{c} \cdot \mathbf{r}\right]}
$$

where $\mathbf{r}$ is the radius vector from the retarded position of the charge to the field point $P$, i.e.,

$$
\mathbf{r}=b \mathbf{e}_{x}-v\left(t-\frac{r}{c}\right) \mathbf{e}_{z}=b \mathbf{e}_{x}-v t^{\prime} \mathbf{e}_{z}
$$

with

$$
t^{\prime}=t-\frac{r}{c} .
$$

Thus

$$
r^{2}=r \cdot r=b^{2}+v^{2}\left(t-\frac{r}{c}\right)^{2}=b^{2}+v^{2}\left(t^{2}-\frac{2 r t}{c}+\frac{r^{2}}{c^{2}}\right)
$$

or

$$
\left(1-\frac{v^{2}}{c^{2}}\right) r^{2}+2 \frac{v^{2} t}{c} r-b^{2}-v^{2} t^{2}=0 .
$$

This is the retardation condition, with the solutions

$$
r=\frac{-\beta v t \pm \sqrt{\left(1-\beta^{2}\right) b^{2}+v^{2} t^{2}}}{1-\beta^{2}}
$$

where $\beta=\frac{v}{c}$.

However the upper sign is to be taken since $r \geq 0$. As $v=v e_{z}$,

$$
\begin{aligned}
r-\frac{\mathbf{v} \cdot \mathbf{r}}{c} &=r+\frac{v^{2} t^{\prime}}{c}=r+v \beta\left(t-\frac{r}{c}\right)=\left(1-\beta^{2}\right) r+v \beta t \\
&=\sqrt{\left(1-\beta^{2}\right) b^{2}+v^{2} t^{2}} .
\end{aligned}
$$

The scalar potential $\phi$ is then

$$
\phi=\frac{e}{4 \pi \varepsilon_{0} \sqrt{\left(1-\beta^{2}\right) b^{2}+v^{2} t^{2}}}
$$
\textbf{Topic} :Relativity, Particle-Field Interactions\\
\textbf{Book} :Problems and Solutions on Electromagnetism\\
\textbf{Final Answer} :\frac{e}{4 \pi \varepsilon_{0} \sqrt{\left(1-\beta^{2}\right) b^{2}+v^{2} t^{2}}}\\


\textbf{Solution} :The Liénard-Wiechert potentials at $P$ due to the charge are given

$$
\phi=\frac{e}{4 \pi \varepsilon_{0}\left[r-\frac{\mathbf{v}}{c} \cdot \mathbf{r}\right]}, \quad \mathbf{A}=\frac{e \mathbf{v}}{4 \pi \varepsilon_{0} c^{2}\left[r-\frac{\mathbf{v}}{c} \cdot \mathbf{r}\right]}
$$

where $\mathbf{r}$ is the radius vector from the retarded position of the charge to the field point $P$, i.e.,

$$
\mathbf{r}=b \mathbf{e}_{x}-v\left(t-\frac{r}{c}\right) \mathbf{e}_{z}=b \mathbf{e}_{x}-v t^{\prime} \mathbf{e}_{z}
$$

with

$$
t^{\prime}=t-\frac{r}{c} .
$$

Thus

$$
r^{2}=r \cdot r=b^{2}+v^{2}\left(t-\frac{r}{c}\right)^{2}=b^{2}+v^{2}\left(t^{2}-\frac{2 r t}{c}+\frac{r^{2}}{c^{2}}\right)
$$

or

$$
\left(1-\frac{v^{2}}{c^{2}}\right) r^{2}+2 \frac{v^{2} t}{c} r-b^{2}-v^{2} t^{2}=0 .
$$

This is the retardation condition, with the solutions

$$
r=\frac{-\beta v t \pm \sqrt{\left(1-\beta^{2}\right) b^{2}+v^{2} t^{2}}}{1-\beta^{2}}
$$

where $\beta=\frac{v}{c}$.

However the upper sign is to be taken since $r \geq 0$. As $v=v e_{z}$,

$$
\begin{aligned}
r-\frac{\mathbf{v} \cdot \mathbf{r}}{c} &=r+\frac{v^{2} t^{\prime}}{c}=r+v \beta\left(t-\frac{r}{c}\right)=\left(1-\beta^{2}\right) r+v \beta t \\
&=\sqrt{\left(1-\beta^{2}\right) b^{2}+v^{2} t^{2}} .
\end{aligned}
$$

The scalar potential $\phi$ is then

$$
\phi=\frac{e}{4 \pi \varepsilon_{0} \sqrt{\left(1-\beta^{2}\right) b^{2}+v^{2} t^{2}}}
$$

 The vector potential $A$ is

$$
\mathbf{A}=\frac{e v}{4 \pi \varepsilon_{0} c^{2} \sqrt{\left(1-\beta^{2}\right) b^{2}+v^{2} t^{2}}} \mathbf{e}_{z} .
$$
\textbf{Topic} :Relativity, Particle-Field Interactions\\
\textbf{Book} :Problems and Solutions on Electromagnetism\\
\textbf{Final Answer} :\frac{e v}{4 \pi \varepsilon_{0} c^{2} \sqrt{\left(1-\beta^{2}\right) b^{2}+v^{2} t^{2}}} \mathbf{e}_{z}\\


\textbf{Solution} :The Liénard-Wiechert potentials at $P$ due to the charge are given

$$
\phi=\frac{e}{4 \pi \varepsilon_{0}\left[r-\frac{\mathbf{v}}{c} \cdot \mathbf{r}\right]}, \quad \mathbf{A}=\frac{e \mathbf{v}}{4 \pi \varepsilon_{0} c^{2}\left[r-\frac{\mathbf{v}}{c} \cdot \mathbf{r}\right]}
$$

where $\mathbf{r}$ is the radius vector from the retarded position of the charge to the field point $P$, i.e.,

$$
\mathbf{r}=b \mathbf{e}_{x}-v\left(t-\frac{r}{c}\right) \mathbf{e}_{z}=b \mathbf{e}_{x}-v t^{\prime} \mathbf{e}_{z}
$$

with

$$
t^{\prime}=t-\frac{r}{c} .
$$

Thus

$$
r^{2}=r \cdot r=b^{2}+v^{2}\left(t-\frac{r}{c}\right)^{2}=b^{2}+v^{2}\left(t^{2}-\frac{2 r t}{c}+\frac{r^{2}}{c^{2}}\right)
$$

or

$$
\left(1-\frac{v^{2}}{c^{2}}\right) r^{2}+2 \frac{v^{2} t}{c} r-b^{2}-v^{2} t^{2}=0 .
$$

This is the retardation condition, with the solutions

$$
r=\frac{-\beta v t \pm \sqrt{\left(1-\beta^{2}\right) b^{2}+v^{2} t^{2}}}{1-\beta^{2}}
$$

where $\beta=\frac{v}{c}$.

However the upper sign is to be taken since $r \geq 0$. As $v=v e_{z}$,

$$
\begin{aligned}
r-\frac{\mathbf{v} \cdot \mathbf{r}}{c} &=r+\frac{v^{2} t^{\prime}}{c}=r+v \beta\left(t-\frac{r}{c}\right)=\left(1-\beta^{2}\right) r+v \beta t \\
&=\sqrt{\left(1-\beta^{2}\right) b^{2}+v^{2} t^{2}} .
\end{aligned}
$$

The scalar potential $\phi$ is then

$$
\phi=\frac{e}{4 \pi \varepsilon_{0} \sqrt{\left(1-\beta^{2}\right) b^{2}+v^{2} t^{2}}}
$$

 The vector potential $A$ is

$$
\mathbf{A}=\frac{e v}{4 \pi \varepsilon_{0} c^{2} \sqrt{\left(1-\beta^{2}\right) b^{2}+v^{2} t^{2}}} \mathbf{e}_{z} .
$$

 The electric field at $P$ is obtained by differentiating the LiénardWiechert potentials:

$$
\mathbf{E}(t)=-\nabla \phi-\frac{\partial \mathbf{A}}{\partial t} .
$$

For the spatial differentiation, $b$ is to be first replaced by $x$. We then have

$$
(\nabla \phi)_{b}=\left(\frac{\partial \phi}{\partial x}\right)_{b} \mathbf{e}_{x}=-\frac{e\left(1-\beta^{2}\right) b}{4 \pi \varepsilon_{0}\left[\left(1-\beta^{2}\right) b^{2}+v^{2} t^{2}\right]^{3 / 2}} \mathbf{e}_{x}
$$

As $A$ is in the $z$-direction, it does not contribute to $E_{x}$. Hence

$$
E_{x}=\frac{e\left(1-\beta^{2}\right) b}{4 \pi \varepsilon_{0}\left[\left(1-\beta^{2}\right) b^{2}+v^{2} t^{2}\right]^{3 / 2}} .
$$

\textbf{Topic} :Relativity, Particle-Field Interactions\\
\textbf{Book} :Problems and Solutions on Electromagnetism\\
\textbf{Final Answer} :\frac{e\left(1-\beta^{2}\right) b}{4 \pi \varepsilon_{0}\left[\left(1-\beta^{2}\right) b^{2}+v^{2} t^{2}\right]^{3 / 2}}\\


\textbf{Solution} :For a non-relativistic particle of charge $e$ the radiation field is given

$$
\begin{gathered}
\mathbf{E}=\frac{e \mathbf{n}^{\prime} \times\left(\mathbf{n}^{\prime} \times \dot{\boldsymbol{\beta}} c\right)}{4 \pi \varepsilon_{0} c^{2} r}=\frac{e}{4 \pi \varepsilon_{0} c r} \mathbf{n}^{\prime} \times\left(\mathbf{n}^{\prime} \times \dot{\boldsymbol{\beta}}\right), \\
\mathbf{B}=\frac{1}{c} \mathbf{n}^{\prime} \times \mathbf{E}
\end{gathered}
$$

where $r$ is the distance of the observer from the charge. The Poynting vector at the observer is then

$$
\mathbf{N}=\mathbf{E} \times \mathbf{H}=\frac{1}{\mu_{0}} \mathbf{E} \times \mathbf{B}=\frac{1}{\mu_{0} c} E^{2} \mathbf{n}^{\prime}=\frac{e^{2}}{16 \pi^{2} \varepsilon_{0} c r^{2}}\left|\mathbf{n}^{\prime} \times\left(\mathbf{n}^{\prime} \times \dot{\boldsymbol{\beta}}\right)\right|^{2} \mathbf{n}^{\prime} .
$$

Let $\theta$ be the angle between $\mathbf{n}^{\prime}$ and $\dot{\boldsymbol{\beta}}$, then

$$
\frac{d P}{d \Omega}=\frac{\mathbf{N} \cdot \mathbf{n}^{\prime}}{r^{-2}}=\frac{e^{2}}{16 \pi^{2} \varepsilon_{0} c}|\dot{\boldsymbol{\beta}}|^{2} \sin ^{2} \theta .
$$

This result is not changed by time averging unless the motion of the charge is periodic.
\\
\textbf{Topic} :Relativity, Particle-Field Interactions\\
\textbf{Book} :Problems and Solutions on Electromagnetism\\
\textbf{Final Answer} :\frac{e^{2}}{16 \pi^{2} \varepsilon_{0} c}|\dot{\boldsymbol{\beta}}|^{2} \sin ^{2} \theta\\


\textbf{Solution} :For a non-relativistic particle of charge $e$ the radiation field is given

$$
\begin{gathered}
\mathbf{E}=\frac{e \mathbf{n}^{\prime} \times\left(\mathbf{n}^{\prime} \times \dot{\boldsymbol{\beta}} c\right)}{4 \pi \varepsilon_{0} c^{2} r}=\frac{e}{4 \pi \varepsilon_{0} c r} \mathbf{n}^{\prime} \times\left(\mathbf{n}^{\prime} \times \dot{\boldsymbol{\beta}}\right), \\
\mathbf{B}=\frac{1}{c} \mathbf{n}^{\prime} \times \mathbf{E}
\end{gathered}
$$

where $r$ is the distance of the observer from the charge. The Poynting vector at the observer is then

$$
\mathbf{N}=\mathbf{E} \times \mathbf{H}=\frac{1}{\mu_{0}} \mathbf{E} \times \mathbf{B}=\frac{1}{\mu_{0} c} E^{2} \mathbf{n}^{\prime}=\frac{e^{2}}{16 \pi^{2} \varepsilon_{0} c r^{2}}\left|\mathbf{n}^{\prime} \times\left(\mathbf{n}^{\prime} \times \dot{\boldsymbol{\beta}}\right)\right|^{2} \mathbf{n}^{\prime} .
$$

Let $\theta$ be the angle between $\mathbf{n}^{\prime}$ and $\dot{\boldsymbol{\beta}}$, then

$$
\frac{d P}{d \Omega}=\frac{\mathbf{N} \cdot \mathbf{n}^{\prime}}{r^{-2}}=\frac{e^{2}}{16 \pi^{2} \varepsilon_{0} c}|\dot{\boldsymbol{\beta}}|^{2} \sin ^{2} \theta .
$$

This result is not changed by time averging unless the motion of the charge is periodic.
 If $z=a \cos \left(\omega_{0} t\right)$, then $\dot{\beta} c=\ddot{z}=-a \omega_{0}^{2} \cos \left(\omega_{0} t\right)$ and as $\frac{1}{T} \int_{0}^{T} \cos ^{2}\left(\omega_{0} t\right) d t=\frac{1}{2}$, where $T$ is the period,

Hence

$$
\left\langle(\dot{\beta} c)^{2}\right\rangle=\frac{1}{2} Q^{2} \omega_{0}^{4} .
$$

$$
\frac{d P}{d \Omega}=\frac{e^{2} a^{2} \omega_{0}^{4}}{32 \pi^{2} \varepsilon_{0} c^{3}} \sin ^{2} \theta
$$
\textbf{Topic} :Relativity, Particle-Field Interactions\\
\textbf{Book} :Problems and Solutions on Electromagnetism\\
\textbf{Final Answer} :\frac{e^{2} a^{2} \omega_{0}^{4}}{32 \pi^{2} \varepsilon_{0} c^{3}} \sin ^{2} \theta\\


\textbf{Solution} :For a non-relativistic particle of charge $e$ the radiation field is given

$$
\begin{gathered}
\mathbf{E}=\frac{e \mathbf{n}^{\prime} \times\left(\mathbf{n}^{\prime} \times \dot{\boldsymbol{\beta}} c\right)}{4 \pi \varepsilon_{0} c^{2} r}=\frac{e}{4 \pi \varepsilon_{0} c r} \mathbf{n}^{\prime} \times\left(\mathbf{n}^{\prime} \times \dot{\boldsymbol{\beta}}\right), \\
\mathbf{B}=\frac{1}{c} \mathbf{n}^{\prime} \times \mathbf{E}
\end{gathered}
$$

where $r$ is the distance of the observer from the charge. The Poynting vector at the observer is then

$$
\mathbf{N}=\mathbf{E} \times \mathbf{H}=\frac{1}{\mu_{0}} \mathbf{E} \times \mathbf{B}=\frac{1}{\mu_{0} c} E^{2} \mathbf{n}^{\prime}=\frac{e^{2}}{16 \pi^{2} \varepsilon_{0} c r^{2}}\left|\mathbf{n}^{\prime} \times\left(\mathbf{n}^{\prime} \times \dot{\boldsymbol{\beta}}\right)\right|^{2} \mathbf{n}^{\prime} .
$$

Let $\theta$ be the angle between $\mathbf{n}^{\prime}$ and $\dot{\boldsymbol{\beta}}$, then

$$
\frac{d P}{d \Omega}=\frac{\mathbf{N} \cdot \mathbf{n}^{\prime}}{r^{-2}}=\frac{e^{2}}{16 \pi^{2} \varepsilon_{0} c}|\dot{\boldsymbol{\beta}}|^{2} \sin ^{2} \theta .
$$

This result is not changed by time averging unless the motion of the charge is periodic.
 If $z=a \cos \left(\omega_{0} t\right)$, then $\dot{\beta} c=\ddot{z}=-a \omega_{0}^{2} \cos \left(\omega_{0} t\right)$ and as $\frac{1}{T} \int_{0}^{T} \cos ^{2}\left(\omega_{0} t\right) d t=\frac{1}{2}$, where $T$ is the period,

Hence

$$
\left\langle(\dot{\beta} c)^{2}\right\rangle=\frac{1}{2} Q^{2} \omega_{0}^{4} .
$$

$$
\frac{d P}{d \Omega}=\frac{e^{2} a^{2} \omega_{0}^{4}}{32 \pi^{2} \varepsilon_{0} c^{3}} \sin ^{2} \theta
$$

 The circular motion of the particle in the $x y$ plane may be considered as superposition of two mutually perpendicular harmonic oscillations:

$$
\mathbf{R}\left(t^{\prime}\right)=R \cos \left(\omega_{0} t^{\prime}\right) \mathbf{e}_{x}+R \sin \left(\omega_{0} t^{\prime}\right) \mathbf{e}_{y} .
$$

In spherical coordinates let the observer have radius vector $r(r, \theta, \varphi)$ from the center of the circle, which is also the origin of the coordinate system. The angles between $\mathbf{r}$ or $\mathbf{n}^{\prime}$ and $\dot{\beta}$ for the two oscillations are given by

$$
\begin{aligned}
&\cos \theta_{1}=\sin \theta \cos \varphi, \\
&\cos \theta_{2}=\sin \theta \cos \left(\frac{\pi}{2}-\varphi\right)=\sin \theta \sin \varphi .
\end{aligned}
$$

Using the results of (b) we have

$$
\begin{aligned}
\frac{d P}{d \Omega} &=\frac{e^{2} R^{2} \omega_{0}^{4}}{32 \pi^{2} \varepsilon_{0} c^{3}}\left(\sin ^{2} \theta_{1}+\sin ^{2} \theta_{2}\right) \\
&=\frac{e^{2} R^{2} \omega_{0}^{4}}{32 \pi^{2} \varepsilon_{0} c^{3}}\left(1+\cos ^{2} \theta\right) .
\end{aligned}
$$
\textbf{Topic} :Relativity, Particle-Field Interactions\\
\textbf{Book} :Problems and Solutions on Electromagnetism\\
\textbf{Final Answer} :\frac{e^{2} R^{2} \omega_{0}^{4}}{32 \pi^{2} \varepsilon_{0} c^{3}}\left(1+\cos ^{2} \theta\right)\\


\textbf{Solution} :For a non-relativistic particle of charge $e$ the radiation field is given

$$
\begin{gathered}
\mathbf{E}=\frac{e \mathbf{n}^{\prime} \times\left(\mathbf{n}^{\prime} \times \dot{\boldsymbol{\beta}} c\right)}{4 \pi \varepsilon_{0} c^{2} r}=\frac{e}{4 \pi \varepsilon_{0} c r} \mathbf{n}^{\prime} \times\left(\mathbf{n}^{\prime} \times \dot{\boldsymbol{\beta}}\right), \\
\mathbf{B}=\frac{1}{c} \mathbf{n}^{\prime} \times \mathbf{E}
\end{gathered}
$$

where $r$ is the distance of the observer from the charge. The Poynting vector at the observer is then

$$
\mathbf{N}=\mathbf{E} \times \mathbf{H}=\frac{1}{\mu_{0}} \mathbf{E} \times \mathbf{B}=\frac{1}{\mu_{0} c} E^{2} \mathbf{n}^{\prime}=\frac{e^{2}}{16 \pi^{2} \varepsilon_{0} c r^{2}}\left|\mathbf{n}^{\prime} \times\left(\mathbf{n}^{\prime} \times \dot{\boldsymbol{\beta}}\right)\right|^{2} \mathbf{n}^{\prime} .
$$

Let $\theta$ be the angle between $\mathbf{n}^{\prime}$ and $\dot{\boldsymbol{\beta}}$, then

$$
\frac{d P}{d \Omega}=\frac{\mathbf{N} \cdot \mathbf{n}^{\prime}}{r^{-2}}=\frac{e^{2}}{16 \pi^{2} \varepsilon_{0} c}|\dot{\boldsymbol{\beta}}|^{2} \sin ^{2} \theta .
$$

This result is not changed by time averging unless the motion of the charge is periodic.
 If $z=a \cos \left(\omega_{0} t\right)$, then $\dot{\beta} c=\ddot{z}=-a \omega_{0}^{2} \cos \left(\omega_{0} t\right)$ and as $\frac{1}{T} \int_{0}^{T} \cos ^{2}\left(\omega_{0} t\right) d t=\frac{1}{2}$, where $T$ is the period,

Hence

$$
\left\langle(\dot{\beta} c)^{2}\right\rangle=\frac{1}{2} Q^{2} \omega_{0}^{4} .
$$

$$
\frac{d P}{d \Omega}=\frac{e^{2} a^{2} \omega_{0}^{4}}{32 \pi^{2} \varepsilon_{0} c^{3}} \sin ^{2} \theta
$$

 The circular motion of the particle in the $x y$ plane may be considered as superposition of two mutually perpendicular harmonic oscillations:

$$
\mathbf{R}\left(t^{\prime}\right)=R \cos \left(\omega_{0} t^{\prime}\right) \mathbf{e}_{x}+R \sin \left(\omega_{0} t^{\prime}\right) \mathbf{e}_{y} .
$$

In spherical coordinates let the observer have radius vector $r(r, \theta, \varphi)$ from the center of the circle, which is also the origin of the coordinate system. The angles between $\mathbf{r}$ or $\mathbf{n}^{\prime}$ and $\dot{\beta}$ for the two oscillations are given by

$$
\begin{aligned}
&\cos \theta_{1}=\sin \theta \cos \varphi, \\
&\cos \theta_{2}=\sin \theta \cos \left(\frac{\pi}{2}-\varphi\right)=\sin \theta \sin \varphi .
\end{aligned}
$$

Using the results of (b) we have

$$
\begin{aligned}
\frac{d P}{d \Omega} &=\frac{e^{2} R^{2} \omega_{0}^{4}}{32 \pi^{2} \varepsilon_{0} c^{3}}\left(\sin ^{2} \theta_{1}+\sin ^{2} \theta_{2}\right) \\
&=\frac{e^{2} R^{2} \omega_{0}^{4}}{32 \pi^{2} \varepsilon_{0} c^{3}}\left(1+\cos ^{2} \theta\right) .
\end{aligned}
$$

 For the cases (a) and (b), the curves $\rho=\frac{d P}{d \Omega}$ vs. $\theta$ are sketched in Figs. $5.11$ and $5.12$ respectively, where $z$ is the direction of $\dot{\beta}$.

MATHPIX IMAGE

Fig. $5.11$ 

MATHPIX IMAGE

Fig. $5.12$
\textbf{Topic} :Relativity, Particle-Field Interactions\\
\textbf{Book} :Problems and Solutions on Electromagnetism\\
\textbf{Final Answer} :\frac{e^{2} R^{2} \omega_{0}^{4}}{32 \pi^{2} \varepsilon_{0} c^{3}}\left(1+\cos ^{2} \theta\right)\\


\textbf{Solution} :For a non-relativistic particle of charge $e$ the radiation field is given

$$
\begin{gathered}
\mathbf{E}=\frac{e \mathbf{n}^{\prime} \times\left(\mathbf{n}^{\prime} \times \dot{\boldsymbol{\beta}} c\right)}{4 \pi \varepsilon_{0} c^{2} r}=\frac{e}{4 \pi \varepsilon_{0} c r} \mathbf{n}^{\prime} \times\left(\mathbf{n}^{\prime} \times \dot{\boldsymbol{\beta}}\right), \\
\mathbf{B}=\frac{1}{c} \mathbf{n}^{\prime} \times \mathbf{E}
\end{gathered}
$$

where $r$ is the distance of the observer from the charge. The Poynting vector at the observer is then

$$
\mathbf{N}=\mathbf{E} \times \mathbf{H}=\frac{1}{\mu_{0}} \mathbf{E} \times \mathbf{B}=\frac{1}{\mu_{0} c} E^{2} \mathbf{n}^{\prime}=\frac{e^{2}}{16 \pi^{2} \varepsilon_{0} c r^{2}}\left|\mathbf{n}^{\prime} \times\left(\mathbf{n}^{\prime} \times \dot{\boldsymbol{\beta}}\right)\right|^{2} \mathbf{n}^{\prime} .
$$

Let $\theta$ be the angle between $\mathbf{n}^{\prime}$ and $\dot{\boldsymbol{\beta}}$, then

$$
\frac{d P}{d \Omega}=\frac{\mathbf{N} \cdot \mathbf{n}^{\prime}}{r^{-2}}=\frac{e^{2}}{16 \pi^{2} \varepsilon_{0} c}|\dot{\boldsymbol{\beta}}|^{2} \sin ^{2} \theta .
$$

This result is not changed by time averging unless the motion of the charge is periodic.
 If $z=a \cos \left(\omega_{0} t\right)$, then $\dot{\beta} c=\ddot{z}=-a \omega_{0}^{2} \cos \left(\omega_{0} t\right)$ and as $\frac{1}{T} \int_{0}^{T} \cos ^{2}\left(\omega_{0} t\right) d t=\frac{1}{2}$, where $T$ is the period,

Hence

$$
\left\langle(\dot{\beta} c)^{2}\right\rangle=\frac{1}{2} Q^{2} \omega_{0}^{4} .
$$

$$
\frac{d P}{d \Omega}=\frac{e^{2} a^{2} \omega_{0}^{4}}{32 \pi^{2} \varepsilon_{0} c^{3}} \sin ^{2} \theta
$$

 The circular motion of the particle in the $x y$ plane may be considered as superposition of two mutually perpendicular harmonic oscillations:

$$
\mathbf{R}\left(t^{\prime}\right)=R \cos \left(\omega_{0} t^{\prime}\right) \mathbf{e}_{x}+R \sin \left(\omega_{0} t^{\prime}\right) \mathbf{e}_{y} .
$$

In spherical coordinates let the observer have radius vector $r(r, \theta, \varphi)$ from the center of the circle, which is also the origin of the coordinate system. The angles between $\mathbf{r}$ or $\mathbf{n}^{\prime}$ and $\dot{\beta}$ for the two oscillations are given by

$$
\begin{aligned}
&\cos \theta_{1}=\sin \theta \cos \varphi, \\
&\cos \theta_{2}=\sin \theta \cos \left(\frac{\pi}{2}-\varphi\right)=\sin \theta \sin \varphi .
\end{aligned}
$$

Using the results of (b) we have

$$
\begin{aligned}
\frac{d P}{d \Omega} &=\frac{e^{2} R^{2} \omega_{0}^{4}}{32 \pi^{2} \varepsilon_{0} c^{3}}\left(\sin ^{2} \theta_{1}+\sin ^{2} \theta_{2}\right) \\
&=\frac{e^{2} R^{2} \omega_{0}^{4}}{32 \pi^{2} \varepsilon_{0} c^{3}}\left(1+\cos ^{2} \theta\right) .
\end{aligned}
$$

 For the cases (a) and (b), the curves $\rho=\frac{d P}{d \Omega}$ vs. $\theta$ are sketched in Figs. $5.11$ and $5.12$ respectively, where $z$ is the direction of $\dot{\beta}$.

MATHPIX IMAGE

Fig. $5.11$ 

MATHPIX IMAGE

Fig. $5.12$

 For $\beta \approx 0$, the direction of maximum intensity is along $\theta=\frac{\pi}{2}$. As $\beta \rightarrow 1$, the direction of maximum intensity tends more and more toward the direction $\theta=0$, i.e., the direction of $\dot{\beta}$. In fact the radiation will be concentrated mainly in a cone with $\Delta \theta \sim \frac{1}{\gamma}$ about the direction of $\dot{\beta}$. However there is no radiation exactly along that direction.

\textbf{Topic} :Relativity, Particle-Field Interactions\\
\textbf{Book} :Problems and Solutions on Electromagnetism\\
\textbf{Final Answer} :\frac{e^{2} R^{2} \omega_{0}^{4}}{32 \pi^{2} \varepsilon_{0} c^{3}}\left(1+\cos ^{2} \theta\right)\\


\textbf{Solution} :As shown in Fig. 5.13, the radiation emitted by the charge at $Q^{\prime}$ at time $t^{\prime}$ arrives at $P$ at time $t$ when the charge is at Q. As the radiation propagates at the speed $c / n$ and the particle has speed $v$ where $v>c / n$, we have

$$
\mathrm{Q}^{\prime} \mathrm{P}=\frac{c}{n}\left(t-t^{\prime}\right), \quad \mathrm{Q}^{\prime} \mathrm{Q}=v\left(t-t^{\prime}\right),
$$

or

$$
\frac{\mathrm{Q}^{\prime} \mathrm{P}}{\mathrm{Q}^{\prime} \mathrm{Q}}=\cos \theta=\frac{c}{v n}=\frac{1}{\beta n},
$$

where $\beta=\frac{v}{c}$. At all the points intermediate between $Q^{\prime}$ and $Q$ the radiation emitted will arrive at the line QP at time $t$. Hence QP forms the wavefront of all radiation emitted prior to $t$.

MATHPIX IMAGE

Fig. $5.13$

 As $|\cos \theta| \leq 1$, we require $\beta \geq \frac{1}{n}$ for emission of Čerenkov radiation. Hence we require

$$
\gamma=\frac{1}{\sqrt{1-\beta^{2}}} \geq\left(1-\frac{1}{n^{2}}\right)^{-\frac{1}{2}}=\frac{n}{\sqrt{n^{2}-1}} .
$$



Thus the particle must have a kinetic energy greater than

$$
\begin{aligned}
T &=(\gamma-1) m_{0} c^{2} \\
&=\left[\frac{n}{\sqrt{(n+1)(n-1)}}-1\right] m_{0} c^{2} \\
& \approx\left(\frac{1}{\sqrt{2 \times 1.35 \times 10^{-4}}}-1\right) \times 0.5 \\
& \approx 29.93 \mathrm{MeV} .
\end{aligned}
$$
\textbf{Topic} :Relativity, Particle-Field Interactions\\
\textbf{Book} :Problems and Solutions on Electromagnetism\\
\textbf{Final Answer} :\left[\frac{n}{\sqrt{(n+1)(n-1)}}-1\right] m_{0} c^{2} \\
& \approx\left(\frac{1}{\sqrt{2 \times 135 \times 10^{-4}}}-1\right) \times 05 \\
& \approx 2993 \mathrm{MeV}\\


\textbf{Solution} :As shown in Fig. 5.13, the radiation emitted by the charge at $Q^{\prime}$ at time $t^{\prime}$ arrives at $P$ at time $t$ when the charge is at Q. As the radiation propagates at the speed $c / n$ and the particle has speed $v$ where $v>c / n$, we have

$$
\mathrm{Q}^{\prime} \mathrm{P}=\frac{c}{n}\left(t-t^{\prime}\right), \quad \mathrm{Q}^{\prime} \mathrm{Q}=v\left(t-t^{\prime}\right),
$$

or

$$
\frac{\mathrm{Q}^{\prime} \mathrm{P}}{\mathrm{Q}^{\prime} \mathrm{Q}}=\cos \theta=\frac{c}{v n}=\frac{1}{\beta n},
$$

where $\beta=\frac{v}{c}$. At all the points intermediate between $Q^{\prime}$ and $Q$ the radiation emitted will arrive at the line QP at time $t$. Hence QP forms the wavefront of all radiation emitted prior to $t$.

MATHPIX IMAGE

Fig. $5.13$

 As $|\cos \theta| \leq 1$, we require $\beta \geq \frac{1}{n}$ for emission of Čerenkov radiation. Hence we require

$$
\gamma=\frac{1}{\sqrt{1-\beta^{2}}} \geq\left(1-\frac{1}{n^{2}}\right)^{-\frac{1}{2}}=\frac{n}{\sqrt{n^{2}-1}} .
$$



Thus the particle must have a kinetic energy greater than

$$
\begin{aligned}
T &=(\gamma-1) m_{0} c^{2} \\
&=\left[\frac{n}{\sqrt{(n+1)(n-1)}}-1\right] m_{0} c^{2} \\
& \approx\left(\frac{1}{\sqrt{2 \times 1.35 \times 10^{-4}}}-1\right) \times 0.5 \\
& \approx 29.93 \mathrm{MeV} .
\end{aligned}
$$

 For a relativistic particle of momentum $P \gg m_{0} c$,

$$
\gamma=\frac{E}{m_{0} c^{2}}=\frac{\sqrt{P^{2} c^{2}+m_{0}^{2} c^{4}}}{m_{0} c^{2}} \approx \frac{P}{m_{0} c} .
$$

With $P$ fixed we have

$$
d \gamma=-\frac{P}{c} \cdot \frac{d m_{0}}{m_{0}^{2}} .
$$

Now $\beta=\frac{\gamma \beta}{\gamma}=\sqrt{\frac{\gamma^{2}-1}{\gamma^{2}}}=\sqrt{1-\frac{1}{\gamma^{2}}} \approx 1-\frac{1}{2 \gamma^{2}}$ for $\gamma \gg 1$, so that

$$
d \beta=\frac{d \gamma}{\gamma^{3}} .
$$

For the Čerenkov radiation emitted by the particle, we have

$$
\cos \theta=\frac{1}{\beta n},
$$

or

$$
d \beta=n \beta^{2} \sin \theta d \theta .
$$

Combining the above we have

$$
\left|\frac{d m_{0}}{m_{0}}\right|=\frac{m_{0} c}{P} d \gamma \approx \frac{d \gamma}{\gamma}=\gamma^{2} \beta=n \beta^{2} \gamma^{2} \sin \theta d \theta=\beta \gamma^{2} \tan \theta d \theta .
$$

With $\gamma=\frac{P c}{m_{0} c^{2}}=\frac{100}{1}=100, n=1+1.35 \times 10^{-4}$, we have

$$
\beta \approx 1-\frac{1}{2 \times 10^{4}}=1-5 \times 10^{-5},
$$

$\cos \theta=\frac{1}{\beta n}=\left(1-5 \times 10^{-5}\right)^{-1}\left(1+1.35 \times 10^{-4}\right)^{-1} \approx 1-8.5 \times 10^{-5}$, 

$$
\begin{aligned}
\tan \theta=\sqrt{\frac{1}{\cos ^{2} \theta}-1} &=\sqrt{\left(1-8.5 \times 10^{-5}\right)^{-2}-1} \\
& \approx \sqrt{1.7 \times 10^{-4}} \approx 1.3 \times 10^{-2},
\end{aligned}
$$

and hence

$$
\left|\frac{d m_{0}}{m_{0}}\right|=\left(1-5 \times 10^{-5}\right) \times 10^{4} \times 1.3 \times 10^{-2} \times 10^{-3}=0.13
$$

\textbf{Topic} :Relativity, Particle-Field Interactions\\
\textbf{Book} :Problems and Solutions on Electromagnetism\\
\textbf{Final Answer} :013\\


\textbf{Solution} :As

$$
\frac{d(m \gamma v)}{d t}=e E, \quad m \gamma v=\int_{0}^{t} e E d t=e E t,
$$

where $E$ is the intensity of the uniform electric field,

$$
\gamma=\left(1-\beta^{2}\right)^{-\frac{1}{2}} \quad \text { with } \quad \beta=\frac{v}{c} .
$$
\textbf{Topic} :Relativity, Particle-Field Interactions\\
\textbf{Book} :Problems and Solutions on Electromagnetism\\
\textbf{Final Answer} :\frac{v}{c}\\


\textbf{Solution} :As

$$
\frac{d(m \gamma v)}{d t}=e E, \quad m \gamma v=\int_{0}^{t} e E d t=e E t,
$$

where $E$ is the intensity of the uniform electric field,

$$
\gamma=\left(1-\beta^{2}\right)^{-\frac{1}{2}} \quad \text { with } \quad \beta=\frac{v}{c} .
$$

 As

$$
m \gamma \beta c=e E t,
$$

or

$$
\gamma \beta=\left(\gamma^{2}-1\right)^{\frac{1}{2}}=\frac{e E t}{m c},
$$

we have

$$
\gamma^{2}=\frac{1}{1-\beta^{2}}=\left(\frac{e E t}{m c}\right)^{2}+1
$$

or

$$
\beta^{2}=\frac{(e E t)^{2}}{(e E t)^{2}+(m c)^{2}},
$$

giving

$$
v=\beta c=\frac{e E c t}{\sqrt{(e E t)^{2}+(m c)^{2}}} \text {. }
$$
frame is
\textbf{Topic} :Relativity, Particle-Field Interactions\\
\textbf{Book} :Problems and Solutions on Electromagnetism\\
\textbf{Final Answer} :\frac{e E c t}{\sqrt{(e E t)^{2}+(m c)^{2}}}\\


\textbf{Solution} :In uniform magnetic field $B$ the motion of an electron is described in Gaussian units by

$$
\frac{d p}{d t}=\frac{e}{c} \mathbf{v} \times \mathbf{B},
$$

where $p$ is the momentum of the electron,

$$
\mathbf{p}=m \boldsymbol{v},
$$

with $\gamma=\left(1-\beta^{2}\right)^{-\frac{1}{2}}, \beta=\frac{v}{c}$. Since $\frac{e}{c} \mathbf{v} \times \mathbf{B} \cdot \mathbf{v}=0$, the magnetic force does no work and the magnitude of the velocity does not change, i.e., $v$, and hence $\gamma$, are constant. For circular motion,

$$
\left|\frac{d v}{d t}\right|=\frac{v^{2}}{R} .
$$

Then

$$
m \gamma\left|\frac{d v}{d t}\right|=\frac{e}{c}|\mathbf{v} \times \mathbf{B}| .
$$

As $\mathbf{v}$ is normal to $B$, we have

$$
m \gamma \frac{v^{2}}{R}=\frac{e}{c} v B
$$

or

$$
B=\frac{p c}{e R} \text {. }
$$

With $E \gg m c^{2}, p c=\sqrt{E^{2}-m^{2} c^{4}} \approx E$ and

$$
B \approx \frac{E}{e R} \approx 0.28 \times 10^{4} \mathrm{Gs} .
$$
\textbf{Topic} :Relativity, Particle-Field Interactions\\
\textbf{Book} :Problems and Solutions on Electromagnetism\\
\textbf{Final Answer} :\frac{e E c t}{\sqrt{(e E t)^{2}+(m c)^{2}}}\\


\textbf{Solution} :In uniform magnetic field $B$ the motion of an electron is described in Gaussian units by

$$
\frac{d p}{d t}=\frac{e}{c} \mathbf{v} \times \mathbf{B},
$$

where $p$ is the momentum of the electron,

$$
\mathbf{p}=m \boldsymbol{v},
$$

with $\gamma=\left(1-\beta^{2}\right)^{-\frac{1}{2}}, \beta=\frac{v}{c}$. Since $\frac{e}{c} \mathbf{v} \times \mathbf{B} \cdot \mathbf{v}=0$, the magnetic force does no work and the magnitude of the velocity does not change, i.e., $v$, and hence $\gamma$, are constant. For circular motion,

$$
\left|\frac{d v}{d t}\right|=\frac{v^{2}}{R} .
$$

Then

$$
m \gamma\left|\frac{d v}{d t}\right|=\frac{e}{c}|\mathbf{v} \times \mathbf{B}| .
$$

As $\mathbf{v}$ is normal to $B$, we have

$$
m \gamma \frac{v^{2}}{R}=\frac{e}{c} v B
$$

or

$$
B=\frac{p c}{e R} \text {. }
$$

With $E \gg m c^{2}, p c=\sqrt{E^{2}-m^{2} c^{4}} \approx E$ and

$$
B \approx \frac{E}{e R} \approx 0.28 \times 10^{4} \mathrm{Gs} .
$$

 The rate of radiation of an accelerated non-relativistic electron is

$$
P=\frac{2}{3} \frac{e^{2}}{c^{3}}|\dot{v}|^{2}=\frac{2}{3} \frac{e^{2}}{m^{2} c^{3}}\left(\frac{d p}{d t} \cdot \frac{d p}{d t}\right)
$$

where $\mathbf{v}$ and $\mathbf{p}$ are respectively the velocity and momentum of the electron. For a relativistic electron, the formula is modified to

$$
P=\frac{2}{3} \frac{e^{2}}{m^{2} c^{5}}\left(\frac{d p_{\mu}}{d \tau} \frac{d p^{\mu}}{d \tau}\right),
$$

where $d \tau=\frac{d t}{\gamma}, p_{\mu}$ and $p^{\mu}$ are respectively the covariant and contravariant momentum-energy four-vector of the electron:

$$
p_{\mu}=(\mathbf{p} c,-E), \quad p^{\mu}=(\mathbf{p} c, E) .
$$

Thus

$$
\frac{d p_{\mu}}{d \tau} \frac{d p^{\mu}}{d \tau}=\left(\frac{d p}{d \tau} \cdot \frac{d p}{d \tau}\right) c^{2}-\left(\frac{d E}{d \tau}\right)^{2}
$$



Since the energy loss of the electron per revolution is very small, we can take approximations $\frac{d E}{d \tau} \approx 0$ and $\gamma \approx$ constant. Then

$$
\frac{d p}{d \tau}=\gamma \frac{d p}{d t}=m \gamma^{2} \frac{d v}{d t} .
$$

Substitution in the expression for $\tau$ gives

$$
P=\frac{2}{3} \frac{e^{2}}{m^{2} c^{5}} m^{2} \gamma^{4} c^{2}\left|\frac{d v}{d t}\right|^{2}=\frac{2}{3} \frac{e^{2} c}{R^{2}}\left(\frac{v}{c}\right)^{4} \gamma^{4} .
$$

The energy loss per revolution is

$$
\Delta E=\frac{2 \pi R}{v} P=\frac{4 \pi}{3} \frac{e^{2}}{R}\left(\frac{v}{c}\right)^{3} \gamma^{4}
$$

As $\gamma=\frac{E}{m c^{2}}$,

$$
\frac{\Delta E}{E}=\frac{4 \pi}{3}\left(\frac{v}{c}\right)^{3}\left(\frac{e^{2}}{m c^{2} R}\right)\left(\frac{E}{m c^{2}}\right)^{3} \approx 5 \times 10^{-4}
$$

\textbf{Topic} :Relativity, Particle-Field Interactions\\
\textbf{Book} :Problems and Solutions on Electromagnetism\\
\textbf{Final Answer} :\frac{4 \pi}{3}\left(\frac{v}{c}\right)^{3}\left(\frac{e^{2}}{m c^{2} R}\right)\left(\frac{E}{m c^{2}}\right)^{3} \approx 5 \times 10^{-4}\\


\textbf{Solution} :The equation of motion for the electron is

$$
m \ddot{x}+m \omega_{0}^{2} x+\gamma \dot{x}=0,
$$

with the initial conditions

$$
\begin{aligned}
&\left.x\right|_{t=0}=x_{0}, \\
&\left.\dot{x}\right|_{t=0}=0 .
\end{aligned}
$$

Its solution is

$$
x=x_{0} e^{-\frac{\gamma}{2 m} t} e^{-i \omega t},
$$

where

$$
\omega=\sqrt{\omega_{0}^{2}-\frac{\gamma^{2}}{4 m^{2}}} .
$$

As $\frac{\gamma}{2 m} \ll \omega_{0}, \omega \approx \omega_{0}$, and $x=x_{0} e^{-\frac{\gamma}{2 m} t} e^{-i \omega_{0} t}$. The oscillation of the electron about the positive nucleus is equivalent to an oscillating dipole of moment $\mathbf{p}=\mathbf{p}_{0} e^{-\frac{\gamma}{2 m} t} e^{-i \omega_{0} t}$, i.e., a dipole oscillator with attenuating amplitude, where $p_{0}=e x_{0}$. Its radiation field at a large distance away is given by

$$
\mathbf{E}(r, t)=\mathbf{E}_{0}(r) e^{-\frac{\gamma}{2 m}\left(t-\frac{r}{c}\right)} e^{-i \omega_{0}\left(t-\frac{r}{c}\right)} .
$$

For simplicty we shall put $t-\frac{r}{c}=t^{\prime}$ and write

$$
\mathbf{E}(t)=\mathbf{E}_{0} e^{-\frac{\gamma}{2 m} t^{\prime}} e^{-i \omega_{0} t^{\prime}}
$$

Note that $t^{\prime}$ is the retarded time. By Fourier transform the oscillations are a superposition of oscillations of a spread of frequencies:

$$
\mathbf{E}(t)=\int_{-\infty}^{+\infty} \mathbf{E}(\omega) e^{-i \omega t^{\prime}} d \omega
$$

where, as $E(t)=0$ for $t<0$,

$$
\begin{aligned}
\mathbf{E}(\omega) &=\frac{1}{2 \pi} \int_{0}^{+\infty} \mathrm{E}(t) e^{i \omega t^{\prime}} d t^{\prime} \\
&=\frac{1}{2 \pi} \int_{0}^{\infty}\left(\mathrm{E}_{0} e^{-\frac{\gamma}{2 m} t^{\prime}} e^{-i \omega_{0} t^{\prime}}\right) e^{i \omega t^{\prime}} d t^{\prime} \\
&=\frac{\mathbf{E}_{0}}{2 \pi} \frac{1}{i\left(\omega-\omega_{0}\right)-\frac{\gamma}{2 m}} .
\end{aligned}
$$

The rate of radiation is then

$$
I(\omega) \propto|\mathbb{E}(\omega)|^{2} \propto \frac{1}{\left(\omega-\omega_{0}\right)^{2}+\frac{\gamma^{2}}{4 m^{2}}}
$$

This is a Lorentz spectrum.
\textbf{Topic} :Relativity, Particle-Field Interactions\\
\textbf{Book} :Problems and Solutions on Electromagnetism\\
\textbf{Final Answer} :\frac{4 \pi}{3}\left(\frac{v}{c}\right)^{3}\left(\frac{e^{2}}{m c^{2} R}\right)\left(\frac{E}{m c^{2}}\right)^{3} \approx 5 \times 10^{-4}\\


\textbf{Solution} :The equation of motion for the electron is

$$
m \ddot{x}+m \omega_{0}^{2} x+\gamma \dot{x}=0,
$$

with the initial conditions

$$
\begin{aligned}
&\left.x\right|_{t=0}=x_{0}, \\
&\left.\dot{x}\right|_{t=0}=0 .
\end{aligned}
$$

Its solution is

$$
x=x_{0} e^{-\frac{\gamma}{2 m} t} e^{-i \omega t},
$$

where

$$
\omega=\sqrt{\omega_{0}^{2}-\frac{\gamma^{2}}{4 m^{2}}} .
$$

As $\frac{\gamma}{2 m} \ll \omega_{0}, \omega \approx \omega_{0}$, and $x=x_{0} e^{-\frac{\gamma}{2 m} t} e^{-i \omega_{0} t}$. The oscillation of the electron about the positive nucleus is equivalent to an oscillating dipole of moment $\mathbf{p}=\mathbf{p}_{0} e^{-\frac{\gamma}{2 m} t} e^{-i \omega_{0} t}$, i.e., a dipole oscillator with attenuating amplitude, where $p_{0}=e x_{0}$. Its radiation field at a large distance away is given by

$$
\mathbf{E}(r, t)=\mathbf{E}_{0}(r) e^{-\frac{\gamma}{2 m}\left(t-\frac{r}{c}\right)} e^{-i \omega_{0}\left(t-\frac{r}{c}\right)} .
$$

For simplicty we shall put $t-\frac{r}{c}=t^{\prime}$ and write

$$
\mathbf{E}(t)=\mathbf{E}_{0} e^{-\frac{\gamma}{2 m} t^{\prime}} e^{-i \omega_{0} t^{\prime}}
$$

Note that $t^{\prime}$ is the retarded time. By Fourier transform the oscillations are a superposition of oscillations of a spread of frequencies:

$$
\mathbf{E}(t)=\int_{-\infty}^{+\infty} \mathbf{E}(\omega) e^{-i \omega t^{\prime}} d \omega
$$

where, as $E(t)=0$ for $t<0$,

$$
\begin{aligned}
\mathbf{E}(\omega) &=\frac{1}{2 \pi} \int_{0}^{+\infty} \mathrm{E}(t) e^{i \omega t^{\prime}} d t^{\prime} \\
&=\frac{1}{2 \pi} \int_{0}^{\infty}\left(\mathrm{E}_{0} e^{-\frac{\gamma}{2 m} t^{\prime}} e^{-i \omega_{0} t^{\prime}}\right) e^{i \omega t^{\prime}} d t^{\prime} \\
&=\frac{\mathbf{E}_{0}}{2 \pi} \frac{1}{i\left(\omega-\omega_{0}\right)-\frac{\gamma}{2 m}} .
\end{aligned}
$$

The rate of radiation is then

$$
I(\omega) \propto|\mathbb{E}(\omega)|^{2} \propto \frac{1}{\left(\omega-\omega_{0}\right)^{2}+\frac{\gamma^{2}}{4 m^{2}}}
$$

This is a Lorentz spectrum.

 For $\gamma=0, p=p_{0} e^{-i \omega_{0} t^{\prime}}$ and the rate of dipole radiation is

$$
\langle P\rangle=\frac{2}{3 c^{3}}\left\langle|\ddot{\mathrm{p}}|^{2}\right\rangle=\frac{2}{3 c^{3}} \frac{1}{2} \operatorname{Re}\left(\ddot{p}^{*} \ddot{p}\right)=\frac{e^{2} \omega_{0}^{4} x_{0}^{2}}{3 c^{3}} .
$$

The total energy of the dipole is $U=\frac{1}{2} m \omega_{0}^{2} x_{0}^{2}$, so $\langle P\rangle=\frac{2 e^{2} \omega_{0}^{2}}{3 e^{3} m} U$. As the loss of energy is due solely to radiation, we have

$$
\frac{d U}{d t^{\prime}}+\frac{2 e^{2} \omega_{0}^{2}}{3 m c^{3}} U=0,
$$

which has solution $U=U_{0} e^{-\Gamma t^{\prime}}$, with $\Gamma=\frac{2 e^{2} \omega_{0}^{2}}{3 m e^{3}}$

 To find the width of the spectral line, we see that, for $\gamma=0, \frac{\gamma}{2 m}$ in Eq.
(1) is to be replaced by $\frac{\Gamma}{2}$. Then if we define $\Delta \omega=\omega_{+}-\omega_{-}$, where $\omega_{\pm}$ are the frequencies at which the intensity is half the maximum intensity, we have

$$
\frac{\Delta \omega}{2}=\frac{\Gamma}{2}
$$

or

$$
\Delta \omega=\Gamma .
$$

Hence

$$
\begin{aligned}
\Delta \lambda &=\lambda_{0} \frac{\Delta \omega}{\omega_{0}}=\lambda_{0} \frac{\Gamma}{\omega_{0}}=\lambda_{0} \frac{2 e^{2} \omega_{0}}{3 m c^{3}}=\frac{2 e^{2}}{3 m c^{3}} 2 \pi c=\frac{4 \pi}{3} r_{0} \\
&=\frac{4 \pi}{3} \times 2.82 \times 10^{-5}=1.2 \times 10^{-4} \AA
\end{aligned}
$$

where $r_{0}=\frac{e^{2}}{m c^{2}}=2.82 \times 10^{-5} \AA$ is the classical radius of electron. The time needed for losing half the energy is $T=\frac{\ln 2}{\Gamma}$ while the time for one oscillation is $\tau=\frac{2 \pi}{\omega_{0}}$. Hence to lose half the energy the number of oscillations required is

$$
\begin{aligned}
N &=\frac{T}{\tau}=\frac{\ln 2}{\Gamma} \cdot \frac{\omega_{0}}{2 \pi}=\frac{3 c}{4 \pi \omega_{0} r_{0}} \ln 2=\frac{3}{8 \pi^{2}} \frac{\lambda_{0}}{r_{0}} \ln 2 \\
&=\frac{3 \ln 2}{8 \pi^{2}} \times \frac{5000}{2.82 \times 10^{-5}}=4.7 \times 10^{6} .
\end{aligned}
$$

\textbf{Topic} :Relativity, Particle-Field Interactions\\
\textbf{Book} :Problems and Solutions on Electromagnetism\\
\textbf{Final Answer} :47 \times 10^{6}\\


\textbf{Solution} :Let the particle's mass, charge, and kinetic energy (at time $t$ ) be $m, q$ and $T$ respectively. As the particle is non-relativistic, the radiation energy loss per revolution is very much smaller than the kinetic energy, so that we may consider the particle as moving along a circle of radius $R$ at time $t$. Its rate of radiation is

$$
P=\frac{q^{2}}{6 \pi \varepsilon_{0} c^{3}} \dot{v}^{2} .
$$

The equation of motion for the charge as it moves along a circular path in an axial uniform magnetic field $B$ is

$$
m|\dot{\mathbf{v}}|=\frac{m v^{2}}{R}=q v B .
$$

The non-relativistic kinetic energy of the particle is $T=\frac{1}{2} m v^{2}$. Thus its rate of radiation is

$$
P=\frac{q^{2}}{6 \pi \varepsilon_{0} c^{3}} \dot{v}^{2}=\frac{q^{4} B^{2} T}{3 \pi \varepsilon_{0} m^{3} c^{3}} .
$$

The magnetic force does no work on the charge since $\mathbf{v} \times \mathbf{B} \cdot \mathbf{v}=\mathbf{0}$. $P$ is therefore equal to the loss of kinetic energy per unit time:

$$
P=-\frac{d T}{d t}=\frac{q^{4} B^{2} T}{3 \pi \varepsilon_{0} m^{3} c^{3}},
$$

which gives

$$
T=T_{0} e^{-\frac{e^{4} B^{3}}{3 \pi \varepsilon_{0} m^{3} c^{3}}},
$$

where $T_{0}$ is the initial kinetic energy of the charge.
\textbf{Topic} :Relativity, Particle-Field Interactions\\
\textbf{Book} :Problems and Solutions on Electromagnetism\\
\textbf{Final Answer} :T_{0} e^{-\frac{e^{4} B^{3}}{3 \pi \varepsilon_{0} m^{3} c^{3}}}\\


\textbf{Solution} :Let the particle's mass, charge, and kinetic energy (at time $t$ ) be $m, q$ and $T$ respectively. As the particle is non-relativistic, the radiation energy loss per revolution is very much smaller than the kinetic energy, so that we may consider the particle as moving along a circle of radius $R$ at time $t$. Its rate of radiation is

$$
P=\frac{q^{2}}{6 \pi \varepsilon_{0} c^{3}} \dot{v}^{2} .
$$

The equation of motion for the charge as it moves along a circular path in an axial uniform magnetic field $B$ is

$$
m|\dot{\mathbf{v}}|=\frac{m v^{2}}{R}=q v B .
$$

The non-relativistic kinetic energy of the particle is $T=\frac{1}{2} m v^{2}$. Thus its rate of radiation is

$$
P=\frac{q^{2}}{6 \pi \varepsilon_{0} c^{3}} \dot{v}^{2}=\frac{q^{4} B^{2} T}{3 \pi \varepsilon_{0} m^{3} c^{3}} .
$$

The magnetic force does no work on the charge since $\mathbf{v} \times \mathbf{B} \cdot \mathbf{v}=\mathbf{0}$. $P$ is therefore equal to the loss of kinetic energy per unit time:

$$
P=-\frac{d T}{d t}=\frac{q^{4} B^{2} T}{3 \pi \varepsilon_{0} m^{3} c^{3}},
$$

which gives

$$
T=T_{0} e^{-\frac{e^{4} B^{3}}{3 \pi \varepsilon_{0} m^{3} c^{3}}},
$$

where $T_{0}$ is the initial kinetic energy of the charge.

 For a proton, $q=1.6 \times 10^{-19} \mathrm{C}, m=1.67 \times 10^{-27} \mathrm{~kg}$. The time it takes to lose 10 percent of its initial energy is

$$
\tau=-\frac{3 \pi \varepsilon_{0} m^{3} c^{3}}{q^{4} B^{2}} \ln (0.9)
$$

As $T=\frac{1}{2} m v^{2}=\frac{1}{2 m} \cdot R^{2} q^{2} B^{2}$, with $T_{0}=100 \mathrm{MeV}, R=10 \mathrm{~m}$, the magnetic field is given by

$$
B^{2}=\frac{2 m T_{0}}{R^{2} q^{2}} \approx 2.09 \times 10^{-2} \mathrm{~Wb}^{2} / \mathrm{m}^{2}
$$

Substituting it in the expression for $\tau$, we find

$$
\tau \approx 8.07 \times 10^{10} \mathrm{~s} .
$$

\textbf{Topic} :Relativity, Particle-Field Interactions\\
\textbf{Book} :Problems and Solutions on Electromagnetism\\
\textbf{Final Answer} :T_{0} e^{-\frac{e^{4} B^{3}}{3 \pi \varepsilon_{0} m^{3} c^{3}}}\\


\textbf{Solution} :As the radiation loss of the positron is much smaller than its kinetic energy, it can be considered as a small perturbation. We therefore first neglect the effect of radiation. By the conservation of energy, when the distance between the positron and the fixed nucleus is $r$ and its velocity is $\boldsymbol{v}$ we have

$$
\frac{1}{2} m v^{2}+\frac{1}{4 \pi \varepsilon_{0}} \frac{Z e^{2}}{r}=\frac{1}{2} m v_{1}^{2} .
$$

When $v=0, r$ reaches its minimum ro. Thus

$$
\frac{1}{4 \pi \varepsilon_{0}} \frac{Z e^{2}}{r_{0}}=\frac{1}{2} m v_{1}^{2},
$$

or

$$
r_{0}=\frac{Z e^{2}}{2 \pi \varepsilon_{0} m v_{1}^{2}},
$$

whence

$$
v^{2}=v_{1}^{2}\left(1-\frac{r_{0}}{r}\right) .
$$

Differentiating the last equation we have

$$
2 \dot{r} \ddot{r}=\frac{v_{1}^{2} r_{0}}{r^{2}} \dot{r},
$$

or

$$
\ddot{r}=\frac{v_{1}^{2} r_{0}}{2 r^{2}} \text {. }
$$

The rate of radiation loss is given by

$$
P=\frac{d W}{d t}=\frac{d W}{d r} \dot{r}=\frac{d W}{d r} v=\frac{2 e^{2}}{3 c^{3}} \ddot{r}^{2}
$$

so that

$$
d W=\frac{e^{2}}{6 c^{3}} \frac{v_{1}^{3} r_{0}^{2}}{r^{4}} \frac{1}{\sqrt{1-\frac{r_{0}}{r}}} d r .
$$

Hence

$$
\Delta W=2 \int_{r_{0}}^{\infty} d W=\frac{e^{2} v_{1}^{3} r_{0}^{2}}{3 c^{3}} \int_{r_{0}}^{\infty} \frac{d r}{r^{3} \sqrt{r\left(r-r_{0}\right)}} .
$$

By putting $r=r_{0} \sec ^{2} \alpha$, we can carry out the integration and find

$$
\Delta W=-\frac{8}{45} \cdot \frac{v_{1}^{3}}{Z c^{3}} m v_{1}^{2}
$$

As $\frac{1}{2} m v_{2}^{2}=\frac{1}{2} m v_{1}^{2}-\Delta W$, we have

$$
v_{2}^{2}=v_{1}^{2}\left(1-\frac{16}{45} \cdot \frac{v_{1}^{3}}{c^{3}} \cdot \frac{1}{Z}\right) .
$$

Hence

$$
v_{2} \approx v_{1}\left(1-\frac{8}{45} \cdot \frac{v_{1}^{3}}{Z c^{3}}\right)
$$

as $v_{1} \ll c$.

Because $v \ll c$, the radiation is dipole in nature so that the angular distribution of its radiated power is given by

$$
\frac{d P}{d \Omega} \propto \sin ^{2} \theta,
$$

$\theta$ being the angle between the directions of the radiation and the particle velocity. The radiation is plane polarized with the electric field vector in the plane containing the directions of the radiation and the acceleration (which is the same as that of the velocity in this case).

\textbf{Topic} :Relativity, Particle-Field Interactions\\
\textbf{Book} :Problems and Solutions on Electromagnetism\\
\textbf{Final Answer} :T_{0} e^{-\frac{e^{4} B^{3}}{3 \pi \varepsilon_{0} m^{3} c^{3}}}\\


\textbf{Solution} :As $b \ll h \ll \lambda$, the small dielectric cylinder can be considered as an elctric dipole of moment $p$ for scattering of the electromagnetic wave. The electric field generated by $p$ is much smaller than the electric field of the incident electromagnetic wave. Since the tangential component of the electric field intensity across the surface of the cylinder is continuous, the electric field inside the cylinder is equal to the electric field $\mathbf{E}=E_{0} e^{i(\mathbf{k} \cdot \mathbf{r}-\omega t)} \mathbf{e}_{z}$ of the incident wave. Take the origin at the location of the dipole, then $r=0$ and the electric dipole moment of the small cylinder is

$$
\mathbf{p}=\pi b^{2} h \varepsilon_{0}(K-1) E_{0} e^{-i \omega t} \mathbf{e}_{z},
$$

the $z$-axis being taken along the axis of the cylinder.

The total power radiated by the oscillating electric dipole, averaged over one cycle, is

$$
P=\frac{|\ddot{\mathrm{p}}|^{2}}{12 \pi \varepsilon_{0} c^{3}}=\frac{\pi b^{4} h^{2} \varepsilon_{0} \omega^{4}(K-1)^{2} E_{0}^{2}}{12 c^{3}} .
$$

The intensity of the incident wave is $I_{0}=\frac{\varepsilon_{0} c}{2} E_{0}^{2}$, so the total scattering cross section of the cylinder is

$$
\sigma=\frac{P}{I_{0}}=\frac{\pi}{6} b^{4} h^{2}(K-1)^{2} \frac{\omega^{4}}{c^{4}} .
$$


\textbf{Topic} :Relativity, Particle-Field Interactions\\
\textbf{Book} :Problems and Solutions on Electromagnetism\\
\textbf{Final Answer} :\frac{\pi}{6} b^{4} h^{2}(K-1)^{2} \frac{\omega^{4}}{c^{4}}\\


\textbf{Solution} :$n=\left(1-\omega_{p}^{2} / \omega^{2}\right)^{\frac{1}{2}}$.
\\
\textbf{Topic} :Relativity, Particle-Field Interactions\\
\textbf{Book} :Problems and Solutions on Electromagnetism\\
\textbf{Final Answer} :\frac{\pi}{6} b^{4} h^{2}(K-1)^{2} \frac{\omega^{4}}{c^{4}}\\


\textbf{Solution} :$n=\left(1-\omega_{p}^{2} / \omega^{2}\right)^{\frac{1}{2}}$.
 For $\omega>\omega_{p}, n<1$.
 For $\omega>\omega_{p}$, the phase velocity in the plasma is

$$
v_{p}=\frac{c}{n}>c .
$$

However, messages or signals are transmitted with the group velocity

$$
v_{\mathrm{g}}=\frac{d \omega}{d k}=\frac{c^{2} k}{\omega}=c\left(1-\omega_{\mathrm{p}}^{2} / \omega^{2}\right)^{\frac{1}{2}} .
$$

As $\omega>\omega_{p}$, it is clear that $v_{g}<c$.
\\
\textbf{Topic} :Relativity, Particle-Field Interactions\\
\textbf{Book} :Problems and Solutions on Electromagnetism\\
\textbf{Final Answer} :c\left(1-\omega_{\mathrm{p}}^{2} / \omega^{2}\right)^{\frac{1}{2}}\\


\textbf{Solution} :Assuming the medium to be linear, $M=\chi_{m} \mathbf{H}$, we have, by the definition of permeability $\mu$

$$
\mathbf{M}(t)=\chi_{m}(\omega) \mathbf{H}(t)=\frac{\mu(\omega)-1}{4 \pi} \mathbf{H}(t) .
$$

 Maxwell's equations for the medium are

$$
\begin{array}{ll}
\nabla \cdot \mathbf{D}=4 \pi \rho, & \nabla \cdot \mathbf{B}=0, \\
\nabla \times \mathbf{E}=-\frac{1}{c} \frac{\partial \mathbf{B}}{\partial t}, & \nabla \times \mathbf{H}=\frac{4 \pi}{c} \mathbf{j}+\frac{1}{c} \frac{\partial \mathbf{D}}{\partial t} .
\end{array}
$$

Note that $B$ in the equations is the superposition of the external field and the time-dependent field produced by the magnetization $M(t)$.

Consider the medium as isolated and uncharged, then $\rho=\mathbf{j}=0$. Also $\mathbf{D}=\mathbf{E}$ as $\varepsilon=1$. Maxwell's equations now reduce to

$$
\begin{array}{ll}
\nabla \cdot \mathbf{E}=0, & \nabla \cdot \mathbf{B}=0, \\
\nabla \times \mathbf{E}=-\frac{1}{c} \frac{\partial \mathbf{B}}{\partial t}, & \nabla \times \mathbf{H}=\frac{1}{c} \frac{\partial \mathbf{E}}{\partial t},
\end{array}
$$

with

$$
\mathbf{B}=\boldsymbol{\mu} \mathbf{H} .
$$

Deduce from these equations the wave equation

$$
\nabla^{2} \mathbf{H}-\frac{\mu}{c^{2}} \frac{\partial^{2} \mathrm{H}}{\partial t^{2}}=0 .
$$

Consider a plane wave solution

$$
\mathbf{H}(t)=\mathbf{H}_{0} e^{i(\mathbf{k} \cdot \mathbf{r}-\omega t)} .
$$

Substitution in the wave equation gives the dispersion relation

$$
k^{2}-\mu \frac{\omega^{2}}{c^{2}}=0,
$$

or

$$
k=\frac{\omega}{v}=\frac{\omega}{c} \sqrt{\mu(\omega)} \text {. }
$$

The magnetization $M$ can then be represented by a plane wave

$$
M(t)=\chi_{m}(\omega) \mathbf{H}(t)=\chi_{m} \mathbf{H}_{0} e^{i(k \cdot r-\omega t)} .
$$

As $M$ satisfies the same wave equation the dispersion relation above remains valid.

\textbf{Topic} :Relativity, Particle-Field Interactions\\
\textbf{Book} :Problems and Solutions on Electromagnetism\\
\textbf{Final Answer} :\chi_{m} \mathbf{H}_{0} e^{i(k \cdot r-\omega t)}\\


\textbf{Solution} :The equation of motion for the electron is

$$
m \ddot{\mathbf{x}}=-m \omega_{0}^{2} \mathbf{x}-e \mathbf{E}_{0} e^{-i \omega t} .
$$

Consider the steady state solution $x=x_{0} e^{-i \omega t}$. Substituting in the equation gives

$$
\mathbf{x}=\frac{e \mathbf{E}_{0}}{m\left(\omega^{2}-\omega_{0}^{2}\right)} e^{-i \omega t} .
$$
\textbf{Topic} :Relativity, Particle-Field Interactions\\
\textbf{Book} :Problems and Solutions on Electromagnetism\\
\textbf{Final Answer} :\frac{e \mathbf{E}_{0}}{m\left(\omega^{2}-\omega_{0}^{2}\right)} e^{-i \omega t}\\\end{document}