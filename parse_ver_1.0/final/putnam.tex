\documentclass[10pt]{article}
\usepackage{hyperref}
\usepackage{amsthm}
\usepackage{amsmath}
\usepackage{amssymb}
\usepackage{amsfonts}
\setlength\parindent{0pt}
\title{OUTPUT.TEX.}
\date{Updated: \today}
\author{}


\begin{document}


\textbf{Problem Statement} :
3. Find the least positive integer $n$ such that any set of $n$ pairwise relatively prime integers greater than 1 and less than 2005 contains at least one prime number. 
\\
\textbf{Solution} :
3. $a>b>c$. In this case $a b c-2=a+b+c<3 a$. Therefore, $a(b c-3)<2$. It follows that $b c-3<2$, that is, $b c<5$. We have the following situations:

(i) $b=2, c=1$, so $a=5$ and we obtain the solution $(1,2,3,10)$.

(ii) $b=3, c=1$, so $a=3$ and we return to case 2 .

(iii) $b=4, c=1$, so $3 a=7$, which is impossible.

In conclusion, we have obtained the solutions $(1,1,1,8),(1,1,2,9),(1,2,3,10)$, and those obtained by permutations of $x, y, z$.

(Romanian Mathematical Olympiad, 1995)
\\
\textbf{Topic} :Probability\\
\textbf{Book} :Putnam and Beyond\\
\textbf{Final Answer} :\\


\textbf{Problem Statement} :
64. Find all odd positive integers $n$ greater than 1 such that for any coprime divisors $a$ and $b$ of $n$, the number $a+b-1$ is also a divisor of $n$. 
\\
\textbf{Solution} :
64. We call a number good if it satisfies the given condition. It is not difficult to see that all powers of primes are good. Suppose $n$ is a good number that has at least two distinct prime factors. Let $n=p^{r} s$, where $p$ is the smallest prime dividing $n$ and $s$ is not divisible by $p$. Because $n$ is good, $p+s-1$ must divide $n$. For any prime $q$ dividing $s, s<p+s-1<s+q$, so $q$ does not divide $p+s-1$. Therefore, the only prime factor of $p+s-1$ is $p$. Then $s=p^{c}-p+1$ for some integer $c>1$. Because $p^{c}$ must also divide $n, p^{c}+s-1=2 p^{c}-p$ divides $n$. Because $2 p^{c-1}-1$ has no factors of $p$, it must divide $s$. But

$$
\begin{aligned}
\frac{p-1}{2}\left(2 p^{c-1}-1\right) &=p^{c}-p^{c-1}-\frac{p-1}{2}<p^{c}-p+1<\frac{p+1}{2}\left(2 p^{c-1}-1\right) \\
&=p^{c}+p^{c-1}-\frac{p+1}{2},
\end{aligned}
$$

a contradiction. It follows that the only good integers are the powers of primes.

(Russian Mathematical Olympiad, 2001)
\\
\textbf{Topic} :Probability\\
\textbf{Book} :Putnam and Beyond\\
\textbf{Final Answer} :\\


\textbf{Problem Statement} :
67. An ordered triple of numbers is given. It is permitted to perform the following operation on the triple: to change two of them, say $a$ and $b$, to $(a+b) / \sqrt{2}$ and $(a-b) / \sqrt{2}$. Is it possible to obtain the triple $(1, \sqrt{2}, 1+\sqrt{2})$ from the triple $(2, \sqrt{2}, 1 / \sqrt{2})$ using this operation?
\\
\textbf{Solution} :
67. Because

$$
a^{2}+b^{2}=\left(\frac{a+b}{\sqrt{2}}\right)^{2}+\left(\frac{a-b}{\sqrt{2}}\right)^{2},
$$

the sum of the squares of the numbers in a triple is invariant under the operation. The sum of squares of the first triple is $\frac{9}{2}$ and that of the second is $6+2 \sqrt{2}$, so the first triple cannot be transformed into the second.

(D. Fomin, S. Genkin, I. Itenberg, Mathematical Circles, AMS, 1996)
\\
\textbf{Topic} :Probability\\
\textbf{Book} :Putnam and Beyond\\
\textbf{Final Answer} :\\


\textbf{Problem Statement} :
86. Factor $5^{1985}-1$ into a product of three integers, each of which is greater than $5^{100}$.
\\
\textbf{Solution} :
86. We use the identity

$$
a^{5}-1=(a-1)\left(a^{4}+a^{3}+a^{2}+a+1\right)
$$

applied for $a=5^{397}$. The difficult part is to factor $a^{4}+a^{3}+a^{2}+a+1$. Note that

$$
a^{4}+a^{3}+a^{2}+a+1=\left(a^{2}+3 a+1\right)^{2}-5 a(a+1)^{2} .
$$

Hence

$$
\begin{aligned}
a^{4}+a^{3}+a^{2}+a+1 &=\left(a^{2}+3 a+1\right)^{2}-5^{398}(a+1)^{2} \\
&=\left(a^{2}+3 a+1\right)^{2}-\left(5^{199}(a+1)\right)^{2} \\
&=\left(a^{2}+3 a+1+5^{199}(a+1)\right)\left(a^{2}+3 a+1-5^{199}(a+1)\right) .
\end{aligned}
$$

It is obvious that $a-1$ and $a^{2}+3 a+1+5^{199}(a+1)$ are both greater than $5^{100}$. As for the third factor, we have

$$
a^{2}+3 a+1-5^{199}(a+1)=a\left(a-5^{199}\right)+3 a-5^{199}+1 \geq a+0+1 \geq 5^{100} .
$$

Hence the conclusion.

(proposed by Russia for the 26th International Mathematical Olympiad, 1985)
\\
\textbf{Topic} :Probability\\
\textbf{Book} :Putnam and Beyond\\
\textbf{Final Answer} :\\


\textbf{Problem Statement} :
90. Solve in real numbers the equation

$$
\sqrt[3]{x-1}+\sqrt[3]{x}+\sqrt[3]{x+1}=0
$$
\\
\textbf{Solution} :
90. First solution: Using the indentity

$$
a^{3}+b^{3}+c^{3}-3 a b c=\frac{1}{2}(a+b+c)\left((a-b)^{2}+(b-c)^{2}+(c-a)^{2}\right)
$$

applied to the (distinct) numbers $a=\sqrt[3]{x-1}, b=\sqrt[3]{x}$, and $c=\sqrt[3]{x+1}$, we transform the equation into the equivalent

$$
(x-1)+x+(x+1)-3 \sqrt[3]{(x-1) x(x+1)}=0 .
$$

We further change this into $x=\sqrt[3]{x^{3}-x}$. Raising both sides to the third power, we obtain $x^{3}=x^{3}-x$. We conclude that the equation has the unique solution $x=0$.

Second solution: The function $f: \mathbb{R} \rightarrow \mathbb{R}, f(x)=\sqrt[3]{x-1}+\sqrt[3]{x}+\sqrt[3]{x+1}$ is strictly increasing, so the equation $f(x)=0$ has at most one solution. Since $x=0$ satisfies this equation, it is the unique solution.
\\
\textbf{Topic} :Probability\\
\textbf{Book} :Putnam and Beyond\\
\textbf{Final Answer} :\\


\textbf{Problem Statement} :
91. Find all triples $(x, y, z)$ of positive integers such that

$$
x^{3}+y^{3}+z^{3}-3 x y z=p,
$$

where $p$ is a prime number greater than 3 .
\\
\textbf{Solution} :
91. The key observation is that the left-hand side of the equation can be factored as

$$
(x+y+z)\left(x^{2}+y^{2}+z^{2}-x y-y z-z x\right)=p .
$$

Since $x+y+z>1$ and $p$ is prime, we must have $x+y+z=p$ and $x^{2}+y^{2}+z^{2}-x y-$ $y z-z x=1$. The second equality can be written as $(x-y)^{2}+(y-z)^{2}+(z-x)^{2}=2$. Without loss of generality, we may assume that $x \geq y \geq z$. If $x>y>z$, then $x-y \geq 1$, $y-z \geq 1$, and $x-z \geq 2$, which would imply that $(x-y)^{2}+(y-z)^{2}+(z-x)^{2} \geq 6>2$.

Therefore, either $x=y=z+1$ or $x-1=y=z$. According to whether the prime $p$ is of the form $3 k+1$ or $3 k+2$, the solutions are $\left(\frac{p-1}{3}, \frac{p-1}{3}, \frac{p+2}{3}\right)$ and the corresponding permutations, or $\left(\frac{p-2}{3}, \frac{p+1}{3}, \frac{p+1}{3}\right)$ and the corresponding permutations.

(T. Andreescu, D. Andrica, An Introduction to Diophantine Equations, GIL 2002)
\\
\textbf{Topic} :Probability\\
\textbf{Book} :Putnam and Beyond\\
\textbf{Final Answer} :\\


\textbf{Problem Statement} :
94. Find

$$
\min _{a, b \in \mathbb{R}} \max \left(a^{2}+b, b^{2}+a\right) .
$$
\\
\textbf{Solution} :
94. Let $M(a, b)=\max \left(a^{2}+b, b^{2}+a\right)$. Then $M(a, b) \geq a^{2}+b$ and $M(a, b) \geq b^{2}+a$, so $2 M(a, b) \geq a^{2}+b+b^{2}+a$. It follows that

$$
2 M(a, b)+\frac{1}{2} \geq\left(a+\frac{1}{2}\right)^{2}+\left(b+\frac{1}{2}\right)^{2} \geq 0,
$$

hence $M(a, b) \geq-\frac{1}{4}$. We deduce that

$$
\min _{a, b \in \mathbb{R}} M(a, b)=-\frac{1}{4},
$$

which, in fact, is attained when $a=b=-\frac{1}{2}$.

(T. Andreescu)
\\
\textbf{Topic} :Probability\\
\textbf{Book} :Putnam and Beyond\\
\textbf{Final Answer} :\\


\textbf{Problem Statement} :
96. Find all positive integers $n$ for which the equation

$$
n x^{4}+4 x+3=0
$$

has a real root.
\\
\textbf{Solution} :
96. Clearly, 0 is not a solution. Solving for $n$ yields $\frac{-4 x-3}{x^{4}} \geq 1$, which reduces to $x^{4}+4 x+3 \leq 0$. The last inequality can be written in its equivalent form,

$$
\left(x^{2}-1\right)^{2}+2(x+1)^{2} \leq 0,
$$

whose only real solution is $x=-1$.

Hence $n=1$ is the unique solution, corresponding to $x=-1$.

(T. Andreescu)
\\
\textbf{Topic} :Probability\\
\textbf{Book} :Putnam and Beyond\\
\textbf{Final Answer} :\\


\textbf{Problem Statement} :
97. Find all triples $(x, y, z)$ of real numbers that are solutions to the system of equations

$$
\begin{aligned}
&\frac{4 x^{2}}{4 x^{2}+1}=y, \\
&\frac{4 y^{2}}{4 y^{2}+1}=z, \\
&\frac{4 z^{2}}{4 z^{2}+1}=x .
\end{aligned}
$$
\\
\textbf{Solution} :
97. If $x=0$, then $y=0$ and $z=0$, yielding the triple $(x, y, z)=(0,0,0)$. If $x \neq 0$, then $y \neq 0$ and $z \neq 0$, so we can rewrite the equations of the system in the form

$$
\begin{aligned}
&1+\frac{1}{4 x^{2}}=\frac{1}{y}, \\
&1+\frac{1}{4 y^{2}}=\frac{1}{z}, \\
&1+\frac{1}{4 z^{2}}=\frac{1}{x} .
\end{aligned}
$$

Summing up the three equations leads to

$$
\left(1-\frac{1}{x}+\frac{1}{4 x^{2}}\right)+\left(1-\frac{1}{y}+\frac{1}{4 y^{2}}\right)+\left(1-\frac{1}{z}+\frac{1}{4 z^{2}}\right)=0 .
$$

This is equivalent to

$$
\left(1-\frac{1}{2 x}\right)^{2}+\left(1-\frac{1}{2 y}\right)^{2}+\left(1-\frac{1}{2 z}\right)^{2}=0 .
$$

It follows that $\frac{1}{2 x}=\frac{1}{2 y}=\frac{1}{2 z}=1$, yielding the triple $(x, y, z)=\left(\frac{1}{2}, \frac{1}{2}, \frac{1}{2}\right)$. Both triples satisfy the equations of the system.

(Canadian Mathematical Olympiad, 1996)
\\
\textbf{Topic} :Probability\\
\textbf{Book} :Putnam and Beyond\\
\textbf{Final Answer} :\\


\textbf{Problem Statement} :
98. Find the minimum of

$$
\log _{x_{1}}\left(x_{2}-\frac{1}{4}\right)+\log _{x_{2}}\left(x_{3}-\frac{1}{4}\right)+\cdots+\log _{x_{n}}\left(x_{1}-\frac{1}{4}\right)
$$

over all $x_{1}, x_{2}, \ldots, x_{n} \in\left(\frac{1}{4}, 1\right)$. 
\\
\textbf{Solution} :
98. First, note that $\left(x-\frac{1}{2}\right)^{2} \geq 0$ implies $x-\frac{1}{4} \leq x^{2}$, for all real numbers $x$. Applying this and using the fact that the $x_{i}$ 's are less than 1 , we find that

$$
\log _{x_{k}}\left(x_{k+1}-\frac{1}{4}\right) \geq \log _{x_{k}}\left(x_{k+1}^{2}\right)=2 \log _{x_{k}} x_{k+1} .
$$

Therefore,

$$
\sum_{k=1}^{n} \log _{x_{k}}\left(x_{k+1}-\frac{1}{4}\right) \geq 2 \sum_{k=1}^{n} \log _{x_{k}} x_{k+1} \geq 2 n \sqrt[n]{\frac{\ln x_{2}}{\ln x_{1}} \cdot \frac{\ln x_{3}}{\ln x_{2}} \cdots \frac{\ln x_{n}}{\ln x_{1}}}=2 n .
$$

So a good candidate for the minimum is $2 n$, which is actually attained for $x_{1}=x_{2}=$ $\cdots=x_{n}=\frac{1}{2}$.

(Romanian Mathematical Olympiad, 1984, proposed by T. Andreescu)
\\
\textbf{Topic} :Probability\\
\textbf{Book} :Putnam and Beyond\\
\textbf{Final Answer} :\\


\textbf{Problem Statement} :
101. Find all pairs $(x, y)$ of real numbers that are solutions to the system

$$
\begin{aligned}
&x^{4}+2 x^{3}-y=-\frac{1}{4}+\sqrt{3}, \\
&y^{4}+2 y^{3}-x=-\frac{1}{4}-\sqrt{3} .
\end{aligned}
$$
\\
\textbf{Solution} :
101. Adding up the two equations yields

$$
\left(x^{4}+2 x^{3}-x+\frac{1}{4}\right)+\left(y^{4}+2 y^{3}-y+\frac{1}{4}\right)=0 .
$$

Here we recognize two perfect squares, and write this as

$$
\left(x^{2}+x-\frac{1}{2}\right)^{2}+\left(y^{2}+y-\frac{1}{2}\right)^{2}=0 .
$$

Equality can hold only if $x^{2}+x-\frac{1}{2}=y^{2}+y-\frac{1}{2}=0$, which then gives $\{x, y\} \subset$ $\left\{-\frac{1}{2}-\frac{\sqrt{3}}{2},-\frac{1}{2}+\frac{\sqrt{3}}{2}\right\}$. Moreover, since $x \neq y,\{x, y\}=\left\{-\frac{1}{2}-\frac{\sqrt{3}}{2},-\frac{1}{2}+\frac{\sqrt{3}}{2}\right\}$. A simple verification leads to $(x, y)=\left(-\frac{1}{2}+\frac{\sqrt{3}}{2},-\frac{1}{2}-\frac{\sqrt{3}}{2}\right)$.

(Mathematical Reflections, proposed by T. Andreescu)
\\
\textbf{Topic} :Probability\\
\textbf{Book} :Putnam and Beyond\\
\textbf{Final Answer} :\\


\textbf{Problem Statement} :
105. Let $a_{1}, a_{2}, \ldots, a_{n}$ be distinct real numbers. Find the maximum of

$$
a_{1} a_{\sigma(a)}+a_{2} a_{\sigma(2)}+\cdots+a_{n} a_{\sigma(n)}
$$

over all permutations of the set $\{1,2, \ldots, n\}$.
\\
\textbf{Solution} :
105. Apply Cauchy-Schwarz:

$$
\begin{aligned}
\left(a_{1} a_{\sigma(a)}+a_{2} a_{\sigma(2)}+\cdots+a_{n} a_{\sigma(n)}\right)^{2} & \leq\left(a_{1}^{2}+a_{2}^{2}+\cdots+a_{n}^{2}\right)\left(a_{\sigma(1)}+a_{\sigma(2)}+\cdots+a_{\sigma(n)}^{2}\right) \\
&=\left(a_{1}^{2}+a_{2}^{2}+\cdots+a_{n}^{2}\right)^{2} .
\end{aligned}
$$

The maximum is $a_{1}^{2}+a_{2}^{2}+\cdots+a_{n}^{2}$. The only permutation realizing it is the identity permutation.
\\
\textbf{Topic} :Probability\\
\textbf{Book} :Putnam and Beyond\\
\textbf{Final Answer} :\\


\textbf{Problem Statement} :
107. Find all positive integers $n, k_{1}, \ldots, k_{n}$ such that $k_{1}+\cdots+k_{n}=5 n-4$ and

$$
\frac{1}{k_{1}}+\cdots+\frac{1}{k_{n}}=1 .
$$
\\
\textbf{Solution} :
107. By the Cauchy-Schwarz inequality,

$$
\left(k_{1}+\cdots+k_{n}\right)\left(\frac{1}{k_{1}}+\cdots+\frac{1}{k_{n}}\right) \geq n^{2} .
$$

We must thus have $5 n-4 \geq n^{2}$, so $n \leq 4$. Without loss of generality, we may suppose that $k_{1} \leq \cdots \leq k_{n}$.

If $n=1$, we must have $k_{1}=1$, which is a solution. Note that hereinafter we cannot have $k_{1}=1$

If $n=2$, we have $\left(k_{1}, k_{2}\right) \in\{(2,4),(3,3)\}$, neither of which satisfies the relation from the statement.

If $n=3$, we have $k_{1}+k_{2}+k_{3}=11$, so $2 \leq k_{1} \leq 3$. Hence $\left(k_{1}, k_{2}, k_{3}\right) \in$ $\{(2,2,7),(2,3,6),(2,4,5),(3,3,5),(3,4,4)\}$, and only $(2,3,6)$ works.

If $n=4$, we must have equality in the Cauchy-Schwarz inequality, and this can happen only if $k_{1}=k_{2}=k_{3}=k_{4}=4$.

Hence the solutions are $n=1$ and $k_{1}=1, n=3$, and $\left(k_{1}, k_{2}, k_{3}\right)$ is a permutation of $(2,3,6)$, and $n=4$ and $\left(k_{1}, k_{2}, k_{3}, k_{4}\right)=(4,4,4,4)$.

(66th W.L. Putnam Mathematical Competition, 2005, proposed by T. Andreescu)
\\
\textbf{Topic} :Probability\\
\textbf{Book} :Putnam and Beyond\\
\textbf{Final Answer} :\\


\textbf{Problem Statement} :
125. Which number is larger,

$$
\prod_{n=1}^{25}\left(1-\frac{n}{365}\right) \text { or } \frac{1}{2} ?
$$
\\
\textbf{Solution} :
125. It is natural to try to simplify the product, and for this we make use of the AM-GM inequality:

$$
\prod_{n=1}^{25}\left(1-\frac{n}{365}\right) \leq\left[\frac{1}{25} \sum_{n=1}^{25}\left(1-\frac{n}{365}\right)\right]^{25}=\left(\frac{352}{365}\right)^{25}=\left(1-\frac{13}{365}\right)^{25} .
$$

We now use Newton's binomial formula to estimate this power. First, note that

$$
\left(\begin{array}{c}
25 \\
k
\end{array}\right)\left(\frac{13}{365}\right)^{k} \geq\left(\begin{array}{c}
25 \\
k+1
\end{array}\right)\left(\frac{13}{365}\right)^{k+1},
$$

since this reduces to

$$
\frac{13}{365} \leq \frac{k+1}{25-k},
$$

and the latter is always true for $1 \leq k \leq 24$. For this reason if we ignore the part of the binomial expansion beginning with the fourth term, we increase the value of the expression. In other words,

$$
\left(1-\frac{13}{365}\right)^{25} \leq 1-\left(\begin{array}{c}
25 \\
1
\end{array}\right) \frac{13}{365}+\left(\begin{array}{c}
25 \\
2
\end{array}\right) \frac{13^{2}}{365^{2}}=1-\frac{65}{73}+\frac{169 \cdot 12}{63^{2}}<\frac{1}{2} .
$$

We conclude that the second number is larger.

(Soviet Union University Student Mathematical Olympiad, 1975) 
\\
\textbf{Topic} :Probability\\
\textbf{Book} :Putnam and Beyond\\
\textbf{Final Answer} :\\


\textbf{Problem Statement} :
148. Determine all polynomials $P(x)$ with real coefficients for which there exists a positive integer $n$ such that for all $x$,

$$
P\left(x+\frac{1}{n}\right)+P\left(x-\frac{1}{n}\right)=2 P(x) .
$$
\\
\textbf{Solution} :
148. First solution: Let $m$ be the degree of $P(x)$, and write

$$
P(x)=a_{m} x^{m}+a_{m-1} x^{m-1}+\cdots+a_{0} .
$$

Using the binomial formula for $\left(x \pm \frac{1}{n}\right)^{m}$ and $\left(x \pm \frac{1}{n}\right)^{m-1}$ we transform the identity from the statement into

$$
\begin{array}{r}
2 a_{m} x^{m}+2 a_{m-1} x^{m-1}+2 a_{m-2} x^{m-2}+a_{m} \frac{m(m-1)}{n^{2}} x^{m-2}+Q(x) \\
=2 a_{m} x^{m}+2 a_{m-1} x^{m-1}+2 a_{m-2} x^{m-2}+R(x),
\end{array}
$$

where $Q$ and $R$ are polynomials of degree at most $m-3$. If we identify the coefficients of the corresponding powers of $x$, we find that $a_{m} \frac{m(m-1)}{n^{2}}=0$. But $a_{m} \neq 0$, being the leading coefficient of the polynomial; hence $m(m-1)=0$. So either $m=0$ or $m=1$. One can check in an instant that all polynomials of degree 0 or 1 satisfy the required condition.

Second solution: Fix a point $x_{0}$. The graph of $P(x)$ has infinitely many points in common with the line that has slope

$$
m=n\left(P\left(x_{0}+\frac{1}{n}\right)-P\left(x_{0}\right)\right)
$$

and passes through the point $\left(x_{0}, P\left(x_{0}\right)\right)$. Therefore, the graph of $P(x)$ is a line, so the polynomial has degree 0 or 1 .

Third solution: If there is such a polynomial of degree $m \geq 2$, differentiating the given relation $m-2$ times we find that there is a quadratic polynomial that satisfies the given relation. But then any point on its graph would be the vertex of the parabola, which of course is impossible. Hence only linear and constant polynomials satisfy the given relation.

(Romanian Team Selection Test for the International Mathematical Olympiad, 1979, proposed by D. Buşneag)
\\
\textbf{Topic} :Probability\\
\textbf{Book} :Putnam and Beyond\\
\textbf{Final Answer} :\\


\textbf{Problem Statement} :
149. Find a polynomial with integer coefficients that has the zero $\sqrt{2}+\sqrt[3]{3}$.
\\
\textbf{Solution} :
149. Let $x=\sqrt{2}+\sqrt[3]{3}$. Then $\sqrt[3]{3}=x-\sqrt{2}$, which raised to the third power yields $3=x^{3}-3 \sqrt{2} x^{2}+6 x-2 \sqrt{2}$, or

$$
x^{3}+6 x-3=\left(3 x^{2}+2\right) \sqrt{2} .
$$

By squaring this equality we deduce that $x$ satisfies the polynomial equation

$$
x^{6}-6 x^{4}-6 x^{3}+12 x^{2}-36 x+1=0 .
$$

(Belgian Mathematical Olympiad, 1978, from a note by P. Radovici-Mărculescu)
\\
\textbf{Topic} :Probability\\
\textbf{Book} :Putnam and Beyond\\
\textbf{Final Answer} :\\


\textbf{Problem Statement} :
151. Let $P(x)$ be a polynomial of degree $n$. Knowing that

$$
P(k)=\frac{k}{k+1}, \quad k=0,1, \ldots, n,
$$

find $P(m)$ for $m>n$.
\\
\textbf{Solution} :
151. Because $P(0)=0$, there exists a polynomial $Q(x)$ such that $P(x)=x Q(x)$. Then

$$
Q(k)=\frac{1}{k+1}, \quad k=1,2, \ldots, n .
$$

Let $H(x)=(x+1) Q(x)-1$. The degree of $H(x)$ is $n$ and $H(k)=0$ for $k=1,2, \ldots, n$. Hence

$$
H(x)=(x+1) Q(x)-1=a_{0}(x-1)(x-2) \cdots(x-n) .
$$

In this equality $H(-1)=-1$ yields $a_{0}=\frac{(-1)^{n+1}}{(n+1) !}$. For $x=m, m>n$, which gives

$$
Q(m)=\frac{(-1)^{n+1}(m-1)(m-2) \cdots(m-n)+1}{(n+1) !(m+1)}+\frac{1}{m+1},
$$

and so 

$$
P(m)=\frac{(-1)^{m+1} m(m-1) \cdots(m-n)}{(n+1) !(m+1)}+\frac{m}{m+1} .
$$

(D. Andrica, published in T. Andreescu, D. Andrica, 360 Problems for Mathematical Contests, GIL, 2003)
\\
\textbf{Topic} :Probability\\
\textbf{Book} :Putnam and Beyond\\
\textbf{Final Answer} :\\


\textbf{Problem Statement} :
156. Solve the system

$$
\begin{array}{r}
x+y+z=1, \\
x y z=1,
\end{array}
$$

knowing that $x, y, z$ are complex numbers of absolute value equal to 1 .
\\
\textbf{Solution} :
156. Taking the conjugate of the first equation, we obtain

$$
\bar{x}+\bar{y}+\bar{z}=1,
$$

and hence

$$
\frac{1}{x}+\frac{1}{y}+\frac{1}{z}=1 .
$$

Combining this with $x y z=1$, we obtain

$$
x y+y z+x z=1 .
$$

Therefore, $x, y, z$ are the roots of the polynomial equation

$$
t^{3}-t^{2}+t-1=0,
$$

which are $1, i,-i$. Any permutation of these three complex numbers is a solution to the original system of equations.
\\
\textbf{Topic} :Probability\\
\textbf{Book} :Putnam and Beyond\\
\textbf{Final Answer} :\\


\textbf{Problem Statement} :
157. Find all real numbers $r$ for which there is at least one triple $(x, y, z)$ of nonzero real numbers such that

$$
x^{2} y+y^{2} z+z^{2} x=x y^{2}+y z^{2}+z x^{2}=r x y z .
$$
\\
\textbf{Solution} :
157. Dividing by the nonzero $x y z$ yields $\frac{x}{z}+\frac{y}{x}+\frac{z}{y}=\frac{y}{z}+\frac{z}{x}+\frac{x}{y}=r$. Let $a=\frac{x}{y}, b=\frac{y}{z}$, $c=\frac{z}{x}$. Then $a b c=1, \frac{1}{a}+\frac{1}{b}+\frac{1}{c}=r, a+b+c=r$. Hence 

$$
\begin{aligned}
a+b+c &=r, \\
a b+b c+c a &=r, \\
a b c &=1 .
\end{aligned}
$$

We deduce that $a, b, c$ are the solutions of the polynomial equation $t^{3}-r t^{2}+r t-1=0$. This equation can be written as

$$
(t-1)\left[t^{2}-(r-1) t+1\right]=0 .
$$

Since it has three real solutions, the discriminant of the quadratic must be positive. This means that $(r-1)^{2}-4 \geq 0$, leading to $r \in(-\infty,-1] \cup[3, \infty)$. Conversely, all such $r$ work.
\\
\textbf{Topic} :Probability\\
\textbf{Book} :Putnam and Beyond\\
\textbf{Final Answer} :\\


\textbf{Problem Statement} :
159. Find all polynomials whose coefficients are equal either to 1 or $-1$ and whose zeros are all real.
\\
\textbf{Solution} :
159. Let $P(x)=a_{n} x^{n}+a_{n-1} x^{n-1}+\cdots+a_{0}$. Denote its zeros by $x_{1}, x_{2}, \ldots, x_{n}$. The first two of Viète's relations give

$$
\begin{aligned}
x_{1}+x_{2}+\cdots+x_{n} &=-\frac{a_{n-1}}{a_{n}}, \\
x_{1} x_{2}+x_{1} x_{3}+\cdots+x_{n-1} x_{n} &=\frac{a_{n-2}}{a_{n}}
\end{aligned}
$$

Combining them, we obtain

$$
x_{1}^{2}+x_{2}^{2}+\cdots+x_{n}^{2}=\left(\frac{a_{n-1}}{a_{n}}\right)^{2}-2\left(\frac{a_{n-2}}{a_{n}}\right) .
$$

The only possibility is $x_{1}^{2}+x_{2}^{2}+\cdots+x_{n}^{2}=3$. Given that $x_{1}^{2} x_{2}^{2} \cdots x_{n}^{2}=1$, the AM-GM inequality yields 

$$
3=x_{1}^{2}+x_{2}^{2}+\cdots+x_{n}^{2} \geq n \sqrt[n]{x_{1}^{2} x_{2}^{2} \cdots x_{n}^{2}}=n .
$$

Therefore, $n \leq 3$. Eliminating case by case, we find among linear polynomials $x+1$ and $x-1$, and among quadratic polynomials $x^{2}+x-1$ and $x^{2}-x-1$. As for the cubic polynomials, we should have equality in the AM-GM inequality. So all zeros should have the same absolute values. The polynomial should share a zero with its derivative. This is the case only for $x^{3}+x^{2}-x-1$ and $x^{3}-x^{2}-x+1$, which both satisfy the required property. Together with their negatives, these are all desired polynomials.

(Indian Olympiad Training Program, 2005)
\\
\textbf{Topic} :Probability\\
\textbf{Book} :Putnam and Beyond\\
\textbf{Final Answer} :\\


\textbf{Problem Statement} :
161. The zeros of the polynomial $P(x)=x^{3}-10 x+11$ are $u$, $v$, and $w$. Determine the value of $\arctan u+\arctan v+\arctan w$.
\\
\textbf{Solution} :
161. First solution: Let $\alpha=\arctan u, \beta=\arctan v$, and $\arctan w$. We are required to determine the sum $\alpha+\beta+\gamma$. The addition formula for the tangent of three angles, 

$$
\tan (\alpha+\beta+\gamma)=\frac{\tan \alpha+\tan \beta+\tan \gamma-\tan \alpha \tan \beta \tan \gamma}{1-(\tan \alpha \tan \beta+\tan \beta \tan \gamma+\tan \alpha \tan \gamma)},
$$

implies

$$
\tan (\alpha+\beta+\gamma)=\frac{u+v+w-u v w}{1-(u v+v w+u v)} .
$$

Using Viète's relations,

$$
u+v+w=0, \quad u v+v w+u w=-10, \quad u v w=-11,
$$

we further transform this into $\tan (\alpha+\beta+\gamma)=\frac{11}{1+10}=1$. Therefore, $\alpha+\beta+\gamma=\frac{\pi}{4}+k \pi$, where $k$ is an integer that remains to be determined.

From Viète's relations we can see the product of the zeros of the polynomial is negative, so the number of negative zeros is odd. And since the sum of the zeros is 0 , two of them are positive and one is negative. Therefore, one of $\alpha, \beta, \gamma$ lies in the interval $\left(-\frac{\pi}{2}, 0\right)$ and two of them lie in $\left(0, \frac{\pi}{2}\right)$. Hence $k$ must be equal to 0 , and $\arctan u+$ $\arctan v+\arctan w=\frac{\pi}{4}$.

Second solution: Because

$$
\operatorname{Im} \ln (1+i x)=\arctan x,
$$

we see that

$$
\begin{aligned}
\arctan u+\arctan v+\arctan w &=\operatorname{Im} \ln (i P(i))=\operatorname{Im} \ln (11+11 i) \\
&=\arctan 1=\frac{\pi}{4} .
\end{aligned}
$$

(Kózépiskolai Matematikai Lapok (Mathematics Magazine for High Schools, Budapest), proposed by K. Bérczi).
\\
\textbf{Topic} :Probability\\
\textbf{Book} :Putnam and Beyond\\
\textbf{Final Answer} :\\


\textbf{Problem Statement} :
164. Determine the maximum value of $\lambda$ such that whenever $P(x)=x^{3}+a x^{2}+b x+c$ is a cubic polynomial with all zeros real and nonnegative, then

$$
P(x) \geq \lambda(x-a)^{3}
$$

for all $x \geq 0$. Find the equality condition.
\\
\textbf{Solution} :
164. Let $\alpha, \beta, \gamma$ be the zeros of $P(x)$. Without loss of generality, we may assume that $0 \leq \alpha \leq \beta \leq \gamma$. Then

$$
x-a=x+\alpha+\beta+\gamma \geq 0 \quad \text { and } \quad P(x)=(x-\alpha)(x-\beta)(x-\gamma) .
$$

If $0 \leq x \leq \alpha$, using the AM-GM inequality, we obtain

$$
-P(x)=(\alpha-x)(\beta-x)(\gamma-x) \leq \frac{1}{27}(\alpha+\beta+\gamma-3 x)^{3}
$$



$$
\leq \frac{1}{27}(x+\alpha+\beta+\gamma)^{3}=\frac{1}{27}(x-a)^{3},
$$

so that $P(x) \geq-\frac{1}{27}(x-a)^{3}$. Equality holds exactly when $\alpha-x=\beta-x=\gamma-x$ in the first inequality and $\alpha+\beta+\gamma-3 x=x+\alpha+\beta+\gamma$ in the second, that is, when $x=0$ and $\alpha=\beta=\gamma$.

If $\beta \leq x \leq \gamma$, then using again the AM-GM inequality, we obtain

$$
\begin{aligned}
-P(x) &=(x-\alpha)(x-\beta)(\gamma-x) \leq \frac{1}{27}(x+\gamma-\alpha-\beta)^{3} \\
& \leq \frac{1}{27}(x+\alpha+\beta+\gamma)^{3}=\frac{1}{27}(x-a)^{3},
\end{aligned}
$$

so that again $P(x) \geq-\frac{1}{27}(x-a)^{3}$. Equality holds exactly when there is equality in both inequalities, that is, when $\alpha=\beta=0$ and $\gamma=2 x$.

Finally, when $\alpha<x<\beta$ or $x>\gamma$, then

$$
P(x)>0 \geq-\frac{1}{27}(x-a)^{3} .
$$

Thus the desired constant is $\lambda=-\frac{1}{27}$, and the equality occurs when $\alpha=\beta=\gamma$ and $x=0$, or when $\alpha=\beta=0, \gamma$ is any nonnegative real, and $x=\frac{\gamma}{2}$.

(Chinese Mathematical Olympiad, 1999)
\\
\textbf{Topic} :Probability\\
\textbf{Book} :Putnam and Beyond\\
\textbf{Final Answer} :\\


\textbf{Problem Statement} :
166. Find all polynomials $P(x)$ with integer coefficients satisfying $P\left(P^{\prime}(x)\right)=$ $P^{\prime}(P(x))$ for all $x \in \mathbb{R}$.
\\
\textbf{Solution} :
166. Let us first consider the case $n \geq 2$. Let $P(x)=a_{n} x^{n}+a_{n-1} x^{n-1}+\cdots+a_{0}$, $a_{n} \neq 0$. Then

$$
P^{\prime}(x)=n a_{n} x^{n-1}+(n-1) a_{n-1} x^{n-2}+\cdots+a_{1} .
$$

Identifying the coefficients of $x^{n(n-1)}$ in the equality $P\left(P^{\prime}(x)\right)=P^{\prime}(P(x))$, we obtain

$$
a_{n}^{n+1} \cdot n^{n}=a_{n}^{n} \cdot n .
$$

This implies $a_{n} n^{n-1}=1$, and so

$$
a_{n}=\frac{1}{n^{n-1}} .
$$

Since $a_{n}$ is an integer, $n$ must be equal to 1 , a contradiction. If $n=1$, say $P(x)=a x+b$, then we should have $a^{2}+b=a$, hence $b=a-a^{2}$. Thus the answer to the problem is the polynomials of the form $P(x)=a x^{2}+a-a^{2}$.

(Revista Matematică din Timişoara (Timişoara Mathematics Gazette), proposed by T. Andreescu)
\\
\textbf{Topic} :Probability\\
\textbf{Book} :Putnam and Beyond\\
\textbf{Final Answer} :\\


\textbf{Problem Statement} :
167. Determine all polynomials $P(x)$ with real coefficients satisfying $(P(x))^{n}=P\left(x^{n}\right)$ for all $x \in \mathbb{R}$, where $n>1$ is a fixed integer. 
\\
\textbf{Solution} :
167. Let $m$ be the degree of $P(x)$, so $P(x)=a_{m} x^{m}+a_{m-1} x^{m-1}+\cdots+a_{0}$. If $P(x)=$ $x^{k} Q(x)$, then

$$
x^{k n} Q^{n}(x)=x^{k n} Q\left(x^{n}\right),
$$

so

$$
Q^{n}(x)=Q\left(x^{n}\right),
$$

which means that $Q(x)$ satisfies the same relation.

Thus we can assume that $P(0) \neq 0$. Substituting $x=0$, we obtain $a_{0}^{n}=a_{0}$, and since $a_{0}$ is a nonzero real number, it must be equal to 1 if $n$ is even, and to $\pm 1$ if $n$ is odd.

Differentiating the relation from the statement, we obtain

$$
n P^{n-1}(x) P^{\prime}(x)=n P^{\prime}\left(x^{n}\right) x^{n-1} .
$$

For $x=0$ we have $P^{\prime}(0)=0$; hence $a_{1}=0$. Differentiating the relation again and reasoning similarly, we obtain $a_{2}=0$, and then successively $a_{3}=a_{4}=\cdots=a_{m}=0$. It follows that $P(x)=1$ if $n$ is even and $P(x)=\pm 1$ if $n$ is odd.

In general, the only solutions are $P(x)=x^{m}$ if $n$ is even, and $P(x)=\pm x^{m}$ if $n$ is odd, $m$ being some nonnegative integer.

(T. Andreescu) 
\\
\textbf{Topic} :Probability\\
\textbf{Book} :Putnam and Beyond\\
\textbf{Final Answer} :\\


\textbf{Problem Statement} :
194. Let $r$ be a positive real number such that $\sqrt[6]{r}+\frac{1}{\sqrt[6]{r}}=6$. Find the maximum value of $\sqrt[4]{r}-\frac{1}{\sqrt[4]{r}}$
\\
\textbf{Solution} :
194. From the identity

$$
x^{3}+\frac{1}{x^{3}}=\left(x+\frac{1}{x}\right)^{3}-3\left(x+\frac{1}{x}\right),
$$

it follows that

$$
\sqrt{r}+\frac{1}{\sqrt{r}}=6^{3}-3 \times 6=198 .
$$

Hence

$$
\left(\sqrt[4]{r}-\frac{1}{\sqrt[4]{r}}\right)^{2}=198-2,
$$

and the maximum value of $\sqrt[4]{r}-\frac{1}{\sqrt[4]{r}}$ is 14 .

(University of Wisconsin at Whitewater Math Meet, 2003, proposed by T. Andreescu)
\\
\textbf{Topic} :Probability\\
\textbf{Book} :Putnam and Beyond\\
\textbf{Final Answer} :\\


\textbf{Problem Statement} :
200. Do there exist $n \times n$ matrices $A$ and $B$ such that $A B-B A=\mathcal{I}_{n}$ ?
\\
\textbf{Solution} :
200. The answer is negative. The trace of $A B-B A$ is zero, while the trace of $\mathcal{I}_{n}$ is $n$; the matrices cannot be equal.

Remark. The equality cannot hold even for continuous linear transformations on an infinite-dimensional vector space. If $P$ and $Q$ are the linear maps that describe the momentum and the position in Heisenberg's matrix model of quantum mechanics, and if $\hbar$ is Planck's constant, then the equality $P Q-Q P=\hbar \mathcal{I}$ is the mathematical expression of Heisenberg's uncertainty principle. We now see that the position and the momentum cannot be modeled using finite-dimensional matrices (not even infinite-dimensional continuous linear transformations). Note on the other hand that the matrices whose entries are residue classes in $\mathbb{Z}_{4}$,

$$
A=\left(\begin{array}{llll}
0 & 1 & 0 & 0 \\
0 & 0 & 1 & 0 \\
0 & 0 & 0 & 1 \\
0 & 0 & 0 & 0
\end{array}\right) \text { and } B=\left(\begin{array}{llll}
0 & 0 & 0 & 0 \\
1 & 0 & 0 & 0 \\
0 & 2 & 0 & 0 \\
0 & 0 & 3 & 0
\end{array}\right) \text {, }
$$

satisfy $A B-B A=\mathcal{I}_{4}$.
\\
\textbf{Topic} :Probability\\
\textbf{Book} :Putnam and Beyond\\
\textbf{Final Answer} :\\


\textbf{Problem Statement} :
211. Compute the determinant of the $n \times n$ matrix $A=\left(a_{i j}\right)_{i j}$, where

$$
a_{i j}= \begin{cases}(-1)^{|i-j|} & \text { if } i \neq j, \\ 2 & \text { if } i=j .\end{cases}
$$
\\
\textbf{Solution} :
211. By adding the second row to the first, the third row to the second, .., the $n$th row to the $(n-1)$ st, the determinant does not change. Hence

$$
\operatorname{det}(A)=\left|\begin{array}{rrrrrr}
2 & -1 & +1 & \cdots & \pm 1 & \mp 1 \\
-1 & 2 & -1 & \cdots & \mp 1 & \pm 1 \\
+1 & -1 & 2 & \cdots & \pm 1 \mp 1 \\
\vdots & \vdots & \vdots & \ddots & \vdots & \vdots \\
\mp 1 \pm 1 & \mp 1 & \cdots & 2 & -1 \\
\pm 1 & \mp 1 \pm 1 & \cdots & -1 & 2
\end{array}\right|=\left|\begin{array}{ccccccc}
1 & 1 & 0 & 0 & \cdots & 0 & 0 \\
0 & 1 & 1 & 0 & \cdots & 0 & 0 \\
0 & 0 & 1 & 1 & \cdots & 0 & 0 \\
\vdots & \vdots & \vdots & \vdots & \ddots & \vdots & \vdots \\
0 & 0 & 0 & 0 & \cdots & 1 & 1 \\
\pm 1 \mp 1 \pm 1 & \mp 1 & \ldots & -1 & 2
\end{array}\right| .
$$

Now subtract the first column from the second, then subtract the resulting column from the third, and so on. This way we obtain 

$$
\operatorname{det}(A)=\left|\begin{array}{cccccc}
1 & 0 & 0 & \cdots & 0 & 0 \\
0 & 1 & 0 & \cdots & 0 & 0 \\
\vdots & \vdots & \vdots & \ddots & \vdots & \vdots \\
0 & 0 & 0 & \cdots & 1 & 0 \\
\pm 1 & \mp 2 & \pm 3 & \cdots & -n+1 & n+1
\end{array}\right|=n+1 .
$$

(9th International Mathematics Competition for University Students, 2002)
\\
\textbf{Topic} :Probability\\
\textbf{Book} :Putnam and Beyond\\
\textbf{Final Answer} :\\


\textbf{Problem Statement} :
223. Determine the matrix $A$ knowing that its adjoint matrix (the one used in the computation of the inverse) is

$$
A^{*}=\left(\begin{array}{ccc}
m^{2}-1 & 1-m & 1-m \\
1-m & m^{2}-1 & 1-m \\
1-m & 1-m & m^{2}-1
\end{array}\right), \quad m \neq 1,-2
$$
\\
\textbf{Solution} :
223. We know that $A A^{*}=A^{*} A=(\operatorname{det} A) \mathcal{I}_{3}$, so if $A$ is invertible then so is $A^{*}$, and $A=\operatorname{det} A\left(A^{*}\right)^{-1}$. Also, $\operatorname{det} A \operatorname{det} A^{*}=(\operatorname{det} A)^{3}$; hence $\operatorname{det} A^{*}=(\operatorname{det} A)^{2}$. Therefore, $A=\pm \sqrt{\operatorname{det} A^{*}}\left(A^{*}\right)^{-1}$.

Because

$$
A^{*}=(1-m)\left(\begin{array}{ccc}
-m-1 & 1 & 1 \\
1 & -m-1 & 1 \\
1 & 1 & -m-1
\end{array}\right)
$$

we have

$$
\operatorname{det} A^{*}=(1-m)^{3}\left[-(m+1)^{3}+2+3(m+1)\right]=(1-m)^{4}(m+2)^{2} .
$$

Using the formula with minors, we compute the inverse of the matrix

$$
\left(\begin{array}{ccc}
-m-1 & 1 & 1 \\
1 & -m-1 & 1 \\
1 & 1 & -m-1
\end{array}\right)
$$

to be

$$
\frac{1}{(1-m)(m+2)^{2}}\left(\begin{array}{ccc}
-m^{2}-m-2 & m+2 & m+2 \\
m+2 & -m^{2}-m-2 & m+2 \\
m+2 & m+2 & -m^{2}-m-2
\end{array}\right) .
$$

Then $\left(A^{*}\right)^{-1}$ is equal to this matrix divided by $(1-m)^{3}$. Consequently, the matrix we are looking for is

$$
\begin{aligned}
A &=\pm \sqrt{\operatorname{det} A^{*}}\left(A^{*}\right)^{-1} \\
&=\pm \frac{1}{(1-m)^{2}(m+2)}\left(\begin{array}{ccc}
-m^{2}-m-2 & m+2 & m+2 \\
m+2 & -m^{2}-m-2 & m+2 \\
m+2 & m+2 & -m^{2}-m-2
\end{array}\right) .
\end{aligned}
$$

(Romanian mathematics competition)
\\
\textbf{Topic} :Probability\\
\textbf{Book} :Putnam and Beyond\\
\textbf{Final Answer} :\\


\textbf{Problem Statement} :
235. Let $P(x)=x^{n}+x^{n-1}+\cdots+x+1$. Find the remainder obtained when $P\left(x^{n+1}\right)$ is divided by $P(x)$.
\\
\textbf{Solution} :
235. First solution: The zeros of $P(x)$ are $\epsilon, \epsilon^{2}, \ldots, \epsilon^{n}$, where $\epsilon$ is a primitive $(n+1)$ st root of unity. As such, the zeros of $P(x)$ are distinct. Let

$$
P\left(x^{n+1}\right)=Q(x) \cdot P(x)+R(x),
$$

where $R(x)=a_{n-1} x^{n-1}+\cdots+a_{1} x+a_{0}$ is the remainder. Replacing $x$ successively by $\epsilon, \epsilon^{2}, \ldots, \epsilon^{n}$, we obtain

$$
\begin{aligned}
a_{n} \epsilon^{n-1}+\cdots+a_{1} \epsilon+a_{0} &=n+1, \\
a_{n}\left(\epsilon^{2}\right)^{n-1}+\cdots+a_{1} \epsilon^{2}+a_{0} &=n+1, \\
& \cdots \\
a_{n}\left(\epsilon^{n}\right)^{n-1}+\cdots+a_{1} \epsilon^{n}+a_{0} &=n+1,
\end{aligned}
$$

or

$$
\begin{aligned}
{\left[a_{0}-(n+1)\right]+a_{1} \epsilon+\cdots+a_{n-1} \epsilon^{n-1} } &=0, \\
{\left[a_{0}-(n+1)\right]+a_{1}\left(\epsilon^{2}\right)+\cdots+a_{n-1}\left(\epsilon^{2}\right)^{n-1} } &=0, \\
& \cdots \\
{\left[a_{0}-(n+1)\right]+a_{1}\left(\epsilon^{n}\right)+\cdots+a_{n-1}\left(\epsilon^{n}\right)^{n-1} } &=0 .
\end{aligned}
$$

This can be interpreted as a homogeneous system in the unknowns $a_{0}-(n+1)$, $a_{1}, a_{2}, \ldots, a_{n-1}$. The determinant of the coefficient matrix is Vandermonde, thus nonzero, and so the system has the unique solution $a_{0}-(n+1)=a_{1}=\cdots=a_{n-1}=0$. We obtain $R(x)=n+1$.

Second solution: Note that

$$
x^{n+1}=(x-1) P(x)+1 ;
$$

hence

$$
x^{k(n+1)}=(x-1)\left(x^{(k-1)(n+1)}+x^{(k-2)(n+1)}+\cdots+1\right) P(x)+1 .
$$

Thus the remainder of any polynomial $F\left(x^{n+1}\right)$ modulo $P(x)$ is $F(1)$. In our situation this is $n+1$, as seen above.

(Gazeta Matematic ă (Mathematics Gazette, Bucharest), proposed by M. Diaconescu) 
\\
\textbf{Topic} :Probability\\
\textbf{Book} :Putnam and Beyond\\
\textbf{Final Answer} :\\


\textbf{Problem Statement} :
236. Find all functions $f: \mathbb{R} \backslash\{-1,1\} \rightarrow \mathbb{R}$ satisfying

$$
f\left(\frac{x-3}{x+1}\right)+f\left(\frac{3+x}{1-x}\right)=x
$$

for all $x \neq \pm 1$.
\\
\textbf{Solution} :
236. The function $\phi(t)=\frac{t-3}{t+1}$ has the property that $\phi \circ \phi \circ \phi$ equals the identity function. And $\phi(\phi(t))=\frac{3+t}{1-t}$. Replace $x$ in the original equation by $\phi(x)$ and $\phi(\phi(x))$ to obtain two more equations. The three equations form a linear system

$$
\begin{aligned}
f\left(\frac{x-3}{x+1}\right)+f\left(\frac{3+x}{1-x}\right) &=x, \\
f\left(\frac{3+x}{1-x}\right)+f(x) &=\frac{x-3}{x+1}, \\
f(x)+f\left(\frac{x-3}{x+1}\right) &=\frac{3+x}{1-x},
\end{aligned}
$$

in the unknowns

$$
f(x), \quad f\left(\frac{x-3}{x+1}\right), \quad f\left(\frac{3+x}{1-x}\right) .
$$

Solving, we find that

$$
f(t)=\frac{4 t}{1-t^{2}}-\frac{t}{2},
$$

which is the unique solution to the functional equation.

(Kvant (Quantum), also appeared at the Korean Mathematical Olympiad, 1999)
\\
\textbf{Topic} :Probability\\
\textbf{Book} :Putnam and Beyond\\
\textbf{Final Answer} :\\


\textbf{Problem Statement} :
237. Find all positive integer solutions $(x, y, z, t)$ to the Diophantine equation

$$
(x+y)(y+z)(z+x)=t x y z
$$

such that $\operatorname{gcd}(x, y)=\operatorname{gcd}(y, z)=\operatorname{gcd}(z, x)=1$.
\\
\textbf{Solution} :
237. It is obvious that $\operatorname{gcd}(x, x+y)=\operatorname{gcd}(x, x+z)=1$. So in the equality from the statement, $x$ divides $y+z$. Similarly, $y$ divides $z+x$ and $z$ divides $x+y$. It follows that there exist integers $a, b, c$ with $a b c=t$ and

$$
\begin{aligned}
&x+y=c z, \\
&y+z=a x,
\end{aligned}
$$



$$
z+x=b y .
$$

View this as a homogeneous system in the variables $x, y, z$. Because we assume that the system admits nonzero solutions, the determinant of the coefficient matrix is zero. Writing down this fact, we obtain a new Diophantine equation in the unknowns $a, b, c$ :

$$
a b c-a-b-c-2=0 .
$$

This can be solved by examining the following cases:

1. $a=b=c$. Then $a=2$ and it follows that $x=y=z$, because these numbers are pairwise coprime. This means that $x=y=z=1$ and $t=8$. We have obtained the solution $(1,1,1,8)$.

2. $a=b, a \neq c$. The equation becomes $a^{2} c-2=2 a+c$, which is equivalent to $c\left(a^{2}-1\right)=2(a+1)$, that is, $c(a-1)=2$. We either recover case 1 , or find the new solution $c=1, a=b=3$. This yields the solution to the original equation $(1,1,2,9)$
\\
\textbf{Topic} :Probability\\
\textbf{Book} :Putnam and Beyond\\
\textbf{Final Answer} :\\


\textbf{Problem Statement} :
243. Let $A$ be the $n \times n$ matrix whose $i, j$ entry is $i+j$ for all $i, j=1,2, \ldots, n$. What is the rank of $A$ ?
\\
\textbf{Solution} :
243. For $n=1$ the rank is 1 . Let us consider the case $n \geq 2$. Observe that the rank does not change under row/column operations. For $i=n, n-1, \ldots, 2$, subtract the $(i-1)$ st row from the $i$ th. Then subtract the second row from all others. Explicitly, we obtain 

$$
\begin{aligned}
& \operatorname{rank}\left(\begin{array}{cccc}2 & 3 & \cdots & n+1 \\3 & 4 & \cdots & n+2 \\\vdots & \vdots & \ddots & \vdots \\n+1 & n+2 & \cdots & 2 n\end{array}\right)=\operatorname{rank}\left(\begin{array}{cccc}2 & 3 & \cdots & n+1 \\1 & 1 & \cdots & 1 \\\vdots & \ddots & \ddots & \vdots \\1 & 1 & \cdots & 1\end{array}\right) \\
& =\operatorname{rank}\left(\begin{array}{cccc}1 & 2 & \cdots & n \\1 & 1 & \cdots & 1 \\0 & 0 & \cdots & 0 \\\vdots & \vdots & \ddots & \vdots \\0 & 0 & \cdots & 0\end{array}\right)=2 \text {. }
\end{aligned}
$$

(12th International Competition in Mathematics for University Students, 2005)
\\
\textbf{Topic} :Probability\\
\textbf{Book} :Putnam and Beyond\\
\textbf{Final Answer} :\\


\textbf{Problem Statement} :
244. For integers $n \geq 2$ and $0 \leq k \leq n-2$, compute the determinant

$$
\left|\begin{array}{ccccc}
1^{k} & 2^{k} & 3^{k} & \cdots & n^{k} \\
2^{k} & 3^{k} & 4^{k} & \cdots & (n+1)^{k} \\
3^{k} & 4^{k} & 5^{k} & \cdots & (n+2)^{k} \\
\vdots & \vdots & \vdots & \ddots & \vdots \\
n^{k} & (n+1)^{k} & (n+2)^{k} & \cdots & (2 n-1)^{k}
\end{array}\right| .
$$
\\
\textbf{Solution} :
244. The polynomials $P_{j}(x)=(x+j)^{k}, j=0,1, \ldots, n-1$, lie in the $(k+1)$-dimensional real vector space of polynomials of degree at most $k$. Because $k+1<n$, they are linearly dependent. The columns consist of the evaluations of these polynomials at $1,2, \ldots, n$, so the columns are linearly dependent. It follows that the determinant is zero.
\\
\textbf{Topic} :Probability\\
\textbf{Book} :Putnam and Beyond\\
\textbf{Final Answer} :\\


\textbf{Problem Statement} :
250. Consider the $n \times n$ matrix $A=\left(a_{i j}\right)$ with $a_{i j}=1$ if $j-i \equiv 1(\bmod n)$ and $a_{i j}=0$ otherwise. For real numbers $a$ and $b$ find the eigenvalues of $a A+b A^{t}$.
\\
\textbf{Solution} :
250. First solution: The eigenvalues are the zeros of the polynomial $\operatorname{det}\left(\lambda \mathcal{I}_{n}-a A-b A^{t}\right)$. The matrix $\lambda \mathcal{I}_{n}-a A-b A^{t}$ is a circulant matrix, and the determinant of a circulant matrix was the subject of problem 211 in Section 2.3.2. According to that formula,

$$
\operatorname{det}\left(\lambda \mathcal{I}_{n}-a A-b A^{t}\right)=(-1)^{n-1} \prod_{j=0}^{n-1}\left(\lambda \zeta^{j}-a \zeta^{2 j}-b\right),
$$

where $\zeta=e^{2 \pi i / n}$ is a primitive $n$th root of unity. We find that the eigenvalues of $a A+b A^{t}$ are $a \zeta^{j}+b \zeta^{-j}, j=0,1, \ldots, n-1$.

Second solution: Simply note that for $\zeta=e^{2 \pi i / n}$ and $j=0,1, \ldots, n-1,\left(1, \zeta^{j}, \zeta^{2 j}\right.$, $\ldots, \zeta^{(n-1) j}$ ) is an eigenvector with eigenvalue $a \zeta^{j}+b \zeta^{-j}$.
\\
\textbf{Topic} :Probability\\
\textbf{Book} :Putnam and Beyond\\
\textbf{Final Answer} :\\


\textbf{Problem Statement} :
266. For a positive integer $n$ and any real number $c$, define $\left(x_{k}\right)_{k \geq 0}$ recursively by $x_{0}=0$, $x_{1}=1$, and for $k \geq 0$,

$$
x_{k+2}=\frac{c x_{k+1}-(n-k) x_{k}}{k+1} .
$$

Fix $n$ and then take $c$ to be the largest value for which $x_{n+1}=0$. Find $x_{k}$ in terms of $n$ and $k, 1 \leq k \leq n$.
\textbf{Solution} :
266. We try some particular cases. For $n=2$, we obtain $c=1$ and the sequence 1 , 1 , or $n=3, c=2$ and the sequence $1,2,1$, and for $n=4, c=3$ and the sequence $1,3,3,1$. We formulate the hypothesis that $c=n-1$ and $x_{k}=\left(\begin{array}{c}n-1 \\ k-1\end{array}\right)$.

The condition $x_{n+1}=0$ makes the recurrence relation from the statement into a linear system in the unknowns $\left(x_{1}, x_{2}, \ldots, x_{n}\right)$. More precisely, the solution is an eigenvector of the matrix $A=\left(a_{i j}\right)_{i j}$ defined by

$$
a_{i j}= \begin{cases}i & \text { if } j=i+1, \\ n-j & \text { if } j=i-1, \\ 0 & \text { otherwise. }\end{cases}
$$

This matrix has nonnegative entries, so the Perron-Frobenius Theorem as stated here does not really apply. But let us first observe that $A$ has an eigenvector with positive coordinates, namely $x_{k}=\left(\begin{array}{c}n-1 \\ k-1\end{array}\right), k=1,2, \ldots, n$, whose eigenvalue is $n-1$. This follows by rewriting the combinatorial identity

$$
\left(\begin{array}{c}
n-1 \\
k
\end{array}\right)=\left(\begin{array}{c}
n-2 \\
k
\end{array}\right)+\left(\begin{array}{l}
n-2 \\
k-1
\end{array}\right)
$$

as

$$
\left(\begin{array}{c}
n-1 \\
k
\end{array}\right)=\frac{k+1}{n-1}\left(\begin{array}{c}
n-1 \\
k+1
\end{array}\right)+\frac{n-k}{n-1}\left(\begin{array}{l}
n-1 \\
k-1
\end{array}\right) .
$$

To be more explicit, this identity implies that for $c=n-1$, the sequence $x_{k}=\left(\begin{array}{l}n-1 \\ k-1\end{array}\right)$ satisfies the recurrence relation from the statement, and $x_{n+1}=0$.

Let us assume that $n-1$ is not the largest value that $c$ can take. For a larger value, consider an eigenvector $v$ of $A$. Then $\left(A+\mathcal{I}_{n}\right) v=(c+1) v$, and $\left(A+\mathcal{I}_{n}\right)^{n} v=(c+1)^{n} v$. The matrix $\left(A+\mathcal{I}_{n}\right)^{n}$ has positive entries, and so by the Perron-Frobenius Theorem has a unique eigenvector with positive coordinates. We already found one such vector, that for which $x_{k}=\left(\begin{array}{c}n-1 \\ k-1\end{array}\right)$. Its eigenvalue has the largest absolute value among all eigenvalues of $\left(A+\mathcal{I}_{n}\right)^{n}$, which means that $n^{n}>(c+1)^{n}$. This implies $n>c+1$, contradicting our assumption. So $n-1$ is the largest value $c$ can take, and the sequence we found is the answer to the problem.

(57th W.L. Putnam Mathematical Competition, 1997, solution by G. Kuperberg and published in K. Kedlaya, B. Poonen, R. Vakil, The William Lowell Putnam Mathematical Competition 1985-2000, MAA, 2002)
\\
\textbf{Topic} :Probability\\
\textbf{Book} :Putnam and Beyond\\
\textbf{Final Answer} :\\


\textbf{Problem Statement} :
268. Invent a binary operation from which $+,-, \times$, and / can be derived.
\\
\textbf{Solution} :
268. Building on the previous problem, we see that it suffices to produce an operation o, from which the subtraction and reciprocal are derivable. A good choice is $\frac{1}{x-y}$. Indeed, $\frac{1}{x}=\frac{1}{x-0}$, and also $x-y=\frac{1}{(1 /(x-y)-0)}$. Success!

(D.J. Newman, A Problem Seminar, Springer-Verlag) 
\\
\textbf{Topic} :Probability\\
\textbf{Book} :Putnam and Beyond\\
\textbf{Final Answer} :\\


\textbf{Problem Statement} :
297. Find a formula for the general term of the sequence

$$
1,2,2,3,3,3,4,4,4,4,5,5,5,5,5, \ldots
$$
\\
\textbf{Solution} :
297. Examining the sequence, we see that the $m$ th term of the sequence is equal to $n$ exactly for those $m$ that satisfy

$$
\frac{n^{2}-n}{2}+1 \leq m \leq \frac{n^{2}+n}{2} .
$$

So the sequence grows about as fast as the square root of twice the index. Let us rewrite the inequality as

$$
n^{2}-n+2 \leq 2 m \leq n^{2}+n,
$$

then try to solve for $n$. We can almost take the square root. And because $m$ and $n$ are integers, the inequality is equivalent to

$$
n^{2}-n+\frac{1}{4}<2 m<n^{2}+n+\frac{1}{4} .
$$

Here it was important that $n^{2}-n$ is even. And now we can take the square root. We obtain

$$
n-\frac{1}{2}<\sqrt{2 m}<n+\frac{1}{2},
$$

or

$$
n<\sqrt{2 m}+\frac{1}{2}<n+1 .
$$

Now this happens if and only if $n=\left\lfloor\sqrt{2 m}+\frac{1}{2}\right\rfloor$, which then gives the formula for the general term of the sequence

$$
a_{m}=\left\lfloor\sqrt{2 m}+\frac{1}{2}\right\rfloor, \quad m \geq 1 .
$$

(R. Graham, D. Knuth, O. Patashnik, Concrete Mathematics: A Foundation for Computer Science, 2nd ed., Addison-Wesley, 1994) 
\\
\textbf{Topic} :Probability\\
\textbf{Book} :Putnam and Beyond\\
\textbf{Final Answer} :\\


\textbf{Problem Statement} :
298. Find a formula in compact form for the general term of the sequence defined recursively by $x_{1}=1, x_{n}=x_{n-1}+n$ if $n$ is odd, and $x_{n}=x_{n-1}+n-1$ if $n$ is even.
\\
\textbf{Solution} :
298. If we were given the recurrence relation $x_{n}=x_{n-1}+n$, for all $n$, the terms of the sequence would be the triangular numbers $T_{n}=\frac{n(n+1)}{2}$. If we were given the recurrence relation $x_{n}=x_{n-1}+n-1$, the terms of the sequence would be $T_{n-1}+1=\frac{n^{2}-n+2}{2}$. In our case,

$$
\frac{n^{2}-n+2}{2} \leq x_{n} \leq \frac{n^{2}+n}{2} .
$$

We expect $x_{n}=P(n) / 2$ for some polynomial $P(n)=n^{2}+a n+b$; in fact, we should have $x_{n}=\lfloor P(n) / 2\rfloor$ because of the jumps. From here one can easily guess that $x_{n}=\left\lfloor\frac{n^{2}+1}{2}\right\rfloor$, and indeed

$$
\left\lfloor\frac{n^{2}+1}{2}\right\rfloor=\left\lfloor\frac{(n-1)^{2}+1}{2}+\frac{2(n-1)+1}{2}\right\rfloor=\left\lfloor\frac{(n-1)^{2}+1}{2}+\frac{1}{2}\right\rfloor+(n-1),
$$

which is equal to $\left\lfloor\frac{(n-1)^{2}+1}{2}\right\rfloor+(n-1)$ if $n$ is even, and to $\left\lfloor\frac{(n-1)^{2}+1}{2}\right\rfloor+n$ if $n$ is odd. 
\\
\textbf{Topic} :Probability\\
\textbf{Book} :Putnam and Beyond\\
\textbf{Final Answer} :\\


\textbf{Problem Statement} :
305. Find the general term of the sequence given by $x_{0}=3, x_{1}=4$, and

$$
(n+1)(n+2) x_{n}=4(n+1)(n+3) x_{n-1}-4(n+2)(n+3) x_{n-2}, \quad n \geq 2 .
$$
\\
\textbf{Solution} :
305. Divide through by the product $(n+1)(n+2)(n+3)$. The recurrence relation becomes

$$
\frac{x_{n}}{n+3}=4 \frac{x_{n-1}}{n+2}+4 \frac{x_{n-2}}{n+1} .
$$

The sequence $y_{n}=x_{n} /(n+3)$ satisfies the recurrence

$$
y_{n}=4 y_{n-1}-4 y_{n-2} .
$$

Its characteristic equation has the double root 2. Knowing that $y_{0}=1$ and $y_{1}=1$, we obtain $y_{n}=2^{n}-n 2^{n-1}$. It follows that the answer to the problem is

$$
x_{n}=(n+3) 2^{n}-n(n+3) 2^{n-1} .
$$

(D. Buşneag, I. Maftei, Teme pentru cercurile şi concursurile de matematic ă (Themes for mathematics circles and contests), Scrisul Românesc, Craiova)
\\
\textbf{Topic} :Probability\\
\textbf{Book} :Putnam and Beyond\\
\textbf{Final Answer} :\\


\textbf{Problem Statement} :
307. Define the sequence $\left(a_{n}\right)_{n}$ recursively by $a_{1}=1$ and

$$
a_{n+1}=\frac{1+4 a_{n}+\sqrt{1+24 a_{n}}}{16}, \quad \text { for } n \geq 1
$$

Find an explicit formula for $a_{n}$ in terms of $n$. 
\\
\textbf{Solution} :
307. A standard idea is to eliminate the square root. If we set $b_{n}=\sqrt{1+24 a_{n}}$, then $b_{n}^{2}=1+24 a_{n}$, and so

$$
\begin{aligned}
b_{n+1}^{2} &=1+24 a_{n+1}=1+\frac{3}{2}\left(1+4 a_{n}+\sqrt{1+24 a_{n}}\right) \\
&=1+\frac{3}{2}\left(1+\frac{1}{6}\left(b_{n}^{2}-1\right)+b_{n}\right) \\
&=\frac{1}{4}\left(b_{n}^{2}+6 b_{n}+9\right)=\left(\frac{b_{n}+3}{2}\right)^{2} .
\end{aligned}
$$

Hence $b_{n+1}=\frac{1}{2} b_{n}+\frac{3}{2}$. This is an inhomogeneous first-order linear recursion. We can solve this by analogy with inhomogeneous linear first-order equations. Recall that if $a, b$ are constants, then the equation $f^{\prime}(x)=a f(x)+b$ has the solution

$$
f(x)=e^{a x} \int e^{-a x} b d x+c e^{a x} .
$$

In our problem the general term should be

$$
b_{n}=\frac{1}{2^{n+1}}+3 \sum_{k=1}^{n} \frac{1}{2^{k}}, \quad n \geq 1 .
$$

Summing the geometric series, we obtain $b_{n}=3+\frac{1}{2^{n-2}}$, and the answer to our problem is

$$
a_{n}=\frac{b_{n}^{2}-1}{24}=\frac{1}{3}+\frac{1}{2^{n}}+\frac{1}{3} \cdot \frac{1}{2^{2 n-1}} .
$$

(proposed by Germany for the 22nd International Mathematical Olympiad, 1981)
\\
\textbf{Topic} :Probability\\
\textbf{Book} :Putnam and Beyond\\
\textbf{Final Answer} :\\


\textbf{Problem Statement} :
313. Compute

$$
\lim _{n \rightarrow \infty}\left|\sin \left(\pi \sqrt{n^{2}+n+1}\right)\right| .
$$
\\
\textbf{Solution} :
313. The function $|\sin x|$ is periodic with period $\pi$. Hence

$$
\lim _{n \rightarrow \infty}\left|\sin \pi \sqrt{n^{2}+n+1}\right|=\lim _{n \rightarrow \infty}\left|\sin \pi\left(\sqrt{n^{2}+n+1}-n\right)\right| .
$$

But

$$
\lim _{n \rightarrow \infty}\left(\sqrt{n^{2}+n+1}-n\right)=\lim _{n \rightarrow \infty} \frac{n^{2}+n+1-n^{2}}{\sqrt{n^{2}+n+1}+n}=\frac{1}{2} .
$$

It follows that the limit we are computing is equal to $\left|\sin \frac{\pi}{2}\right|$, which is 1 .
\\
\textbf{Topic} :Probability\\
\textbf{Book} :Putnam and Beyond\\
\textbf{Final Answer} :\\


\textbf{Problem Statement} :
321. Compute

$$
\lim _{n \rightarrow \infty} \sum_{k=1}^{n}\left(\frac{k}{n^{2}}\right)^{\frac{k}{n^{2}}+1} .
$$
\\
\textbf{Solution} :
321. We use the fact that

$$
\lim _{x \rightarrow 0^{+}} x^{x}=1 .
$$

As a consequence, we have

$$
\lim _{x \rightarrow 0^{+}} \frac{x^{x+1}}{x}=1 .
$$

For our problem, let $\epsilon>0$ be a fixed small positive number. There exists $n(\epsilon)$ such that for any integer $n \geq n(\epsilon)$,

$$
1-\epsilon<\frac{\left(\frac{k}{n^{2}}\right)^{\frac{k}{n^{2}}+1}}{\frac{k}{n^{2}}}<1+\epsilon, \quad k=1,2, \ldots, n .
$$

From this, using properties of ratios, we obtain

$$
1-\epsilon<\frac{\sum_{k=1}^{n}\left(\frac{k}{n^{2}}\right)^{\frac{k}{n^{2}}+1}}{\sum_{k=1}^{n} \frac{k}{n^{2}}}<1+\epsilon, \quad \text { for } n \geq n(\epsilon) .
$$

Knowing that $\sum_{k=1}^{n} k=\frac{n(n+1)}{2}$, this implies

$$
(1-\epsilon) \frac{n+1}{2 n}<\sum_{k=1}^{n}\left(\frac{k}{n^{2}}\right)^{\frac{k}{n^{2}}+1}<(1+\epsilon) \frac{n+1}{2 n}, \quad \text { for } n \geq n(\epsilon) .
$$

It follows that

$$
\lim _{n \rightarrow \infty} \sum_{k-1}^{n}\left(\frac{k}{n^{2}}\right)^{\frac{k}{n^{2}}+1}=\frac{1}{2} .
$$

(D. Andrica)
\\
\textbf{Topic} :Probability\\
\textbf{Book} :Putnam and Beyond\\
\textbf{Final Answer} :\\


\textbf{Problem Statement} :
340. If $\left(u_{n}\right)_{n}$ is a sequence of positive real numbers and if $\lim _{n \rightarrow \infty} \frac{u_{n+1}}{u_{n}}=u>0$, then $\lim _{n \rightarrow \infty} \sqrt[n]{u_{n}}=u$
\\
\textbf{Solution} :
340. The solution is a direct application of the Cesàro-Stolz theorem. Indeed, if we let $a_{n}=\ln u_{n}$ and $b_{n}=n$, then

$$
\ln \frac{u_{n+1}}{u_{n}}=\ln u_{n+1}-\ln u_{n}=\frac{a_{n+1}-a_{n}}{b_{n+1}-b_{n}}
$$

and

$$
\ln \sqrt[n]{u_{n}}=\frac{1}{n} \ln u_{n}=\frac{a_{n}}{b_{n}} .
$$

The conclusion follows.
\\
\textbf{Topic} :Probability\\
\textbf{Book} :Putnam and Beyond\\
\textbf{Final Answer} :\\


\textbf{Problem Statement} :
343. Let $x_{0} \in[-1,1]$ and $x_{n+1}=x_{n}-\arcsin \left(\sin ^{2} x_{n}\right)$ for $n \geq 0$. Compute $\lim _{n \rightarrow \infty} \sqrt{n} x_{n}$
\\
\textbf{Solution} :
343. It is not difficult to see that $\lim _{n \rightarrow \infty} x_{n}=0$. Because of this fact,

$$
\lim _{n \rightarrow \infty} \frac{x_{n}}{\sin x_{n}}=1 .
$$

If we are able to find the limit of

$$
\frac{n}{\frac{1}{\sin ^{2} x_{n}}},
$$

then this will equal the square of the limit under discussion. We use the Cesàro-Stolz theorem.

Suppose $0<x_{0} \leq 1$ (the cases $x_{0}<0$ and $x_{0}=0$ being trivial; see above). If $0<x_{n} \leq 1$, then $0<\arcsin \left(\sin ^{2} x_{n}\right)<\arcsin \left(\sin x_{n}\right)=x_{n}$, so $0<x_{n+1}<x_{n}$. It follows by induction on $n$ that $x_{n} \in(0,1]$ for all $n$ and $x_{n}$ decreases to 0 . Rewriting the recurrence as $\sin x_{n+1}=\sin x_{n} \sqrt{1-\sin ^{4} x_{n}}-\sin ^{2} x_{n} \cos x_{n}$ gives

$$
\begin{aligned}
\frac{1}{\sin x_{n+1}}-\frac{1}{\sin x_{n}} &=\frac{\sin x_{n}-\sin x_{n+1}}{\sin x_{n} \sin x_{n+1}} \\
&=\frac{\sin x_{n}-\sin x_{n} \sqrt{1-\sin ^{4} x_{n}}+\sin ^{2} x_{n} \cos x_{n}}{\sin x_{n}\left(\sin x_{n} \sqrt{1-\sin ^{4} x_{n}}-\sin ^{2} x_{n} \cos x_{n}\right)} \\
&=\frac{1-\sqrt{1-\sin ^{4} x_{n}}+\sin x_{n} \cos x_{n}}{\sin x_{n} \sqrt{1-\sin ^{4} x_{n}}-\sin ^{2} x_{n} \cos x_{n}} \\
&=\frac{\frac{\sin ^{4} x_{n}}{1+\sqrt{1-\sin ^{4} x_{n}}}+\sin x_{n} \cos x_{n}}{\sin x_{n} \sqrt{1-\sin ^{4} x_{n}}-\sin ^{2} x_{n} \cos x_{n}} \\
&=\frac{\frac{\sin ^{3} x_{n}}{1+\sqrt{1-\sin ^{4} x_{n}}}+\cos x_{n}}{\sqrt{1-\sin ^{4} x_{n}}-\sin x_{n} \cos x_{n}} .
\end{aligned}
$$

Hence

$$
\lim _{n \rightarrow \infty}\left(\frac{1}{\sin x_{n+1}}-\frac{1}{\sin x_{n}}\right)=1
$$

From the Cesàro-Stolz theorem it follows that $\lim _{n \rightarrow \infty} \frac{1}{n \sin x_{n}}=1$, and so we have $\lim _{n \rightarrow \infty} n x_{n}=1$.

(Gazeta Matematica (Mathematics Gazette, Bucharest), 2002, proposed by T. Andreescu)
\\
\textbf{Topic} :Probability\\
\textbf{Book} :Putnam and Beyond\\
\textbf{Final Answer} :\\


\textbf{Problem Statement} :
344. For an arbitrary number $x_{0} \in(0, \pi)$ define recursively the sequence $\left(x_{n}\right)_{n}$ by $x_{n+1}=\sin x_{n}, n \geq 0$. Compute $\lim _{n \rightarrow \infty} \sqrt{n} x_{n}$.
\\
\textbf{Solution} :
344. We compute the square of the reciprocal of the limit, namely $\lim _{n \rightarrow \infty} \frac{1}{n x_{n}^{2}}$. To this end, we apply the Cesàro-Stolz theorem to the sequences $a_{n}=\frac{1}{x_{n}^{2}}$ and $b_{n}=n$. First, note that $\lim _{n \rightarrow \infty} x_{n}=0$. Indeed, in view of the inequality $0<\sin x<x$ on $(0, \pi)$, the sequence is bounded and decreasing, and the limit $L$ satisfies $L=\sin L$, so $L=0$. We then have

$$
\begin{aligned}
\lim _{n \rightarrow \infty}\left(\frac{1}{x_{n+1}^{2}}-\frac{1}{x_{n}^{2}}\right) &=\lim _{n \rightarrow \infty}\left(\frac{1}{\sin ^{2} x_{n}}-\frac{1}{x_{n}^{2}}\right)=\lim _{n \rightarrow \infty} \frac{x_{n}^{2}-\sin ^{2} x_{n}}{x_{n}^{2} \sin ^{2} x_{n}} \\
&=\lim _{x_{n} \rightarrow 0} \frac{x_{n}^{2}-\frac{1}{2}\left(1-\cos 2 x_{n}\right)}{\frac{1}{2} x_{n}^{2}\left(1-\cos 2 x_{n}\right)}=\lim _{x_{n} \rightarrow 0} \frac{2 x_{n}^{2}-\left[\frac{\left(2 x_{n}\right)^{2}}{2 !}-\frac{\left(2 x_{n}\right)^{4}}{4 !}+\cdots\right]}{x_{n}^{2}\left[\frac{\left(2 x_{n}\right)^{2}}{2 !}-\frac{\left(2 x_{n}\right)^{4}}{4 !}+\cdots\right]}
\end{aligned}
$$



$$
=\frac{2^{4} / 4 !}{2^{2} / 2 !}=\frac{1}{3} .
$$

We conclude that the original limit is $\sqrt{3}$.

(J. Dieudonné, Infinitesimal Calculus, Hermann, 1962, solution by Ch. Radoux)
\\
\textbf{Topic} :Probability\\
\textbf{Book} :Putnam and Beyond\\
\textbf{Final Answer} :\\


\textbf{Problem Statement} :
357. Does the series $\sum_{n=1}^{\infty} \sin \pi \sqrt{n^{2}+1}$ converge?
\\
\textbf{Solution} :
357. We have

$$
\sin \pi \sqrt{n^{2}+1}=(-1)^{n} \sin \pi\left(\sqrt{n^{2}+1}-n\right)=(-1)^{n} \sin \frac{\pi}{\sqrt{n^{2}+1}+n} .
$$

Clearly, the sequence $x_{n}=\frac{\pi}{\sqrt{n^{2}+1}+n}$ lies entirely in the interval $\left(0, \frac{\pi}{2}\right)$, is decreasing, and converges to zero. It follows that $\sin x_{n}$ is positive, decreasing, and converges to zero. By Riemann's convergence criterion, $\sum_{k \geq 1}(-1)^{n} \sin x_{n}$, which is the series in question, is convergent.

(Gh. Sireţchi, Calcul Diferential si Integral (Differential and Integral Calculus), Editura Ştiinţifică şi Enciclopedică, 1985)
\\
\textbf{Topic} :Probability\\
\textbf{Book} :Putnam and Beyond\\
\textbf{Final Answer} :\\


\textbf{Problem Statement} :
372. Evaluate in closed form

$$
\sum_{m=0}^{\infty} \sum_{n=0}^{\infty} \frac{m ! n !}{(m+n+2) !} .
$$
\\
\textbf{Solution} :
372. Let us look at the summation over $n$ first. Multiplying each term by $(m+n+2)-$ $(n+1)$ and dividing by $m+1$, we obtain

$$
\frac{m !}{m+1} \sum_{n=0}^{\infty}\left(\frac{n !}{(m+n+1) !}-\frac{(n+1) !}{(m+n+2) !}\right)
$$

This is a telescopic sum that adds up to

$$
\frac{m !}{m+1} \cdot \frac{0 !}{(m+1) !} .
$$

Consequently, the expression we are computing is equal to

$$
\sum_{m=0}^{\infty} \frac{1}{(m+1)^{2}}=\frac{\pi^{2}}{6} .
$$

(Mathematical Mayhem, 1995)
\\
\textbf{Topic} :Probability\\
\textbf{Book} :Putnam and Beyond\\
\textbf{Final Answer} :\\


\textbf{Problem Statement} :
374. Evaluate in closed form

$$
\sum_{k=0}^{n}(-1)^{k}(n-k) !(n+k) !
$$
\\
\textbf{Solution} :
374. First solution: Let $S_{n}=\sum_{k=0}^{n}(-1)^{k}(n-k) !(n+k)$ !. Reordering the terms of the sum, we have

$$
\begin{aligned}
S_{n} &=(-1)^{n} \sum_{k=0}^{n}(-1)^{k} k !(2 n-k) ! \\
&=(-1)^{n} \frac{1}{2}\left((-1)^{n} n ! n !+\sum_{k=0}^{2 n}(-1)^{k} k !(2 n-k) !\right) \\
&=\frac{(n !)^{2}}{2}+(-1)^{n} \frac{T_{n}}{2},
\end{aligned}
$$

where $T_{n}=\sum_{k=0}^{2 n}(-1)^{k} k !(2 n-k)$ !. We now focus on the sum $T_{n}$. Observe that

$$
\frac{T_{n}}{(2 n) !}=\sum_{k=0}^{2 n} \frac{(-1)^{k}}{\left(\begin{array}{c}
2 n \\
k
\end{array}\right)}
$$

and

$$
\frac{1}{\left(\begin{array}{c}
2 n \\
k
\end{array}\right)}=\frac{2 n+1}{2(n+1)}\left[\frac{1}{\left(\begin{array}{c}
2 n+1 \\
k
\end{array}\right)}+\frac{1}{\left(\begin{array}{c}
2 n+1 \\
k+1
\end{array}\right)}\right] .
$$

Hence

$$
\frac{T_{n}}{(2 n) !}=\frac{2 n+1}{2(n+1)}\left[\frac{1}{\left(\begin{array}{c}
2 n+1 \\
0
\end{array}\right)}+\frac{1}{\left(\begin{array}{c}
2 n+1 \\
1
\end{array}\right)}-\frac{1}{\left(\begin{array}{c}
2 n+1 \\
1
\end{array}\right)}-\frac{1}{\left(\begin{array}{c}
2 n+1 \\
2
\end{array}\right)}+\cdots+\frac{1}{\left(\begin{array}{c}
2 n+1 \\
2 n
\end{array}\right)}+\frac{1}{\left(\begin{array}{c}
2 n+1 \\
2 n+1
\end{array}\right)}\right] .
$$

This sum telescopes to

$$
\frac{2 n+1}{2(n+1)}\left[\frac{1}{\left(\begin{array}{c}
2 n+1 \\
0
\end{array}\right)}+\frac{1}{\left(\begin{array}{c}
2 n+1 \\
2 n+1
\end{array}\right)}\right]=\frac{2 n+1}{n+1} .
$$

Thus $T_{n}=\frac{(2 n+1) !}{n+1}$, and therefore

$$
S_{n}=\frac{(n !)^{2}}{2}+(-1)^{n} \frac{(2 n+1) !}{2(n+1)} .
$$

Second solution: Multiply the $k$ th term in $S_{n}$ by $(n-k+1)+(n+k+1)$ and divide by $2(n+1)$ to obtain

$$
S_{n}=\frac{1}{2(n+1)} \sum_{k=0}^{n}\left[(-1)^{k}(n-k+1) !(n+k) !+(-1)^{k}(n-k) !(n+k+1) !\right] .
$$

This telescopes to

$$
\frac{1}{2(n+1)}\left[n !(n+1) !+(-1)^{n}(2 n+1) !\right] .
$$

(T. Andreescu, second solution by R. Stong)
\\
\textbf{Topic} :Probability\\
\textbf{Book} :Putnam and Beyond\\
\textbf{Final Answer} :\\


\textbf{Problem Statement} :
377. Compute the product

$$
\left(1-\frac{4}{1}\right)\left(1-\frac{4}{9}\right)\left(1-\frac{4}{25}\right) \cdots
$$
\\
\textbf{Solution} :
377. For $N \geq 2$, define

$$
a_{N}=\left(1-\frac{4}{1}\right)\left(1-\frac{4}{9}\right)\left(1-\frac{4}{25}\right) \cdots\left(1-\frac{4}{(2 N-1)^{2}}\right) .
$$

The problem asks us to find $\lim _{N \rightarrow \infty} a_{N}$. The defining product for $a_{N}$ telescopes as follows:

$$
\begin{aligned}
a_{N} &=\left[\left(1-\frac{2}{1}\right)\left(1+\frac{2}{1}\right)\right]\left[\left(1-\frac{2}{3}\right)\left(1+\frac{2}{3}\right)\right] \cdots\left[\left(1-\frac{2}{2 N-1}\right)\left(1+\frac{2}{2 N-1}\right)\right] \\
&=(-1 \cdot 3)\left(\frac{1}{3} \cdot \frac{5}{3}\right)\left(\frac{3}{5} \cdot \frac{7}{5}\right) \cdots\left(\frac{2 N-3}{2 N-1} \cdot \frac{2 N+1}{2 N-1}\right)=-\frac{2 N+1}{2 N-1} .
\end{aligned}
$$

Hence the infinite product is equal to

$$
\lim _{N \rightarrow \infty} a_{N}=-\lim _{N \rightarrow \infty} \frac{2 N+1}{2 N-1}=-1 .
$$
\\
\textbf{Topic} :Probability\\
\textbf{Book} :Putnam and Beyond\\
\textbf{Final Answer} :\\


\textbf{Problem Statement} :
378. Let $x$ be a positive number less than 1 . Compute the product

$$
\prod_{n=0}^{\infty}\left(1+x^{2^{n}}\right) .
$$
\\
\textbf{Solution} :
378. Define the sequence $\left(a_{N}\right)_{N}$ by

$$
a_{N}=\prod_{n=1}^{N}\left(1+x^{2^{n}}\right) .
$$

Note that $(1-x) a_{N}$ telescopes as

$$
\begin{aligned}
(1-x) &(1+x)\left(1+x^{2}\right)\left(1+x^{4}\right) \cdots\left(1+x^{2^{N}}\right) \\
=&\left(1-x^{2}\right)\left(1+x^{2}\right)\left(1+x^{4}\right) \cdots\left(1+x^{2^{N}}\right) \\
=&\left(1-x^{4}\right)\left(1+x^{4}\right) \cdots\left(1+x^{2^{N}}\right) \\
=& \cdots=\left(1-x^{2^{N+1}}\right) .
\end{aligned}
$$

Hence $(1-x) a_{N} \rightarrow 1$ as $N \rightarrow \infty$, and therefore

$$
\prod_{n \geq 0}\left(1+x^{2^{n}}\right)=\frac{1}{1-x} .
$$
\\
\textbf{Topic} :Probability\\
\textbf{Book} :Putnam and Beyond\\
\textbf{Final Answer} :\\


\textbf{Problem Statement} :
380. Find the real parameters $m$ and $n$ such that the graph of the function $f(x)=$ $\sqrt[3]{8 x^{3}+m x^{2}}-n x$ has the horizontal asymptote $y=1$. 
\\
\textbf{Solution} :
380. We are supposed to find $m$ and $n$ such that

$$
\lim _{x \rightarrow \infty} \sqrt[3]{8 x^{3}+m x^{2}}-n x=1 \quad \text { or } \quad \lim _{x \rightarrow-\infty} \sqrt[3]{8 x^{3}+m x}-n x=1 .
$$

We compute

$$
\sqrt[3]{8 x^{3}+m x^{2}}-n x=\frac{\left(8-n^{3}\right) x^{3}+m x^{2}}{\sqrt[3]{\left(8 x^{3}+m x^{2}\right)^{2}}+n x \sqrt[3]{8 x^{3}+m x^{2}}+n^{2} x^{2}} .
$$

For this to have a finite limit at either $+\infty$ or $-\infty, 8-n^{3}$ must be equal to 0 (otherwise the highest degree of $x$ in the numerator would be greater than the highest degree of $x$ in the denominator). We have thus found that $n=2$.

Next, factor out and cancel an $x^{2}$ to obtain

$$
f(x)=\frac{m}{\sqrt[3]{\left(8+\frac{m}{x}\right)^{2}}+2 \sqrt[3]{8+\frac{m}{x}}+4} .
$$

We see that $\lim _{x \rightarrow \infty} f(x)=\frac{m}{12}$. For this to be equal to $1, m$ must be equal to 12 . Hence the answer to the problem is $(m, n)=(12,2)$.
\\
\textbf{Topic} :Probability\\
\textbf{Book} :Putnam and Beyond\\
\textbf{Final Answer} :\\


\textbf{Problem Statement} :
381. Does

$$
\lim _{x \rightarrow \pi / 2}(\sin x)^{\frac{1}{\cos x}}
$$

exist?
\\
\textbf{Solution} :
381. This is a limit of the form $1^{\infty}$. It can be computed as follows:

$$
\begin{aligned}
\lim _{x \rightarrow \pi / 2}(\sin x)^{\frac{1}{\cos x}} &=\lim _{x \rightarrow \pi / 2}(1+\sin x-1)^{\frac{1}{\sin x-1} \cdot \frac{\sin x-1}{\cos x}} \\
&=\left(\lim _{t \rightarrow 0}(1+t)^{1 / t}\right)^{\lim _{x \rightarrow \pi / 2} \frac{\sin x-1}{\cos x}}=\exp \left(\lim _{u \rightarrow 0} \frac{\cos u-1}{\sin u}\right)
\end{aligned}
$$



$$
=\exp \left(\frac{\cos u-1}{u} \cdot \frac{u}{\sin u}\right)=e^{0.1}=e^{0}=1 .
$$

The limit therefore exists.
\\
\textbf{Topic} :Probability\\
\textbf{Book} :Putnam and Beyond\\
\textbf{Final Answer} :\\


\textbf{Problem Statement} :
382. For two positive integers $m$ and $n$, compute

$$
\lim _{x \rightarrow 0} \frac{\sqrt[m]{\cos x}-\sqrt[n]{\cos x}}{x^{2}} .
$$
\\
\textbf{Solution} :
382. Without loss of generality, we may assume that $m>n$. Write the limit as

$$
\lim _{x \rightarrow 0} \frac{\sqrt[m n]{\cos ^{n} x}-\sqrt[m n]{\cos ^{m} x}}{x^{2}} .
$$

Now we can multiply by the conjugate and obtain

$$
\begin{aligned}
\lim _{x \rightarrow 0} & \frac{\cos ^{n} x-\cos ^{m} x}{x^{2}\left(\sqrt[m n]{\left(\cos ^{n} x\right)^{m n-1}}+\cdots+\sqrt[m n]{\left(\cos ^{m} x\right)^{m n-1}}\right)} \\
&=\lim _{x \rightarrow 0} \frac{\cos ^{n} x\left(1-\cos ^{m-n} x\right)}{m n x^{2}}=\lim _{x \rightarrow 0} \frac{1-\cos ^{m-n} x}{m n x^{2}} \\
&=\lim _{x \rightarrow 0} \frac{(1-\cos x)\left(1+\cos x+\cdots+\cos ^{m-n-1} x\right)}{m n x^{2}} \\
&=\frac{m-n}{m n} \lim _{x \rightarrow 0} \frac{1-\cos x}{x^{2}}=\frac{m-n}{2 m n} .
\end{aligned}
$$

We are done.
\\
\textbf{Topic} :Probability\\
\textbf{Book} :Putnam and Beyond\\
\textbf{Final Answer} :\\


\textbf{Problem Statement} :
383. Does there exist a nonconstant function $f:(1, \infty) \rightarrow \mathbb{R}$ satisfying the relation $f(x)=f\left(\frac{x^{2}+1}{2}\right)$ for all $x>1$ and such that $\lim _{x \rightarrow \infty} f(x)$ exists?
\\
\textbf{Solution} :
383. For $x>1$ define the sequence $\left(x_{n}\right)_{n \geq 0}$ by $x_{0}=x$ and $x_{n+1}=\frac{x_{n}^{2}+1}{2}, n \geq 0$. The sequence is increasing because of the AM-GM inequality. Hence it has a limit $L$, finite or infinite. Passing to the limit in the recurrence relation, we obtain $L=\frac{L^{2}+1}{2}$; hence either $L=1$ or $L=\infty$. Since the sequence is increasing, $L \geq x_{0}>1$, so $L=\infty$. We therefore have

$$
f(x)=f\left(x_{0}\right)=f\left(x_{1}\right)=f\left(x_{2}\right)=\cdots=\lim _{n \rightarrow \infty} f\left(x_{n}\right)=\lim _{x \rightarrow \infty} f(x) .
$$

This implies that $f$ is constant, which is ruled out by the hypothesis. So the answer to the question is negative.
\\
\textbf{Topic} :Probability\\
\textbf{Book} :Putnam and Beyond\\
\textbf{Final Answer} :\\


\textbf{Problem Statement} :
387. Does there exist a continuous function $f:[0,1] \rightarrow \mathbb{R}$ that assumes every element of its range an even (finite) number of times?
\\
\textbf{Solution} :
387. The answer is yes, there is a tooth function with this property. We construct $f$ to have local maxima at $\frac{1}{2^{2 n+1}}$ and local minima at 0 and $\frac{1}{2^{2 n}}, n \geq 0$. The values of the function at the extrema are chosen to be $f(0)=f(1)=0, f\left(\frac{1}{2}\right)=\frac{1}{2}$, and $f\left(\frac{1}{2^{2 n+1}}\right)=\frac{1}{2^{n}}$ and $f\left(\frac{1}{2^{2 n}}\right)=\frac{1}{2^{n+1}}$ for $n \geq 1$. These are connected through segments. The graph from Figure 66 convinces the reader that $f$ has the desired properties. dapest $)$ )

(Kozépiskolai Matematikai Lapok (Mathematics Gazette for High Schools, Bu-
\\
\textbf{Topic} :Probability\\
\textbf{Book} :Putnam and Beyond\\
\textbf{Final Answer} :\\


\textbf{Problem Statement} :
399. Let $A$ and $B$ be two cities connected by two different roads. Suppose that two cars can travel from $A$ to $B$ on different roads keeping a distance that does not exceed one mile between them. Is it possible for the cars to travel the first one from $A$ to $B$ and the second one from $B$ to $A$ in such a way that the distance between them is always greater than one mile?
\\
\textbf{Solution} :
399. Without loss of generality, we may assume that the cars traveled on one day from $A$ to $B$ keeping a distance of at most one mile between them, and on the next day they traveled in opposite directions in the same time interval, which we assume to be of length one unit of time.

Since the first car travels in both days on the same road and in the same direction, it defines two parametrizations of that road. Composing the motions of both cars during the second day of travel with a homeomorphism (continuous bijection) of the time interval $[0,1]$, we can ensure that the motion of the first car yields the same parametrization of the road on both days. Let $f(t)$ be the distance from the second car to $A$ when the first is at $t$ on the first day, and $g(t)$ the distance from the second car to $A$ when the first is at $t$ on the second day. These two functions are continuous, so their difference is also continuous. But $f(0)-g(0)=-\operatorname{dist}(A, B)$, and $f(1)-g(1)=\operatorname{dist}(A, B)$, where $\operatorname{dist}(A, B)$ is the distance between the cities.

The intermediate value property implies that there is a moment $t$ for which $f(t)-$ $g(t)=0$. At that moment the two cars are in the same position as they were the day before, so they are at distance at most one mile. Hence the answer to the problem is no. 
\\
\textbf{Topic} :Probability\\
\textbf{Book} :Putnam and Beyond\\
\textbf{Final Answer} :\\


\textbf{Problem Statement} :
405. Find all positive real solutions to the equation $2^{x}=x^{2}$.
\\
\textbf{Solution} :
405. Taking the logarithm, transform the equation into the equivalent $x \ln 2=2 \ln x$. Define the function $f: \mathbb{R} \rightarrow \mathbb{R}, f(x)=x \ln 2-2 \ln x$. We are to find the zeros of $f$. Differentiating, we obtain

$$
f^{\prime}(x)=\ln 2-\frac{2}{x},
$$

which is strictly increasing. The unique zero of the derivative is $\frac{2}{\ln 2}$, and so $f^{\prime}$ is negative for $x<2 / \ln 2$ and positive for $x>\frac{2}{\ln 2}$. Note also that $\lim _{x \rightarrow 0} f(x)=\lim _{x \rightarrow \infty} f(x)=$ $\infty$. There are two possibilities: either $f\left(\frac{2}{\ln 2}\right)>0$, in which case the equation $f(x)=0$ has no solutions, or $f\left(\frac{2}{\ln 2}\right)<0$, in which case the equation $f(x)=0$ has exactly two solutions. The latter must be true, since $f(2)=f(4)=0$. Therefore, $x=2$ and $x=4$ are the only solutions to $f(x)=0$, and hence also to the original equation.
\\
\textbf{Topic} :Probability\\
\textbf{Book} :Putnam and Beyond\\
\textbf{Final Answer} :\\


\textbf{Problem Statement} :
407. Determine

$$
\max _{z \in \mathbb{C},|z|=1}\left|z^{3}-z+2\right| .
$$
\\
\textbf{Solution} :
407. Let $f: \mathbb{C} \rightarrow \mathbb{C}, f(z)=z^{3}-z+2$. We have to determine $\max _{|z|=1}|f(z)|^{2}$. For this, we switch to real coordinates. If $|z|=1$, then $z=x+i y$ with $y^{2}=1-x^{2}$, $-1 \leq x \leq 1$. View the restriction of $|f(z)|^{2}$ to the unit circle as a function depending on the real variable $x$ :

$$
\begin{aligned}
|f(z)|^{2} &=\left|(x+i y)^{3}-(x+i y)+2\right|^{2} \\
&=\left|\left(x^{3}-3 x y^{2}-x+2\right)+i y\left(3 x^{2}-y^{2}-1\right)\right|^{2} \\
&=\left|\left(x^{3}-3 x\left(1-x^{2}\right)-x+2\right)+i y\left(3 x^{2}-\left(1-x^{2}\right)-1\right)\right|^{2} \\
&=\left(4 x^{3}-4 x+2\right)^{2}+\left(1-x^{2}\right)\left(4 x^{2}-2\right)^{2} \\
&=16 x^{3}-4 x^{2}-16 x+8 .
\end{aligned}
$$

Call this last expression $g(x)$. Its maximum on $[-1,1]$ is either at a critical point or at an endpoint of the interval. The critical points are the roots of $g^{\prime}(x)=48 x^{2}-8 x-16=0$, namely, $x=\frac{2}{3}$ and $x=-\frac{1}{2}$. We compute $g(-1)=4, g\left(-\frac{1}{2}\right)=13, g\left(\frac{2}{3}\right)=\frac{8}{27}$, $g(1)=4$. The largest of them is 13 , which is therefore the answer to the problem. It is attained when $z=-\frac{1}{2} \pm \frac{\sqrt{3}}{2} i$

(8th W.L. Putnam Mathematical Competition, 1947)
\\
\textbf{Topic} :Probability\\
\textbf{Book} :Putnam and Beyond\\
\textbf{Final Answer} :\\


\textbf{Problem Statement} :
408. Find the minimum of the function $f: \mathbb{R} \rightarrow \mathbb{R}$,

$$
f(x)=\frac{\left(x^{2}-x+1\right)^{3}}{x^{6}-x^{3}+1} .
$$
\\
\textbf{Solution} :
408. After we bring the function into the form

$$
f(x)=\frac{\left(x-1+\frac{1}{x}\right)^{3}}{x^{3}-1+\frac{1}{x^{3}}},
$$

the substitution $x+\frac{1}{x}=s$ becomes natural. We are to find the minimum of the function

$$
h(s)=\frac{(s-1)^{3}}{s^{3}-3 s-1}=1+\frac{-3 s^{2}+6 s}{s^{3}-3 s-1}
$$

over the domain $(-\infty,-2] \cup[2, \infty)$. Setting the first derivative equal to zero yields the equation

$$
3(s-1)\left(s^{3}-3 s^{2}+2\right)=0 .
$$

The roots are $s=1$ (double root) and $s=1 \pm \sqrt{3}$. Of these, only $s=1+\sqrt{3}$ lies in the domain of the function.

We compute

$$
\lim _{x \rightarrow \pm \infty} h(s)=1, \quad h(2)=1, \quad h(-2)=9, \quad h(1+\sqrt{3})=\frac{\sqrt{3}}{2+\sqrt{3}} .
$$

Of these the last is the least. Hence the minimum of $f$ is $\sqrt{3} /(2+\sqrt{3})$, which is attained when $x+\frac{1}{x}=1+\sqrt{3}$, that is, when $x=(1+\sqrt{3} \pm \sqrt[4]{12}) / 2$.

(Mathematical Reflections, proposed by T. Andreescu)
\\
\textbf{Topic} :Probability\\
\textbf{Book} :Putnam and Beyond\\
\textbf{Final Answer} :\\


\textbf{Problem Statement} :
409. How many real solutions does the equation

$$
\sin (\sin (\sin (\sin (\sin x))))=\frac{x}{3}
$$

have?
\\
\textbf{Solution} :
409. Let $f(x)=\sin (\sin (\sin (\sin (\sin (x)))))$. The first solution is $x=0$. We have

$$
\begin{aligned}
f^{\prime}(0) &=\cos 0 \cos (\sin 0) \cos (\sin (\sin 0)) \cos (\sin (\sin (\sin 0))) \cos (\sin (\sin (\sin (\sin 0)))) \\
&=1>\frac{1}{3} .
\end{aligned}
$$

Therefore, $f(x)>\frac{x}{3}$ in some neighborhood of 0 . On the other hand, $f(x)<1$, whereas $\frac{x}{3}$ is not bounded as $x \rightarrow \infty$. Therefore, $f\left(x_{0}\right)=\frac{x_{0}}{3}$ for some $x_{0}>0$. Because $f$ is odd, $-x_{0}$ is also a solution. The second derivative of $f$ is

$$
\begin{aligned}
&-\cos (\sin x) \cos (\sin (\sin x)) \cos (\sin (\sin (\sin x))) \cos (\sin (\sin (\sin (\sin x)))) \sin x \\
&\quad-\cos ^{2} x \cos (\sin (\sin x)) \cos (\sin (\sin (\sin x))) \cos (\sin (\sin (\sin (\sin x)))) \sin (\sin x)
\end{aligned}
$$



$$
\begin{aligned}
&-\cos ^{2} x \cos ^{2}(\sin x) \cos (\sin (\sin (\sin x))) \cos (\sin (\sin (\sin (\sin x)))) \sin (\sin (\sin x)) \\
&-\cos ^{2} x \cos ^{2}(\sin x) \cos ^{2}(\sin (\sin x)) \cos (\sin (\sin (\sin (\sin x))) \sin (\sin (\sin (\sin x))) \\
&\left.-\cos ^{2} x \cos ^{2}(\sin x)\right) \cos ^{2}(\sin (\sin x)) \cos ^{2}(\sin (\sin (\sin x))) \sin (\sin (\sin (\sin (\sin x)))
\end{aligned}
$$

which is clearly nonpositive for $0 \leq x \leq 1$. This means that $f^{\prime}(x)$ is monotonic. Therefore, $f^{\prime}(x)$ has at most one root $x^{\prime}$ in $[0,+\infty)$. Then $f(x)$ is monotonic at $\left[0, x^{\prime}\right]$ and $\left[x^{\prime},+\infty\right)$ and has at most two nonnegative roots. Because $f(x)$ is an odd function, it also has at most two nonpositive roots. Therefore, $-x_{0}, 0, x_{0}$ are the only solutions. 
\\
\textbf{Topic} :Probability\\
\textbf{Book} :Putnam and Beyond\\
\textbf{Final Answer} :\\


\textbf{Problem Statement} :
413. Let $f$ and $g$ be $n$-times continuously differentiable functions in a neighborhood of a point $a$, such that $f(a)=g(a)=\alpha, f^{\prime}(a)=g^{\prime}(a), \ldots, f^{(n-1)}(a)=g^{(n-1)}(a)$, and $f^{(n)}(a) \neq g^{(n)}(a)$. Find, in terms of $\alpha$,

$$
\lim _{x \rightarrow a} \frac{e^{f(x)}-e^{g(x)}}{f(x)-g(x)} .
$$
\\
\textbf{Solution} :
413. Let us examine the function $F(x)=f(x)-g(x)$. Because $F^{(n)}(a) \neq 0$, we have $F^{(n)}(x) \neq 0$ for $x$ in a neighborhood of $a$. Hence $F^{(n-1)}(x) \neq 0$ for $x \neq a$ and $x$ in a neighborhood of $a$ (otherwise, this would contradict Rolle's theorem). Then $F^{(n-2)}(x)$ is monotonic to the left, and to the right of $a$, and because $F^{(n-2)}(a)=0, F^{(n-2)}(x) \neq 0$ for $x \neq a$ and $x$ in a neighborhood of $a$. Inductively, we obtain $F^{\prime}(x) \neq 0$ and $f(x) \neq 0$ in some neighborhood of $a$.

The limit from the statement can be written as

$$
\lim _{x \rightarrow a} e^{g(x)} \frac{e^{f(x)-g(x)}-1}{f(x)-g(x)} .
$$

We only have to compute the limit of the fraction, since $g(x)$ is a continuous function. We are in a $\frac{0}{0}$ situation, and can apply L'Hôpital's theorem:

$$
\lim _{x \rightarrow a} \frac{e^{f(x)-g(x)}-1}{f(x)-g(x)}=\lim _{x \rightarrow a} \frac{\left(f^{\prime}(x)-g^{\prime}(x)\right) e^{f(x)-g(x)}}{f^{\prime}(x)-g^{\prime}(x)}=e^{0}=1 .
$$

Hence the limit from the statement is equal to $e^{g(a)}=e^{\alpha}$.

(N. Georgescu-Roegen)
\\
\textbf{Topic} :Probability\\
\textbf{Book} :Putnam and Beyond\\
\textbf{Final Answer} :\\


\textbf{Problem Statement} :
423. Find all real solutions to the equation

$$
6^{x}+1=8^{x}-27^{x-1} .
$$
\\
\textbf{Solution} :
423. The equation is $a^{3}+b^{3}+c^{3}=3 a b c$, with $a=2^{x}, b=-3^{x-1}$, and $c=-1$. Using the factorization

$$
a^{3}+b^{3}+c^{3}-3 a b c=\frac{1}{2}(a+b+c)\left[(a-b)^{2}+(b-c)^{2}+(c-a)^{2}\right]
$$

we find that $a+b+c=0$ (the other factor cannot be zero since, for example, $2^{x}$ cannot equal $-1)$. This yields the simpler equation

$$
2^{x}=3^{x-1}+1 .
$$

Rewrite this as

$$
3^{x-1}-2^{x-1}=2^{x-1}-1 .
$$

We immediately notice the solutions $x=1$ and $x=2$. Assume that another solution exists, and consider the function $f(t)=t^{x-1}$. Because $f(3)-f(2)=f(2)-f(1)$, by the mean value theorem there exist $t_{1} \in(2,3)$ and $t_{2} \in(1,2)$ such that $f^{\prime}\left(t_{1}\right)=f^{\prime}\left(t_{2}\right)$. This gives rise to the impossible equality $(x-1) t_{1}^{x-2}=(x-1) t_{2}^{x-2}$. We conclude that there are only two solutions: $x=1$ and $x=2$.

(Mathematical Reflections, proposed by T. Andreescu) 
\\
\textbf{Topic} :Probability\\
\textbf{Book} :Putnam and Beyond\\
\textbf{Final Answer} :\\


\textbf{Problem Statement} :
425. Let $x_{1}, x_{2}, \ldots, x_{n}$ be real numbers. Find the real numbers $a$ that minimize the expression

$$
\left|a-x_{1}\right|+\left|a-x_{2}\right|+\cdots+\left|a-x_{n}\right| .
$$
\\
\textbf{Solution} :
425. Arrange the $x_{i}$ 's in increasing order $x_{1} \leq x_{2} \leq \cdots \leq x_{n}$. The function

$$
f(a)=\left|a-x_{1}\right|+\left|a-x_{2}\right|+\cdots+\left|a-x_{n}\right|
$$

is convex, being the sum of convex functions. It is piecewise linear. The derivative at a point $a$, in a neighborhood of which $f$ is linear, is equal to the difference between the number of $x_{i}$ 's that are less than $a$ and the number of $x_{i}$ 's that are greater than $a$. The global minimum is attained where the derivative changes sign. For $n$ odd, this happens precisely at $x_{\lfloor n / 2\rfloor+1}$. If $n$ is even, the minimum is achieved at any point of the interval $\left[x_{\lfloor n / 2\rfloor}, x_{\lfloor n / 2\rfloor+1}\right]$ at which the first derivative is zero and the function is constant.

So the answer to the problem is $a=x_{\lfloor n / 2\rfloor+1}$ if $n$ is odd, and $a$ is any number in the interval $\left[x_{\lfloor n / 2\rfloor}, x_{\lfloor n / 2\rfloor+1}\right]$ if $n$ is even.

Remark. The required number $x$ is called the median of $x_{1}, x_{2}, \ldots, x_{n}$. In general, if the numbers $x \in \mathbb{R}$ occur with probability distribution $d \mu(x)$ then their median $a$ minimizes

$$
E(|x-a|)=\int_{-\infty}^{\infty}|x-a| d \mu(x) .
$$

The median is any number such that

$$
\int_{-\infty}^{a} d \mu(x)=P(x \leq a) \geq \frac{1}{2}
$$

and

$$
\int_{a}^{\infty} d \mu(x)=P(x \geq a) \geq \frac{1}{2} .
$$

In the particular case of our problem, the numbers $x_{1}, x_{2}, \ldots, x_{n}$ occur with equal probability, so the median lies in the middle.
\\
\textbf{Topic} :Probability\\
\textbf{Book} :Putnam and Beyond\\
\textbf{Final Answer} :\\


\textbf{Problem Statement} :
434. Let $\alpha, \beta$, and $\gamma$ be three fixed positive numbers and $[a, b]$ a given interval. Find $x, y, z$ in $[a, b]$ for which the expression

$$
E(x, y, z)=\alpha(x-y)^{2}+\beta(y-z)^{2}+\gamma(z-x)^{2}
$$

has maximal value.
\\
\textbf{Solution} :
434. We assume that $\alpha \leq \beta \leq \gamma$, the other cases being similar. The expression is a convex function in each of the variables, so it attains its maximum for some $x, y, z=a$ or $b$.

Now let us fix three numbers $x, y, z \in[a, b]$, with $x \leq y \leq z$. We have

$$
E(x, y, z)-E(x, z, y)=(\gamma-\alpha)\left((z-x)^{2}-(y-z)^{2}\right) \geq 0,
$$

and hence $E(x, y, z) \geq E(x, z, y)$. Similarly, $E(x, y, z) \geq E(y, x, z)$ and $E(z, y, x) \geq$ $E(y, z, x)$. So it suffices to consider the cases $x=a, z=b$ or $x=b$ and $z=a$. For these cases we have

$$
E(a, a, b)=E(b, b, a)=(\beta+\gamma)(b-a)^{2}
$$

and

$$
E(a, b, b)=E(b, a, a)=(\alpha+\gamma)(b-a)^{2} .
$$

We deduce that the maximum of the expression under discussion is $(\beta+\gamma)(b-a)^{2}$, which is attained for $x=y=a, z=b$ and for $x=y=b, z=a$.

(Revista Matematică din Timişoara (Timişoara Mathematics Gazette), proposed by D. Andrica and I. Raşa)
\\
\textbf{Topic} :Probability\\
\textbf{Book} :Putnam and Beyond\\
\textbf{Final Answer} :\\


\textbf{Problem Statement} :
444. Compute the integral

$$
\int\left(1+2 x^{2}\right) e^{x^{2}} d x
$$
\\
\textbf{Solution} :
444. Split the integral as

$$
\int e^{x^{2}} d x+\int 2 x^{2} e^{x^{2}} d x .
$$

Denote the first integral by $I_{1}$. Then use integration by parts to transform the second integral as

$$
\int 2 x^{2} e^{x^{2}} d x=x e^{x^{2}}-\int e^{x^{2}} d x=x e^{x^{2}}-I_{1} .
$$

The integral from the statement is therefore equal to

$$
I_{1}+x e^{x^{2}}-I_{1}=x e^{x^{2}}+C .
$$
\\
\textbf{Topic} :Probability\\
\textbf{Book} :Putnam and Beyond\\
\textbf{Final Answer} :\\


\textbf{Problem Statement} :
445. Compute
\\
\textbf{Solution} :
445. Adding and subtracting $e^{x}$ in the numerator, we obtain

$$
\begin{aligned}
\int \frac{x+\sin x-\cos x-1}{x+e^{x}+\sin x} d x &=\int \frac{x+e^{x}+\sin x-1-e^{x}-\cos x}{x+e^{x}+\sin x} d x \\
&=\int \frac{x+e^{x}+\sin x}{x+e^{x}+\sin x} d x-\int \frac{1+e^{x}+\cos x}{x+e^{x}+\sin x} d x \\
&=x+\ln \left(x+e^{x}+\sin x\right)+C .
\end{aligned}
$$

(Romanian college entrance exam)
\\
\textbf{Topic} :Probability\\
\textbf{Book} :Putnam and Beyond\\
\textbf{Final Answer} :\\


\textbf{Problem Statement} :
446. Find

$$
\int \frac{x+\sin x-\cos x-1}{x+e^{x}+\sin x} d x .
$$

$$
\int\left(x^{6}+x^{3}\right) \sqrt[3]{x^{3}+2} d x
$$
\\
\textbf{Solution} :
446. The trick is to bring a factor of $x$ inside the cube root:

$$
\int\left(x^{6}+x^{3}\right) \sqrt[3]{x^{3}+2} d x=\int\left(x^{5}+x^{2}\right) \sqrt[3]{x^{6}+2 x^{3}} d x .
$$

The substitution $u=x^{6}+2 x^{3}$ now yields the answer

$$
\frac{1}{6}\left(x^{6}+2 x^{3}\right)^{4 / 3}+C .
$$

(G.T. Gilbert, M.I. Krusemeyer, L.C. Larson, The Wohascum County Problem Book, MAA, 1993)
\\
\textbf{Topic} :Probability\\
\textbf{Book} :Putnam and Beyond\\
\textbf{Final Answer} :\\


\textbf{Problem Statement} :
447. Compute the integral

$$
\int \frac{x^{2}+1}{x^{4}-x^{2}+1} d x
$$
\\
\textbf{Solution} :
447. We want to avoid the lengthy method of partial fraction decomposition. To this end, we rewrite the integral as

$$
\int \frac{x^{2}\left(1+\frac{1}{x^{2}}\right)}{x^{2}\left(x^{2}-1+\frac{1}{x^{2}}\right)} d x=\int \frac{1+\frac{1}{x^{2}}}{x^{2}-1+\frac{1}{x^{2}}} d x .
$$

With the substitution $x-\frac{1}{x}=t$ we have $\left(1+\frac{1}{x^{2}}\right) d x=d t$, and the integral takes the form

$$
\int \frac{1}{t^{2}+1} d t=\arctan t+C
$$

We deduce that the integral from the statement is equal to

$$
\arctan \left(x-\frac{1}{x}\right)+C
$$
\\
\textbf{Topic} :Probability\\
\textbf{Book} :Putnam and Beyond\\
\textbf{Final Answer} :\\


\textbf{Problem Statement} :
448. Compute

$$
\int \sqrt{\frac{e^{x}-1}{e^{x}+1}} d x, \quad x>0 .
$$
\\
\textbf{Solution} :
448. Substitute $u=\sqrt{\frac{e^{x}-1}{e^{x}+1}}, 0<u<1$. Then $x=\ln \left(1+u^{2}\right)-\ln \left(1-u^{2}\right)$, and $d x=\left(\frac{2 u}{1+u^{2}}+\frac{2 u}{1-u^{2}}\right) d u$. The integral becomes

$$
\begin{aligned}
\int u\left(\frac{2 u}{u^{2}+1}+\frac{2 u}{u^{2}-1}\right) d u &=\int\left(4-\frac{2}{u^{2}+1}+\frac{2}{u^{2}-1}\right) d u \\
&=4 u-2 \arctan u+\int\left(\frac{1}{u+1}+\frac{1}{1-u}\right) d u \\
&=4 u-2 \arctan u+\ln (u+1)-\ln (u-1)+C .
\end{aligned}
$$

In terms of $x$, this is equal to

$$
4 \sqrt{\frac{e^{x}-1}{e^{x}+1}}-2 \arctan \sqrt{\frac{e^{x}-1}{e^{x}+1}}+\ln \left(\sqrt{\frac{e^{x}-1}{e^{x}+1}}+1\right)-\ln \left(\sqrt{\frac{e^{x}-1}{e^{x}+1}}-1\right)+C
$$
\\
\textbf{Topic} :Probability\\
\textbf{Book} :Putnam and Beyond\\
\textbf{Final Answer} :\\


\textbf{Problem Statement} :
449. Find the antiderivatives of the function $f:[0,2] \rightarrow \mathbb{R}$,

$$
f(x)=\sqrt{x^{3}+2-2 \sqrt{x^{3}+1}}+\sqrt{x^{3}+10-6 \sqrt{x^{3}+1}} .
$$
\\
\textbf{Solution} :
449. If we naively try the substitution $t=x^{3}+1$, we obtain

$$
f(t)=\sqrt{t+1-2 \sqrt{t}}+\sqrt{t+9-6 \sqrt{t}}
$$

Now we recognize the perfect squares, and we realize that

$$
f(x)=\sqrt{\left(\sqrt{x^{3}+1}-1\right)^{2}}+\sqrt{\left(\sqrt{x^{3}+1}-3\right)^{2}}=\left|\sqrt{x^{3}+1}-1\right|+\left|\sqrt{x^{3}+1}-3\right| .
$$

When $x \in[0,2], 1 \leq \sqrt{x^{3}+1} \leq 3$. Therefore,

$$
f(x)=\sqrt{x^{3}+1}-1+3-\sqrt{x^{3}+1}=2 .
$$

The antiderivatives of $f$ are therefore the linear functions $f(x)=2 x+C$, where $C$ is a constant.

(communicated by E. Craina)
\\
\textbf{Topic} :Probability\\
\textbf{Book} :Putnam and Beyond\\
\textbf{Final Answer} :\\


\textbf{Problem Statement} :
450. For a positive integer $n$, compute the integral

$$
\int \frac{x^{n}}{1+x+\frac{x^{2}}{2 !}+\cdots+\frac{x^{n}}{n !}} d x
$$
\\
\textbf{Solution} :
450. Let $f_{n}=1+x+\frac{x^{2}}{2 !}+\cdots+\frac{x^{n}}{n !}$. Then $f^{\prime}(x)=1+x+\cdots+\frac{x^{n-1}}{(n-1) !}$. The integral in the statement becomes 

$$
\begin{aligned}
I_{n} &=\int \frac{n !\left(f_{n}(x)-f_{n}^{\prime}(x)\right)}{f_{n}(x)} d x=n ! \int\left(1-\frac{f_{n}^{\prime}(x)}{f_{n}(x)}\right) d x=n ! x-n ! \ln f_{n}(x)+C \\
&=n ! x-n ! \ln \left(1+x+\frac{x^{2}}{2 !}+\cdots+\frac{x^{n}}{n !}\right)+C .
\end{aligned}
$$

(C. Mortici, Probleme Pregătitoare pentru Concursurile de Matematic $\breve{a}$ (Training Problems for Mathematics Contests), GIL, 1999)
\\
\textbf{Topic} :Probability\\
\textbf{Book} :Putnam and Beyond\\
\textbf{Final Answer} :\\


\textbf{Problem Statement} :
451. Compute the integral

$$
\int \frac{d x}{\left(1-x^{2}\right) \sqrt[4]{2 x^{2}-1}}
$$
\\
\textbf{Solution} :
451. The substitution is

$$
u=\frac{x}{\sqrt[4]{2 x^{2}-1}}
$$

for which

$$
d u=\frac{x^{2}-1}{\left(2 x^{2}-1\right) \sqrt[4]{2 x^{2}-1}} d x
$$

We can transform the integral as follows:

$$
\begin{aligned}
\int \frac{2 x^{2}-1}{-\left(x^{2}-1\right)^{2}} \cdot \frac{x^{2}-1}{\left(2 x^{2}-1\right) \sqrt[4]{2 x^{2}-1}} d x &=\int \frac{1}{\frac{-x^{4}+2 x^{2}-1}{2 x^{2}-1}} \cdot \frac{x^{2}-1}{\left(2 x^{2}-1\right) \sqrt[4]{2 x^{2}-1}} d x \\
&=\int \frac{1}{1-\frac{x^{4}}{2 x^{2}-1}} \cdot \frac{x^{2}-1}{\left(2 x^{2}-1\right) \sqrt[4]{2 x^{2}-1}} d x \\
&=\int \frac{1}{1-u^{4}} d u
\end{aligned}
$$

This is computed using Jacobi's method for rational functions, giving the final answer to the problem

$$
\frac{1}{4} \ln \frac{\sqrt[4]{2 x^{2}-1}+x}{\sqrt[4]{2 x^{2}-1}-x}-\frac{1}{2} \arctan \frac{\sqrt[4]{2 x^{2}-1}}{x}+C
$$
\\
\textbf{Topic} :Probability\\
\textbf{Book} :Putnam and Beyond\\
\textbf{Final Answer} :\\


\textbf{Problem Statement} :
452. Compute

$$
\int \frac{x^{4}+1}{x^{6}+1} d x .
$$

Give the answer in the form $\alpha \arctan \frac{P(x)}{Q(x)}+C, \alpha \in \mathbb{Q}$, and $P(x), Q(x) \in \mathbb{Z}[x]$. 
\textbf{Solution} :
452. Of course, Jacobi's partial fraction decomposition method can be applied, but it is more laborious. However, in the process of applying it we factor the denominator as $x^{6}+1=\left(x^{2}+1\right)\left(x^{4}-x^{2}+1\right)$, and this expression can be related somehow to the numerator. Indeed, if we add and subtract an $x^{2}$ in the numerator, we obtain

$$
\frac{x^{4}+1}{x^{6}+1}=\frac{x^{4}-x^{2}+1}{x^{6}+1}+\frac{x^{2}}{x^{6}+1}
$$

Now integrate as follows:

$$
\int \frac{x^{4}+1}{x^{6}+1} d x=\int \frac{x^{4}-x^{2}+1}{x^{6}+1} d x+\int \frac{x^{2}}{x^{6}+1} d x=\int \frac{1}{x^{2}+1} d x+\int \frac{1}{3} \frac{\left(x^{3}\right)^{\prime}}{\left(x^{3}\right)^{2}+1} d x
$$



$$
=\arctan x+\frac{1}{3} \arctan x^{3} .
$$

To write the answer in the required form we should have

$$
3 \arctan x+\arctan x^{3}=\arctan \frac{P(x)}{Q(x)} .
$$

Applying the tangent function to both sides, we deduce

$$
\frac{\frac{3 x-x^{3}}{1-3 x^{2}}+x^{3}}{1-\frac{3 x-x^{3}}{1-3 x^{2}} \cdot x^{3}}=\tan \left(\arctan \frac{P(x)}{Q(x)}\right) .
$$

From here

$$
\arctan \frac{P(x)}{Q(x)}=\arctan \frac{3 x-3 x^{5}}{1-3 x^{2}-3 x^{4}+x^{6}},
$$

and hence $P(x)=3 x-3 x^{5}, Q(x)=1-3 x^{2}-3 x^{4}+x^{6}$. The final answer is

$$
\frac{1}{3} \arctan \frac{3 x-3 x^{5}}{1-3 x^{2}-3 x^{4}+x^{6}}+C .
$$
\\
\textbf{Topic} :Probability\\
\textbf{Book} :Putnam and Beyond\\
\textbf{Final Answer} :\\


\textbf{Problem Statement} :
453. Compute the integral

$$
\int_{-1}^{1} \frac{\sqrt[3]{x}}{\sqrt[3]{1-x}+\sqrt[3]{1+x}} d x
$$
\\
\textbf{Solution} :
453. The function $f:[-1,1] \rightarrow \mathbb{R}$,

$$
f(x)=\frac{\sqrt[3]{x}}{\sqrt[3]{1-x}+\sqrt[3]{1+x}},
$$

is odd; therefore, the integral is zero.
\\
\textbf{Topic} :Probability\\
\textbf{Book} :Putnam and Beyond\\
\textbf{Final Answer} :\\


\textbf{Problem Statement} :
454. Compute

$$
\int_{0}^{\pi} \frac{x \sin x}{1+\sin ^{2} x} d x .
$$
\\
\textbf{Solution} :
454. We use the example from the introduction for the particular function $f(x)=\frac{x}{1+x^{2}}$ to transform the integral into

$$
\pi \int_{0}^{\frac{\pi}{2}} \frac{\sin x}{1+\sin ^{2} x} d x .
$$

This is the same as

$$
\pi \int_{0}^{\frac{\pi}{2}}-\frac{d(\cos x)}{2-\cos ^{2} x},
$$

which with the substitution $t=\cos x$ becomes

$$
\pi \int_{0}^{1} \frac{1}{2-t^{2}} d t=\left.\frac{\pi}{2 \sqrt{2}} \ln \frac{\sqrt{2}+t}{\sqrt{2}-t}\right|_{0} ^{1}=\frac{\pi}{2 \sqrt{2}} \ln \frac{\sqrt{2}+1}{\sqrt{2}-1} .
$$
\\
\textbf{Topic} :Probability\\
\textbf{Book} :Putnam and Beyond\\
\textbf{Final Answer} :\\


\textbf{Problem Statement} :
455. Let $a$ and $b$ be positive real numbers. Compute

$$
\int_{a}^{b} \frac{e^{\frac{x}{a}}-e^{\frac{b}{x}}}{x} d x .
$$
\\
\textbf{Solution} :
455. Denote the value of the integral by $I$. With the substitution $t=\frac{a b}{x}$ we have

$$
I=\int_{a}^{b} \frac{e^{\frac{b}{t}}-e^{\frac{t}{a}}}{\frac{a b}{t}} \cdot \frac{-a b}{t^{2}} d t=-\int_{a}^{b} \frac{e^{\frac{t}{a}}-e^{\frac{b}{t}}}{t} d t=-I .
$$

Hence $I=0$.
\\
\textbf{Topic} :Probability\\
\textbf{Book} :Putnam and Beyond\\
\textbf{Final Answer} :\\


\textbf{Problem Statement} :
456. Compute the integral

$$
I=\int_{0}^{1} \sqrt[3]{2 x^{3}-3 x^{2}-x+1} d x
$$
\\
\textbf{Solution} :
456. The substitution $t=1-x$ yields

$$
I=\int_{0}^{1} \sqrt[3]{2(1-t)^{3}-3(1-t)^{2}-(1-t)+1} d t=-\int_{0}^{1} \sqrt[3]{2 t^{3}-3 t^{2}-t+1} d t=-I .
$$

Hence $I=0$.

(Mathematical Reflections, proposed by T. Andreescu)
\\
\textbf{Topic} :Probability\\
\textbf{Book} :Putnam and Beyond\\
\textbf{Final Answer} :\\


\textbf{Problem Statement} :
457. Compute the integral

$$
\int_{0}^{a} \frac{d x}{x+\sqrt{a^{2}-x^{2}}} \quad(a>0) .
$$
\\
\textbf{Solution} :
457. Using the substitutions $x=a \sin t$, respectively, $x=a \cos t$, we find the integral to be equal to both the integral

$$
L_{1}=\int_{0}^{\pi / 2} \frac{\sin t}{\sin t+\cos t} d t
$$

and the integral

$$
L_{2}=\int_{0}^{\pi / 2} \frac{\cos t}{\sin t+\cos t} d t .
$$

Hence the desired integral is equal to

$$
\frac{1}{2}\left(L_{1}+L_{2}\right)=\frac{1}{2} \int_{0}^{\frac{\pi}{2}} 1 d t=\frac{\pi}{4} .
$$
\\
\textbf{Topic} :Probability\\
\textbf{Book} :Putnam and Beyond\\
\textbf{Final Answer} :\\


\textbf{Problem Statement} :
458. Compute the integral

$$
\int_{0}^{\frac{\pi}{4}} \ln (1+\tan x) d x
$$
\\
\textbf{Solution} :
458. Denote the integral by $I$. With the substitution $t=\frac{\pi}{4}-x$ the integral becomes

$$
\begin{aligned}
I &=\int_{\frac{\pi}{4}}^{0} \ln \left(1+\tan \left(\frac{\pi}{4}-t\right)\right)(-1) d t=\int_{0}^{\frac{\pi}{4}} \ln \left(1+\frac{1-\tan t}{1+\tan t}\right) d t \\
&=\int_{0}^{\frac{\pi}{4}} \ln \frac{2}{1+\tan t} d t=\frac{\pi}{4} \ln 2-I .
\end{aligned}
$$

Solving for $I$, we obtain $I=\frac{\pi}{8} \ln 2$.
\\
\textbf{Topic} :Probability\\
\textbf{Book} :Putnam and Beyond\\
\textbf{Final Answer} :\\


\textbf{Problem Statement} :
459. Find

$$
\int_{0}^{1} \frac{\ln (1+x)}{1+x^{2}} d x
$$
\\
\textbf{Solution} :
459. With the substitution $\arctan x=t$ the integral takes the form

$$
I=\int_{0}^{\frac{\pi}{4}} \ln (1+\tan t) d t .
$$

This we already computed in the previous problem. ("Happiness is longing for repetition," says M. Kundera.) So the answer to the problem is $\frac{\pi}{8} \ln 2$.

(66th W.L. Putnam Mathematical Competition, 2005, proposed by T. Andreescu)
\\
\textbf{Topic} :Probability\\
\textbf{Book} :Putnam and Beyond\\
\textbf{Final Answer} :\\


\textbf{Problem Statement} :
461. Compute the integral

$$
\int_{0}^{\frac{\pi}{2}} \frac{x \cos x-\sin x}{x^{2}+\sin ^{2} x} d x
$$
\\
\textbf{Solution} :
461. The statement is misleading. There is nothing special about the limits of integration! The indefinite integral can be computed as follows:

$$
\begin{aligned}
\int \frac{x \cos x-\sin x}{x^{2}+\sin ^{2} x} d x &=\int \frac{\frac{\cos x}{x}-\frac{\sin x}{x^{2}}}{1+\left(\frac{\sin x}{x}\right)^{2}} d x=\int \frac{1}{1+\left(\frac{\sin x}{x}\right)^{2}}\left(\frac{\sin x}{x}\right)^{\prime} d x \\
&=\arctan \left(\frac{\sin x}{x}\right)+C .
\end{aligned}
$$

Therefore,

$$
\int_{0}^{\frac{\pi}{2}} \frac{x \cos x-\sin x}{x^{2}+\sin ^{2} x} d x=\arctan \frac{2}{\pi}-\frac{\pi}{4} .
$$

(Z. Ahmed)
\\
\textbf{Topic} :Probability\\
\textbf{Book} :Putnam and Beyond\\
\textbf{Final Answer} :\\


\textbf{Problem Statement} :
462. Let $\alpha$ be a real number. Compute the integral

$$
I(\alpha)=\int_{-1}^{1} \frac{\sin \alpha d x}{1-2 x \cos \alpha+x^{2}} .
$$
\\
\textbf{Solution} :
462. If $\alpha$ is a multiple of $\pi$, then $I(\alpha)=0$. Otherwise, use the substitution $x=$ $\cos \alpha+t \sin \alpha$. The indefinite integral becomes

$$
\int \frac{\sin \alpha d x}{1-2 x \cos \alpha+x^{2}}=\int \frac{d t}{1+t^{2}}=\arctan t+C .
$$

It follows that the definite integral $I(\alpha)$ has the value 

$$
\arctan \left(\frac{1-\cos \alpha}{\sin \alpha}\right)-\arctan \left(\frac{-1-\cos \alpha}{\sin \alpha}\right)
$$

where the angles are to be taken between $-\frac{\pi}{2}$ and $\frac{\pi}{2}$. But

$$
\frac{1-\cos \alpha}{\sin \alpha} \times \frac{-1-\cos \alpha}{\sin \alpha}=-1 \text {. }
$$

Hence the difference between these angles is $\pm \frac{\pi}{2}$. Notice that the sign of the integral is the same as the sign of $\alpha$. Hence $I(\alpha)=\frac{\pi}{2}$ if $\alpha \in(2 k \pi,(2 k+1) \pi)$ and $-\frac{\pi}{2}$ if $\alpha \in((2 k+1) \pi,(2 k+2) \pi)$ for some integer $k$.

Remark. This is an example of an integral with parameter that does not depend continuously on the parameter.

(E. Goursat, A Course in Mathematical Analysis, Dover, NY, 1904)
\\
\textbf{Topic} :Probability\\
\textbf{Book} :Putnam and Beyond\\
\textbf{Final Answer} :\\


\textbf{Problem Statement} :
467. Compute

$$
\int_{-\pi}^{\pi} \frac{\sin n x}{\left(1+2^{x}\right) \sin x} d x, \quad n \geq 0 .
$$
\textbf{Solution} :
467. Denote the integral from the statement by $I_{n}, n \geq 0$. We have

$$
I_{n}=\int_{-\pi}^{0} \frac{\sin n x}{\left(1+2^{x}\right) \sin x} d x+\int_{0}^{\pi} \frac{\sin n x}{\left(1+2^{x}\right) \sin x} d x .
$$

In the first integral change $x$ to $-x$ to further obtain

$$
\begin{aligned}
I_{n} &=\int_{0}^{\pi} \frac{\sin n x}{\left(1+2^{-x}\right) \sin x} d x+\int_{0}^{\pi} \frac{\sin n x}{\left(1+2^{x}\right) \sin x} d x \\
&=\int_{0}^{\pi} \frac{2^{x} \sin n x}{\left(1+2^{x}\right) \sin x} d x+\int_{0}^{\pi} \frac{\sin n x}{\left(1+2^{x}\right) \sin x} d x \\
&=\int_{0}^{\pi} \frac{\left(1+2^{x}\right) \sin n x}{\left(1+2^{x}\right) \sin x} d x=\int_{0}^{\pi} \frac{\sin n x}{\sin x} d x .
\end{aligned}
$$

And these integrals can be computed recursively. Indeed, for $n \geq 0$ we have

$$
I_{n+2}-I_{n}=\int_{0}^{\pi} \frac{\sin (n+2) x-\sin n x}{\sin x} d x=2 \int_{0}^{\pi} \cos (n-1) x d x=0,
$$

a very simple recurrence. Hence for $n$ even, $I_{n}=I_{0}=0$, and for $n$ odd, $I_{n}=I_{1}=\pi$.

(3rd International Mathematics Competition for University Students, 1996) 
\\
\textbf{Topic} :Probability\\
\textbf{Book} :Putnam and Beyond\\
\textbf{Final Answer} :\\


\textbf{Problem Statement} :
468. Compute

$$
\lim _{n \rightarrow \infty}\left[\frac{1}{\sqrt{4 n^{2}-1^{2}}}+\frac{1}{\sqrt{4 n^{2}-2^{2}}}+\cdots+\frac{1}{\sqrt{4 n^{2}-n^{2}}}\right] .
$$
\\
\textbf{Solution} :
468. We have

$$
\begin{aligned}
s_{n} &=\frac{1}{\sqrt{4 n^{2}-1^{2}}}+\frac{1}{\sqrt{4 n^{2}-2^{2}}}+\cdots+\frac{1}{\sqrt{4 n^{2}-n^{2}}} \\
&=\frac{1}{n}\left[\frac{1}{\sqrt{4-\left(\frac{1}{n}\right)^{2}}}+\frac{1}{\sqrt{4-\left(\frac{2}{n}\right)^{2}}}+\cdots+\frac{1}{\sqrt{4-\left(\frac{n}{n}\right)^{2}}}\right] .
\end{aligned}
$$

Hence $s_{n}$ is the Riemann sum of the function $f:[0,1] \rightarrow \mathbb{R}, f(x)=\frac{1}{\sqrt{4-x^{2}}}$ associated to the subdivision $x_{0}=0<x_{1}=\frac{1}{n}<x_{2}=\frac{2}{n}<\cdots<x_{n}=\frac{n}{n}=1$, with the intermediate points $\xi_{i}=\frac{i}{n} \in\left[x_{i}, x_{i+1}\right]$. The answer to the problem is therefore

$$
\lim _{n \rightarrow \infty} s_{n}=\int_{0}^{1} \frac{1}{\sqrt{4-x^{2}}} d x=\left.\arcsin \frac{x}{2}\right|_{0} ^{1}=\frac{\pi}{6} .
$$
\\
\textbf{Topic} :Probability\\
\textbf{Book} :Putnam and Beyond\\
\textbf{Final Answer} :\\


\textbf{Problem Statement} :
470. Compute

$$
\lim _{n \rightarrow \infty}\left(\frac{2^{1 / n}}{n+1}+\frac{2^{2 / n}}{n+\frac{1}{2}}+\cdots+\frac{2^{n / n}}{n+\frac{1}{n}}\right) .
$$
\\
\textbf{Solution} :
470. We would like to recognize the general term of the sequence as being a Riemann sum. This, however, does not seem to happen, since we can only write

$$
\sum_{i=1}^{n} \frac{2^{i / n}}{n+\frac{1}{i}}=\frac{1}{n} \sum_{i=1}^{n} \frac{2^{i / n}}{1+\frac{1}{n i}} .
$$

But for $i \geq 2$,

$$
2^{i / n}>\frac{2^{i / n}}{1+\frac{1}{n i}},
$$

and, using the inequality $e^{x}>1+x$,

$$
\frac{2^{i / n}}{1+\frac{1}{n i}}=2^{(i-1) / n} \frac{2^{1 / n}}{1+\frac{1}{n i}}=2^{(i-1) / n} \frac{e^{\ln 2 / n}}{1+\frac{1}{n i}}>2^{(i-1) / n} \frac{1+\frac{\ln 2}{n}}{1+\frac{1}{n i}}>2^{(i-1) / n},
$$

for $i \geq 2$. By the intermediate value property, for each $i \geq 2$ there exists $\xi_{i} \in\left[\frac{i-1}{n}, \frac{i}{n}\right]$ such that

$$
\frac{2^{i / n}}{1+\frac{1}{n i}}=2^{\xi_{i}} .
$$

Of course, the term corresponding to $i=1$ can be neglected when $n$ is large. Now we see that our limit is indeed the Riemann sum of the function $2^{x}$ integrated over the interval $[0,1]$. We obtain

$$
\lim _{n \rightarrow \infty}\left(\frac{2^{1 / n}}{n+1}+\frac{2^{2 / n}}{n+\frac{1}{2}}+\cdots+\frac{2^{n / n}}{n+\frac{1}{n}}\right)=\int_{0}^{1} 2^{x} d x=\frac{1}{\ln 2}
$$

(Soviet Union University Student Mathematical Olympiad, 1976)
\\
\textbf{Topic} :Probability\\
\textbf{Book} :Putnam and Beyond\\
\textbf{Final Answer} :\\


\textbf{Problem Statement} :
471. Compute the integral

$$
\int_{0}^{\pi} \ln \left(1-2 a \cos x+a^{2}\right) d x .
$$
\\
\textbf{Solution} :
471. This is an example of an integral that is determined using Riemann sums. Divide the interval $[0, \pi]$ into $n$ equal parts and consider the Riemann sum

$$
\begin{array}{r}
\frac{\pi}{n}\left[\ln \left(a^{2}-2 a \cos \frac{\pi}{n}+1\right)+\ln \left(a^{2}-2 a \cos \frac{2 \pi}{n}+1\right)+\cdots\right. \\
\left.+\ln \left(a^{2}-2 a \cos \frac{(n-1) \pi}{n}+1\right)\right] .
\end{array}
$$

This expression can be written as

$$
\frac{\pi}{n} \ln \left(a^{2}-2 a \cos \frac{\pi}{n}+1\right)\left(a^{2}-2 a \cos \frac{2 \pi}{n}+1\right) \ldots\left(a^{2}-2 a \cos \frac{(n-1) \pi}{n}+1\right) .
$$

The product inside the natural logarithm factors as

$$
\prod_{k=1}^{n-1}\left[a-\left(\cos \frac{k \pi}{n}+i \sin \frac{k \pi}{n}\right)\right]\left[a-\left(\cos \frac{k \pi}{n}-i \sin \frac{k \pi}{n}\right)\right] .
$$

These are exactly the factors in $a^{2 n}-1$, except for $a-1$ and $a+1$. The Riemann sum is therefore equal to

$$
\frac{\pi}{n} \ln \frac{a^{2 n}-1}{a^{2}-1} .
$$

We are left to compute the limit of this expression as $n$ goes to infinity. If $a \leq 1$, this limit is equal to 0 . If $a>1$, the limit is

$$
\lim _{n \rightarrow \infty} \pi \ln \sqrt[n]{\frac{a^{2 n}-1}{a^{2}-1}}=2 \pi \ln a .
$$

(S.D. Poisson)
\\
\textbf{Topic} :Probability\\
\textbf{Book} :Putnam and Beyond\\
\textbf{Final Answer} :\\


\textbf{Problem Statement} :
472. Find all continuous functions $f: \mathbb{R} \rightarrow[1, \infty)$ for which there exist $a \in \mathbb{R}$ and $k$ a positive integer such that

$$
f(x) f(2 x) \cdots f(n x) \leq a n^{k},
$$

for every real number $x$ and positive integer $n$. 
\textbf{Solution} :
472. The condition $f(x) f(2 x) \cdots f(n x) \leq a n^{k}$ can be written equivalently as

$$
\sum_{j=1}^{n} \ln f(j x) \leq \ln a+k \ln n, \quad \text { for all } x \in \mathbb{R}, n \geq 1 .
$$

Taking $\alpha>0$ and $x=\frac{\alpha}{n}$, we obtain

$$
\sum_{j=1}^{n} \ln f\left(\frac{j \alpha}{n}\right) \leq \ln a+k \ln n
$$

or

$$
\sum_{j=1}^{n} \frac{\alpha}{n} \ln f\left(\frac{j \alpha}{n}\right) \leq \frac{\alpha \ln a+k \alpha \ln n}{n} .
$$

The left-hand side is a Riemann sum for the function $\ln f$ on the interval $[0, \alpha]$. Because $f$ is continuous, so is $\ln f$, and thus $\ln f$ is integrable. Letting $n$ tend to infinity, we obtain

$$
\int_{0}^{1} \ln f(x) d x \leq \lim _{n \rightarrow \infty} \frac{\alpha \ln a+k \alpha \ln n}{n}=0 .
$$

The fact that $f(x) \geq 1$ implies that $\ln f(x) \geq 0$ for all $x$. Hence $\ln f(x)=0$ for all $x \in[0, \alpha]$. Since $\alpha$ is an arbitrary positive number, $f(x)=1$ for all $x \geq 0$. A similar argument yields $f(x)=1$ for $x<0$. So there is only one such function, the constant function equal to 1 .

(Romanian Mathematical Olympiad, 1999, proposed by R. Gologan)
\\
\textbf{Topic} :Probability\\
\textbf{Book} :Putnam and Beyond\\
\textbf{Final Answer} :\\


\textbf{Problem Statement} :
473. Determine the continuous functions $f:[0,1] \rightarrow \mathbb{R}$ that satisfy

$$
\int_{0}^{1} f(x)(x-f(x)) d x=\frac{1}{12} \text {. }
$$
\\
\textbf{Solution} :
473. The relation from the statement can be rewritten as

$$
\int_{0}^{1}\left(x f(x)-f^{2}(x)\right) d x=\int_{0}^{1} \frac{x^{2}}{4} d x .
$$

Moving everything to one side, we obtain

$$
\int_{0}^{1}\left(f^{2}(x)-x f(x)+\frac{x^{2}}{4}\right) d x=0 .
$$

We now recognize a perfect square and write this as

$$
\int_{0}^{1}\left(f(x)-\frac{x}{2}\right)^{2} d x=0 .
$$

The integral of the nonnegative continuous function $\left(f(x)-\frac{x}{2}\right)^{2}$ is strictly positive, unless the function is identically equal to zero. It follows that the only function satisfying the condition from the statement is $f(x)=\frac{x}{2}, x \in[0,1]$.

(Revista de Matematica din Timişoara (Timişoara Mathematics Gazette), proposed by T. Andreescu)
\\
\textbf{Topic} :Probability\\
\textbf{Book} :Putnam and Beyond\\
\textbf{Final Answer} :\\


\textbf{Problem Statement} :
474. Let $n$ be an odd integer greater than 1. Determine all continuous functions $f$ : $[0,1] \rightarrow \mathbb{R}$ such that

$$
\int_{0}^{1}\left(f\left(x^{\frac{1}{k}}\right)\right)^{n-k} d x=\frac{k}{n}, \quad k=1,2, \ldots, n-1 .
$$
\\
\textbf{Solution} :
474. Performing the substitution $x^{\frac{1}{k}}=t$, the given conditions become

$$
\int_{0}^{1}(f(t))^{n-k} t^{k-1} d t=\frac{1}{n}, \quad k=1,2, \ldots, n-1 .
$$

Observe that this equality also holds for $k=n$. With this in mind we write

$$
\begin{aligned}
\int_{0}^{1}(f(t)-t)^{n-1} d t &=\int_{0}^{1} \sum_{k=0}^{n-1}\left(\begin{array}{c}
n-1 \\
k
\end{array}\right)(-1)^{k}(f(t))^{n-1-k} t^{k} d t \\
&=\int_{0}^{1} \sum_{k=1}^{n}\left(\begin{array}{c}
n-1 \\
k-1
\end{array}\right)(-1)^{k-1}(f(t))^{n-k} t^{k-1} d t \\
&=\sum_{k=1}^{n}(-1)^{k-1}\left(\begin{array}{c}
n-1 \\
k-1
\end{array}\right) \int_{0}^{1}(f(t))^{n-k} t^{k-1} d t \\
&=\sum_{k=1}^{n}(-1)^{k-1}\left(\begin{array}{c}
n-1 \\
k-1
\end{array}\right) \frac{1}{n}=\frac{1}{n}(1-1)^{n-1}=0 .
\end{aligned}
$$

Because $n-1$ is even, $(f(t)-t)^{n-1} \geq 0$. The integral of this function can be zero only if $f(t)-t=0$ for all $t \in[0,1]$. Hence the only solution to the problem is $f:[0,1] \rightarrow \mathbb{R}$, $f(x)=x$.

(Romanian Mathematical Olympiad, 2002, proposed by T. Andreescu)
\\
\textbf{Topic} :Probability\\
\textbf{Book} :Putnam and Beyond\\
\textbf{Final Answer} :\\


\textbf{Problem Statement} :
476. For each continuous function $f:[0,1] \rightarrow \mathbb{R}$, we define $I(f)=\int_{0}^{1} x^{2} f(x) d x$ and $J(f)=\int_{0}^{1} x(f(x))^{2} d x$. Find the maximum value of $I(f)-J(f)$ over all such functions $f$.
\\
\textbf{Solution} :
476. We change this into a minimum problem, and then relate the latter to an inequality of the form $x \geq 0$. Completing the square, we see that

$$
\left.x(f(x))^{2}-x^{2} f(x)=\sqrt{x} f(x)\right)^{2}-2 \sqrt{x} f(x) \frac{x^{\frac{3}{2}}}{2}=\left(\sqrt{x} f(x)-\frac{x^{\frac{3}{2}}}{2}\right)^{2}-\frac{x^{3}}{4}
$$

Hence, indeed,

$$
J(f)-I(f)=\int_{0}^{1}\left(\sqrt{x} f(x)-\frac{x^{\frac{3}{2}}}{2}\right)^{2} d x-\int_{0}^{1} \frac{x^{3}}{4} d x \geq-\frac{1}{16}
$$

It follows that $I(f)-J(f) \leq \frac{1}{16}$ for all $f$. The equality holds, for example, for $f:[0,1] \rightarrow \mathbb{R}, f(x)=\frac{x}{2}$. We conclude that

$$
\max _{f \in \mathcal{C}^{0}([0,1])}(I(f)-J(f))=\frac{1}{16} .
$$

(49th W.L. Putnam Mathematical Competition, 2006, proposed by T. Andreescu) 
\\
\textbf{Topic} :Probability\\
\textbf{Book} :Putnam and Beyond\\
\textbf{Final Answer} :\\


\textbf{Problem Statement} :
479. Find the maximal value of the ratio

$$
\left(\int_{0}^{3} f(x) d x\right)^{3} / \int_{0}^{3} f^{3}(x) d x,
$$

as $f$ ranges over all positive continuous functions on $[0,1]$. 
\\
\textbf{Solution} :
479. By Hölder's inequality,

$$
\int_{0}^{3} f(x) \cdot 1 d x \leq\left(\int_{0}^{3}|f(x)|^{3} d x\right)^{\frac{1}{3}}\left(\int_{0}^{3} 1^{\frac{3}{2}} d x\right)^{\frac{2}{3}}=3^{\frac{2}{3}}\left(\int_{0}^{3}|f(x)|^{3} d x\right)^{\frac{1}{3}} .
$$

Raising everything to the third power, we obtain

$$
\left(\int_{0}^{3} f(x) d x\right)^{3} / \int_{0}^{3} f^{3}(x) d x \leq 9 .
$$

To see that the maximum 9 can be achieved, choose $f$ to be constant.
\\
\textbf{Topic} :Probability\\
\textbf{Book} :Putnam and Beyond\\
\textbf{Final Answer} :\\


\textbf{Problem Statement} :
487. Compute the ratio

$$
\frac{1+\frac{\pi^{4}}{5 !}+\frac{\pi^{8}}{9 !}+\frac{\pi^{12}}{13 !}+\cdots}{\frac{1}{3 !}+\frac{\pi^{4}}{7 !}+\frac{\pi^{8}}{11 !}+\frac{\pi^{12}}{15 !}+\cdots} .
$$
\\
\textbf{Solution} :
487. Denote by $p$ the numerator and by $q$ the denominator of this fraction. Recall the Taylor series expansion of the sine function,

$$
\sin x=\frac{x}{1 !}-\frac{x^{3}}{3 !}+\frac{x^{5}}{5 !}-\frac{x^{7}}{7 !}+\frac{x^{9}}{9 !}+\cdots .
$$

We recognize the denominators of these fractions inside the expression that we are computing, and now it is not hard to see that $p \pi-q \pi^{3}=\sin \pi=0$. Hence $p \pi=q \pi^{3}$, and the value of the expression from the statement is $\pi^{2}$.

(Soviet Union University Student Mathematical Olympiad, 1975) 
\\
\textbf{Topic} :Probability\\
\textbf{Book} :Putnam and Beyond\\
\textbf{Final Answer} :\\


\textbf{Problem Statement} :
490. Compute to three decimal places

$$
\int_{0}^{1} \cos \sqrt{x} d x .
$$
\\
\textbf{Solution} :
490. The Taylor series expansion of $\cos \sqrt{x}$ around 0 is

$$
\cos \sqrt{x}=1-\frac{x}{2 !}+\frac{x^{2}}{4 !}-\frac{x^{3}}{6 !}+\frac{x^{4}}{8 !}-\cdots .
$$

Integrating term by term, we obtain

$$
\int_{0}^{1} \cos \sqrt{x} d x=\left.\sum_{n=1}^{\infty} \frac{(-1)^{n-1} x^{n}}{(n+1)(2 n) !}\right|_{0} ^{1}=\sum_{n=0}^{\infty} \frac{(-1)^{n-1}}{(n+1)(2 n) !} .
$$

Grouping consecutive terms we see that

$$
\begin{aligned}
\left(\frac{1}{5 \cdot 8 !}-\frac{1}{6 \cdot 10 !}\right)+\left(\frac{1}{7 \cdot 12 !}-\frac{1}{8 \cdot 14 !}\right)+\cdots &<\frac{1}{2 \cdot 10^{4}}+\frac{1}{2 \cdot 10^{5}}+\frac{1}{2 \cdot 10^{6}}+\cdots \\
&<\frac{1}{10^{4}} .
\end{aligned}
$$

Also, truncating to the fourth decimal place yields

$$
0.7638<1-\frac{1}{4}+\frac{1}{72}-\frac{1}{2880}<0.7639 .
$$

We conclude that

$$
\int_{0}^{1} \cos \sqrt{x} d x \approx 0.763 .
$$
\\
\textbf{Topic} :Probability\\
\textbf{Book} :Putnam and Beyond\\
\textbf{Final Answer} :\\


\textbf{Problem Statement} :
496. For a positive integer $n$ find the Fourier series of the function

$$
f(x)=\frac{\sin ^{2} n x}{\sin ^{2} x} .
$$
\\
\textbf{Solution} :
496. We will use only trigonometric considerations, and compute no integrals. A first remark is that the function is even, so only terms involving cosines will appear. Using Euler's formula

$$
e^{i \alpha}=\cos \alpha+i \sin \alpha
$$

we can transform the identity

$$
\sum_{k=1}^{n} e^{2 i k x}=\frac{e^{2 i(n+1) x}-1}{e^{2 i x}-1}
$$

into the corresponding identities for the real and imaginary parts:

$$
\begin{aligned}
&\cos 2 x+\cos 4 x+\cdots+\cos 2 n x=\frac{\sin n x \cos (n+1) x}{\sin x} \\
&\sin 2 x+\sin 4 x+\cdots+\sin 2 n x=\frac{\sin n x \sin (n+1) x}{\sin x}
\end{aligned}
$$

These two relate to our function as

$$
\frac{\sin ^{2} n x}{\sin ^{2} x}=\left(\frac{\sin n x \cos (n+1) x}{\sin x}\right)^{2}+\left(\frac{\sin n x \sin (n+1) x}{\sin x}\right)^{2},
$$

which allows us to write the function as an expression with no fractions:

$$
f(x)=(\cos 2 x+\cos 4 x+\cdots+\cos 2 n x)^{2}+(\sin 2 x+\sin 4 x+\cdots+\sin 2 n x)^{2} .
$$

Expanding the squares, we obtain

$$
\begin{aligned}
f(x) &=n+\sum_{1 \leq l<k \leq n}(2 \sin 2 l x \sin 2 k x+2 \cos 2 l x \cos 2 k x) \\
&=n+2 \sum_{1 \leq l<k \leq n} \cos 2(k-l) x=n+\sum_{m=1}^{n-1} 2(n-m) \cos 2 m x .
\end{aligned}
$$

In conclusion, the nonzero Fourier coefficients of $f$ are $a_{0}=n$ and $a_{2 m}=2(n-m)$, $m=1,2, \ldots, n-1$.

(D. Andrica)
\\
\textbf{Topic} :Probability\\
\textbf{Book} :Putnam and Beyond\\
\textbf{Final Answer} :\\


\textbf{Problem Statement} :
502. Find the global minimum of the function $f: \mathbb{R}^{2} \rightarrow \mathbb{R}$, 

$$
f(x, y)=x^{4}+6 x^{2} y^{2}+y^{4}-\frac{9}{4} x-\frac{7}{4} y .
$$
\\
\textbf{Solution} :
502. First, observe that if $|x|+|y| \rightarrow \infty$ then $f(x, y) \rightarrow \infty$, hence the function indeed has a global minimum. The critical points of $f$ are solutions to the system of equations

$$
\begin{aligned}
&\frac{\partial f}{\partial x}(x, y)=4 x^{3}+12 x y^{2}-\frac{9}{4}=0, \\
&\frac{\partial f}{\partial y}(x, y)=12 x^{2} y+4 y^{3}-\frac{7}{4}=0 .
\end{aligned}
$$

If we divide the two equations by 4 and then add, respectively, subtract them, we obtain $x^{3}+3 x^{2} y+3 x y^{2}+y^{3}-1=0$ and $x^{3}-3 x^{2} y+3 x y^{3}-y^{3}=\frac{1}{8}$. Recognizing the perfect cubes, we write these as $(x+y)^{3}=1$ and $(x-y)^{3}=\frac{1}{8}$, from which we obtain $x+y=1$ and $x-y=\frac{1}{2}$. We find a unique critical point $x=\frac{3}{4}, y=\frac{1}{4}$. The minimum of $f$ is attained at this point, and it is equal to $f\left(\frac{3}{4}, \frac{1}{4}\right)=-\frac{51}{32}$.

(R. Gelca)
\\
\textbf{Topic} :Probability\\
\textbf{Book} :Putnam and Beyond\\
\textbf{Final Answer} :\\


\textbf{Problem Statement} :
503. Find the equation of the smallest sphere that is tangent to both of the lines (i) $x=$ $t+1, y=2 t+4, z=-3 t+5$, and (ii) $x=4 t-12, y=-t+8, z=t+17$.
\\
\textbf{Solution} :
503. The diameter of the sphere is the segment that realizes the minimal distance between the lines. So if $P(t+1,2 t+4,-3 t+5)$ and $Q(4 s-12,-t+8, t+17)$, we have to minimize the function

$$
\begin{aligned}
|P Q|^{2} &=(s-4 t+13)^{2}+(2 s+t-4)^{2}+(-3 s-t-12)^{2} \\
&=14 s^{2}+2 s t+18 t^{2}+82 s-88 t+329 .
\end{aligned}
$$

To minimize this function we set its partial derivatives equal to zero:

$$
\begin{aligned}
&28 s+2 t+82=0, \\
&2 s+36 t-88=0 .
\end{aligned}
$$

This system has the solution $t=-782 / 251, s=657 / 251$. Substituting into the equation of the line, we deduce that the two endpoints of the diameter are $P\left(-\frac{531}{251},-\frac{560}{251}, \frac{3601}{251}\right)$ and $Q\left(-\frac{384}{251}, \frac{1351}{251}, \frac{4924}{251}\right)$. The center of the sphere is $\frac{1}{502}(-915,791,8252)$, and the radius $\frac{147}{\sqrt{1004}}$. The equation of the sphere is

$$
(502 x+915)^{2}+(502 y-791)^{2}+(502 z-8525)^{2}=251(147)^{2} .
$$

(20th W.L. Putnam Competition, 1959)
\\
\textbf{Topic} :Probability\\
\textbf{Book} :Putnam and Beyond\\
\textbf{Final Answer} :\\


\textbf{Problem Statement} :
504. Determine the maximum and the minimum of $\cos A+\cos B+\cos C$ when $A, B$, and $C$ are the angles of a triangle.
\\
\textbf{Solution} :
504. Writing $C=\pi-A-B$, the expression can be viewed as a function in the independent variables $A$ and $B$, namely,

$$
f(A, B)=\cos A+\cos B-\cos (A+B) .
$$

And because $A$ and $B$ are angles of a triangle, they are constrained to the domain $A, B>0$, $A+B<\pi$. We extend the function to the boundary of the domain, then study its extrema. The critical points satisfy the system of equations 

$$
\begin{aligned}
&\frac{\partial f}{\partial A}(A, B)=-\sin A+\sin (A+B)=0, \\
&\frac{\partial f}{\partial B}(A, B)=-\sin B+\sin (A+B)=0 .
\end{aligned}
$$

From here we obtain $\sin A=\sin B=\sin (A+B)$, which can happen only if $A=B=\frac{\pi}{3}$. This is the unique critical point, for which $f\left(\frac{\pi}{3}, \frac{\pi}{3}\right)=\frac{3}{2}$. On the boundary, if $A=0$ or $B=0$, then $f(A, B)=1$. Same if $A+B=\pi$. We conclude that the maximum of $\cos A+\cos B+\cos C$ is $\frac{3}{2}$, attained for the equilateral triangle, while the minimum is 1 , which is attained only for a degenerate triangle in which two vertices coincide.
\\
\textbf{Topic} :Probability\\
\textbf{Book} :Putnam and Beyond\\
\textbf{Final Answer} :\\


\textbf{Problem Statement} :
511. Of all triangles circumscribed about a given circle, find the one with the smallest area.
\\
\textbf{Solution} :
511. Without loss of generality, we may assume that the circle has radius 1 . If $a, b, c$ are the sides, and $S(a, b, c)$ the area, then (because of the formula $S=p r$, where $p$ is the semiperimeter) the constraint reads $S=\frac{a+b+c}{2}$. We will maximize the function $f(a, b, c)=S(a, b, c)^{2}$ with the constraint $g(a, b, c)=S(a, b, c)^{2}-\left(\frac{a+b+c}{2}\right)^{2}=0$. Using Hero's formula, we can write

$$
\begin{aligned}
f(a, b, c) &=\frac{a+b+c}{2} \cdot \frac{-a+b+c}{2} \cdot \frac{a-b+c}{2} \cdot \frac{a+b-c}{2} \\
&=\frac{-a^{4}-b^{4}-c^{4}+2\left(a^{2} b^{2}+b^{2} c^{2}+a^{2} c^{2}\right)}{16} .
\end{aligned}
$$

The method of Lagrange multipliers gives rise to the system of equations

$$
\begin{aligned}
(\lambda-1) \frac{-a^{3}+a\left(b^{2}+c^{2}\right)}{4} &=\frac{a+b+c}{2}, \\
(\lambda-1) \frac{-b^{3}+b\left(a^{2}+c^{2}\right)}{4} &=\frac{a+b+c}{2}, \\
(\lambda-1) \frac{-c^{3}+c\left(a^{2}+b^{2}\right)}{4} &=\frac{a+b+c}{2}, \\
g(a, b, c) &=0 .
\end{aligned}
$$

Because $a+b+c \neq 0, \lambda$ cannot be 1 , so this further gives

$$
-a^{3}+a\left(b^{2}+c^{2}\right)=-b^{3}+b\left(a^{2}+c^{2}\right)=-c^{3}+c\left(a^{2}+b^{2}\right) .
$$

The first equality can be written as $(b-a)\left(a^{2}+b^{2}-c^{2}\right)=0$. This can happen only if either $a=b$ or $a^{2}+b^{2}=c^{2}$, so either the triangle is isosceles, or it is right. Repeating this for all three pairs of sides we find that either $b=c$ or $b^{2}+c^{2}=a^{2}$, and also that either $a=c$ or $a^{2}+c^{2}=b^{2}$. Since at most one equality of the form $a^{2}+b^{2}=c^{2}$ can hold, we see that, in fact, all three sides must be equal. So the critical point given by the method of Lagrange multipliers is the equilateral triangle.

Is this the global minimum? We just need to observe that as the triangle degenerates, the area becomes infinite. So the answer is yes, the equilateral triangle minimizes the area.
\\
\textbf{Topic} :Probability\\
\textbf{Book} :Putnam and Beyond\\
\textbf{Final Answer} :\\


\textbf{Problem Statement} :
514. Compute the integral $\iint_{D} x d x d y$, where

$$
D=\left\{(x, y) \in \mathbb{R}^{2} \mid x \geq 0,1 \leq x y \leq 2,1 \leq \frac{y}{x} \leq 2\right\} .
$$
\\
\textbf{Solution} :
514. The domain is bounded by the hyperbolas $x y=1, x y=2$ and the lines $y=x$ and $y=2 x$. This domain can mapped into a rectangle by the transformation

$$
T: \quad u=x y, \quad v=\frac{y}{x} .
$$

Thus it is natural to consider the change of coordinates

$$
T^{-1}: \quad x=\sqrt{\frac{u}{v}}, \quad y=\sqrt{u v} .
$$

The domain becomes the rectangle $D^{*}=\left\{(u, v) \in \mathbb{R}^{2} \mid 1 \leq u \leq 2,1 \leq v \leq 2\right\}$. The Jacobian of $T^{-1}$ is $\frac{1}{2 v} \neq 0$. The integral becomes

$$
\int_{1}^{2} \int_{1}^{2} \sqrt{\frac{u}{v}} \frac{1}{2 v} d u d v=\frac{1}{2} \int_{1}^{2} u^{1 / 2} d u \int_{1}^{2} v^{-3 / 2} d v=\frac{1}{3}(5 \sqrt{2}-6) .
$$

(Gh. Bucur, E. Câmpu, S. Găină, Culegere de Probleme de Calcul Diferenţial şi Integral (Collection of Problems in Differential and Integral Calculus), Editura Tehnică, Bucharest, 1967)
\\
\textbf{Topic} :Probability\\
\textbf{Book} :Putnam and Beyond\\
\textbf{Final Answer} :\\


\textbf{Problem Statement} :
515. Find the integral of the function

$$
f(x, y, z)=\frac{x^{4}+2 y^{4}}{x^{4}+4 y^{4}+z^{4}}
$$

over the unit ball $B=\left\{(x, y, z) \mid x^{2}+y^{2}+z^{2} \leq 1\right\}$.
\\
\textbf{Solution} :
515. Denote the integral by $I$. The change of variable $(x, y, z) \rightarrow(z, y, x)$ transforms the integral into

$$
\iiint_{B} \frac{z^{4}+2 y^{4}}{x^{4}+4 y^{4}+z^{4}} d x d y d z .
$$

Hence

$$
\begin{aligned}
2 I &=\iiint_{B} \frac{x^{4}+2 y^{4}}{x^{4}+4 y^{4}+z^{4}} d x d y d z+\iiint_{B} \frac{2 y^{4}+z^{4}}{x^{4}+4 y^{4}+z^{4}} d x d y d z \\
&=\iiint_{B} \frac{x^{4}+4 y^{4}+z^{4}}{x^{4}+4 y^{4}+z^{4}} d x d y d z=\frac{4 \pi}{3} .
\end{aligned}
$$

It follows that $I=\frac{2 \pi}{3}$.
\\
\textbf{Topic} :Probability\\
\textbf{Book} :Putnam and Beyond\\
\textbf{Final Answer} :\\


\textbf{Problem Statement} :
517. Compute the integral

$$
I=\iint_{D}|x y| d x d y,
$$

where

$$
D=\left\{(x, y) \in \mathbb{R}^{2} \mid x \geq 0, \quad\left(\frac{x^{2}}{a^{2}}+\frac{y^{2}}{b^{2}}\right)^{2} \leq \frac{x^{2}}{a^{2}}-\frac{y^{2}}{b^{2}}\right\}, \quad a, b>0 .
$$
\\
\textbf{Solution} :
517. In the equation of the curve that bounds the domain

$$
\left(\frac{x^{2}}{a^{2}}+\frac{y^{2}}{b^{2}}\right)^{2}=\frac{x^{2}}{a^{2}}-\frac{y^{2}}{b^{2}},
$$

the expression on the left suggests the use of generalized polar coordinates, which are suited for elliptical domains. And indeed, if we set $x=a r \cos \theta$ and $y=b r \sin \theta$, the equation of the curve becomes $r^{4}=r^{2} \cos 2 \theta$, or $r=\sqrt{\cos 2 \theta}$. The condition $x \geq 0$ becomes $-\frac{\pi}{2} \leq \theta \leq \frac{\pi}{2}$, and because $\cos 2 \theta$ should be positive we should further have $-\frac{\pi}{4} \leq \theta \leq \frac{\pi}{4}$. Hence the domain of integration is

$$
\left\{(r, \theta) ; \quad 0 \leq r \leq \sqrt{\cos 2 \theta},-\frac{\pi}{4} \leq \theta \leq \frac{\pi}{4}\right\} .
$$

The Jacobian of the transformation is $J=a b r$. Applying the formula for the change of variables, the integral becomes

$$
\int_{-\frac{\pi}{4}}^{\frac{\pi}{4}} \int_{0}^{\sqrt{\cos 2 \theta}} a^{2} b^{2} r^{3} \cos \theta|\sin \theta| d r d \theta=\frac{a^{2} b^{2}}{4} \int_{0}^{\frac{\pi}{4}} \cos ^{2} 2 \theta \sin 2 \theta d \theta=\frac{a^{2} b^{2}}{24} .
$$

(Gh. Bucur, E. Câmpu, S. Găină, Culegere de Probleme de Calcul Diferenţial şi Integral (Collection of Problems in Differential and Integral Calculus), Editura Tehnică, Bucharest, 1967)
\\
\textbf{Topic} :Probability\\
\textbf{Book} :Putnam and Beyond\\
\textbf{Final Answer} :\\


\textbf{Problem Statement} :
519. Evaluate

$$
\int_{0}^{1} \int_{0}^{1} \int_{0}^{1}\left(1+u^{2}+v^{2}+w^{2}\right)^{-2} d u d v d w
$$
\\
\textbf{Solution} :
519. Call the integral $I$. By symmetry, we may compute it over the domain $\{(u, v, w) \in$ $\left.\mathbb{R}^{3} \mid 0 \leq v \leq u \leq 1\right\}$, then double the result. We substitute $u=r \cos \theta, v=r \sin \theta, w=$ $\tan \phi$, taking into account that the limits of integration become $0 \leq \theta, \phi \leq \frac{\pi}{4}$, and $0 \leq r \leq \sec \theta$. We have

$$
\begin{aligned}
I &=2 \int_{0}^{\frac{\pi}{4}} \int_{0}^{\frac{\pi}{4}} \int_{0}^{\sec \theta} \frac{r \sec ^{2} \phi}{\left(1+r^{2} \cos ^{2} \theta+r^{2} \sin ^{2} \theta+\tan ^{2} \phi\right)^{2}} d r d \theta d \phi \\
&=2 \int_{0}^{\frac{\pi}{4}} \int_{0}^{\frac{\pi}{4}} \int_{0}^{\sec \theta} \frac{r \sec ^{2} \phi}{\left(r^{2}+\sec ^{2} \phi\right)^{2}} d r d \theta d \phi \\
&=\left.2 \int_{0}^{\frac{\pi}{4}} \int_{0}^{\frac{\pi}{4}} \sec ^{2} \phi \frac{-1}{2\left(r^{2}+\sec ^{2} \phi\right)}\right|_{r=0} ^{r=\sec \theta} d \theta d \phi \\
&=-\int_{0}^{\frac{\pi}{4}} \int_{0}^{\frac{\pi}{4}} \frac{\sec ^{2} \phi}{\sec ^{2} \theta+\sec ^{2} \phi} d \theta d \phi+\left(\frac{\pi}{4}\right)^{2} .
\end{aligned}
$$

But notice that this is the same as

$$
\int_{0}^{\frac{\pi}{4}} \int_{0}^{\frac{\pi}{4}}\left(1-\frac{\sec ^{2} \phi}{\sec ^{2} \theta+\sec ^{2} \phi}\right) d \theta d \phi=\int_{0}^{\frac{\pi}{4}} \int_{0}^{\frac{\pi}{4}} \frac{\sec ^{2} \theta}{\sec ^{2} \theta+\sec ^{2} \phi} d \theta d \phi .
$$

If we exchange the roles of $\theta$ and $\phi$ in this last integral we see that

$$
-\int_{0}^{\frac{\pi}{4}} \int_{0}^{\frac{\pi}{4}} \frac{\sec ^{2} \phi}{\sec ^{2} \theta+\sec ^{2} \phi} d \theta d \phi+\left(\frac{\pi}{4}\right)^{2}=\int_{0}^{\frac{\pi}{4}} \int_{0}^{\frac{\pi}{4}} \frac{\sec ^{2} \phi}{\sec ^{2} \theta+\sec ^{2} \phi} d \theta d \phi .
$$

Hence

$$
\int_{0}^{\frac{\pi}{4}} \int_{0}^{\frac{\pi}{4}} \frac{\sec ^{2} \phi}{\sec ^{2} \theta+\sec ^{2} \phi} d \theta d \phi=\frac{\pi^{2}}{32} .
$$

Consequently, the integral we are computing is equal to $\frac{\pi^{2}}{32}$.

(American Mathematical Monthly, proposed by M. Hajja and P. Walker)
\\
\textbf{Topic} :Probability\\
\textbf{Book} :Putnam and Beyond\\
\textbf{Final Answer} :\\


\textbf{Problem Statement} :
525. Let $F(x)=\sum_{n=1}^{\infty} \frac{1}{x^{2}+n^{4}}, x \in \mathbb{R}$. Compute $\int_{0}^{\infty} F(t) d t$.
\textbf{Solution} :
525. We can apply Tonelli's theorem to the function $f(x, n)=\frac{1}{x^{2}+n^{4}}$. Integrating term by term, we obtain

$$
\int_{0}^{x} F(t) d t=\int_{0}^{x} \sum_{n=1}^{\infty} f(t, n) d t=\sum_{n=1}^{\infty} \int_{0}^{x} \frac{d t}{t^{2}+n^{4}}=\sum_{n=1}^{\infty} \frac{1}{n^{2}} \arctan \frac{x}{n^{2}} .
$$

This series is bounded from above by $\sum_{n=1}^{\infty} \frac{1}{n^{2}}=\frac{\pi^{2}}{6}$. Hence the summation commutes with the limit as $x$ tends to infinity. We have

$$
\int_{0}^{\infty} F(t) d t=\lim _{x \rightarrow \infty} \int_{0}^{x} F(t) d t=\lim _{x \rightarrow \infty} \sum_{n=1}^{\infty} \frac{1}{n^{2}} \arctan \frac{x}{n^{2}}=\sum_{n=1}^{\infty} \frac{1}{n^{2}} \cdot \frac{\pi}{2} .
$$

Using the identity $\sum_{n \geq 1} \frac{1}{n^{2}}=\frac{\pi^{2}}{6}$, we obtain

$$
\int_{0}^{\infty} F(t) d t=\frac{\pi^{3}}{12} .
$$

(Gh. Sireţchi, Calcul Diferenţial şi Integral (Differential and Integral Calculus), Editura Ştiinţifică şi Enciclopedică, Bucharest, 1985)
\\
\textbf{Topic} :Probability\\
\textbf{Book} :Putnam and Beyond\\
\textbf{Final Answer} :\\


\textbf{Problem Statement} :
527. Compute the flux of the vector field

$$
\vec{F}(x, y, z)=x\left(e^{x y}-e^{z x}\right) \vec{i}+y\left(e^{y z}-e^{x y}\right) \vec{j}+z\left(e^{z x}-e^{y z}\right) \vec{k}
$$

across the upper hemisphere of the unit sphere.
\\
\textbf{Solution} :
527. It can be checked that $\operatorname{div} \vec{F}=0$ (in fact, $\vec{F}$ is the curl of the vector field $e^{y z} \vec{i}+$ $e^{z x} \vec{j}+e^{x y} \vec{k}$ ). If $S$ be the union of the upper hemisphere and the unit disk in the $x y$-plane, then by the divergence theorem $\iint_{S} \vec{F} \cdot \vec{n} d S=0$. And on the unit disk $\vec{F} \cdot \vec{n}=0$, which means that the flux across the unit disk is zero. It follows that the flux across the upper hemisphere is zero as well.
\\
\textbf{Topic} :Probability\\
\textbf{Book} :Putnam and Beyond\\
\textbf{Final Answer} :\\


\textbf{Problem Statement} :
528. Compute

$$
\oint_{C} y^{2} d x+z^{2} d y+x^{2} d z,
$$

where $C$ is the Viviani curve, defined as the intersection of the sphere $x^{2}+y^{2}+z^{2}=$ $a^{2}$ with the cylinder $x^{2}+y^{2}=a x$.
\\
\textbf{Solution} :
528. We simplify the computation using Stokes' theorem:

$$
\oint_{C} y^{2} d x+z^{2} d y+x^{2} d z=-2 \iint_{S} y d x d y+z d y d z+x d z d x
$$

where $S$ is the portion of the sphere bounded by the Viviani curve. We have

$$
-2 \iint_{S} y d x d y+z d y d z+x d z d x=-2 \iint_{S}(z, x, y) \cdot \vec{n} d \sigma,
$$

where $(z, x, y)$ denotes the three-dimensional vector with coordinates $z, x$, and $y$, while $\vec{n}$ denotes the unit vector normal to the sphere at the point of coordinates $(x, y, z)$. We parametrize the portion of the sphere in question by the coordinates $(x, y)$, which range inside the circle $x^{2}+y^{2}-a x=0$. This circle is the projection of the Viviani curve onto the $x y$-plane.

The unit vector normal to the sphere is

$$
\vec{n}=\left(\frac{x}{a}, \frac{y}{a}, \frac{z}{a}\right)=\left(\frac{x}{a}, \frac{y}{a}, \frac{\sqrt{a^{2}-x^{2}-y^{2}}}{a}\right),
$$

while the area element is

$$
d \sigma=\frac{1}{\cos \alpha} d x d y,
$$

$\alpha$ being the angle formed by the normal to the sphere with the $x y$-plane. It is easy to see that $\cos \alpha=\frac{z}{a}=\frac{\sqrt{a^{2}-x^{2}-y^{2}}}{a}$. Hence the integral is equal to

$$
-2 \iint_{D}\left(z \frac{x}{a}+x \frac{y}{a}+y \frac{z}{a}\right) \frac{a}{z} d x d y=-2 \iint_{D}\left(x+y+\frac{x y}{\sqrt{a^{2}-x^{2}-y^{2}}}\right) d x d y
$$

the domain of integration $D$ being the disk $x^{2}+y^{2}-a x \leq 0$. Split the integral as

$$
-2 \iint_{D}(x+y) d x d y-2 \iint_{D} \frac{x y}{\sqrt{a^{2}-x^{2}-y^{2}}} d x d y .
$$

Because the domain of integration is symmetric with respect to the $y$-axis, the second double integral is zero. The first double integral can be computed using polar coordinates: $x=\frac{a}{2}+r \cos \theta, y=r \sin \theta, 0 \leq r \leq \frac{a}{2}, 0 \leq \theta \leq 2 \pi$. Its value is $-\frac{\pi a^{3}}{4}$, which is the answer to the problem.

(D. Flondor, N. Donciu, Algebră şi Analiză Matematică (Algebra and Mathematical Analysis), Editura Didactică şi Pedagogică, Bucharest, 1965)
\\
\textbf{Topic} :Probability\\
\textbf{Book} :Putnam and Beyond\\
\textbf{Final Answer} :\\


\textbf{Problem Statement} :
535. Find all functions $f: \mathbb{R} \rightarrow \mathbb{R}$ satisfying

$$
f\left(x^{2}-y^{2}\right)=(x-y)(f(x)+f(y)) .
$$
\\
\textbf{Solution} :
535. Plugging in $x=y$, we find that $f(0)=0$, and plugging in $x=-1, y=0$, we find that $f(1)=-f(-1)$. Also, plugging in $x=a, y=1$, and then $x=a, y=-1$, we obtain

$$
\begin{aligned}
&f\left(a^{2}-1\right)=(a-1)(f(a)+f(1)), \\
&f\left(a^{2}-1\right)=(a+1)(f(a)-f(1)) .
\end{aligned}
$$

Equating the right-hand sides and solving for $f(a)$ gives $f(a)=f(1) a$ for all $a$.

So any such function is linear. Conversely, a function of the form $f(x)=k x$ clearly satisfies the equation.

(Korean Mathematical Olympiad, 2000)
\\
\textbf{Topic} :Probability\\
\textbf{Book} :Putnam and Beyond\\
\textbf{Final Answer} :\\


\textbf{Problem Statement} :
536. Find all complex-valued functions of a complex variable satisfying

$$
f(z)+z f(1-z)=1+z, \quad \text { for all } z .
$$
\\
\textbf{Solution} :
536. Replace $z$ by $1-z$ to obtain

$$
f(1-z)+(1-z) f(z)=2-z .
$$

Combine this with $f(z)+z f(1-z)=1+z$, and eliminate $f(1-z)$ to obtain

$$
\left(1-z+z^{2}\right) f(z)=1-z+z^{2} .
$$

Hence $f(z)=1$ for all $z$ except maybe for $z=e^{\pm \pi i / 3}$, when $1-z+z^{2}=0$. For $\alpha=e^{i \pi / 3}, \bar{\alpha}=\alpha^{2}=1-\alpha$; hence $f(\alpha)+\alpha f(\bar{\alpha})=1+\alpha$. We therefore have only one constraint, namely $f(\bar{\alpha})=[1+\alpha-f(\alpha)] / \alpha=\bar{\alpha}+1-\bar{\alpha} f(\alpha)$. Hence the solution to the functional equation is of the form

$$
f(z)=1 \quad \text { for } z \neq e^{\pm i \pi / 3},
$$



$$
\begin{aligned}
f\left(e^{i \pi / 3}\right) &=\beta \\
f\left(e^{-i \pi / 3}\right) &=\bar{\alpha}+1-\bar{\alpha} \beta
\end{aligned}
$$

where $\beta$ is an arbitrary complex parameter.

(20th W.L. Putnam Competition, 1959)
\\
\textbf{Topic} :Probability\\
\textbf{Book} :Putnam and Beyond\\
\textbf{Final Answer} :\\


\textbf{Problem Statement} :
538. Find all functions $f: \mathbb{R} \rightarrow \mathbb{R}$ that satisfy the inequality

$$
f(x+y)+f(y+z)+f(z+x) \geq 3 f(x+2 y+3 z)
$$

for all $x, y, z \in \mathbb{R}$.
\\
\textbf{Solution} :
538. Plugging in $x=t, y=0, z=0$ gives

$$
f(t)+f(0)+f(t) \geq 3 f(t),
$$

or $f(0) \geq f(t)$ for all real numbers $t$. Plugging in $x=\frac{t}{2}, y=\frac{t}{2}, z=-\frac{t}{2}$ gives

$$
f(t)+f(0)+f(0) \geq 3 f(0),
$$

or $f(t) \geq f(0)$ for all real numbers $t$. Hence $f(t)=f(0)$ for all $t$, so $f$ must be constant. Conversely, any constant function $f$ clearly satisfies the given condition.

(Russian Mathematical Olympiad, 2000)
\\
\textbf{Topic} :Probability\\
\textbf{Book} :Putnam and Beyond\\
\textbf{Final Answer} :\\


\textbf{Problem Statement} :
540. Find all functions $f: \mathbb{R} \rightarrow \mathbb{R}$ satisfying

$$
f(x+y)=f(x) f(y)-c \sin x \sin y,
$$

for all real numbers $x$ and $y$, where $c$ is a constant greater than 1 .
\\
\textbf{Solution} :
540. The standard approach is to substitute particular values for $x$ and $y$. The solution found by the student S.P. Tungare does quite the opposite. It introduces an additional variable $z$. The solution proceeds as follows:

$$
\begin{aligned}
f(x+&y+z) \\
&=f(x) f(y+z)-c \sin x \sin (y+z) \\
&=f(x)[f(y) f(z)-c \sin y \sin z]-c \sin x \sin y \cos z-c \sin x \cos y \sin z \\
&=f(x) f(y) f(z)-c f(x) \sin y \sin z-c \sin x \sin y \cos z-c \sin x \cos y \sin z .
\end{aligned}
$$

Because obviously $f(x+y+z)=f(y+x+z)$, it follows that we must have

$$
\sin z[f(x) \sin y-f(y) \sin x]=\sin z[\cos x \sin y-\cos y \sin x] .
$$

Substitute $z=\frac{\pi}{2}$ to obtain

$$
f(x) \sin y-f(y) \sin x=\cos x \sin y-\cos y \sin x .
$$

For $x=\pi$ and $y$ not an integer multiple of $\pi$, we obtain $\sin y[f(\pi)+1]=0$, and hence $f(\pi)=-1$.

Then, substituting in the original equation $x=y=\frac{\pi}{2}$ yields

$$
f(\pi)=\left[f\left(\frac{\pi}{2}\right)\right]-c,
$$

whence $f\left(\frac{\pi}{2}\right)=\pm \sqrt{c-1}$. Substituting in the original equation $y=\pi$ we also obtain $f(x+\pi)=-f(x)$. We then have

$$
\begin{aligned}
-f(x) &=f(x+\pi)=f\left(x+\frac{\pi}{2}\right) f\left(\frac{\pi}{2}\right)-c \cos x \\
&=f\left(\frac{\pi}{2}\right)\left(f(x) f\left(\frac{\pi}{2}\right)-c \sin x\right)-c \cos x
\end{aligned}
$$

whence

$$
f(x)\left[\left(f\left(\frac{\pi}{2}\right)\right)^{2}-1\right]=c f\left(\frac{\pi}{2}\right) \sin x-c \cos x .
$$

It follows that $f(x)=f\left(\frac{\pi}{2}\right) \sin x+\cos x$. We find that the functional equation has two solutions, namely,

$$
f(x)=\sqrt{c-1} \sin x+\cos x \text { and } f(x)=-\sqrt{c-1} \sin x+\cos x .
$$

(Indian Team Selection Test for the International Mathematical Olympiad, 2004)
\\
\textbf{Topic} :Probability\\
\textbf{Book} :Putnam and Beyond\\
\textbf{Final Answer} :\\


\textbf{Problem Statement} :
546. Find all continuous functions $f: \mathbb{R} \rightarrow \mathbb{R}$ satisfying

$$
f(x+y)=f(x)+f(y)+f(x) f(y), \quad \text { for all } x, y \in \mathbb{R} .
$$
\\
\textbf{Solution} :
546. Adding 1 to both sides of the functional equation and factoring, we obtain

$$
f(x+y)+1=(f(x)+1)(f(y)+1) .
$$

The continuous function $g(x)=f(x)+1$ satisfies the functional equation $g(x+y)=$ $g(x) g(y)$, and we have seen in the previous problem that $g(x)=c^{x}$ for some nonnegative constant $c$. We conclude that $f(x)=c^{x}-1$ for all $x$.
\\
\textbf{Topic} :Probability\\
\textbf{Book} :Putnam and Beyond\\
\textbf{Final Answer} :\\


\textbf{Problem Statement} :
554. Find the functions $f, g: \mathbb{R} \rightarrow \mathbb{R}$ with continuous derivatives satisfying

$$
f^{2}+g^{2}=f^{\prime 2}+g^{\prime 2}, \quad f+g=g^{\prime}-f^{\prime},
$$

and such that the equation $f=g$ has two real solutions, the smaller of them being zero.
\\
\textbf{Solution} :
554. Rewrite the equation $f^{2}+g^{2}=f^{\prime 2}+g^{\prime 2}$ as

$$
(f+g)^{2}+(f-g)^{2}=\left(f^{\prime}+g^{\prime}\right)^{2}+\left(g^{\prime}-f^{\prime}\right)^{2} .
$$

This, combined with $f+g=g^{\prime}-f^{\prime}$, implies that $(f-g)^{2}=\left(f^{\prime}+g^{\prime}\right)^{2}$.

Let $x_{0}$ be the second root of the equation $f(x)=g(x)$. On the intervals $I_{1}=$ $(-\infty, 0), I_{2}=\left(0, x_{0}\right)$, and $I_{3}=\left(x_{0}, \infty\right)$ the function $f-g$ is nonzero; hence so is $f^{\prime}+g^{\prime}$. These two functions maintain constant sign on the three intervals; hence $f-g=\epsilon_{j}\left(f^{\prime}+g^{\prime}\right)$ on $I_{j}$, for some $\epsilon_{j} \in\{-1,1\}, j=1,2,3$. If on any of these intervals $f-g=f^{\prime}+g^{\prime}$, then since $f+g=g^{\prime}-f^{\prime}$ it follows that $f=g^{\prime}$ on that interval, and so $g^{\prime}+g=g^{\prime}-g^{\prime \prime}$. This implies that $g$ satisfies the equation $g^{\prime \prime}+g=0$, or that $g(x)=A \sin x+B \cos x$ on that interval. Also, $f(x)=g^{\prime}(x)=A \cos x-B \sin x$.

If $f-g=-f^{\prime}-g^{\prime}$ on some interval, then using again $f+g=g^{\prime}-f^{\prime}$, we find that $g=g^{\prime}$ on that interval. Hence $g(x)=C_{1} e^{x}$. From the fact that $f=-f^{\prime}$, we obtain $f(x)=C_{2} e^{-x}$.

Assuming that $f$ and $g$ are exponentials on the interval $\left(0, x_{0}\right)$, we deduce that $C_{1}=g(0)=f(0)=C_{2}$ and that $C_{1} e^{x_{0}}=g\left(x_{0}\right)=f\left(x_{0}\right)=C_{2} e^{-x}$. These two inequalities cannot hold simultaneously, unless $f$ and $g$ are identically zero, ruled out by the hypothesis of the problem. Therefore, $f(x)=A \cos x-B \sin x$ and $g(x)=$ $A \sin x+B \cos x$ on $\left(0, x_{0}\right)$, and consequently $x_{0}=\pi$.

On the intervals $(-\infty, 0]$ and $\left[x_{0}, \infty\right)$ the functions $f$ and $g$ cannot be periodic, since then the equation $f=g$ would have infinitely many solutions. So on these intervals the functions are exponentials. Imposing differentiability at 0 and $\pi$, we obtain $B=A$, $C_{1}=A$ on $I_{1}$ and $C_{1}=-A e^{-\pi}$ on $I_{3}$ and similarly $C_{2}=A$ on $I_{1}$ and $C_{2}=-A e^{\pi}$ on $I_{3}$. Hence the answer to the problem is

$$
\begin{aligned}
&f(x)= \begin{cases}A e^{-x} & \text { for } x \in(-\infty, 0], \\
A(\sin x+\cos x) & \text { for } x \in(0, \pi], \\
-A e^{-x+\pi} & \text { for } x \in(\pi, \infty),\end{cases} \\
&g(x)= \begin{cases}A e^{x} & \text { for } x \in(-\infty, 0], \\
A(\sin x-\cos x) & \text { for } x \in(0, \pi], \\
-A e^{x-\pi} & \text { for } x \in(\pi, \infty),\end{cases}
\end{aligned}
$$

where $A$ is some nonzero constant.

(Romanian Mathematical Olympiad, 1976, proposed by V. Matrosenco)
\\
\textbf{Topic} :Probability\\
\textbf{Book} :Putnam and Beyond\\
\textbf{Final Answer} :\\


\textbf{Problem Statement} :
556. Let $A, B, C, D, m, n$ be real numbers with $A D-B C \neq 0$. Solve the differential equation

$$
y\left(B+C x^{m} y^{n}\right) d x+x\left(A+D x^{m} y^{n}\right) d y=0 .
$$
\\
\textbf{Solution} :
556. The idea is to write the equation as

$$
B y d x+A x d y+x^{m} y^{n}(D y d x+C x d y)=0,
$$

then find an integrating factor that integrates simultaneously $B y d x+A x d y$ and $x^{m} y^{n}(D y d x+C x d y)$. An integrating factor of $B y d x+A x d y$ will be of the form $x^{-1} y^{-1} \phi_{1}\left(x^{B} y^{A}\right)$, while an integrating factor of $x^{m} y^{n}(D y d x+C x d y)=D x^{m} y^{n+1} d x+$ $C x^{m+1} y^{n} d y$ will be of the form $x^{-m-1} y^{-n-1} \phi_{2}\left(x^{D} y^{C}\right)$, where $\phi_{1}$ and $\phi_{2}$ are one-variable functions. To have the same integrating factor for both expressions, we should have

$$
x^{m} y^{n} \phi_{1}\left(x^{B} y^{A}\right)=\phi_{2}\left(x^{D} y^{C}\right) .
$$

It is natural to try power functions, say $\phi_{1}(t)=t^{p}$ and $\phi_{2}(t)=t^{q}$. The equality condition gives rise to the system

$$
\begin{aligned}
&A p-C q=-n, \\
&B p-D q=-m
\end{aligned}
$$

which according to the hypothesis can be solved for $p$ and $q$. We find that

$$
p=\frac{B n-A m}{A D-B C}, \quad q=\frac{D n-C m}{A D-B C} .
$$

Multiplying the equation by $x^{-1} y^{-1}\left(x^{B} y^{A}\right)^{p}=x^{-1-m} y^{-1-n}\left(x^{D} y^{C}\right)^{q}$ and integrating, we obtain

$$
\frac{1}{p+1}\left(x^{B} y^{A}\right)^{p+1}+\frac{1}{q+1}\left(x^{D} y^{C}\right)^{q+1}=\text { constant },
$$

which gives the solution in implicit form.

(M. Ghermănescu, Ecuaţii Diferenţiale (Differential Equations), Editura Didactică şi Pedagogică, Bucharest, 1963)
\\
\textbf{Topic} :Probability\\
\textbf{Book} :Putnam and Beyond\\
\textbf{Final Answer} :\\


\textbf{Problem Statement} :
560. Solve the differential equation

$$
x y^{\prime \prime}+2 y^{\prime}+x y=0 .
$$
\\
\textbf{Solution} :
560. The equation can be rewritten as

$$
(x y)^{\prime \prime}+(x y)=0 .
$$

Solving, we find $x y=C_{1} \sin x+C_{2} \cos x$, and hence

$$
y=C_{1} \frac{\sin x}{x}+C_{2} \frac{\cos x}{x}
$$

on intervals that do not contain 0 .
\\
\textbf{Topic} :Probability\\
\textbf{Book} :Putnam and Beyond\\
\textbf{Final Answer} :\\


\textbf{Problem Statement} :
561. Find all twice-differentiable functions defined on the entire real axis that satisfy $f^{\prime}(x) f^{\prime \prime}(x)=0$ for all $x$.
\\
\textbf{Solution} :
561. The function $f^{\prime}(x) f^{\prime \prime}(x)$ is the derivative of $\frac{1}{2}\left(f^{\prime}(x)\right)^{2}$. The equation is therefore equivalent to

$$
\left(f^{\prime}(x)\right)^{2}=\text { constant }
$$

And because $f^{\prime}(x)$ is continuous, $f^{\prime}(x)$ itself must be constant, which means that $f(x)$ is linear. Clearly, all linear functions are solutions. 
\\
\textbf{Topic} :Probability\\
\textbf{Book} :Putnam and Beyond\\
\textbf{Final Answer} :\\


\textbf{Problem Statement} :
562. Find all continuous functions $f: \mathbb{R} \rightarrow \mathbb{R}$ that satisfy

$$
f(x)+\int_{0}^{x}(x-t) f(t) d t=1, \quad \text { for all } x \in \mathbb{R} .
$$
\\
\textbf{Solution} :
562. The relation from the statement implies right away that $f$ is differentiable. Differentiating

$$
f(x)+x \int_{0}^{x} f(t) d t-\int_{0}^{x} t f(t) d t=1,
$$

we obtain

$$
f^{\prime}(x)+\int_{0}^{x} f(t) d t+x f(x)-x f(x)=0,
$$

that is, $f^{\prime}(x)+\int_{0}^{x} f(t) d t=0$. Again we conclude that $f$ is twice differentiable, and so we can transform this equality into the differential equation $f^{\prime \prime}+f=0$. The general solution is $f(x)=A \cos x+B \sin x$. Substituting in the relation from the statement, we obtain $A=1, B=0$, that is, $f(x)=\cos x$.

(E. Popa, Analiza Matematică, Culegere de Probleme (Mathematical Analysis, Collection of Problems), Editura GIL, 2005)
\\
\textbf{Topic} :Probability\\
\textbf{Book} :Putnam and Beyond\\
\textbf{Final Answer} :\\


\textbf{Problem Statement} :
563. Solve the differential equation

$$
(x-1) y^{\prime \prime}+(4 x-5) y^{\prime}+(4 x-6) y=x e^{-2 x} .
$$
\\
\textbf{Solution} :
563. The equation is of Laplace type, but we can bypass the standard method once we make the following observation. The associated homogeneous equation can be written as

$$
x\left(y^{\prime \prime}+4 y^{\prime}+4 y\right)-\left(y^{\prime \prime}+5 y^{\prime}+6 y\right)=0,
$$

and the equations $y^{\prime \prime}+4 y^{\prime}+4 y=0$ and $y^{\prime \prime}+5 y^{\prime}+6 y=0$ have the common solution $y(x)=e^{-2 x}$. This will therefore be a solution to the homogeneous equation, as well. To find a solution to the inhomogeneous equation, we use the method of variation of the constant. Set $y(x)=C(x) e^{-2 x}$. The equation becomes

$$
(x-1) C^{\prime \prime}-C^{\prime}=x \text {, }
$$

with the solution

$$
C^{\prime}(x)=\lambda(x-1)+(x-1) \ln |x-1|-1 .
$$

Integrating, we obtain

$$
C(x)=\frac{1}{2}(x-1)^{2} \ln |x-1|+\left(\frac{\lambda}{2}-\frac{1}{4}\right)(x-1)^{2}-x+C_{1} .
$$

If we set $c_{2}=\frac{\lambda}{2}-\frac{1}{4}$, then the general solution to the equation is

$$
y(x)=e^{-2 x}\left[C_{1}+C_{2}(x-1)^{2}+\frac{1}{2}(x-1)^{2} \ln |x-1|-x\right] .
$$

(D. Flondor, N. Donciu, Algebră şi Analiză Matematică (Algebra and Mathematical Analysis), Editura Didactică şi Pedagogică, Bucharest, 1965) 
\\
\textbf{Topic} :Probability\\
\textbf{Book} :Putnam and Beyond\\
\textbf{Final Answer} :\\


\textbf{Problem Statement} :
565. Find the one-to-one, twice-differentiable solutions $y$ to the equation

$$
\frac{d^{2} y}{d x^{2}}+\frac{d^{2} x}{d y^{2}}=0 .
$$
\\
\textbf{Solution} :
565. We interpret the differential equation as being posed for a function $y$ of $x$. In this perspective, we need to write $\frac{d^{2} x}{d y^{2}}$ in terms of the derivatives of $y$ with respect to $x$. We have

$$
\frac{d x}{d y}=\frac{1}{\frac{d y}{d x}},
$$

and using this fact and the chain rule yields

$$
\begin{aligned}
\frac{d^{2} x}{d y^{2}} &=\frac{d}{d y}\left(\frac{1}{\frac{d y}{d x}}\right)=\frac{d}{d x}\left(\frac{1}{\frac{d y}{d x}}\right) \cdot \frac{d x}{d y} \\
&=-\frac{1}{\left(\frac{d y}{d x}\right)^{2}} \cdot \frac{d^{2} y}{d x^{2}} \cdot \frac{d x}{d y}=-\frac{1}{\left(\frac{d y}{d x}\right)^{3}} \cdot \frac{d^{2} y}{d x^{2}} .
\end{aligned}
$$

The equation from the statement takes the form

$$
\frac{d^{2} y}{d x^{2}}\left(1-\frac{1}{\left(\frac{d y}{d x}\right)^{3}}\right)=0 .
$$

This splits into

$$
\frac{d^{2} y}{d x^{2}}=0 \quad \text { and } \quad\left(\frac{d y}{d x}\right)^{3}=1 .
$$

The first of these has the solutions $y=a x+b$, with $a \neq 0$, because $y$ has to be one-toone, while the second reduces to $y^{\prime}=1$, whose family of solutions $y=x+c$ is included in the first. Hence the answer to the problem consists of the nonconstant linear functions.

(M. Ghermănescu, Ecuaţii Diferenţiale (Differential Equations), Editura Didactică şi Pedagogică, Bucharest, 1963)
\\
\textbf{Topic} :Probability\\
\textbf{Book} :Putnam and Beyond\\
\textbf{Final Answer} :\\


\textbf{Problem Statement} :
570. Does there exist a continuously differentiable function $f: \mathbb{R} \rightarrow \mathbb{R}$ satisfying $f(x)>0$ and $f^{\prime}(x)=f(f(x))$ for every $x \in \mathbb{R}$ ? 
\\
\textbf{Solution} :
570. Assume that such a function exists. Because $f^{\prime}(x)=f(f(x))>0$, the function is strictly increasing.

The monotonicity and the positivity of $f$ imply that $f(f(x))>f(0)$ for all $x$. Thus $f(0)$ is a lower bound for $f^{\prime}(x)$. Integrating the inequality $f(0)<f^{\prime}(x)$ for $x<0$, we obtain

$$
f(x)<f(0)+f(0) x=(x+1) f(0) .
$$

But then for $x \leq-1$, we would have $f(x) \leq 0$, contradicting the hypothesis that $f(x)>0$ for all $x$. We conclude that such a function does not exist.

(9th International Mathematics Competition for University Students, 2002)
\\
\textbf{Topic} :Probability\\
\textbf{Book} :Putnam and Beyond\\
\textbf{Final Answer} :\\


\textbf{Problem Statement} :
575. Let $\vec{a}, \vec{b}, \vec{c}$ be vectors such that $\vec{b}$ and $\vec{c}$ are perpendicular, but $\vec{a}$ and $\vec{b}$ are not. Let $m$ be a real number. Solve the system

$$
\begin{aligned}
\vec{x} \cdot \vec{a} &=m, \\
\vec{x} \times \vec{b} &=\vec{c} .
\end{aligned}
$$
\\
\textbf{Solution} :
575. Multiply the second equation on the left by $\vec{a}$ to obtain

$$
\vec{a} \times(\vec{x} \times \vec{b})=\vec{a} \times \vec{c} .
$$

Using the formula for the double cross-product, also known as the cab-bac formula, we transform this into

$$
(\vec{a} \cdot \vec{b}) \vec{x}-(\vec{a} \cdot \vec{x}) \vec{b}=\vec{a} \times \vec{c}
$$

Hence the solution to the equation is

$$
\vec{x}=\frac{m}{\vec{a} \cdot \vec{b}} \vec{b}+\frac{1}{\vec{a} \cdot \vec{b}} \vec{a} \times \vec{c} .
$$

(C. Coşniţă, I. Sager, I. Matei, I. Dragotă, Culegere de probleme de Geometrie Analitica (Collection of Problems in Analytical Geometry), Editura Didactică şi Pedagogică, Bucharest, 1963)
\\
\textbf{Topic} :Probability\\
\textbf{Book} :Putnam and Beyond\\
\textbf{Final Answer} :\\


\textbf{Problem Statement} :
579. Does there exist a bijection $f$ of (a) a plane with itself or (b) three-dimensional space with itself such that for any distinct points $A, B$ the lines $A B$ and $f(A) f(B)$ are perpendicular?
\\
\textbf{Solution} :
579. (a) Yes: simply rotate the plane $90^{\circ}$ about some axis perpendicular to it. For example, in the $x y$-plane we could map each point $(x, y)$ to the point $(y,-x)$.

(b) Suppose such a bijection existed. In vector notation, the given condition states that

$$
(\vec{a}-\vec{b}) \cdot(f(\vec{a})-f(\vec{b}))=0
$$

for any three-dimensional vectors $\vec{a}$ and $\vec{b}$.

Assume without loss of generality that $f$ maps the origin to itself; otherwise, $g(\vec{p})=$ $f(\vec{p})-f(\overrightarrow{0})$ is still a bijection and still satisfies the above equation. Plugging $\vec{b}=$ $(0,0,0)$ into the above equation, we obtain that $\vec{a} \cdot f(\vec{a})=0$ for all $\vec{a}$. The equation reduces to

$$
\vec{a} \cdot f(\vec{b})-\vec{b} \cdot f(\vec{a})=0
$$

Given any vectors $\vec{a}, \vec{b}, \vec{c}$ and any real numbers $m, n$, we then have

$$
m(\vec{a} \cdot f(\vec{b})+\vec{b} \cdot f(\vec{a}))=0,
$$



$$
\begin{array}{r}
n(\vec{a} \cdot f(\vec{c})+\vec{c} \cdot f(\vec{a}))=0, \\
a \cdot f(m \vec{b}+n \vec{c})+(m \vec{b}+n \vec{c}) \cdot f(\vec{a})=0 .
\end{array}
$$

Adding the first two equations and subtracting the third gives

$$
\vec{a} \cdot(m f(\vec{b})+n f(\vec{c})-f(m \vec{b}+n \vec{c}))=0 .
$$

Because this is true for any vector $\vec{a}$, we must have

$$
f(m \vec{b}+n \vec{c})=m f(\vec{b})+n f(\vec{c}) .
$$

Therefore, $f$ is linear, and it is determined by the images of the unit vectors $\vec{i}=(1,0,0)$, $\vec{j}=(0,1,0)$, and $\vec{k}=(0,0,1)$. If

$$
f(\vec{i})=\left(a_{1}, a_{2}, a_{3}\right), \quad f(\vec{j})=\left(b_{1}, b_{2}, b_{3}\right), \quad \text { and } \quad f(\vec{k})=\left(c_{1}, c_{2}, c_{3}\right),
$$

then for a vector $\vec{x}$ we have

$$
f(\vec{x})=\left[\begin{array}{lll}
a_{1} & b_{1} & c_{1} \\
a_{2} & b_{2} & c_{2} \\
a_{3} & b_{3} & c_{3}
\end{array}\right] \vec{x} .
$$

Substituting in $f(\vec{a}) \cdot \vec{a}=0$ successively $\vec{a}=\vec{i}, \vec{j}, \vec{k}$, we obtain $a_{1}=b_{2}=c_{3}=$ 0 . Then substituting in $\vec{a} \cdot f(\vec{b})+\vec{b} \cdot f(\vec{a}),(\vec{a}, \vec{b})=(\vec{i}, \vec{j}),(\vec{j}, \vec{k}),(\vec{k}, \vec{i})$, we obtain $b_{1}=-a_{2}, c_{2}=-b_{3}, c_{1}=-a_{3}$.

Setting $k_{1}=c_{2}, k_{2}=-c_{1}$, and $k_{3}=b_{1}$ yields

$$
f\left(k_{1} \vec{i}+k_{2} \vec{j}+k_{3} \vec{k}\right)=k_{1} f(\vec{i})+k_{2} f(\vec{j})+k_{3} f(\vec{k})=\overrightarrow{0} .
$$

Because $f$ is injective and $f(\overrightarrow{0})=\overrightarrow{0}$, this implies that $k_{1}=k_{2}=k_{3}=0$. Then $f(\vec{x})=0$ for all $\vec{x}$, contradicting the assumption that $f$ was a surjection. Therefore, our original assumption was false, and no such bijection exists.

(Team Selection Test for the International Mathematical Olympiad, Belarus, 1999)
\\
\textbf{Topic} :Probability\\
\textbf{Book} :Putnam and Beyond\\
\textbf{Final Answer} :\\


\textbf{Problem Statement} :
591. Find the locus of points $P$ in the interior of a triangle $A B C$ such that the distances from $P$ to the lines $A B, B C$, and $C A$ are the side lengths of some triangle.
\\
\textbf{Solution} :
591. Denote by $\delta(P, M N)$ the distance from $P$ to the line $M N$. The problem asks for the locus of points $P$ for which the inequalities

$$
\begin{aligned}
&\delta(P, A B)<\delta(P, B C)+\delta(P, C A), \\
&\delta(P, B C)<\delta(P, C A)+\delta(P, A B), \\
&\delta(P, C A)<\delta(P, A B)+\delta(P, B C)
\end{aligned}
$$

are simultaneously satisfied.

Let us analyze the first inequality, written as $f(P)=\delta(P, B C)+\delta(P, C A)-$ $\delta(P, A B)>0$. As a function of the coordinates $(x, y)$ of $P$, the distance from $P$ to a line is of the form $m x+n y+p$. Combining three such functions, we see that $f(P)=f(x, y)$ is of the same form, $f(x, y)=\alpha x+\beta y+\gamma$. To solve the inequality $f(x, y)>0$ it suffices to find the line $f(x, y)=0$ and determine on which side of the line the function is positive. The line intersects the side $B C$ where $\delta(P, C A)=\delta(P, A B)$, hence at the point $E$ where the angle bisector from $A$ intersects this side. It intersects side $C A$ at the point $F$ where the bisector from $B$ intersects the side. Also, $f(x, y)>0$ on side $A B$, hence on the same side of the line $E F$ as the segment $A B$. Arguing similarly for the other two inequalities, we deduce that the locus is the interior of the triangle formed by the points where the angle bisectors meet the opposite sides.
\\
\textbf{Topic} :Probability\\
\textbf{Book} :Putnam and Beyond\\
\textbf{Final Answer} :\\


\textbf{Problem Statement} :
592. Let $A_{1}, A_{2}, \ldots, A_{n}$ be distinct points in the plane, and let $m$ be the number of midpoints of all the segments they determine. What is the smallest value that $m$ can have?
\\
\textbf{Solution} :
592. Consider an affine system of coordinates such that none of the segments determined by the $n$ points is parallel to the $x$-axis. If the coordinates of the midpoints are $\left(x_{i}, y_{i}\right)$, $i=1,2, \ldots, m$, then $x_{i} \neq x_{j}$ for $i \neq j$. Thus we have reduced the problem to the onedimensional situation. So let $A_{1}, A_{2}, \ldots, A_{n}$ lie on a line in this order. The midpoints of $A_{1} A_{2}, A_{1} A_{3}, \ldots, A_{1} A_{n}$ are all distinct and different from the (also distinct) midpoints of $A_{2} A_{n}, A_{3} A_{n}, \ldots, A_{n-1} A_{n}$. Hence there are at least $(n-1)+(n-2)=2 n-3$ midpoints. This bound can be achieved for $A_{1}, A_{2}, \ldots, A_{n}$ the points $1,2, \ldots, n$ on the real axis.

(Középiskolai Matematikai Lapok (Mathematics Magazine for High Schools, Budapest), proposed by M. Salát)
\\
\textbf{Topic} :Probability\\
\textbf{Book} :Putnam and Beyond\\
\textbf{Final Answer} :\\


\textbf{Problem Statement} :
600. On the sides of a convex quadrilateral $A B C D$ one draws outside the equilateral triangles $A B M$ and $C D P$ and inside the equilateral triangles $B C N$ and $A D Q$. Describe the shape of the quadrilateral $M N P Q$.
\\
\textbf{Solution} :
600. With the convention that the lowercase letter denotes the complex coordinate of the point denoted by the same letter in uppercase, we translate the geometric conditions from the statement into the algebraic equations 

$$
\frac{m-a}{b-a}=\frac{n-c}{b-c}=\frac{p-c}{d-c}=\frac{q-a}{d-a}=\epsilon
$$

where $\epsilon=\cos \frac{\pi}{3}+i \sin \frac{\pi}{3}$. Therefore,

$$
\begin{aligned}
m=a+(b-a) \epsilon, & n=c+(b-c) \epsilon, \\
p=c+(d-c) \epsilon, & q=a+(d-a) \epsilon .
\end{aligned}
$$

It is now easy to see that $\frac{1}{2}(m+p)=\frac{1}{2}(n+q)$, meaning that $M P$ and $N Q$ have the same midpoint. So either the four points are collinear, or they form a parallelogram.

(short list of the 23rd International Mathematical Olympiad, 1982)
\\
\textbf{Topic} :Probability\\
\textbf{Book} :Putnam and Beyond\\
\textbf{Final Answer} :\\


\textbf{Problem Statement} :
602. Let $A_{1} A_{2} \ldots A_{n}$ be a regular polygon with circumradius equal to 1 . Find the maximum value of $\prod_{k=1}^{n} P A_{k}$ as $P$ ranges over the circumcircle. 
\\
\textbf{Solution} :
602. In the language of complex numbers we are required to find the maximum of $\prod_{k=1}^{n}\left|z-\epsilon^{k}\right|$ as $z$ ranges over the unit disk, where $\epsilon=\cos \frac{2 \pi}{n}+i \sin \frac{2 \pi}{n}$. We have

$$
\prod_{k=1}^{n}\left|z-\epsilon^{k}\right|=\left|\prod_{k=1}^{n}\left(z-\epsilon^{k}\right)\right|=\left|z^{n}-1\right| \leq\left|z^{n}\right|+1=2 .
$$

The maximum is 2 , attained when $z$ is an $n$th root of $-1$.

(Romanian Mathematics Competition "Grigore Moisil," 1992, proposed by D. Andrica)
\\
\textbf{Topic} :Probability\\
\textbf{Book} :Putnam and Beyond\\
\textbf{Final Answer} :\\


\textbf{Problem Statement} :
608. Consider the parabola $y^{2}=4 p x$. Find the locus of the points such that the tangents to the parabola from those points make a constant angle $\phi$.
\\
\textbf{Solution} :
608. The condition that a line through $\left(x_{0}, y_{0}\right)$ be tangent to the parabola is that the system

$$
\begin{aligned}
y^{2} &=4 p x, \\
y-y_{0} &=m\left(x-x_{0}\right)
\end{aligned}
$$

have a unique solution. This means that the discriminant of the quadratic equation in $x$ obtained by eliminating $y,\left(m x-m x_{0}+y_{0}\right)^{2}-4 p x=0$, is equal to zero. This translates into the condition

$$
m^{2} x_{0}-m y_{0}+p=0 .
$$

The slopes $m$ of the two tangents are therefore the solutions to this quadratic equation. They satisfy 

$$
\begin{aligned}
m_{1}+m_{2} &=\frac{y_{0}}{x_{0}}, \\
m_{1} m_{2} &=\frac{p}{x_{0}} .
\end{aligned}
$$

We also know that the angle between the tangents is $\phi$. We distinguish two situations. First, if $\phi=90^{\circ}$, then $m_{1} m_{2}=-1$. This implies $\frac{p}{x_{0}}=-1$, so the locus is the line $x=-p$, which is the directrix of the parabola.

If $\phi \neq 90^{\circ}$, then

$$
\tan \phi=\frac{m_{1}-m_{2}}{1+m_{1} m_{2}}=\frac{m_{1}-m_{2}}{1+\frac{p}{x_{0}}} .
$$

We thus have

$$
\begin{aligned}
&m_{1}+m_{2}=\frac{y_{0}}{x_{0}}, \\
&m_{1}-m_{2}=\tan \phi+\frac{p}{x_{0}} \tan \phi .
\end{aligned}
$$

We can compute $m_{1} m_{2}$ by squaring the equations and then subtracting them, and we obtain

$$
m_{1} m_{2}=\frac{y_{0}^{2}}{4 x_{0}^{2}}-\left(1+\frac{p}{x_{0}}\right)^{2} \tan ^{2} \phi .
$$

This must equal $\frac{p}{x_{0}}$. We obtain the equation of the locus to be

$$
-y^{2}+(x+p)^{2} \tan ^{2} \phi+4 p x=0,
$$

which is a hyperbola. One branch of the hyperbola contains the points from which the parabola is seen under the angle $\phi$, and one branch contains the points from which the parabola is seen under an angle equal to the suplement of $\phi$.

(A. Myller, Geometrie Analitic ă (Analytical Geometry), 3rd ed., Editura Didactică şi Pedagogică, Bucharest, 1972)
\\
\textbf{Topic} :Probability\\
\textbf{Book} :Putnam and Beyond\\
\textbf{Final Answer} :\\


\textbf{Problem Statement} :
609. Let $T_{1}, T_{2}, T_{3}$ be points on a parabola, and $t_{1}, t_{2}, t_{3}$ the tangents to the parabola at these points. Compute the ratio of the area of triangle $T_{1} T_{2} T_{3}$ to the area of the triangle determined by the tangents.
\\
\textbf{Solution} :
609. Choose a Cartesian system of coordinates such that the equation of the parabola is $y^{2}=4 p x$. The coordinates of the three points are $T_{i}\left(4 p \alpha_{i}^{2}, 4 p \alpha_{i}\right)$, for appropriately chosen $\alpha_{i}, i=1,2,3$. Recall that the equation of the tangent to the parabola at a point $\left(x_{0}, y_{0}\right)$ is $y y_{0}=2 p\left(x+x_{0}\right)$. In our situation the three tangents are given by

$$
2 \alpha_{i} y=x+4 p \alpha_{i}^{2}, \quad i=1,2,3 .
$$

If $P_{i j}$ is the intersection of $t_{i}$ and $t_{j}$, then its coordinates are $\left(4 p \alpha_{i} \alpha_{j}, 2 p\left(\alpha_{i}+\alpha_{j}\right)\right)$. The area of triangle $T_{1} T_{2} T_{3}$ is given by a Vandermonde determinant: 

$$
\pm \frac{1}{2}\left|\begin{array}{lll}
4 p \alpha_{1}^{2} & 4 p \alpha_{1} & 1 \\
4 p \alpha_{2}^{2} & 4 p \alpha_{2} & 1 \\
4 p \alpha_{3}^{2} & 4 p \alpha_{3} & 1
\end{array}\right|=\pm 8 p^{2}\left|\begin{array}{lll}
\alpha_{1}^{2} & \alpha_{1} & 1 \\
\alpha_{2}^{2} & \alpha_{2} & 1 \\
\alpha_{3}^{2} & \alpha_{3} & 1
\end{array}\right|=8 p^{2}\left|\left(\alpha_{1}-\alpha_{2}\right)\left(\alpha_{1}-\alpha_{3}\right)\left(\alpha_{2}-\alpha_{3}\right)\right| .
$$

The area of the triangle $P_{12} P_{23} P_{31}$ is given by

$$
\begin{aligned}
& \pm \frac{1}{2}\left|\begin{array}{lll}4 p \alpha_{1} \alpha_{2} & 2 p\left(\alpha_{1}+\alpha_{2}\right) & 1 \\4 p \alpha_{2} \alpha_{3} & 2 p\left(\alpha_{2}+\alpha_{3}\right) & 1 \\4 p \alpha_{3} \alpha_{1} & 2 p\left(\alpha_{3}+\alpha_{1}\right) & 1\end{array}\right| \\
& =\pm 4 p^{2}\left|\begin{array}{ll}\alpha_{1} \alpha_{2}\left(\alpha_{1}+\alpha_{2}\right) & 1 \\\alpha_{2} \alpha_{3}\left(\alpha_{2}+\alpha_{3}\right) & 1 \\\alpha_{3} \alpha_{1}\left(\alpha_{3}+\alpha_{1}\right) & 1\end{array}\right|=\pm 4 p^{2}\left|\begin{array}{rr}\left(\alpha_{1}-\alpha_{3}\right) \alpha_{2}\left(\alpha_{1}-\alpha_{3}\right) & 0 \\\left(\alpha_{2}-\alpha_{1}\right) \alpha_{3}\left(\alpha_{2}-\alpha_{1}\right) & 0 \\\alpha_{3} \alpha_{1}\left(\alpha_{3}+\alpha_{1}\right) & 1\end{array}\right| \\
& =4 p^{2}\left|\left(\alpha_{1}-\alpha_{3}\right)\left(\alpha_{1}-\alpha_{2}\right)\left(\alpha_{2}-\alpha_{3}\right)\right| \text {. }
\end{aligned}
$$

We conclude that the ratio of the two areas is 2 , regardless of the location of the three points or the shape of the parabola.

(Gh. Călugăriţa, V. Mangu, Probleme de Matematică pentru Treapta I şi a II-a de Liceu (Mathematics Problems for High School), Editura Albatros, Bucharest, 1977)
\\
\textbf{Topic} :Probability\\
\textbf{Book} :Putnam and Beyond\\
\textbf{Final Answer} :\\


\textbf{Problem Statement} :
614. Compute the integral

$$
\int \frac{d x}{a+b \cos x+c \sin x},
$$

where $a, b, c$ are real numbers, not all equal to zero.
\\
\textbf{Solution} :
614. The interesting case occurs of course when $b$ and $c$ are not both equal to zero. Set $d=\sqrt{b^{2}+c^{2}}$ and define the angle $\alpha$ by the conditions $\cos \alpha=\frac{b}{\sqrt{b^{2}+c^{2}}}$ and $\sin \alpha=$ $\frac{c}{\sqrt{b^{2}+c^{2}}}$. The integral takes the form

$$
\int \frac{d x}{a+d \cos (x-\alpha)},
$$

which, with the substitution $u=x-\alpha$, becomes the simpler

$$
\int \frac{d u}{a+d \cos u} \text {. }
$$

The substitution $t=\tan \frac{u}{2}$ changes this into

$$
\frac{2}{a+d} \int \frac{d t}{1+\frac{a-d}{a+d} t^{2}} .
$$

If $a=d$ the answer to the problem is $\frac{1}{a} \tan \frac{x-\alpha}{2}+C$. If $\frac{a-d}{a+d}>0$, the answer is

$$
\frac{2}{\sqrt{a^{2}-d^{2}}} \arctan \left(\sqrt{\frac{a-d}{a+d}} \tan \frac{x-\alpha}{2}+C\right),
$$

while if $\frac{a-d}{a+d}<0$, the answer is

$$
\frac{1}{\sqrt{d^{2}-a^{2}}} \ln \left|\frac{1+\sqrt{\frac{d-a}{d+a}} \tan \frac{x-\alpha}{2}}{1-\sqrt{\frac{d-a}{d+a}} \tan \frac{x-\alpha}{2}}\right|+C .
$$
\\
\textbf{Topic} :Probability\\
\textbf{Book} :Putnam and Beyond\\
\textbf{Final Answer} :\\


\textbf{Problem Statement} :
617. Given a circle of diameter $A B$, a variable secant through $A$ intersects the circle at $C$ and the tangent through $B$ at $D$. On the half-line $A C$ a point $M$ is chosen such that $A M=C D$. Find the locus of $M$. 
\\
\textbf{Solution} :
617. Let $A B=a$ and consider a system of polar coordinates with pole $A$ and axis $A B$. The equation of the curve traced by $M$ is obtained as follows. We have $A M=r$, $A D=\frac{a}{\cos \theta}$, and $A C=a \cos \theta$. The equality $A M=A D-A C$ yields the equation

$$
r=\frac{a}{\cos \theta}-a \cos \theta .
$$

The equation of the locus is therefore $r=\frac{a \sin ^{2} \theta}{\cos \theta}$. This curve is called the cisoid of Diocles (Figure 80).
\\
\textbf{Topic} :Probability\\
\textbf{Book} :Putnam and Beyond\\
\textbf{Final Answer} :\\


\textbf{Problem Statement} :
618. Find the locus of the projection of a fixed point on a circle onto the tangents to the circle.
\\
\textbf{Solution} :
618. Let $O$ be the center and $a$ the radius of the circle, and let $M$ be the point on the circle. Choose a system of polar coordinates with $M$ the pole and $M O$ the axis. For an arbitrary tangent, let $I$ be its intersection with $M O, T$ the tangency point, and $P$ the projection of $M$ onto the tangent. Then

$$
O I=\frac{O T}{\cos \theta}=\frac{a}{\cos \theta} .
$$

Hence

$$
M P=r=(M O+O I) \cos \theta=\left(a+\frac{a}{\cos \theta}\right) \cos \theta .
$$

We obtain $r=a(1+\cos \theta)$, which is the equation of a cardioid (Figure 80).
\\
\textbf{Topic} :Probability\\
\textbf{Book} :Putnam and Beyond\\
\textbf{Final Answer} :\\


\textbf{Problem Statement} :
620. The endpoints of a variable segment $A B$ lie on two perpendicular lines that intersect at $O$. Find the locus of the projection of $O$ onto $A B$, provided that the segment $A B$ maintains a constant length.
\\
\textbf{Solution} :
620. As before, we work with polar coordinates, choosing $O$ as the pole and $O A$ as the axis. Denote by $a$ the length of the segment $A B$ and by $P(r, \theta)$ the projection of $O$ onto this segment. Then $O A=\frac{r}{\cos \theta}$ and $O A=A B \sin \theta$, which yield the equation of the locus

$$
r=a \sin \theta \cos \theta=\frac{a}{2} \sin 2 \theta .
$$

This is a four-leaf rose.
\\
\textbf{Topic} :Probability\\
\textbf{Book} :Putnam and Beyond\\
\textbf{Final Answer} :\\


\textbf{Problem Statement} :
622. Find a transformation of the plane that maps the unit circle $x^{2}+y^{2}=1$ into a cardioid. (Recall that the general equation of a cardioid is $r=2 a(1+\cos \theta)$.)
\\
\textbf{Solution} :
622. The solution uses complex and polar coordinates. Our goal is to map the circle onto a cardioid of the form

$$
r=a(1+\cos \theta), \quad a>0 .
$$

Because this cardioid passes through the origin, it is natural to work with a circle that itself passes through the origin, for example $|z-1|=1$. If $\phi: \mathbb{C} \rightarrow \mathbb{C}$ maps this circle into the cardioid, then the equation of the cardioid will have the form

$$
\left|\phi^{-1}(z)-1\right|=1 .
$$

So we want to bring the original equation of the cardioid into this form. First, we change it to

$$
r=a \cdot 2 \cos ^{2} \frac{\theta}{2} ;
$$

then we take the square root,

$$
\sqrt{r}=\sqrt{2 a} \cos \frac{\theta}{2} .
$$

Multiplying by $\sqrt{r}$, we obtain

$$
r=\sqrt{2 a} \sqrt{r} \cos \frac{\theta}{2},
$$

or

$$
r-\sqrt{2 a} \sqrt{r} \cos \frac{\theta}{2}=0 .
$$

This should look like the equation of a circle. We modify the expression as follows:

$$
\begin{aligned}
r-\sqrt{2 a} \sqrt{r} \cos \frac{\theta}{2} &=r\left(\cos ^{2} \frac{\theta}{2}+\sin ^{2} \frac{\theta}{2}\right)-\sqrt{2 a} \sqrt{r} \cos \frac{\theta}{2}+1-1 \\
&=\left(\sqrt{r} \cos \frac{\theta}{2}\right)^{2}-\sqrt{2 a} \sqrt{r} \cos \frac{\theta}{2}+1+\left(\sqrt{r} \sin \frac{\theta}{2}\right)^{2}-1 .
\end{aligned}
$$

If we set $a=2$, we have a perfect square, and the equation becomes

$$
\left(\sqrt{r} \cos \frac{\theta}{2}-1\right)^{2}+\left(\sqrt{r} \sin \frac{\theta}{2}\right)^{2}=1,
$$

which in complex coordinates reads $|\sqrt{z}-1|=1$. Of course, there is an ambiguity in taking the square root, but we are really interested in the transformation $\phi$, not in $\phi^{-1}$. Therefore, we can choose $\phi(z)=z^{2}$, which maps the circle $|z-1|=1$ into the cardioid $r=2(1+\cos \theta)$.

Remark. Of greater practical importance is the Zhukovski transformation $z \rightarrow \frac{1}{2}\left(z+\frac{1}{z}\right)$, which maps the unit circle onto the profile of the airplane wing (the so-called aerofoil). Because the Zhukovski map preserves angles, it helps reduce the study of the air flow around an airplane wing to the much simpler study of the air flow around a circle.
\\
\textbf{Topic} :Probability\\
\textbf{Book} :Putnam and Beyond\\
\textbf{Final Answer} :\\


\textbf{Problem Statement} :
641. In a triangle $A B C$ for a variable point $P$ on $B C$ with $P B=x$ let $t(x)$ be the measure of $\angle P A B$. Compute

$$
\int_{0}^{a} \cos t(x) d x
$$

in terms of the sides and angles of triangle $A B C$.
\\
\textbf{Solution} :
641. The law of cosines in triangle $A P B$ gives

$$
A P^{2}=x^{2}+c^{2}-2 x c \cos B
$$

and

$$
x^{2}=c^{2}+A P^{2}=x^{2}+c^{2}-2 x c \cos B-2 c \sqrt{x^{2}+c^{2}-2 x c \cos B} \cos t,
$$

whence 

$$
\cos t=\frac{c-x \cos B}{\sqrt{x^{2}+c^{2}-2 x c \cos B}}
$$

The integral from the statement is

$$
\int_{0}^{a} \cos t(x) d x=\int_{0}^{a} \frac{c-x \cos B}{\sqrt{x^{2}+c^{2}-2 x c \cos B}} d x
$$

Using the standard integration formulas

$$
\begin{aligned}
&\int \frac{d x}{\sqrt{x^{2}+\alpha x+\beta}}=\ln \left(2 x+\alpha+2 \sqrt{x^{2}+\alpha x+\beta}\right) \\
&\int \frac{x d x}{\sqrt{x^{2}+\alpha x+\beta}}=\sqrt{x^{2}+\alpha x+\beta}-\frac{\alpha}{2} \ln \left(2 x+\alpha+2 \sqrt{x^{2}+\alpha x+\beta}\right)
\end{aligned}
$$

we obtain

$$
\begin{aligned}
\int_{0}^{a} \cos t(x) d x=&\left.c \sin ^{2} B \ln \left(2 x+2 c \cos B+2 \sqrt{x^{2}-2 c x \cos B+c^{2}}\right)\right|_{0} ^{a} \\
&-\left.\cos B \sqrt{x^{2}-2 c x \cos B+c^{2}}\right|_{0} ^{a} \\
=& c \sin ^{2} B \ln \frac{a-c \cos B+b}{c(1-\cos B)}+\cos B(c-b)
\end{aligned}
$$
\\
\textbf{Topic} :Probability\\
\textbf{Book} :Putnam and Beyond\\
\textbf{Final Answer} :\\


\textbf{Problem Statement} :
642. Let $f:[0, a] \rightarrow \mathbb{R}$ be a continuous and increasing function such that $f(0)=0$. Define by $R$ the region bounded by $f(x)$ and the lines $x=a$ and $y=0$. Now consider the solid of revolution obtained when $R$ is rotated around the $y$-axis as a sort of dish. Determine $f$ such that the volume of water the dish can hold is equal to the volume of the dish itself, this happening for all $a$.
\\
\textbf{Solution} :
642. It is equivalent to ask that the volume of the dish be half of that of the solid of revolution obtained by rotating the rectangle $0 \leq x \leq a$ and $0 \leq y \leq f(a)$. Specifically, this condition is

$$
\int_{0}^{a} 2 \pi x f(x) d x=\frac{1}{2} \pi a^{2} f(a)
$$

Because the left-hand side is differentiable with respect to $a$ for all $a>0$, the right-hand side is differentiable, too. Differentiating, we obtain

$$
2 \pi a f(a)=\pi a f(a)+\frac{1}{2} \pi a^{2} f^{\prime}(a) .
$$

This is a differential equation in $f$, which can be written as $f^{\prime}(a) / f(a)=\frac{2}{a}$. Integrating, we obtain $\ln f(a)=2 \ln a$, or $f(a)=c a^{2}$ for some constant $c>0$. This solves the problem.

\section{(Math Horizons)}\
\textbf{Topic} :Probability\\
\textbf{Book} :Putnam and Beyond\\
\textbf{Final Answer} :\\


\textbf{Problem Statement} :
659. Find the range of the function $f: \mathbb{R} \rightarrow \mathbb{R}, f(x)=(\sin x+1)(\cos x+1)$.
\\
\textbf{Solution} :
659. We have

$$
\begin{aligned}
f(x) &=\sin x \cos x+\sin x+\cos x+1=\frac{1}{2}(\sin x+\cos x)^{2}-\frac{1}{2}+\sin x+\cos x+1 \\
&=\frac{1}{2}\left[(\sin x+\cos x)^{2}+2(\sin x+\cos x)+1\right]=\frac{1}{2}[(\sin x+\cos x)+1]^{2} .
\end{aligned}
$$

This is a function of $y=\sin x+\cos x$, namely $f(y)=\frac{1}{2}(y+1)^{2}$. Note that

$$
y=\cos \left(\frac{\pi}{2}-x\right)+\cos x=2 \cos \frac{\pi}{4} \cos \left(x-\frac{\pi}{4}\right)=\sqrt{2} \cos \left(x-\frac{\pi}{4}\right) .
$$

So $y$ ranges between $-\sqrt{2}$ and $\sqrt{2}$. Hence $f(y)$ ranges between 0 and $\frac{1}{2}(\sqrt{2}+1)^{2}$.
\\
\textbf{Topic} :Probability\\
\textbf{Book} :Putnam and Beyond\\
\textbf{Final Answer} :\\


\textbf{Problem Statement} :
661. Compute the integral

$$
\int \sqrt{\frac{1-x}{1+x}} d x, \quad x \in(-1,1) .
$$
\\
\textbf{Solution} :
661. We would like to eliminate the square root, and for that reason we recall the trigonometric identity

$$
\frac{1-\sin t}{1+\sin t}=\frac{\cos ^{2} t}{(1+\sin t)^{2}} .
$$

The proof of this identity is straightforward if we express the cosine in terms of the sine and then factor the numerator. Thus if we substitute $x=\sin t$, then $d x=\cos t d t$ and the integral becomes

$$
\int \frac{\cos ^{2} t}{1+\sin t} d t=\int 1-\sin t d t=t+\cos t+C .
$$

Since $t=\arcsin x$, this is equal to $\arcsin x+\sqrt{1-x^{2}}+C$.

(Romanian high school textbook)
\\
\textbf{Topic} :Probability\\
\textbf{Book} :Putnam and Beyond\\
\textbf{Final Answer} :\\


\textbf{Problem Statement} :
668. Compute the sum

$$
\left(\begin{array}{l}
n \\
1
\end{array}\right) \cos x+\left(\begin{array}{l}
n \\
2
\end{array}\right) \cos 2 x+\cdots+\left(\begin{array}{l}
n \\
n
\end{array}\right) \cos n x .
$$
\\
\textbf{Solution} :
668. Denote the sum in question by $S_{1}$ and let

$$
S_{2}=\left(\begin{array}{l}
n \\
1
\end{array}\right) \sin x+\left(\begin{array}{l}
n \\
2
\end{array}\right) \sin 2 x+\cdots+\left(\begin{array}{l}
n \\
n
\end{array}\right) \sin n x .
$$

Using Euler's formula, we can write

$$
1+S_{1}+i S_{2}=\left(\begin{array}{l}
n \\
0
\end{array}\right)+\left(\begin{array}{l}
n \\
1
\end{array}\right) e^{i x}+\left(\begin{array}{l}
n \\
2
\end{array}\right) e^{2 i x}+\cdots+\left(\begin{array}{l}
n \\
n
\end{array}\right) e^{i n x} .
$$

By the multiplicative property of the exponential we see that this is equal to

$$
\sum_{k=0}^{n}\left(\begin{array}{l}
n \\
k
\end{array}\right)\left(e^{i x}\right)^{k}=\left(1+e^{i x}\right)^{n}=\left(2 \cos \frac{x}{2}\right)^{n}\left(e^{i \frac{x}{2}}\right)^{n} .
$$

The sum in question is the real part of this expression less 1 , which is equal to

$$
2^{n} \cos ^{n} \frac{x}{2} \cos \frac{n x}{2}-1 .
$$
\\
\textbf{Topic} :Probability\\
\textbf{Book} :Putnam and Beyond\\
\textbf{Final Answer} :\\


\textbf{Problem Statement} :
669. Find the Taylor series expansion at 0 of the function

$$
f(x)=e^{x \cos \theta} \cos (x \sin \theta),
$$

where $\theta$ is a parameter.
\\
\textbf{Solution} :
669. Combine $f(x)$ with the function $g(x)=e^{x \cos \theta} \sin (x \sin \theta)$ and write

$$
\begin{aligned}
f(x)+i g(x) &=e^{x \cos \theta}(\cos (x \sin \theta)+i \sin (x \sin \theta)) \\
&=e^{x \cos \theta} \cdot e^{i x \sin \theta}=e^{x(\cos \theta+i \sin \theta)} .
\end{aligned}
$$

Using the de Moivre formula we expand this in a Taylor series as

$$
1+\frac{x}{1 !}(\cos \theta+i \sin \theta)+\frac{x^{2}}{2 !}(\cos 2 \theta+i \sin 2 \theta)+\cdots+\frac{x^{n}}{n !}(\cos n \theta+i \sin n \theta)+\cdots .
$$

Consequently, the Taylor expansion of $f(x)$ around 0 is the real part of this series, i.e.,

$$
f(x)=1+\frac{\cos \theta}{1 !} x+\frac{\cos 2 \theta}{2 !} x^{2}+\cdots+\frac{\cos n \theta}{n !} x^{n}+\cdots .
$$
\\
\textbf{Topic} :Probability\\
\textbf{Book} :Putnam and Beyond\\
\textbf{Final Answer} :\\


\textbf{Problem Statement} :
672. Find $(\cos \alpha)(\cos 2 \alpha)(\cos 3 \alpha) \cdots(\cos 999 \alpha)$ with $\alpha=\frac{2 \pi}{1999}$. 
\\
\textbf{Solution} :
672. More generally, for an odd integer $n$, let us compute

$$
S=(\cos \alpha)(\cos 2 \alpha) \cdots(\cos n \alpha)
$$

with $\alpha=\frac{2 \pi}{2 n+1}$. We can let $\zeta=e^{i \alpha}$ and then $S=2^{-n} \prod_{k=1}^{n}\left(\zeta^{k}+\zeta^{-k}\right)$. Since $\zeta^{k}+\zeta^{-k}=$ $\zeta^{2 n+1-k}+\zeta^{-(2 n+1-k)}, k=1,2, \ldots, n$, we obtain 

$$
S^{2}=2^{-2 n} \prod_{k=1}^{2 n}\left(\zeta^{k}+\zeta^{-k}\right)=2^{-2 n} \times \prod_{k=1}^{2 n} \zeta^{-k} \times \prod_{k=1}^{2 n}\left(1+\zeta^{2 k}\right)
$$

The first of the two products is just $\zeta^{-(1+2+\cdots+2 n)}$. Because $1+2+\cdots+2 n=n(2 n+1)$, which is a multiple of $2 n+1$, this product equals 1 .

As for the product $\prod_{k=1}^{2 n}\left(1+\zeta^{2 k}\right)$, note that it can be written as $\prod_{k=1}^{2 n}\left(1+\zeta^{k}\right)$, since the numbers $\zeta^{2 k}$ range over the $(2 n+1)$ st roots of unity other than 1 itself, taking each value exactly once. We compute this using the factorization

$$
z^{n+1}-1=(z-1) \prod_{k=1}^{2 n}\left(z-\zeta^{k}\right)
$$

Substituting $z=-1$ and dividing both sides by $-2$ gives $\prod_{k=1}^{2 n}\left(-1-\zeta^{k}\right)=1$, so $\prod_{k=1}^{2 n}\left(1+\zeta^{k}\right)=1$. Hence $S^{2}=2^{-2 n}$, and so $S=\pm 2^{-n}$. We need to determine the sign.

For $1 \leq k \leq n, \cos k \alpha<0$ when $\frac{\pi}{2}<k \alpha<\pi$. The values of $k$ for which this happens are $\left\lceil\frac{n+1}{2}\right\rceil$ through $n$. The number of such $k$ is odd if $n \equiv 1$ or $2(\bmod 4)$, and even if $n \equiv 0$ or $3(\bmod 4)$. Hence

$$
S= \begin{cases}+2^{-n} & \text { if } n \equiv 1 \text { or } 2(\bmod 4) \\ -2^{-n} & \text { if } n \equiv 0 \text { or } 3(\bmod 4)\end{cases}
$$

Taking $n=999 \equiv 3(\bmod 4)$, we obtain the answer to the problem, $-2^{-999}$.

(proposed by J. Propp for the USA Mathematical Olympiad, 1999)
\\
\textbf{Topic} :Probability\\
\textbf{Book} :Putnam and Beyond\\
\textbf{Final Answer} :\\


\textbf{Problem Statement} :
676. Solve the equation $x^{3}-3 x=\sqrt{x+2}$ in real numbers. 
\\
\textbf{Solution} :
676. First, note that if $x>2$, then $x^{3}-3 x>4 x-3 x=x>\sqrt{x+2}$, so all solutions $x$ should satisfy $-2 \leq x \leq 2$. Therefore, we can substitute $x=2 \cos a$ for some $a \in[0, \pi]$. Then the given equation becomes

$$
2 \cos 3 a=\sqrt{2(1+\cos a)}=2 \cos \frac{a}{2},
$$

So

$$
2 \sin \frac{7 a}{4} \sin \frac{5 a}{4}=0
$$

meaning that $a=0, \frac{4 \pi}{7}, \frac{4 \pi}{5}$. It follows that the solutions to the original equation are $x=2,2 \cos \frac{4 \pi}{7},-\frac{1}{2}(1+\sqrt{5})$.
\\
\textbf{Topic} :Probability\\
\textbf{Book} :Putnam and Beyond\\
\textbf{Final Answer} :\\


\textbf{Problem Statement} :
677. Find the maximum value of

$$
S=\left(1-x_{1}\right)\left(1-y_{1}\right)+\left(1-x_{2}\right)\left(1-y_{2}\right)
$$

if $x_{1}^{2}+x_{2}^{2}=y_{1}^{2}+y_{2}^{2}=c^{2}$, where $c$ is some positive number.
\\
\textbf{Solution} :
677. The points $\left(x_{1}, x_{2}\right)$ and $\left(y_{1}, y_{2}\right)$ lie on the circle of radius $c$ centered at the origin. Parametrizing the circle, we can write $\left(x_{1}, x_{2}\right)=(c \cos \phi, c \sin \phi)$ and $\left(y_{1}, y_{2}\right)=$ $(c \cos \psi, c \sin \psi)$. Then

$$
\begin{aligned}
S &=2-c(\cos \phi+\sin \phi+\cos \psi+\sin \psi)+c^{2}(\cos \phi \cos \psi+\sin \phi \sin \psi) \\
&=2+c \sqrt{2}\left(-\sin \left(\phi+\frac{\pi}{4}\right)-\sin \left(\psi+\frac{\pi}{4}\right)\right)+c^{2} \cos (\phi-\psi)
\end{aligned}
$$

We can simultaneously increase each of $-\sin \left(\phi+\frac{\pi}{4}\right),-\sin \left(\psi+\frac{\pi}{4}\right)$, and $\cos (\phi-\psi)$ to 1 by choosing $\phi=\psi=\frac{5 \pi}{4}$. Hence the maximum of $S$ is $2+2 c \sqrt{2}+c^{2}=(c+\sqrt{2})^{2}$. (proposed by C. Rousseau for the USA Mathematical Olympiad, 2002)
\\
\textbf{Topic} :Probability\\
\textbf{Book} :Putnam and Beyond\\
\textbf{Final Answer} :\\


\textbf{Problem Statement} :
682. Solve the following system of equations in real numbers:

$$
\begin{aligned}
&\frac{3 x-y}{x-3 y}=x^{2}, \\
&\frac{3 y-z}{y-3 z}=y^{2}, \\
&\frac{3 z-x}{z-3 x}=z^{2} .
\end{aligned}
$$
\\
\textbf{Solution} :
682. From the first equation, it follows that if $x$ is 0 , then so is $y$, making $x^{2}$ indeterminate; hence $x$, and similarly $y$ and $z$, cannot be 0 . Solving the equations, respectively, for $y, z$, and $x$, we obtain the equivalent system

$$
\begin{gathered}
y=\frac{3 x-x^{3}}{1-3 x^{2}}, \\
z=\frac{3 y-y^{3}}{1-3 y^{2}}, \\
x=\frac{3 z-z^{3}}{1-3 z^{2}},
\end{gathered}
$$

where $x, y, z$ are real numbers different from 0 .

There exists a unique number $u$ in the interval $\left(-\frac{\pi}{2}, \frac{\pi}{2}\right)$ such that $x=\tan u$. Then

$$
\begin{aligned}
&y=\frac{3 \tan u-\tan ^{3} u}{1-3 \tan ^{2} u}=\tan 3 u, \\
&z=\frac{3 \tan 3 u-\tan ^{3} 3 u}{1-3 \tan ^{2} 3 u}=\tan 9 u, \\
&x=\frac{3 \tan 9 u-\tan ^{3} 9 u}{1-3 \tan ^{2} 9 u}=\tan 27 u .
\end{aligned}
$$

The last equality yields $\tan u=\tan 27 u$, so $u$ and $27 u$ differ by an integer multiple of $\pi$. Therefore, $u=\frac{k \pi}{26}$ for some $k$ satisfying $-\frac{\pi}{2}<\frac{k \pi}{26}<\frac{\pi}{2}$. Besides, $k$ must not be 0 , since $x \neq 0$. Hence the possible values of $k$ are $\pm 1, \pm 2, \ldots, \pm 12$, each of them generating the corresponding triple

$$
x=\tan \frac{k \pi}{26}, \quad y=\tan \frac{3 k \pi}{26}, \quad z=\tan \frac{9 k \pi}{26} .
$$

It is immediately checked that all of these triples are solutions of the initial system.
\\
\textbf{Topic} :Probability\\
\textbf{Book} :Putnam and Beyond\\
\textbf{Final Answer} :\\


\textbf{Problem Statement} :
686. Compute the integral

$$
\int \frac{d x}{x+\sqrt{x^{2}-1}} .
$$
\\
\textbf{Solution} :
686. With the substitution $x=\cosh t$, the integral becomes

$$
\begin{aligned}
&\int \frac{1}{\sinh t+\cosh t} \sinh t d t \\
&\quad=\int \frac{e^{t}-e^{-t}}{2 e^{t}} d t=\frac{1}{2} \int\left(1-e^{-2 t}\right) d t=\frac{1}{2} t+\frac{e^{-2 t}}{4}+C \\
&\quad=\frac{1}{2} \ln \left(x+\sqrt{x^{2}-1}\right)+\frac{1}{4} \cdot \frac{1}{2 x^{2}-1+2 x \sqrt{x^{2}-1}}+C .
\end{aligned}
$$
\\
\textbf{Topic} :Probability\\
\textbf{Book} :Putnam and Beyond\\
\textbf{Final Answer} :\\


\textbf{Problem Statement} :
691. Obtain explicit values for the following series:
(a) $\sum_{n=1}^{\infty} \arctan \frac{2}{n^{2}}$,
(b) $\sum_{n=1}^{\infty} \arctan \frac{8 n}{n^{4}-2 n^{2}+5}$.
\\
\textbf{Solution} :
691. The formula

$$
\tan (a-b)=\frac{\tan a-\tan b}{1+\tan a \tan b}
$$

translates into

$$
\arctan \frac{x-y}{1+x y}=\arctan x-\arctan y .
$$

Applied to $x=n+1$ and $y=n-1$, it gives

$$
\arctan \frac{2}{n^{2}}=\arctan \frac{(n+1)-(n-1)}{1+(n+1)(n-1)}=\arctan (n+1)-\arctan (n-1) .
$$

The sum in part (a) telescopes as follows:

$$
\sum_{n=1}^{\infty} \arctan \frac{2}{n^{2}}=\lim _{N \rightarrow \infty} \sum_{n=1}^{N} \arctan \frac{2}{n^{2}}=\lim _{N \rightarrow \infty} \sum_{n=1}^{N}(\arctan (n+1)-\arctan (n-1))
$$



$$
\begin{aligned}
&=\lim _{N \rightarrow \infty}(\arctan (N+1)+\arctan N-\arctan 1-\arctan 0) \\
&=\frac{\pi}{2}+\frac{\pi}{2}-\frac{\pi}{4}=\frac{3 \pi}{4}
\end{aligned}
$$

The sum in part (b) is only slightly more complicated. In the above-mentioned formula for the difference of arctangents we have to substitute $x=\left(\frac{n+1}{\sqrt{2}}\right)^{2}$ and $y=\left(\frac{n-1}{\sqrt{2}}\right)^{2}$. This is because

$$
\frac{8 n}{n^{4}-2 n^{2}+5}=\frac{8 n}{4+\left(n^{2}-1\right)^{2}}=\frac{2\left[(n+1)^{2}-(n-1)^{2}\right]}{4-(n+1)^{2}(n-1)^{2}}=\frac{\left(\frac{n+1}{\sqrt{2}}\right)^{2}-\left(\frac{n-1}{\sqrt{2}}\right)^{2}}{1-\left(\frac{n+1}{\sqrt{2}}\right)^{2}\left(\frac{n-1}{\sqrt{2}}\right)^{2}} .
$$

The sum telescopes as

$$
\begin{aligned}
&\sum_{n=1}^{\infty} \arctan \frac{8 n}{n^{4}-2 n^{2}+5} \\
&=\lim _{N \rightarrow \infty} \sum_{n=1}^{N} \arctan \frac{8 n}{n^{4}-2 n^{2}+5}=\lim _{N \rightarrow \infty} \sum_{n=1}^{N}\left[\arctan \left(\frac{n+1}{\sqrt{2}}\right)^{2}-\arctan \left(\frac{n-1}{\sqrt{2}}\right)^{2}\right] \\
&=\lim _{N \rightarrow \infty}\left[\arctan \left(\frac{N+1}{\sqrt{2}}\right)^{2}+\arctan \left(\frac{N}{\sqrt{2}}\right)^{2}-\arctan 0-\arctan \frac{1}{2}\right]=\pi-\arctan \frac{1}{2} .
\end{aligned}
$$

(American Mathematical Monthly, proposed by J. Anglesio)
\\
\textbf{Topic} :Probability\\
\textbf{Book} :Putnam and Beyond\\
\textbf{Final Answer} :\\


\textbf{Problem Statement} :
693. In a circle of radius 1 a square is inscribed. A circle is inscribed in the square and then a regular octagon in the circle. The procedure continues, doubling each time the number of sides of the polygon. Find the limit of the lengths of the radii of the circles.
\\
\textbf{Solution} :
693. The radii of the circles satisfy the recurrence relation $R_{1}=1, R_{n+1}=R_{n} \cos \frac{\pi}{2^{n+1}}$. Hence 

$$
\lim _{n \rightarrow \infty} R_{n}=\prod_{n=1}^{\infty} \cos \frac{\pi}{2^{n}} .
$$

The product can be made to telescope if we use the double-angle formula for sine written as $\cos x=\frac{\sin 2 x}{2 \sin x}$. We then have

$$
\begin{aligned}
\prod_{n=2}^{\infty} \cos \frac{\pi}{2^{n}} &=\lim _{N \rightarrow \infty} \prod_{n=2}^{N} \cos \frac{\pi}{2^{n}}=\lim _{N \rightarrow \infty} \prod_{n=2}^{N} \frac{1}{2} \cdot \frac{\sin \frac{\pi}{2^{n-1}}}{\sin \frac{\pi}{2^{n}}} \\
&=\lim _{N \rightarrow \infty} \frac{1}{2^{N}} \frac{\sin \frac{\pi}{2}}{\sin \frac{\pi}{2^{N}}}=\frac{2}{\pi} \lim _{N \rightarrow \infty} \frac{\frac{\pi}{2^{N}}}{\sin \frac{\pi}{2^{N}}}=\frac{2}{\pi} .
\end{aligned}
$$

Thus the answer to the problem is $\frac{2}{\pi}$.

Remark. As a corollary, we obtain the formula

$$
\frac{2}{\pi}=\frac{\sqrt{2}}{2} \cdot \frac{\sqrt{2+\sqrt{2}}}{2} \cdot \frac{\sqrt{2+\sqrt{2+\sqrt{2}}}}{2} \cdots .
$$

This formula is credited to F. Viète, although Archimedes already used this approximation of the circle by regular polygons to compute $\pi$.
\\
\textbf{Topic} :Probability\\
\textbf{Book} :Putnam and Beyond\\
\textbf{Final Answer} :\\


\textbf{Problem Statement} :
695. Evaluate the product

$$
\left(1-\cot 1^{\circ}\right)\left(1-\cot 2^{\circ}\right) \cdots\left(1-\cot 44^{\circ}\right) \text {. }
$$
\\
\textbf{Solution} :
695. We have

$$
\begin{aligned}
\left(1-\cot 1^{\circ}\right)\left(1-\cot 2^{\circ}\right) \cdots\left(1-\cot 44^{\circ}\right) \\
&=\left(1-\frac{\cos 1^{\circ}}{\sin 1^{\circ}}\right)\left(1-\frac{\cos 2^{\circ}}{\sin 2^{\circ}}\right) \cdots\left(1-\frac{\cos 44^{\circ}}{\sin 44^{\circ}}\right) \\
&=\frac{\left(\sin 1^{\circ}-\cos 1^{\circ}\right)\left(\sin 2^{\circ}-\cos 2^{\circ}\right) \cdots\left(\sin 44^{\circ}-\cos 44^{\circ}\right)}{\sin 1^{\circ} \sin 2^{\circ} \cdots \sin 44^{\circ}}
\end{aligned}
$$

Using the identity $\sin a-\cos a=\sqrt{2} \sin \left(a-45^{\circ}\right)$ in the numerators, we transform this further into

$$
\begin{gathered}
\frac{\sqrt{2} \sin \left(1^{\circ}-45^{\circ}\right) \cdot \sqrt{2} \sin \left(2^{\circ}-45^{\circ}\right) \cdots \sqrt{2} \sin \left(44^{\circ}-45^{\circ}\right)}{\sin 1^{\circ} \sin 2^{\circ} \cdots \sin 44^{\circ}} \\
=\frac{(\sqrt{2})^{44}(-1)^{44} \sin 44^{\circ} \sin 43^{\circ} \cdots \sin 1^{\circ}}{\sin 44^{\circ} \sin 43^{\circ} \cdots \sin 1^{\circ}} .
\end{gathered}
$$

After cancellations, we obtain $2^{22}$.
\\
\textbf{Topic} :Probability\\
\textbf{Book} :Putnam and Beyond\\
\textbf{Final Answer} :\\


\textbf{Problem Statement} :
696. Compute the product

$$
\left(\sqrt{3}+\tan 1^{\circ}\right)\left(\sqrt{3}+\tan 2^{\circ}\right) \cdots\left(\sqrt{3}+\tan 29^{\circ}\right) .
$$
\\
\textbf{Solution} :
696. We can write

$$
\begin{aligned}
\sqrt{3}+\tan n^{\circ} &=\tan 60^{\circ}+\tan n^{\circ}=\frac{\sin 60^{\circ}}{\cos 60^{\circ}}+\frac{\sin n^{\circ}}{\cos n^{\circ}} \\
&=\frac{\sin \left(60^{\circ}+n^{\circ}\right)}{\cos 60^{\circ} \cos n^{\circ}}=2 \cdot \frac{\sin \left(60^{\circ}+n^{\circ}\right)}{\cos n^{\circ}}=2 \cdot \frac{\cos \left(30^{\circ}-n^{\circ}\right)}{\cos n^{\circ}} .
\end{aligned}
$$

And the product telescopes as follows:

$$
\prod_{n=1}^{29}\left(\sqrt{3}+\tan n^{\circ}\right)=2^{29} \prod_{n=1}^{29} \frac{\cos \left(30^{\circ}-n^{\circ}\right)}{\cos n^{\circ}}=2^{29} \cdot \frac{\cos 29^{\circ} \cos 28^{\circ} \cdots \cos 1^{\circ}}{\cos 1^{\circ} \cos 2^{\circ} \cdots \cos 29^{\circ}}=2^{29} .
$$

\section{(T. Andreescu)}\
\textbf{Topic} :Probability\\
\textbf{Book} :Putnam and Beyond\\
\textbf{Final Answer} :\\


\textbf{Problem Statement} :
699. Let $k$ be a positive integer. The sequence $\left(a_{n}\right)_{n}$ is defined by $a_{1}=1$, and for $n \geq 2$, $a_{n}$ is the $n$th positive integer greater than $a_{n-1}$ that is congruent to $n$ modulo $k$. Find $a_{n}$ in closed form.
\\
\textbf{Solution} :
699. Because $a_{n-1} \equiv n-1(\bmod k)$, the first positive integer greater than $a_{n-1}$ that is congruent to $n$ modulo $k$ must be $a_{n-1}+1$. The $n$th positive integer greater than $a_{n-1}$ that is congruent to $n$ modulo $k$ is simply $(n-1) k$ more than the first positive integer greater than $a_{n-1}$ that satisfies this condition. Therefore, $a_{n}=a_{n-1}+1+(n-1) k$. Solving this recurrence gives

$$
a_{n}=n+\frac{(n-1) n k}{2} .
$$

(Austrian Mathematical Olympiad, 1997)
\\
\textbf{Topic} :Probability\\
\textbf{Book} :Putnam and Beyond\\
\textbf{Final Answer} :\\


\textbf{Problem Statement} :
701. Find all functions $f: \mathbb{N} \rightarrow \mathbb{N}$ satisfying

$$
f(n)+2 f(f(n))=3 n+5, \quad \text { for all } n \in \mathbb{N} .
$$
\\
\textbf{Solution} :
701. From $f(1)+2 f(f(1))=8$ we deduce that $f(1)$ is an even number between 1 and 6 , that is, $f(1)=2,4$, or 6 . If $f(1)=2$ then $2+2 f(2)=8$, so $f(2)=3$. Continuing with $3+2 f(3)=11$, we obtain $f(3)=4$, and formulate the conjecture that $f(n)=n+1$ for all $n \geq 1$. And indeed, in an inductive manner we see that $f(n)=n+1$ implies $n+1+2 f(n+1)=3 n+5$; hence $f(n+1)=n+2$.

The case $f(1)=4$ gives $4+2 f(4)=8$, so $f(4)=2$. But then $2+2 f(f(4))=17$, which cannot hold for reasons of parity. Also, if $f(1)=6$, then $6+2 f(6)=8$, so $f(6)=1$. This cannot happen, because $f(6)+2 f(f(6))=1+2 \cdot 6$, which does not equal $3 \cdot 6+5$

We conclude that $f(n)=n+1, n \geq 1$, is the unique solution to the functional equation.
\\
\textbf{Topic} :Probability\\
\textbf{Book} :Putnam and Beyond\\
\textbf{Final Answer} :\\


\textbf{Problem Statement} :
702. Find all functions $f: \mathbb{Z} \rightarrow \mathbb{Z}$ with the property that

$$
2 f(f(x))-3 f(x)+x=0, \quad \text { for all } x \in \mathbb{Z} .
$$
\\
\textbf{Solution} :
702. Let $g(x)=f(x)-x$. The given equation becomes $g(x)=2 g(f(x))$. Iterating, we obtain that $g(x)=2^{n} f^{(n)}(x)$ for all $x \in \mathbb{Z}$, where $f^{(n)}(x)$ means $f$ composed $n$ times with itself. It follows that for every $x \in \mathbb{Z}, g(x)$ is divisible by all powers of 2 , so $g(x)=0$. Therefore, the only function satisfying the condition from the statement is $f(x)=x$ for all $x$.

(Revista Matematică din Timişoara (Timişoara Mathematics Gazette), proposed by L. Funar)
\\
\textbf{Topic} :Probability\\
\textbf{Book} :Putnam and Beyond\\
\textbf{Final Answer} :\\


\textbf{Problem Statement} :
715. For a positive integer $n$ and a real number $x$, compute the sum

$$
\sum_{0 \leq i<j \leq n}\left\lfloor\frac{x+i}{j}\right\rfloor .
$$
\\
\textbf{Solution} :
715. Denote the sum in question by $S_{n}$. Observe that

$$
\begin{aligned}
S_{n}-S_{n-1} &=\left\lfloor\frac{x}{n}\right\rfloor+\left\lfloor\frac{x+1}{n}\right\rfloor+\cdots+\left\lfloor\frac{x+n-1}{n}\right\rfloor \\
&=\left\lfloor\frac{x}{n}\right\rfloor+\left\lfloor\frac{x}{n}+\frac{1}{n}\right\rfloor+\cdots+\left\lfloor\frac{x}{n}+\frac{n-1}{n}\right\rfloor,
\end{aligned}
$$

and, according to Hermite's identity,

$$
S_{n}-S_{n-1}=\left\lfloor n \frac{x}{n}\right\rfloor=\lfloor x\rfloor .
$$

Because $S_{1}=\lfloor x\rfloor$, it follows that $S_{n}=n\lfloor x\rfloor$ for all $n \geq 1$.

(S. Savchev, T. Andreescu, Mathematical Miniatures, MAA, 2002)
\\
\textbf{Topic} :Probability\\
\textbf{Book} :Putnam and Beyond\\
\textbf{Final Answer} :\\


\textbf{Problem Statement} :
735. Solve in positive integers the equation

$$
x^{x+y}=y^{y-x} .
$$
\\
\textbf{Solution} :
735. The numbers $x$ and $y$ have the same prime factors,

$$
x=\prod_{i=1}^{k} p_{i}^{\alpha_{i}}, \quad y=\prod_{i=1}^{k} p_{i}^{\beta_{i}} .
$$

The equality from the statement can be written as

$$
\prod_{i=1}^{k} p_{i}^{\alpha_{i}(x+y)}=\prod_{i=1}^{k} p_{i}^{\beta_{i}(y-x)} ;
$$

hence $\alpha_{i}(y+x)=\beta_{i}(y-x)$ for $i=1,2, \ldots, k$. From here we deduce that $\alpha_{i}<\beta_{i}$, $i=1,2, \ldots, k$, and therefore $x$ divides $y$. Writing $y=z x$, the equation becomes $x^{x(z+1)}=(x z)^{x(z-1)}$, which implies $x^{2}=z^{z-1}$ and then $y^{2}=(x z)^{2}=z^{z+1}$. A power is a perfect square if either the base is itself a perfect square or if the exponent is even. For $z=t^{2}, t \geq 1$, we have $x=t^{t^{2}-1}, y=t^{t^{2}+1}$, which is one family of solutions. For $z-1=2 s, s \geq 0$, we obtain the second family of solutions $x=(2 s+1)^{s}, y=(2 s+1)^{s+1}$.

(Austrian-Polish Mathematics Competition, 1999, communicated by I. Cucurezeanu)
\\
\textbf{Topic} :Probability\\
\textbf{Book} :Putnam and Beyond\\
\textbf{Final Answer} :\\


\textbf{Problem Statement} :
737. Find all composite positive integers $n$ for which it is possible to arrange all divisors of $n$ that are greater than 1 in a circle such that no two adjacent divisors are relatively prime.
\\
\textbf{Solution} :
737. The only numbers that do not have this property are the products of two distinct primes.

Let $n$ be the number in question. If $n=p q$ with $p, q$ primes and $p \neq q$, then any cycle formed by $p, q, p q$ will have $p$ and $q$ next to each other. This rules out numbers of this form.

For any other number $n=p_{1}^{\alpha_{1}} p_{2}^{\alpha_{2}} \cdots p_{k}^{\alpha_{k}}$, with $k \geq 1, \alpha_{i} \geq 1$ for $i=1,2, \ldots, k$ and $\alpha_{1}+\alpha_{2} \geq 3$ if $k=2$, arrange the divisors of $n$ around the circle according to the following algorithm. First, we place $p_{1}, p_{2}, \ldots, p_{k}$ arranged clockwise around the circle in increasing order of their indices. Second, we place $p_{i} p_{i+1}$ between $p_{i}$ and $p_{i+1}$ for $i=1, \ldots, k-1$. (Note that the text has $p_{i+i}$, which is a typo and lets $i$ go up to $k$, which is a problem if $k=2$, since $p_{1} p_{2}$ gets placed twice.) Third, we place $n$ between $p_{k}$ and $p_{1}$. Note that at this point every pair of consecutive numbers has a common factor and each prime $p_{i}$ occurs as a common factor for some pair of adjacent numbers. Now for any remaining divisor of $n$ we choose a prime $p_{i}$ that divides it and place it between $p_{i}$ and one of its neighbors.

(USA Mathematical Olympiad, 2005, proposed by Z. Feng) 
\\
\textbf{Topic} :Probability\\
\textbf{Book} :Putnam and Beyond\\
\textbf{Final Answer} :\\


\textbf{Problem Statement} :
744. Find all positive integers $n$ such that $n$ ! ends in exactly 1000 zeros.
\\
\textbf{Solution} :
744. There are clearly more 2's than 5's in the prime factorization of $n$ !, so it suffices to solve the equation

$$
\left\lfloor\frac{n}{5}\right\rfloor+\left\lfloor\frac{n}{5^{2}}\right\rfloor+\left\lfloor\frac{n}{5^{3}}\right\rfloor+\cdots=1000 .
$$

On the one hand,

$$
\left\lfloor\frac{n}{5}\right\rfloor+\left\lfloor\frac{n}{5^{2}}\right\rfloor+\left\lfloor\frac{n}{5^{3}}\right\rfloor+\cdots<\frac{n}{5}+\frac{n}{5^{2}}+\frac{n}{5^{3}}+\cdots=\frac{n}{5} \cdot \frac{1}{1-\frac{1}{5}}=\frac{n}{4},
$$

and hence $n>4000$. On the other hand, using the inequality $\lfloor a\rfloor>a-1$, we have

$$
\begin{aligned}
1000 &>\left(\frac{n}{5}-1\right)+\left(\frac{n}{5^{2}}-1\right)+\left(\frac{n}{5^{3}}-1\right)+\left(\frac{n}{5^{4}}-1\right)+\left(\frac{n}{5^{5}}-1\right) \\
&=\frac{n}{5}\left(1+\frac{1}{5}+\frac{1}{5^{2}}+\frac{1}{5^{3}}+\frac{1}{5^{4}}\right)-5=\frac{n}{5} \cdot \frac{1-\left(\frac{1}{5}\right)^{5}}{1-\frac{1}{5}}-5,
\end{aligned}
$$

so

$$
n<\frac{1005 \cdot 4 \cdot 3125}{3124}<4022 .
$$

We have narrowed down our search to $\{4001,4002, \ldots, 4021\}$. Checking each case with Polignac's formula, we find that the only solutions are $n=4005,4006,4007$, 4008, and 4009.
\\
\textbf{Topic} :Probability\\
\textbf{Book} :Putnam and Beyond\\
\textbf{Final Answer} :\\


\textbf{Problem Statement} :
752. Solve in positive integers the equation

$$
2^{x} \cdot 3^{y}=1+5^{z} .
$$
\\
\textbf{Solution} :
752. Reducing modulo 4 , the right-hand side of the equation becomes equal to 2 . So the left-hand side is not divisible by 4 , which means that $x=1$. If $y>1$, then reducing modulo 9 we find that $z$ has to be divisible by 6 . A reduction modulo 6 makes the lefthand side 0 , while the right-hand side would be $1+(-1)^{z}=2$. This cannot happen. Therefore, $y=1$, and we obtain the unique solution $x=y=z=1$.

(Matematika v Škole (Mathematics in Schools), 1979, proposed by I. Mihailov)
\\
\textbf{Topic} :Probability\\
\textbf{Book} :Putnam and Beyond\\
\textbf{Final Answer} :\\


\textbf{Problem Statement} :
769. For each positive integer $n$, find the greatest common divisor of $n !+1$ and $(n+1) !$.
\\
\textbf{Solution} :
769. If $n+1$ is composite, then each prime divisor of $(n+1)$ ! is less than $n$, which also divides $n$ !. Then it does not divide $n !+1$. In this case the greatest common divisor is 1 .

If $n+1$ is prime, then by the same argument the greatest common divisor can only be a power of $n+1$. Wilson's theorem implies that $n+1$ divides $n !+1$. However, $(n+1)^{2}$ does not divide $(n+1)$ !, and thus the greatest common divisor is $(n+1)$.

(Irish Mathematical Olympiad, 1996)
\\
\textbf{Topic} :Probability\\
\textbf{Book} :Putnam and Beyond\\
\textbf{Final Answer} :\\


\textbf{Problem Statement} :
783. Devise a scheme by which a bank can transmit to its customers secure information over the Internet. Only the bank (and not the customers) is in the possession of the secret prime numbers $p$ and $q$.
\\
\textbf{Solution} :
783. The customer picks a number $k$ and transmits it securely to the bank using the algorithm described in the essay. Using the two large prime numbers $p$ and $q$, the bank finds $m$ such that $k m \equiv 1(\bmod (p-1)(q-1))$. If $\alpha$ is the numerical information that the customer wants to receive, the bank computes $\alpha^{m}(\bmod n)$, then transmits the answer $\beta$ to the customer. The customer computes $\beta^{k}(\bmod n)$. By Euler's theorem, this is $\alpha$. Success!
\\
\textbf{Topic} :Probability\\
\textbf{Book} :Putnam and Beyond\\
\textbf{Final Answer} :\\


\textbf{Problem Statement} :
784. A group of United Nations experts is investigating the nuclear program of a country. While they operate in that country, their findings should be handed over to the Ministry of Internal Affairs of the country, which verifies the document for leaks of classified information, then submits it to the United Nations. Devise a scheme by which the country can read the document but cannot modify its contents without destroying the information.
\textbf{Solution} :
784. As before, let $p$ and $q$ be two large prime numbers known by the United Nations experts alone. Let also $k$ be an arbitrary secret number picked by these experts with the property that $\operatorname{gcd}(k,(p-1)(q-1))=1$. The number $n=p q$ and the inverse $m$ of $k$ modulo $\phi(n)=(p-1)(q-1)$ are provided to both the country under investigation and to the United Nations.

The numerical data $\alpha$ that comprises the findings of the team of experts is raised to the power $k$, then reduced modulo $n$. The answer $\beta$ is handed over to the country. Computing $\beta^{m}$ modulo $n$, the country can read the data. But it cannot encrypt fake data, since it does not know the number $k$.
\\
\textbf{Topic} :Probability\\
\textbf{Book} :Putnam and Beyond\\
\textbf{Final Answer} :\\


\textbf{Problem Statement} :
785. An old woman went to the market and a horse stepped on her basket and smashed her eggs. The rider offered to pay for the eggs and asked her how many there were. She did not remember the exact number, but when she had taken them two at a time there was one egg left, and the same happened when she took three, four, five, and six at a time. But when she took them seven at a time, they came out even. What is the smallest number of eggs she could have had? 
\\
\textbf{Solution} :
785. We are to find the smallest positive solution to the system of congruences

$$
\begin{aligned}
&x \equiv 1(\bmod 60), \\
&x \equiv 0(\bmod 7) .
\end{aligned}
$$

The general solution is $7 b_{1}+420 t$, where $b_{1}$ is the inverse of 7 modulo 60 and $t$ is an integer. Since $b_{1}$ is a solution to the Diophantine equation $7 b_{1}+60 y=1$, we find it using Euclid's algorithm. Here is how to do it: $60=8 \cdot 7+4,7=1 \cdot 4+3,4=1 \cdot 3+1$. Then

$$
\begin{aligned}
1 &=4-1 \cdot 3=4-1 \cdot(7-1 \cdot 4)=2 \cdot 4-7=2 \cdot(60-8 \cdot 7)-7 \\
&=2 \cdot 60-17 \cdot 7
\end{aligned}
$$

Hence $b_{1}=-17$, and the smallest positive number of the form $7 b_{1}+420 t$ is $-7 \cdot 17+$ $420 \cdot 1=301$.

(Brahmagupta)
\\
\textbf{Topic} :Probability\\
\textbf{Book} :Putnam and Beyond\\
\textbf{Final Answer} :\\


\textbf{Problem Statement} :
790. Is there a sequence of positive integers in which every positive integer occurs exactly once and for every $k=1,2,3, \ldots$ the sum of the first $k$ terms is divisible by $k$ ?
\\
\textbf{Solution} :
790. We construct such a sequence recursively. Suppose that $a_{1}, a_{2}, \ldots, a_{m}$ have been chosen. Set $s=a_{1}+a_{2}+\cdots+a_{m}$, and let $n$ be the smallest positive integer that is not yet a term of the sequence. By the Chinese Remainder Theorem, there exists $t$ such that $t \equiv-s(\bmod (m+1))$, and $t \equiv-s-n(\bmod (m+2))$. We can increase $t$ by a suitably large multiple of $(m+1)(m+2)$ to ensure that it does not equal any of $a_{1}, a_{2}, \ldots, a_{m}$. Then $a_{1}, a_{2}, \ldots, a_{m}, t, n$ is also a sequence with the desired property. Indeed, $a_{1}+a_{2}+\cdots+a_{m}+t=s+t$ is divisible by $m+1$ and $a_{1}+\cdots+a_{m}+t+n=s+t+n$ is divisible by $m+2$. Continue the construction inductively. Observe that the algorithm ensures that $1, \ldots, m$ all occur among the first $2 m$ terms.

(Russian Mathematical Olympiad, 1995)
\\
\textbf{Topic} :Probability\\
\textbf{Book} :Putnam and Beyond\\
\textbf{Final Answer} :\\


\textbf{Problem Statement} :
804. Solve the following equation in positive integers:

$$
x^{2}+y^{2}=1997(x-y) .
$$
\textbf{Solution} :
804. Here is how to transform the equation from the statement into a Pythagorean equation:

$$
\begin{aligned}
x^{2}+y^{2} &=1997(x-y), \\
2\left(x^{2}+y^{2}\right) &=2 \cdot 1997(x-y), \\
(x+y)^{2}+(x-y)^{2}-2 \cdot 1997(x-y) &=0, \\
(x+y)^{2}+(1997-x+y)^{2} &=1997^{2} .
\end{aligned}
$$

Because $x$ and $y$ are positive integers, $0<x+y<1997$, and for the same reason $0<1997-x+y<1997$. The problem reduces to solving the Pythagorean equation $a^{2}+b^{2}=1997^{2}$ in positive integers. Since 1997 is prime, the greatest common divisor of $a$ and $b$ is 1 . Hence there exist coprime positive integers $u>v$ with the greatest common divisor equal to 1 such that

$$
1997=u^{2}+v^{2}, \quad a=2 u v, \quad b=u^{2}-v^{2} .
$$

Because $u$ is the larger of the two numbers, $\frac{1997}{2}<u^{2}<1997$; hence $33 \leq u \leq 44$. There are 12 cases to check. Our task is simplified if we look at the equality $1997=u^{2}+v^{2}$ and realize that neither $u$ nor $v$ can be divisible by 3 . Moreover, looking at the same equality modulo 5 , we find that $u$ and $v$ can only be 1 or $-1$ modulo 5 . We are left with the cases $m=34,41$, or 44 . The only solution is $(m, n)=(34,29)$. Solving $x+y=2 \cdot 34 \cdot 29$ and $1997-x+y=34^{2}-29^{2}$, we obtain $x=1827, y=145$. Solving $x+y=34^{2}-29^{2}$, $1997-x+y=2 \cdot 34 \cdot 29$, we obtain $(x, y)=(170,145)$. These are the two solutions to the equation.

(Bulgarian Mathematical Olympiad, 1997) 
\\
\textbf{Topic} :Probability\\
\textbf{Book} :Putnam and Beyond\\
\textbf{Final Answer} :\\


\textbf{Problem Statement} :
805. Find a solution to the Diophantine equation

$$
x^{2}-\left(m^{2}+1\right) y^{2}=1,
$$

where $m$ is a positive integer.
\\
\textbf{Solution} :
805. One can verify that $x=2 m^{2}+1$ and $y=2 m$ is a solution. (Diophantus)
\\
\textbf{Topic} :Probability\\
\textbf{Book} :Putnam and Beyond\\
\textbf{Final Answer} :\\


\textbf{Problem Statement} :
810. Find the positive integer solutions to the equation

$$
(x-y)^{5}=x^{3}-y^{3} .
$$
\\
\textbf{Solution} :
810. One family of solutions is of course $(n, n), n \in \mathbb{N}$. Let us see what other solutions the equation might have. Denote by $t$ the greatest common divisor of $x$ and $y$, and let $u=\frac{x}{t}, v=\frac{y}{t}$. The equation becomes $t^{5}(u-v)^{5}=t^{3}\left(u^{3}-v^{3}\right)$. Hence

$$
t^{2}(u-v)^{4}=\frac{u^{3}-v^{3}}{u-v}=u^{2}+u v+v^{2}=(u-v)^{2}+3 u v
$$

or $(u-v)^{2}\left[t^{2}(u-v)^{2}-1\right]=3 u v$. It follows that $(u-v)^{2}$ divides $3 u v$, and since $u$ and $v$ are relatively prime and $u>v$, this can happen only if $u-v=1$. We obtain the equation $3 v(v+1)=t^{2}-1$, which is the same as

$$
(v+1)^{3}-v^{3}=t^{2}
$$

This was solved in the previous problem. The solutions to the original equation are then given by $x=(v+1) t, y=v t$, for any solution $(v, t)$ to this last equation.

\section{(A. Rotkiewicz)}\
\textbf{Topic} :Probability\\
\textbf{Book} :Putnam and Beyond\\
\textbf{Final Answer} :\\


\textbf{Problem Statement} :
813. Find all integer solutions $(x, y)$ to the equation

$$
x^{2}+3 x y+4006(x+y)+2003^{2}=0 .
$$
\\
\textbf{Solution} :
813. First solution: This solution is based on an idea that we already encountered in the section on factorizations and divisibility. Solving for $y$, we obtain

$$
y=-\frac{x^{2}+4006 x+2003^{2}}{3 x+4006} .
$$

To make the expression on the right easier to handle we multiply both sides by 9 and write

$$
9 y=-3 x-8012-\frac{2003^{2}}{3 x+4006} .
$$

If $(x, y)$ is an integer solution to the given equation, then $3 x+4006$ divides $2003^{2}$. Because 2003 is a prime number, we have $3 x+4006 \in\left\{\pm 1, \pm 2003, \pm 2003^{2}\right\}$. Working modulo 3 we see that of these six possibilities, only $1,-2003$, and $2003^{2}$ yield integer solutions for $x$. We deduce that the equation from the statement has three solutions: $(-1334,-446224),(-2003,0)$, and $(1336001,-446224)$.

Second solution: Rewrite the equation as

$$
(3 x+4006)(3 x+9 y+8012)=-2003^{2} .
$$

This yields a linear system

$$
\begin{aligned}
3 x+4006 &=d, \\
3 x+9 y+8012 &=-\frac{2003^{2}}{d},
\end{aligned}
$$

where $d$ is a divisor of $-2003^{2}$. Since 2003 is prime, one has to check the cases $d=$ $\pm 1, \pm 2003, \pm 2003^{2}$, which yield the above solutions.

(American Mathematical Monthly, proposed by Wu Wei Chao) 
\\
\textbf{Topic} :Probability\\
\textbf{Book} :Putnam and Beyond\\
\textbf{Final Answer} :\\


\textbf{Problem Statement} :
816. Find all nonnegative integers $x, y, z, w$ satisfying

$$
4^{x}+4^{y}+4^{z}=w^{2} .
$$
\\
\textbf{Solution} :
816. If $x \leq y \leq z$, then since $4^{x}+4^{y}+4^{z}$ is a perfect square, it follows that the number $1+4^{y-x}+4^{z-x}$ is also a perfect square. Then there exist an odd integer $t$ and a positive integer $m$ such that

$$
1+4^{y-x}+4^{z-x}=\left(1+2^{m} t\right)^{2} .
$$

It follows that

$$
4^{y-x}\left(1+4^{z-x}\right)=2^{m+1} t\left(1+2^{m-1} t\right) ;
$$

hence $m=2 y-2 x-1$. From $1+4^{z-x}=t+2^{m-1} t^{2}$, we obtain

$$
t-1=4^{y-x-1}\left(4^{z-2 y+x+1}-t^{2}\right)=4^{y-x-1}\left(2^{z-2 y+x+1}+t\right)\left(2^{z-2 y+x+1}-t\right) .
$$

Since $2^{z-2 y+x+1}+t>t$, this equality can hold only if $t=1$ and $z=2 y-x-1$. The solutions are of the form $(x, y, 2 y-x-1)$ with $x, y$ nonnegative integers.
\\
\textbf{Topic} :Probability\\
\textbf{Book} :Putnam and Beyond\\
\textbf{Final Answer} :\\


\textbf{Problem Statement} :
821. Let $A$ and $B$ be two sets. Find all sets $X$ with the property that

$$
\begin{aligned}
A \cap X &=B \cap X=A \cap B, \\
A \cup B \cup X=A \cup B .
\end{aligned}
$$
\\
\textbf{Solution} :
821. The relation from the statement implies

$$
(A \cap X) \cup(B \cap X)=A \cap B .
$$

Applying de Morgan's law, we obtain

$$
(A \cup B) \cap X=A \cap B .
$$

But the left-hand side is equal to $(A \cup B \cup X) \cap X$, and this is obviously equal to $X$. Hence $X=A \cap B$.

(Russian Mathematics Competition, 1977)
\\
\textbf{Topic} :Probability\\
\textbf{Book} :Putnam and Beyond\\
\textbf{Final Answer} :\\


\textbf{Problem Statement} :
824. Let $M$ be a subset of $\{1,2,3, \ldots, 15\}$ such that the product of any three distinct elements of $M$ is not a square. Determine the maximum number of elements in $M$.
\\
\textbf{Solution} :
824. Note that the product of the three elements in each of the sets $\{1,4,9\},\{2,6,12\}$, $\{3,5,15\}$, and $\{7,8,14\}$ is a square. Hence none of these sets is a subset of $M$. Because they are disjoint, it follows that $M$ has at most $15-4=11$ elements.

Since 10 is not an element of the aforementioned sets, if $10 \notin M$, then $M$ has at most 10 elements. Suppose $10 \in M$. Then none of $\{2,5\},\{6,15\},\{1,4,9\}$, and $\{7,8,14\}$ is a subset of $M$. If $\{3,12\} \not \subset M$, it follows again that $M$ has at most 10 elements. If $\{3,12\} \subset M$, then none of $\{1\},\{4\},\{9\},\{2,6\},\{5,15\}$, and $\{7,8,14\}$ is a subset of $M$, and then $M$ has at most 9 elements. We conclude that $M$ has at most 10 elements in any case.

Finally, it is easy to verify that the subset

$$
M=\{1,4,5,6,7,10,11,12,13,14\}
$$

has the desired property. Hence the maximum number of elements in $M$ is 10 .

(short list of the 35th International Mathematical Olympiad, 1994, proposed by Bulgaria)
\\
\textbf{Topic} :Probability\\
\textbf{Book} :Putnam and Beyond\\
\textbf{Final Answer} :\\


\textbf{Problem Statement} :
829. For each permutation $a_{1}, a_{2}, \ldots, a_{10}$ of the integers $1,2,3, \ldots, 10$, form the sum

$$
\left|a_{1}-a_{2}\right|+\left|a_{3}-a_{4}\right|+\left|a_{5}-a_{6}\right|+\left|a_{7}-a_{8}\right|+\left|a_{9}-a_{10}\right| .
$$

Find the average value of all such sums.
\\
\textbf{Solution} :
829. We solve the more general case of the permutations of the first $2 n$ positive integers, $n \geq 1$. The average of the sum

$$
\sum_{k=1}^{n}\left|a_{2 k-1}-a_{2 k}\right|
$$

is just $n$ times the average value of $\left|a_{1}-a_{2}\right|$, because the average value of $\left|a_{2 i-1}-a_{2 i}\right|$ is the same for all $i=1,2, \ldots, n$. When $a_{1}=k$, the average value of $\left|a_{1}-a_{2}\right|$ is

$$
\begin{aligned}
\frac{(k-1)+(k-2)+\cdots+1+1+2+\cdots+(2 n-k)}{2 n-1} \\
=& \frac{1}{2 n-1}\left[\frac{k(k-1)}{2}+\frac{(2 n-k)(2 n-k+1)}{2}\right] \\
=& \frac{k^{2}-(2 n+1) k+n(2 n+1)}{2 n-1} .
\end{aligned}
$$

It follows that the average value of the sum is

$$
\begin{aligned}
n \cdot \frac{1}{2 n} & \sum_{k=1}^{2 n} \frac{k^{2}-(2 n+1) k+n(2 n+1)}{2 n-1} \\
&=\frac{1}{4 n-2}\left[\frac{2 n(2 n+1)(4 n+1)}{6}-(2 n+1) \frac{2 n(2 n+1)}{2}+2 n^{2}(2 n+1)\right] \\
&=\frac{n(2 n+1)}{3} .
\end{aligned}
$$

For our problem $n=5$ and the average of the sums is $\frac{55}{3}$.

(American Invitational Mathematics Examination, 1996)
\\
\textbf{Topic} :Probability\\
\textbf{Book} :Putnam and Beyond\\
\textbf{Final Answer} :\\


\textbf{Problem Statement} :
830. Find the number of permutations $a_{1}, a_{2}, a_{3}, a_{4}, a_{5}, a_{6}$ of the numbers $1,2,3,4,5,6$ that can be transformed into $1,2,3,4,5,6$ through exactly four transpositions (and not fewer).
\\
\textbf{Solution} :
830. The condition from the statement implies that any such permutation has exactly two disjoint cycles, say $\left(a_{i_{1}}, \ldots, a_{i_{r}}\right)$ and $\left(a_{i_{r+1}}, \ldots, a_{i_{6}}\right)$. This follows from the fact that in order to transform a cycle of length $r$ into the identity $r-1$, transpositions are needed. Moreover, we can only have $r=5,4$, or 3 .

When $r=5$, there are $\left(\begin{array}{l}6 \\ 1\end{array}\right)$ choices for the number that stays unpermuted. There are $(5-1) !$ possible cycles, so in this case we have $6 \times 4 !=144$ possibilities. When $r=4$, there are $\left(\begin{array}{l}6 \\ 4\end{array}\right)$ ways to split the numbers into the two cycles (two cycles are needed and not just one). One cycle is a transposition. There are $(4-1) !=6$ choices for the other. Hence in this case the number is 90 . Note that here exactly four transpositions are needed.

Finally, when $r=3$, then there are $\left(\begin{array}{l}6 \\ 3\end{array}\right) \times(3-1) ! \times(3-1) !=40$ cases. Therefore, the answer to the problem is $144+90+40=274$.

(Korean Mathematical Olympiad, 1999)
\\
\textbf{Topic} :Probability\\
\textbf{Book} :Putnam and Beyond\\
\textbf{Final Answer} :\\


\textbf{Problem Statement} :
833. Let $a_{1}, a_{2}, \ldots, a_{n}$ be a permutation of the numbers $1,2, \ldots, n$. We call $a_{i}$ a large integer if $a_{i}>a_{j}$ for all $i<j<n$. Find the average number of large integers over all permutations of the first $n$ positive integers.
\\
\textbf{Solution} :
833. Let $N(\sigma)$ be the number we are computing. Denote by $N_{i}(\sigma)$ the average number of large integers $a_{i}$. Taking into account the fact that after choosing the first $i-1$ numbers, the $i$ th is completely determined by the condition of being large, for any choice of the first $i-1$ numbers there are $(n-i+1)$ ! choices for the last $n-i+1$, from which $(n-i)$ ! contain a large integer in the $i$ th position. We deduce that $N_{i}(\sigma)=\frac{1}{n-i+1}$. The answer to the problem is therefore

$$
N(\sigma)=\sum_{i=1}^{n} N_{i}(\sigma)=1+\frac{1}{2}+\cdots+\frac{1}{n} .
$$

(19th W.L. Putnam Mathematical Competition, 1958)
\\
\textbf{Topic} :Probability\\
\textbf{Book} :Putnam and Beyond\\
\textbf{Final Answer} :\\


\textbf{Problem Statement} :
836. Let $n$ be an odd integer greater than 1 . Find the number of permutations $\sigma$ of the set $\{1,2, \ldots, n\}$ for which

$$
|\sigma(1)-1|+|\sigma(2)-2|+\cdots+|\sigma(n)-n|=\frac{n^{2}-1}{2} .
$$
\textbf{Solution} :
836. Expanding $|\sigma(k)-k|$ as $\pm \sigma(k) \pm k$ and reordering, we see that

$$
|\sigma(1)-1|+|\sigma(2)-2|+\cdots+|\sigma(n)-n|=\pm 1 \pm 1 \pm 2 \pm 2 \pm \cdots \pm n \pm n,
$$

for some choices of signs. The maximum of $|\sigma(1)-1|+|\sigma(2)-2|+\cdots+|\sigma(n)-n|$ is reached by choosing the smaller of the numbers to be negative and the larger to be positive, and is therefore equal to

$$
\begin{gathered}
2\left(-1-2-\cdots-\frac{n-1}{2}\right)-\frac{n+1}{2}+\frac{n+1}{2}+2\left(\frac{n+3}{2}+\cdots+n\right) \\
=-\left(1+\frac{n-1}{2}\right) \frac{n-1}{2}+\left(\frac{n+3}{2}+n\right) \frac{n-1}{2}=\frac{n^{2}-1}{2} .
\end{gathered}
$$

Therefore, in order to have $|\sigma(1)-1|+\cdots+|\sigma(n)-n|=\frac{n^{2}-1}{2}$, we must have

$$
\left\{\sigma(1), \ldots, \sigma\left(\frac{n-1}{2}\right)\right\} \subset\left\{\frac{n+1}{2}, \frac{n+3}{2}, \ldots, n\right\}
$$

and 

$$
\left\{\sigma\left(\frac{n+3}{2}\right), \sigma\left(\frac{n+5}{2}\right), \ldots, \sigma(n)\right\} \subset\left\{1,2, \ldots, \frac{n+1}{2}\right\} .
$$

Let $\sigma\left(\frac{n+1}{2}\right)=k$. If $k \leq \frac{n+1}{2}$, then

$$
\left\{\sigma(1), \ldots, \sigma\left(\frac{n-1}{2}\right)\right\}=\left\{\frac{n+3}{2}, \frac{n+5}{2}, \ldots, n\right\}
$$

and

$$
\left\{\sigma\left(\frac{n+3}{2}\right), \sigma\left(\frac{n+5}{2}\right), \ldots, \sigma(n)\right\}=\left\{1,2, \ldots, \frac{n+1}{2}\right\}-\{k\} .
$$

If $k \geq \frac{n+1}{2}$, then

$$
\left\{\sigma(1), \ldots, \sigma\left(\frac{n-1}{2}\right)\right\}=\left\{\frac{n+1}{2}, \frac{n+3}{2}, \ldots, n\right\}-\{k\}
$$

and

$$
\left\{\sigma\left(\frac{n+3}{2}\right), \sigma\left(\frac{n+5}{2}\right), \ldots, \sigma(n)\right\}=\left\{1,2, \ldots, \frac{n-1}{2}\right\} .
$$

For any value of $k$, there are $\left[\left(\frac{n-1}{2}\right) !\right]^{2}$ choices for the remaining values of $\sigma$, so there are

$$
n\left[\left(\frac{n-1}{2}\right) !\right]^{2}
$$

such permutations.

\section{(T. Andreescu)}\
\textbf{Topic} :Probability\\
\textbf{Book} :Putnam and Beyond\\
\textbf{Final Answer} :\\


\textbf{Problem Statement} :
837. In how many regions do $n$ great circles, any three nonintersecting, divide the surface of a sphere?
\\
\textbf{Solution} :
837. Let $f(n)$ be the desired number. We count immediately $f(1)=2, f(2)=4$. For the general case we argue inductively. Assume that we already have constructed $n$ circles. When adding the $(n+1)$ st, it intersects the other circles in $2 n$ points. Each of the $2 n$ arcs determined by those points splits some region in two. This produces the recurrence relation $f(n+1)=f(n)+2 n$. Iterating, we obtain

$$
f(n)=2+2+4+6+\cdots+2(n-1)=n^{2}-n+2
$$

(25th W.L. Putnam Mathematical Competition, 1965)
\\
\textbf{Topic} :Probability\\
\textbf{Book} :Putnam and Beyond\\
\textbf{Final Answer} :\\


\textbf{Problem Statement} :
838. In how many regions do $n$ spheres divide the three-dimensional space if any two intersect along a circle, no three intersect along a circle, and no four intersect at one point?
\\
\textbf{Solution} :
838. Again we try to derive a recursive formula for the number $F(n)$ of regions. But this time counting the number of regions added by a new sphere is not easy at all. The previous problem comes in handy. The first $n$ spheres determine on the $(n+1)$ st exactly $n^{2}-n+2$ regions. This is because the conditions from the statement give rise on the last sphere to a configuration of circles in which any two, but no three, intersect. And this is the only condition that we used in the solution to the previous problem. Each of the $n^{2}-n+2$ spherical regions divides some spatial region into two parts. This allows us to write the recursive formula

$$
F(n+1)=F(n)+n^{2}-n+2, \quad F(1)=2 .
$$

Iterating, we obtain

$$
\begin{aligned}
F(n) &=2+4+8+\cdots+\left[(n-1)^{2}-(n-1)+2\right]=\sum_{k=1}^{n-1}\left(k^{2}-k+2\right) \\
&=\frac{n^{3}-3 n^{2}+8 n}{3}
\end{aligned}
$$
\\
\textbf{Topic} :Probability\\
\textbf{Book} :Putnam and Beyond\\
\textbf{Final Answer} :\\


\textbf{Problem Statement} :
840. An equilateral triangle of side length $n$ is drawn with sides along a triangular grid of side length 1 . What is the maximum number of grid segments on or inside the triangle that can be marked so that no three marked segments form a triangle?
\\
\textbf{Solution} :
840. The grid is made up of $\frac{n(n+1)}{2}$ small equilateral triangles of side length 1 . In each of these triangles, at most 2 segments can be marked, so we can mark at most $\frac{2}{3} \cdot \frac{3 n(n+1)}{2}=$ $n(n+1)$ segments in all. Every segment points in one of three directions, so we can achieve the maximum $n(n+1)$ by marking all the segments pointing in two of the three directions.

(Russian Mathematical Olympiad, 1999)
\\
\textbf{Topic} :Probability\\
\textbf{Book} :Putnam and Beyond\\
\textbf{Final Answer} :\\


\textbf{Problem Statement} :
843. A circle of radius 1 rolls without slipping on the outside of a circle of radius $\sqrt{2}$. The contact point of the circles in the initial position is colored. Any time a point of one circle touches a colored point of the other, it becomes itself colored. How many colored points will the moving circle have after 100 revolutions?
\\
\textbf{Solution} :
843. It seems that the situation is complicated by successive colorings. But it is not! Observe that each time the moving circle passes through the original position, a new point will be colored. But this point will color the same points on the fixed circle. In short, only the first colored point on one circle contributes to newly colored points on the other; all other colored points follow in its footsteps. So there will be as many colored points on the small circle as there are points of coordinate $2 \pi k, k$ an integer, on the segment $[0,200 \pi \sqrt{2}]$. Their number is 

$$
\left\lfloor\frac{200 \pi \sqrt{2}}{2 \pi}\right\rfloor=\lfloor 100 \sqrt{2}\rfloor=141 \text {. }
$$

(Ukrainian Mathematical Olympiad)
\\
\textbf{Topic} :Probability\\
\textbf{Book} :Putnam and Beyond\\
\textbf{Final Answer} :\\


\textbf{Problem Statement} :
867. Find the general-term formula for the sequence $\left(y_{n}\right)_{n \geq 0}$ with $y_{0}=1$ and $y_{n}=$ $a y_{n-1}+b^{n}$ for $n \geq 1$, where $a$ and $b$ are two fixed distinct real numbers.
\\
\textbf{Solution} :
867. Let $G(x)=\sum_{n} y_{n} x^{n}$ be the generating function of the sequence. It satisfies the functional equation

$$
(1-a x) G(x)=1+b x+b x^{2}+\cdots=\frac{1}{1-b x}
$$

We find that

$$
G(x)=\frac{1}{(1-a x)(1-b x)}=\frac{A}{1-a x}+\frac{B}{1-b x}=\sum_{n}\left(A a^{n}+B b^{n}\right) x^{n}
$$

for some $A$ and $B$. It follows that $y_{n}=A a^{n}+B b^{n}$, and because $y_{0}=1$ and $y_{1}=a+b$, $A=\frac{a}{a-b}$ and $B=-\frac{b}{a-b}$. The general term of the sequence is therefore

$$
\frac{1}{a-b}\left(a^{n+1}-b^{n+1}\right) .
$$
\\
\textbf{Topic} :Probability\\
\textbf{Book} :Putnam and Beyond\\
\textbf{Final Answer} :\\


\textbf{Problem Statement} :
868. Compute the sums

$$
\sum_{k=1}^{n} k\left(\begin{array}{l}
n \\
k
\end{array}\right) \text { and } \sum_{k=1}^{n} \frac{1}{k+1}\left(\begin{array}{l}
n \\
k
\end{array}\right) \text {. }
$$
\\
\textbf{Solution} :
868. The first identity is obtained by differentiating $(x+1)^{n}=\sum_{k=1}^{n}\left(\begin{array}{c}n \\ k\end{array}\right) x^{k}$, then setting $x=1$. The answer is $n 2^{n-1}$. The second identity is obtained by integrating the same equality and then setting $x=1$, in which case the answer is $\frac{2^{n+1}}{n+1}$.
\\
\textbf{Topic} :Probability\\
\textbf{Book} :Putnam and Beyond\\
\textbf{Final Answer} :\\


\textbf{Problem Statement} :
870. Compute the sum

$$
\left(\begin{array}{l}
n \\
0
\end{array}\right)-\left(\begin{array}{l}
n \\
1
\end{array}\right)+\left(\begin{array}{l}
n \\
2
\end{array}\right)-\cdots+(-1)^{m}\left(\begin{array}{l}
n \\
m
\end{array}\right) .
$$
\\
\textbf{Solution} :
870. The sum is equal to the coefficient of $x^{n}$ in the expansion of

$$
x^{n}(1-x)^{n}+x^{n-1}(1-x)^{n}+\cdots+x^{n-m}(1-x)^{n} .
$$

This expression is equal to

$$
x^{n-m} \cdot \frac{1-x^{m+1}}{1-x}(1-x)^{n},
$$

which can be written as $\left(x^{n-m}-x^{n+1}\right)(1-x)^{n-1}$. Hence the sum is equal to $(-1)^{m}\left(\begin{array}{c}n-1 \\ m\end{array}\right)$ if $m<n$, and to 0 if $m=n$.
\\
\textbf{Topic} :Probability\\
\textbf{Book} :Putnam and Beyond\\
\textbf{Final Answer} :\\


\textbf{Problem Statement} :
871. Write in short form the sum

$$
\left(\begin{array}{l}
n \\
k
\end{array}\right)+\left(\begin{array}{c}
n+1 \\
k
\end{array}\right)+\left(\begin{array}{c}
n+2 \\
k
\end{array}\right)+\cdots+\left(\begin{array}{c}
n+m \\
k
\end{array}\right) .
$$
\\
\textbf{Solution} :
871. The sum from the statement is equal to the coefficient of $x^{k}$ in the expansion of $(1+x)^{n}+(1+x)^{n+1}+\cdots+(1+x)^{n+m}$. This expression can be written in compact form as 

$$
\frac{1}{x}\left((1+x)^{n+m+1}-(1+x)^{n}\right) .
$$

We deduce that the sum is equal to $\left(\begin{array}{c}n+m+1 \\ k+1\end{array}\right)-\left(\begin{array}{c}n \\ k+1\end{array}\right)$ for $k<n$ and to $\left(\begin{array}{c}n+m+1 \\ n+1\end{array}\right)$ for $k=n$. 
\\
\textbf{Topic} :Probability\\
\textbf{Book} :Putnam and Beyond\\
\textbf{Final Answer} :\\


\textbf{Problem Statement} :
875. Let $p$ be an odd prime number. Find the number of subsets of $\{1,2, \ldots, p\}$ with the sum of elements divisible by $p$.
\\
\textbf{Solution} :
875. The number of subsets with the sum of the elements equal to $n$ is the coefficient of $x^{n}$ in the product

$$
G(x)=(1+x)\left(1+x^{2}\right) \cdots\left(1+x^{p}\right) .
$$

We are asked to compute the sum of the coefficients of $x^{n}$ for $n$ divisible by $p$. Call this number $s(p)$. There is no nice way of expanding the generating function; instead we compute $s(p)$ using particular values of $G$. It is natural to try $p$ th roots of unity.

The first observation is that if $\xi$ is a $p$ th root of unity, then $\sum_{k=1}^{p} \xi^{p}$ is zero except when $\xi=1$. Thus if we sum the values of $G$ at the $p$ th roots of unity, only those terms with exponent divisible by $p$ will survive. To be precise, if $\xi$ is a $p$ th root of unity different from 1, then

$$
\sum_{k=1}^{p} G\left(\xi^{k}\right)=p s(p) .
$$

We are left with the problem of computing $G\left(\xi^{k}\right), k=1,2, \ldots, p$. For $k=p$, this is just $2^{p}$. For $k=1,2, \ldots, p-1$,

$$
\begin{aligned}
G\left(\xi^{k}\right) &=\prod_{j=1}^{p}\left(1+\xi^{k j}\right)=\prod_{j=1}^{p}\left(1+\xi^{j}\right)=(-1)^{p} \prod_{j=1}^{p}\left((-1)-\xi^{j}\right)=(-1)^{p}\left((-1)^{p}-1\right) \\
&=2
\end{aligned}
$$

We therefore have $p s(p)=2^{p}+2(p-1)=2^{p}+2 p-2$. The answer to the problem is $s(p)=\frac{2^{p}-2}{p}+2$. The expression is an integer because of Fermat's little theorem.

(T. Andreescu, Z. Feng, A Path to Combinatorics for Undergraduates, Birkhäuser $2004)$
\\
\textbf{Topic} :Probability\\
\textbf{Book} :Putnam and Beyond\\
\textbf{Final Answer} :\\


\textbf{Problem Statement} :
878. Let $A_{1}, A_{2}, \ldots, A_{n}, \ldots$ and $B_{1}, B_{2}, \ldots, B_{n}, \ldots$ be sequences of sets defined by $A_{1}=\emptyset, B_{1}=\{0\}, A_{n+1}=\left\{x+1 \mid x \in B_{n}\right\}, B_{n+1}=\left(A_{n} \cup B_{n}\right) \backslash\left(A_{n} \cap B_{n}\right)$. Determine all positive integers $n$ for which $B_{n}=\{0\}$.
\textbf{Solution} :
878. We use the same generating functions as in the previous problem. So to the set $A_{n}$ we associate the function 

$$
a_{n} x=\sum_{a=1}^{\infty} c_{a} x^{a}
$$

with $c_{a}=1$ if $a \in A_{n}$ and $c_{a}=0$ if $a \notin A_{n}$. To $B_{n}$ we associate the function $b_{n}(x)$ in a similar manner. These functions satisfy the recurrence $a_{1}(x)=0, b_{1}(x)=1$,

$$
\begin{aligned}
a_{n+1}(x) &=x b_{n}(x) \\
b_{n+1} & \equiv a_{n}(x)+b_{n}(x)(\bmod 2)
\end{aligned}
$$

From now on we understand all equalities modulo 2. Let us restrict our attention to the sequence of functions $b_{n}(x), n=1,2, \ldots$ It satisfies $b_{1}(x)=b_{2}(x)=1$,

$$
b_{n+1}(x)=b_{n}(x)+x b_{n-1}(x)
$$

We solve this recurrence the way one usually solves second-order recurrences, namely by finding two linearly independent solutions $p_{1}(x)$ and $p_{2}(x)$ satisfying

$$
p_{i}(x)^{n+1}=p_{i}(x)^{n}+x p_{i}(x)^{n-1}, \quad i=1,2 .
$$

Again the equality is to be understood modulo 2. The solutions $p_{1}(x)$ and $p_{2}(x)$ are formal power series whose coefficients are residue classes modulo 2 . They satisfy the "characteristic" equation

$$
p(x)^{2}=p(x)+x
$$

which can be rewritten as

$$
p(x)(p(x)+1)=x
$$

So $p_{1}(x)$ and $p_{2}(x)$ can be chosen as the factors of this product, and thus we may assume that $p_{1}(x)=x h(x)$ and $p_{2}(x)=1+p_{1}(x)$, where $h(x)$ is again a formal power series. Writing $p_{1}(x)=\sum \alpha_{a} x^{a}$ and substituting in the characteristic equation, we find that $\alpha_{1}=1, \alpha_{2 k}=\alpha_{k}^{2}$, and $\alpha_{2 k+1}=0$ for $k>1$. Therefore,

$$
p_{1}(x)=\sum_{k=0}^{\infty} x^{2^{k}}
$$

Since $p_{1}(x)+p_{2}(x)=p_{1}(x)^{2}+p_{2}(x)^{2}=1$, it follows that in general,

$$
b_{n}(x)=p_{1}(x)^{n}+p_{2}(x)^{n}=\left(\sum_{k=0}^{\infty} x^{2^{k}}\right)^{n}+\left(1+\sum_{k=0}^{\infty} x^{2^{k}}\right)^{n}, \quad \text { for } n \geq 1 .
$$

We emphasize again that this is to be considered modulo 2. In order for $b_{n}(x)$ to be identically equal to 1 modulo 2 , we should have

$$
\left(\left(\sum_{k=0}^{\infty} x^{2^{k}}\right)+1\right)^{n} \equiv\left(\sum_{k=0}^{\infty} x^{2^{k}}\right)^{n}+1(\bmod 2) .
$$

This obviously happens if $n$ is a power of 2 , since all binomial coefficients in the expansion are even.

If $n$ is not a power of 2 , say $n=2^{i}(2 j+1), j \geq 1$, then the smallest $m$ for which $\left(\begin{array}{l}n \\ m\end{array}\right)$ is odd is $2^{j}$. The left-hand side will contain an $x^{2^{j}}$ with coefficient equal to 1 , while the smallest nonzero power of $x$ on the right is $n$. Hence in this case equality cannot hold.

We conclude that $B_{n}=\{0\}$ if and only if $n$ is a power of 2 .

(Chinese Mathematical Olympiad)
\\
\textbf{Topic} :Probability\\
\textbf{Book} :Putnam and Beyond\\
\textbf{Final Answer} :\\


\textbf{Problem Statement} :
879. Find in closed form

$$
1 \cdot 2\left(\begin{array}{l}
n \\
2
\end{array}\right)+2 \cdot 3\left(\begin{array}{l}
n \\
3
\end{array}\right)+\cdots+(n-1) \cdot n\left(\begin{array}{l}
n \\
n
\end{array}\right) .
$$
\\
\textbf{Solution} :
879. We will count the number of committees that can be chosen from $n$ people, each committee having a president and a vice-president.

Choosing first a committee of $k$ people, the president and the vice-president can then be elected in $k(k-1)$ ways. It is necessary that $k \geq 2$. The committees with president and vice-president can therefore be chosen in

$$
1 \cdot 2\left(\begin{array}{l}
n \\
2
\end{array}\right)+2 \cdot 3\left(\begin{array}{l}
n \\
3
\end{array}\right)+\cdots+(n-1) \cdot n\left(\begin{array}{l}
n \\
n
\end{array}\right)
$$

ways.

But we can start by selecting first the president and the vice-president, and then adding the other members to the committee. From the $n$ people, the president and the vice-president can be selected in $n(n-1)$ ways. The remaining members of the committee can be selected in $2^{n-2}$ ways, since they are some subset of the remaining $n-2$ people. We obtain

$$
1 \cdot 2\left(\begin{array}{l}
n \\
2
\end{array}\right)+2 \cdot 3\left(\begin{array}{l}
n \\
3
\end{array}\right)+\cdots+(n-1) \cdot n\left(\begin{array}{l}
n \\
n
\end{array}\right)=n(n-1) 2^{n-2} .
$$
\\
\textbf{Topic} :Probability\\
\textbf{Book} :Putnam and Beyond\\
\textbf{Final Answer} :\\


\textbf{Problem Statement} :
890. A set $S$ containing four positive integers is called connected if for every $x \in S$ at least one of the numbers $x-1$ and $x+1$ belongs to $S$. Let $C_{n}$ be the number of connected subsets of the set $\{1,2, \ldots, n\}$.

(a) Evaluate $C_{7}$.

(b) Find a general formula for $C_{n}$.
\\
\textbf{Solution} :
890. Let $a<b<c<d$ be the members of a connected set $S$. Because $a-1$ does not belong to the set, it follows that $a+1 \in S$, hence $b=a+1$. Similarly, since $d+1 \notin S$, we deduce that $d-1 \in S$; hence $c=d-1$. Therefore, a connected set has the form $\{a, a+1, d-1, d\}$, with $d-a>2$.

(a) There are 10 connected subsets of the set $\{1,2,3,4,5,6,7\}$, namely,

$$
\begin{aligned}
&\{1,2,3,4\} ;\{1,2,4,5\} ;\{1,2,5,6\} ;\{1,2,6,7\}, \\
&\{2,3,4,5\} ;\{2,3,5,6\} ;\{2,3,6,7\}\{3,4,5,6\} ;\{2,4,6,7\} ; \text { and }\{4,5,6,7\} .
\end{aligned}
$$

(b) Call $D=d-a+1$ the diameter of the set $\{a, a+1, d-1, d\}$. Clearly, $D>3$ and $D \leq n-1+1=n$. For $D=4$ there are $n-3$ connected sets, for $D=5$ there are $n-4$ connected sets, and so on. Adding up yields

$$
C_{n}=1+2+3+\cdots+n-3=\frac{(n-3)(n-2)}{2},
$$

which is the desired formula.

(Romanian Mathematical Olympiad, 2006)
\\
\textbf{Topic} :Probability\\
\textbf{Book} :Putnam and Beyond\\
\textbf{Final Answer} :\\


\textbf{Problem Statement} :
893. Let $S$ be a finite set of points in the plane. A linear partition of $S$ is an unordered pair $\{A, B\}$ of subsets of $S$ such that $A \cup B=S, A \cap B=\emptyset$, and $A$ and $B$ lie on opposite sides of some straight line disjoint from $S$ ( $A$ or $B$ may be empty). Let $L_{S}$ be the number of linear partitions of $S$. For each positive integer $n$, find the maximum of $L_{S}$ over all sets $S$ of $n$ points.
\\
\textbf{Solution} :
893. First, it is not hard to see that a configuration that maximizes the number of partitions should have no three collinear points. After examining several cases we guess that the maximal number of partitions is $\left(\begin{array}{l}n \\ 2\end{array}\right)$. This is exactly the number of lines determined by two points, and we will use these lines to count the number of partitions. By pushing such a line slightly so that the two points lie on one of its sides or the other, we obtain a partition. Moreover, each partition can be obtained this way. There are $2\left(\begin{array}{l}n \\ 2\end{array}\right)$ such lines, obtained by pushing the lines through the $n$ points to one side or the other. However, each partition is counted at least twice this way, except for the partitions that come from the sides of the polygon that is the convex hull of the $n$ points, but those can be paired with the partitions that cut out one vertex of the convex hull from the others. Hence we have at most $2\left(\begin{array}{l}n \\ 2\end{array}\right) / 2=\left(\begin{array}{l}n \\ 2\end{array}\right)$ partitions.

Equality is achieved when the points form a convex $n$-gon, in which case $\left(\begin{array}{l}n \\ 2\end{array}\right)$ counts the pairs of sides that are intersected by the separating line.

(67th W.L. Putnam Mathematical Competition, 2006)
\\
\textbf{Topic} :Probability\\
\textbf{Book} :Putnam and Beyond\\
\textbf{Final Answer} :\\


\textbf{Problem Statement} :
896. A sheet of paper in the shape of a square is cut by a line into two pieces. One of the pieces is cut again by a line, and so on. What is the minimum number of cuts one should perform such that among the pieces one can find one hundred polygons with twenty sides.
\\
\textbf{Solution} :
896. At every cut the number of pieces grows by 1 , so after $n$ cuts we will have $n+1$ pieces.

Let us evaluate the total number of vertices of the polygons after $n$ cuts. After each cut the number of vertices grows by 2 if the cut went through two vertices, by 3 if the cut went through a vertex and a side, or by 4 if the cut went through two sides. So after $n$ cuts there are at most $4 n+4$ vertices.

Assume now that after $N$ cuts we have obtained the one hundred polygons with 20 sides. Since altogether there are $N+1$ pieces, besides the one hundred polygons there are $N+1-100$ other pieces. Each of these other pieces has at least 3 vertices, so the total number of vertices is $100 \cdot 20+(N-99) \cdot 3$. This number does not exceed $4 N+4$. Therefore,

$$
4 N+4 \geq 100 \cdot 20+(N-99) \cdot 3=3 N+1703 .
$$

We deduce that $N \geq 1699$. We can obtain one hundred polygons with twenty sides by making 1699 cuts in the following way. First, cut the square into 100 rectangles ( 99 cuts needed). Each rectangle is then cut through 16 cuts into a polygon with twenty sides and some triangles. We have performed a total of $99+100 \cdot 16=1699$ cuts.

(Kvant (Quantum), proposed by I. Bershtein)
\\
\textbf{Topic} :Probability\\
\textbf{Book} :Putnam and Beyond\\
\textbf{Final Answer} :\\


\textbf{Problem Statement} :
898. Let $m, n, p, q, r, s$ be positive integers such that $p<r<m$ and $q<s<n$. In how many ways can one travel on a rectangular grid from $(0,0)$ to $(m, n)$ such that at each step one of the coordinates increases by one unit and such that the path avoids the points $(p, q)$ and $(r, s)$ ? 
\\
\textbf{Solution} :
898. First, let us forget about the constraint and count the number of paths from $(0,0)$ and $(m, n)$ such that at each step one of the coordinates increases by 1 . There are a total of $m+n$ steps, out of which $n$ go up. These $n$ can be chosen in $\left(\begin{array}{c}m+n \\ n\end{array}\right)$ ways from the total of $m+n$. Therefore, the number of paths is $\left(\begin{array}{c}m+n \\ n\end{array}\right)$.

How many of these go through $(p, q)$ ? There are $\left(\begin{array}{c}p+q \\ q\end{array}\right)$ paths from $(0,0)$ to $(p, q)$ and $\left(\begin{array}{c}m+n-p-q \\ n-q\end{array}\right)$ paths from $(p, q)$ to $(m, n)$. Hence

$$
\left(\begin{array}{c}
p+q \\
q
\end{array}\right) \cdot\left(\begin{array}{c}
m+n-p-q \\
n-q
\end{array}\right)
$$

of all the paths pass through $(p, q)$. And, of course,

$$
\left(\begin{array}{c}
r+s \\
s
\end{array}\right) \cdot\left(\begin{array}{c}
m+n-r-s \\
n-s
\end{array}\right)
$$

paths pass through $(r, s)$. To apply the inclusion-exclusion principle, we also need to count the number of paths that go simultaneously through $(p, q)$ and $(r, s)$. This number is

$$
\left(\begin{array}{c}
p+q \\
q
\end{array}\right) \cdot\left(\begin{array}{c}
r+s-p-q \\
s-q
\end{array}\right) \cdot\left(\begin{array}{c}
m+n-r-s \\
n-s
\end{array}\right) .
$$

Hence, by the inclusion-exclusion principle, the number of paths avoiding $(p, q)$ and $(r, s)$ is

$$
\begin{aligned}
\left(\begin{array}{c}
m+n \\
n
\end{array}\right) &-\left(\begin{array}{c}
p+q \\
q
\end{array}\right) \cdot\left(\begin{array}{c}
m+n-p-q \\
n-q
\end{array}\right)-\left(\begin{array}{c}
r+s \\
s
\end{array}\right) \cdot\left(\begin{array}{c}
m+n-r-s \\
n-s
\end{array}\right) \\
&+\left(\begin{array}{c}
p+q \\
q
\end{array}\right) \cdot\left(\begin{array}{c}
r+s-p-q \\
s-q
\end{array}\right) \cdot\left(\begin{array}{c}
m+n-r-s \\
n-s
\end{array}\right) .
\end{aligned}
$$
\\
\textbf{Topic} :Probability\\
\textbf{Book} :Putnam and Beyond\\
\textbf{Final Answer} :\\


\textbf{Problem Statement} :
899. Let $E$ be a set with $n$ elements and $F$ a set with $p$ elements, $p \leq n$. How many surjective (i.e., onto) functions $f: E \rightarrow F$ are there?
\\
\textbf{Solution} :
899. Let $E=\{1,2, \ldots, n\}$ and $F=\{1,2, \ldots, p\}$. There are $p^{n}$ functions from $E$ to $F$. The number of surjective functions is $p^{n}-N$, where $N$ is the number of functions that are not surjective. We compute $N$ using the inclusion-exclusion principle.

Define the sets

$$
A_{i}=\{f: E \rightarrow F \mid i \notin f(E)\} .
$$

Then

$$
N=\left|\cup_{i=1}^{p} A_{i}\right|=\sum_{i}\left|A_{i}\right|-\sum_{i \neq j}\left|A_{i} \cap A_{j}\right|+\cdots+(-1)^{p-1}\left|\cap_{i=1}^{p} A_{i}\right| .
$$

But $A_{i}$ consists of the functions from $E$ to $F \backslash\{i\}$; hence $\left|A_{i}\right|=(p-1)^{n}$. Similarly, for all $k, 2 \leq k \leq p-1, A_{i_{1}} \cap A_{i_{2}} \cap \cdots \cap A_{i_{k}}$ is the set of functions from $E$ to $F \backslash\left\{i_{1}, i_{2}, \ldots, i_{k}\right\}$; hence $\left|A_{i_{1}} \cap A_{i_{2}} \cap \cdots \cap A_{i_{k}}\right|=(p-k)^{n}$. Also, note that for a certain $k$, there are $\left(\begin{array}{l}p \\ k\end{array}\right)$ terms of the form $\left|A_{i_{1}} \cap A_{i_{2}} \cap \cdots \cap A_{i_{k}}\right|$. It follows that

$$
N=\left(\begin{array}{l}
p \\
1
\end{array}\right)(p-1)^{n}-\left(\begin{array}{l}
p \\
2
\end{array}\right)(p-2)^{n}+\cdots+(-1)^{p-1}\left(\begin{array}{c}
p \\
p-1
\end{array}\right) .
$$

We conclude that the total number of surjections from $E$ to $F$ is

$$
p^{n}-\left(\begin{array}{l}
p \\
1
\end{array}\right)(p-1)^{n}+\left(\begin{array}{l}
p \\
2
\end{array}\right)(p-2)^{n}-\cdots+(-1)^{p}\left(\begin{array}{c}
p \\
p-1
\end{array}\right) .
$$
\\
\textbf{Topic} :Probability\\
\textbf{Book} :Putnam and Beyond\\
\textbf{Final Answer} :\\


\textbf{Problem Statement} :
900. A permutation $\sigma$ of a set $S$ is called a derangement if it does not have fixed points, i.e., if $\sigma(x) \neq x$ for all $x \in S$. Find the number of derangements of the set $\{1,2, \ldots, n\}$.
\\
\textbf{Solution} :
900. We count instead the permutations that are not derangements. Denote by $A_{i}$ the set of permutations $\sigma$ with $\sigma(i)=i$. Because the elements in $A_{i}$ have the value at $i$ already prescribed, it follows that $\left|A_{i}\right|=(n-1)$ !. And for the same reason, $\left|A_{i_{1}} \cup A_{i_{2}} \cup \cdots \cup A_{i_{k}}\right|=$ $(n-k)$ ! for any distinct $i_{1}, i_{2}, \ldots, i_{k}, 1 \leq k \leq n$. Applying the inclusion-exclusion principle, we find that 

$$
\left|A_{1} \cup A_{2} \cup \cdots \cup A_{n}\right|=\left(\begin{array}{l}
n \\
1
\end{array}\right)(n-1) !-\left(\begin{array}{l}
n \\
2
\end{array}\right)(n-2) !+\cdots+(-1)^{n}\left(\begin{array}{l}
n \\
n
\end{array}\right) 1 !
$$

The number of derangements is therefore $n !-\left|A_{1} \cup A_{2} \cup \cdots \cup A_{n}\right|$, which is

$$
n !-\left(\begin{array}{l}
n \\
1
\end{array}\right)(n-1) !+\left(\begin{array}{l}
n \\
2
\end{array}\right)(n-2) !-\cdots+(-1)^{n}\left(\begin{array}{l}
n \\
n
\end{array}\right) 0 !
$$

This number can also be written as

$$
n !\left[1-\frac{1}{1 !}+\frac{1}{2 !}-\cdots+\frac{(-1)^{n}}{n !}\right]
$$

This number is approximately equal to $\frac{n !}{e}$.
\\
\textbf{Topic} :Probability\\
\textbf{Book} :Putnam and Beyond\\
\textbf{Final Answer} :\\


\textbf{Problem Statement} :
902. Let $m \geq 5$ and $n$ be given positive integers, and suppose that $\mathcal{P}$ is a regular ( $2 n+1)$ gon. Find the number of convex $m$-gons having at least one acute angle and having vertices exclusive among the vertices of $\mathcal{P}$.
\\
\textbf{Solution} :
902. If the $m$-gon has three acute angles, say at vertices $A, B, C$, then with a fourth vertex $D$ they form a cyclic quadrilateral $A B C D$ that has three acute angles, which is impossible. Similarly, if the $m$-gon has two acute angles that do not share a side, say at vertices $A$ and $C$, then they form with two other vertices $B$ and $D$ of the $m$-gon a cyclic quadrilateral $A B C D$ that has two opposite acute angles, which again is impossible. Therefore, the $m$-gon has either exactly one acute angle, or has two acute angles and they share a side.

To count the number of such $m$-gons we employ the principle of inclusion and exclusion. Thus we first find the number of $m$-gons with at least one acute angle, then subtract the number of $m$-gons with two acute angles (which were counted twice the first time). If the acute angle of the $m$-gon is $A_{k} A_{1} A_{k+r}$, the condition that this angle is acute translates into $r \leq n$. The other vertices of the $m$-gon lie between $A_{k}$ and $A_{k+r}$; hence $m-2 \leq r$, and these vertices can be chosen in $\left(\begin{array}{c}r-1 \\ m-3\end{array}\right)$ ways. Note also that $1 \leq k \leq 2 n-r$. Thus the number of $m$-gons with an acute angle at $A_{1}$ is

$$
\begin{aligned}
\sum_{r=m-2}^{n} \sum_{k=1}^{2 n-r}\left(\begin{array}{c}
r-1 \\
m-3
\end{array}\right) &=2 n \sum_{m-2}^{n}\left(\begin{array}{c}
r-1 \\
m-3
\end{array}\right)-\sum_{r=m-2}^{n} r\left(\begin{array}{c}
r-1 \\
m-3
\end{array}\right) \\
&=2 n\left(\begin{array}{c}
n \\
m-2
\end{array}\right)-(m-2)\left(\begin{array}{c}
n+1 \\
m-1
\end{array}\right)
\end{aligned}
$$

There are as many polygons with an acute angle at $A_{2}, A_{3}, \ldots, A_{2 n+1}$.

To count the number of $m$-gons with two acute angles, let us first assume that these acute angles are $A_{s} A_{1} A_{k}$ and $A_{1} A_{k} A_{r}$. The other vertices lie between $A_{r}$ and $A_{s}$. We have the restrictions $2 \leq k \leq 2 n-m+3, n+2 \leq r<s \leq k+n$ if $k \leq n$ and no restriction on $r$ and $s$ otherwise. The number of such $m$-gons is

$$
\begin{aligned}
\sum_{k=1}^{n}\left(\begin{array}{c}
k-1 \\
m-2
\end{array}\right)+\sum_{k=n+1}^{2 n+1-(m-2)}\left(\begin{array}{c}
2 n+1-k \\
m-2
\end{array}\right) &=\sum_{k=m-1}^{n}\left(\begin{array}{c}
k-1 \\
m-2
\end{array}\right)+\sum_{s=m-2}^{n}\left(\begin{array}{c}
s \\
m-2
\end{array}\right) \\
&=\left(\begin{array}{c}
n+1 \\
m-1
\end{array}\right)+\left(\begin{array}{c}
n \\
m-1
\end{array}\right)
\end{aligned}
$$

This number has to be multiplied by $2 n+1$ to take into account that the first acute vertex can be at any other vertex of the regular $n$-gon.

We conclude that the number of $m$-gons with at least one acute angle is

$$
(2 n+1)\left(2 n\left(\begin{array}{c}
n \\
m-2
\end{array}\right)-(m-1)\left(\begin{array}{c}
n+1 \\
m-1
\end{array}\right)-\left(\begin{array}{c}
n \\
m-1
\end{array}\right)\right) .
$$
\\
\textbf{Topic} :Probability\\
\textbf{Book} :Putnam and Beyond\\
\textbf{Final Answer} :\\


\textbf{Problem Statement} :
903. Let $S^{1}=\{z \in \mathbb{C}|| z \mid=1\}$. For all functions $f: S^{1} \rightarrow S^{1}$ set $f^{1}=f$ and $f^{n+1}=f \circ f^{n}, n \geq 1$. Call $w \in S^{1}$ a periodic point of $f$ of period $n$ if $f^{i}(w) \neq w$ for $i=1, \ldots, n-1$ and $f^{n}(w)=w$. If $f(z)=z^{m}, m$ a positive integer, find the number of periodic points of $f$ of period 1989.
\\
\textbf{Solution} :
903. Denote by $U_{n}$ the set of $z \in S^{1}$ such that $f^{n}(z)=z$. Because $f^{n}(z)=z^{m^{n}}, U_{n}$ is the set of the roots of unity of order $m^{n}-1$. In our situation $n=1989$, and we are looking for those elements of $U_{1989}$ that do not have period less than 1989. The periods of the elements of $U_{1989}$ are divisors of 1989. Note that $1989=3^{2} \times 13 \times 17$. The elements we are looking for lie in the complement of $U_{1989 / 3} \cup U_{1989 / 13} \cup U_{1989 / 17}$. Using the inclusion-exclusion principle, we find that the answer to the problem is

$$
\begin{aligned}
\left|U_{1989}\right| &-\left|U_{1989 / 3}\right|-\left|U_{1989 / 13}\right|-\left|U_{1989 / 17}\right|+\left|U_{1989 / 3} \cap U_{1898 / 13}\right|+\left|U_{1989 / 3} \cap U_{1989 / 17}\right| \\
&+\left|U_{1989 / 13} \cap U_{1989 / 17}\right|+\left|U_{1989 / 3} \cap U_{1989 / 13} \cap U_{1989 / 17}\right|,
\end{aligned}
$$

i.e.,

$$
\left|U_{1989}\right|-\left|U_{663}\right|-\left|U_{153}\right|-\left|U_{117}\right|+\left|U_{51}\right|+\left|U_{39}\right|+\left|U_{9}\right|-\left|U_{3}\right| .
$$

This number is equal to

$$
m^{1989}-m^{663}-m^{153}-m^{117}+m^{51}+m^{39}+m^{9}-m^{3},
$$

since the $-1$ 's in the formula for the cardinalities of the $U_{n}$ 's cancel out.

(Chinese Mathematical Olympiad, 1989)
\\
\textbf{Topic} :Probability\\
\textbf{Book} :Putnam and Beyond\\
\textbf{Final Answer} :\\


\textbf{Problem Statement} :
906. Let $v$ and $w$ be distinct, randomly chosen roots of the equation $z^{1997}-1=0$. Find the probability that $\sqrt{2+\sqrt{3}} \leq|v+w|$.
\\
\textbf{Solution} :
906. Because the 1997 roots of the equation are symmetrically distributed in the complex plane, there is no loss of generality to assume that $v=1$. We are required to find the probability that

$$
|1+w|^{2}=|(1+\cos \theta)+i \sin \theta|^{2}=2+2 \cos \theta \geq 2+\sqrt{3} .
$$

This is equivalent to $\cos \theta \geq \frac{1}{2} \sqrt{3}$, or $|\theta| \leq \frac{\pi}{6}$. Because $w \neq 1, \theta$ is of the form $\pm \frac{2 k \pi}{1997} k$, $1 \leq k \leq\left\lfloor\frac{1997}{12}\right\rfloor$. There are $2 \cdot 166=332$ such angles, and hence the probability is $\frac{332}{1996}=\frac{83}{499} \approx 0.166$

(American Invitational Mathematics Examination, 1997)
\\
\textbf{Topic} :Probability\\
\textbf{Book} :Putnam and Beyond\\
\textbf{Final Answer} :\\


\textbf{Problem Statement} :
907. Find the probability that in a group of $n$ people there are two with the same birthday. Ignore leap years.
\\
\textbf{Solution} :
907. It is easier to compute the probability that no two people have the same birthday. Arrange the people in some order. The first is free to be born on any of the 365 days. But only 364 dates are available for the second, 363 for the third, and so on. The probability that no two people have the same birthday is therefore

$$
\frac{364}{365} \cdot \frac{363}{365} \cdots \frac{365-n+1}{365}=\frac{365 !}{(365-n) ! 365^{n}} .
$$

And the probability that two people have the same birthday is

$$
1-\frac{365 !}{(365-n) ! 365^{n}} .
$$

Remark. Starting with $n=23$ the probability becomes greater than $\frac{1}{2}$, while when $n>365$ the probability is clearly 1 by the pigeonhole principle. 
\\
\textbf{Topic} :Probability\\
\textbf{Book} :Putnam and Beyond\\
\textbf{Final Answer} :\\


\textbf{Problem Statement} :
908. A solitaire game is played as follows. Six distinct pairs of matched tiles are placed in a bag. The player randomly draws tiles one at a time from the bag and retains them, except that matching tiles are put aside as soon as they appear in the player's hand. The game ends if the player ever holds three tiles, no two of which match; otherwise, the drawing continues until the bag is empty. Find the probability that the bag will be emptied.
\\
\textbf{Solution} :
908. Denote by $P(n)$ the probability that a bag containing $n$ distinct pairs of tiles will be emptied, $n \geq 2$. Then $P(n)=Q(n) P(n-1)$ where $Q(n)$ is the probability that two of the first three tiles selected make a pair. The latter one is

$$
\begin{aligned}
Q(n) &=\frac{\text { number of ways to select three tiles, two of which match }}{\text { number of ways to select three tiles }} \\
&=\frac{n(2 n-2)}{\left(\begin{array}{c}
2 n \\
3
\end{array}\right)}=\frac{3}{2 n-1} .
\end{aligned}
$$

The recurrence

$$
P(n)=\frac{3}{2 n-1} P(n-1)
$$

yields

$$
P(n)=\frac{3^{n-2}}{(2 n-1)(2 n-3) \cdots 5} P(2) .
$$

Clearly, $P(2)=1$, and hence the answer to the problem is

$$
P(6)=\frac{3^{4}}{11 \cdot 9 \cdot 7 \cdot 5}=\frac{9}{385} \approx 0.023 .
$$

(American Invitational Mathematics Examination, 1994)
\\
\textbf{Topic} :Probability\\
\textbf{Book} :Putnam and Beyond\\
\textbf{Final Answer} :\\


\textbf{Problem Statement} :
909. An urn contains $n$ balls numbered $1,2, \ldots, n$. A person is told to choose a ball and then extract $m$ balls among which is the chosen one. Suppose he makes two independent extractions, where in each case he chooses the remaining $m-1$ balls at random. What is the probability that the chosen ball can be determined?
\\
\textbf{Solution} :
909. Because there are two extractions each of with must contain a certain ball, the total number of cases is $\left(\begin{array}{c}n-1 \\ m-1\end{array}\right)^{2}$. The favorable cases are those for which the balls extracted the second time differ from those extracted first (except of course the chosen ball). For the first extraction there are $\left(\begin{array}{c}n-1 \\ m-1\end{array}\right)$ cases, while for the second there are $\left(\begin{array}{c}n-m \\ m-1\end{array}\right)$, giving a total number of cases $\left(\begin{array}{c}n-1 \\ m-1\end{array}\right)\left(\begin{array}{c}n-m \\ m-1\end{array}\right)$. Taking the ratio, we obtain the desired probability as

$$
P=\frac{\left(\begin{array}{c}
n-1 \\
m-1
\end{array}\right)\left(\begin{array}{c}
n-m \\
m-1
\end{array}\right)}{\left(\begin{array}{c}
n-1 \\
m-1
\end{array}\right)}=\frac{\left(\begin{array}{c}
n-m \\
m-1
\end{array}\right)}{\left(\begin{array}{c}
n-1 \\
m-1
\end{array}\right)} .
$$

(Gazeta Matematic $\breve{a}$ (Mathematics Gazette, Bucharest), proposed by C. Marinescu)
\\
\textbf{Topic} :Probability\\
\textbf{Book} :Putnam and Beyond\\
\textbf{Final Answer} :\\


\textbf{Problem Statement} :
910. A bag contains 1993 red balls and 1993 black balls. We remove two balls at a time repeatedly and

(i) discard them if they are of the same color,

(ii) discard the black ball and return to the bag the red ball if they are of different colors.

What is the probability that this process will terminate with one red ball in the bag?
\\
\textbf{Solution} :
910. First, observe that since at least one ball is removed during each stage, the process will eventually terminate, leaving no ball or one ball in the bag. Because red balls are removed 2 at a time and since we start with an odd number of red balls, the number of red balls in the bag at any time is odd. Hence the process will always leave red balls in the bag, and so it must terminate with exactly one red ball. The probability we are computing is therefore 1 .

(Mathematics and Informatics Quarterly, proposed by D. Macks) 
\\
\textbf{Topic} :Probability\\
\textbf{Book} :Putnam and Beyond\\
\textbf{Final Answer} :\\


\textbf{Problem Statement} :
912. What is the probability that a permutation of the first $n$ positive integers has the numbers 1 and 2 within the same cycle.
\\
\textbf{Solution} :
912. The total number of permutations is of course $n$ !. We will count instead the number of permutations for which 1 and 2 lie in different cycles.

If the cycle that contains 1 has length $k$, we can choose the other $k-1$ elements in $\left(\begin{array}{c}n-2 \\ k-1\end{array}\right)$ ways from the set $\{3,4, \ldots, n\}$. There exist $(k-1)$ ! circular permutations of these elements, and $(n-k)$ ! permutations of the remaining $n-k$ elements. Hence the total number of permutations for which 1 and 2 belong to different cycles is equal to

$$
\sum_{k=1}^{n-1}\left(\begin{array}{l}
n-2 \\
k-1
\end{array}\right)(k-1) !(n-k) !=(n-2) ! \sum_{k=1}^{n-1}(n-k)=(n-2) ! \frac{n(n-1)}{2}=\frac{n !}{2} .
$$

It follows that exactly half of all permutations contain 1 and 2 in different cycles, and so half contain 1 and 2 in the same cycle. The probability is $\frac{1}{2}$.

(I. Tomescu Problems in Combinatorics, Wiley, 1985)
\\
\textbf{Topic} :Probability\\
\textbf{Book} :Putnam and Beyond\\
\textbf{Final Answer} :\\


\textbf{Problem Statement} :
913. An unbiased coin is tossed $n$ times. Find a formula, in closed form, for the expected value of $|H-T|$, where $H$ is the number of heads, and $T$ is the number of tails.
\\
\textbf{Solution} :
913. There are $\left(\begin{array}{l}n \\ k\end{array}\right)$ ways in which exactly $k$ tails appear, and in this case the difference is $n-2 k$. Hence the expected value of $|H-T|$ is

$$
\frac{1}{2^{n}} \sum_{k=0}^{n}|n-2 k|\left(\begin{array}{l}
n \\
k
\end{array}\right) .
$$

Evaluate the sum as follows:

$$
\begin{aligned}
\frac{1}{2^{n}} \sum_{m=0}^{n}|n-2 m|\left(\begin{array}{c}
n \\
m
\end{array}\right) &=\frac{1}{2^{n-1}} \sum_{m=0}^{\lfloor n / 2\rfloor}(n-2 m)\left(\begin{array}{c}
n \\
m
\end{array}\right) \\
&=\frac{1}{2^{n-1}}\left(\sum_{m=0}^{\lfloor n / 2\rfloor}(n-m)\left(\begin{array}{c}
n \\
m
\end{array}\right)-\sum_{m=0}^{\lfloor n / 2\rfloor} m\left(\begin{array}{c}
n \\
m
\end{array}\right)\right) \\
&=\frac{1}{2^{n-1}}\left(\sum_{m=0}^{\lfloor n / 2\rfloor} n\left(\begin{array}{c}
n-1 \\
m
\end{array}\right)-\sum_{m=1}^{\lfloor n / 2\rfloor} n\left(\begin{array}{c}
n-1 \\
m-1
\end{array}\right)\right) \\
&=\frac{n}{2^{n-1}}\left(\begin{array}{c}
n-1 \\
\left\lfloor\frac{n}{2}\right\rfloor
\end{array}\right) .
\end{aligned}
$$

(35th W.L. Putnam Mathematical Competition, 1974)
\\
\textbf{Topic} :Probability\\
\textbf{Book} :Putnam and Beyond\\
\textbf{Final Answer} :\\


\textbf{Problem Statement} :
915. An exam consists of 3 problems selected randomly from a list of $2 n$ problems, where $n$ is an integer greater than 1 . For a student to pass, he needs to solve correctly at least two of the three problems. Knowing that a certain student knows how to solve exactly half of the $2 n$ problems, find the probability that the student will pass the exam.
\\
\textbf{Solution} :
915. Denote by $A_{i}$ the event "the student solves correctly exactly $i$ of the three proposed problems, '' $i=0,1,2,3$. The event $A$ whose probability we are computing is

$$
A=A_{2} \cup A_{3} \text {, }
$$

and its probability is

$$
P(A)=P\left(A_{2}\right)+P\left(A_{3}\right),
$$

since $A_{2}$ and $A_{3}$ exclude each other.

Because the student knows how to solve half of all the problems,

$$
P\left(A_{0}\right)=P\left(A_{3}\right) \quad \text { and } \quad P\left(A_{1}\right)=P\left(A_{2}\right) .
$$

The equality

$$
P\left(A_{0}\right)+P\left(A_{1}\right)+P\left(A_{2}\right)+P\left(A_{3}\right)=1
$$

becomes

$$
2\left[P\left(A_{2}\right)+P\left(A_{3}\right)\right]=1 .
$$

It follows that the probability we are computing is

$$
P(A)=P\left(A_{2}\right)+P\left(A_{3}\right)=\frac{1}{2} .
$$

(N. Negoescu, Probleme cu... Probleme (Problems with... Problems), Editura Facla, 1975)
\\
\textbf{Topic} :Probability\\
\textbf{Book} :Putnam and Beyond\\
\textbf{Final Answer} :\\


\textbf{Problem Statement} :
916. The probability that a woman has breast cancer is $1 \%$. If a woman has breast cancer, the probability is $60 \%$ that she will have a positive mammogram. However, if a woman does not have breast cancer, the mammogram might still come out positive, with a probability of $7 \%$. What is the probability for a woman with positive mammogram to actually have cancer?
\\
\textbf{Solution} :
916. For the solution we will use Bayes' theorem for conditional probabilities. We denote by $P(A)$ the probability that the event $A$ holds, and by $P\left(\frac{B}{A}\right)$ the probability that the event $B$ holds given that $A$ in known to hold. Bayes' theorem states that 

$$
P(B / A)=\frac{P(B)}{P(A)} \cdot P(A / B) .
$$

For our problem $A$ is the event that the mammogram is positive and $B$ the event that the woman has breast cancer. Then $P(B)=0.01$, while $P(A / B)=0.60$. We compute $P(A)$ from the formula

$$
P(A)=P(A / B) P(B)+P(A / \operatorname{not} B) P(\text { not } B)=0.6 \cdot 0.01+0.07 \cdot 0.99=0.0753 .
$$

The answer to the question is therefore

$$
P(B / A)=\frac{0.01}{0.0753} \cdot 0.6=0.0795 \approx 0.08
$$

The chance that the woman has breast cancer is only $8 \%$ !
\\
\textbf{Topic} :Probability\\
\textbf{Book} :Putnam and Beyond\\
\textbf{Final Answer} :\\


\textbf{Problem Statement} :
917. Find the probability that in the process of repeatedly flipping a coin, one will encounter a run of 5 heads before one encounters a run of 2 tails.
\\
\textbf{Solution} :
917. We call a successful string a sequence of $H$ 's and $T$ 's in which $H H H H H$ appears before $T T$ does. Each successful string must belong to one of the following three types:

(i) those that begin with $T$, followed by a successful string that begins with $H$;

(ii) those that begin with $H, H H, H H H$, or $H H H H$, followed by a successful string that begins with $T$;

(iii) the string $H H H H H$.

Let $P_{H}$ denote the probability of obtaining a successful string that begins with $H$, and let $P_{T}$ denote the probability of obtaining a successful string that begins with $T$. Then

$$
\begin{aligned}
P_{T} &=\frac{1}{2} P_{H}, \\
P_{H} &=\left(\frac{1}{2}+\frac{1}{4}+\frac{1}{8}+\frac{1}{16}\right) P_{T}+\frac{1}{32} .
\end{aligned}
$$

Solving these equations simultaneously, we find that

$$
P_{H}=\frac{1}{17} \quad \text { and } \quad P_{T}=\frac{1}{34} .
$$

Hence the probability of obtaining five heads before obtaining two tails is $\frac{3}{34}$.

(American Invitational Mathematics Examination, 1995)
\\
\textbf{Topic} :Probability\\
\textbf{Book} :Putnam and Beyond\\
\textbf{Final Answer} :\\


\textbf{Problem Statement} :
918. The temperatures in Chicago and Detroit are $x^{\circ}$ and $y^{\circ}$, respectively. These temperatures are not assumed to be independent; namely, we are given the following:

(i) $P\left(x^{\circ}=70^{\circ}\right)=a$, the probability that the temperature in Chicago is $70^{\circ}$,

(ii) $P\left(y^{\circ}=70^{\circ}\right)=b$, and

(iii) $P\left(\max \left(x^{\circ}, y^{\circ}\right)=70^{\circ}\right)=c$.

Determine $P\left(\min \left(x^{\circ}, y^{\circ}\right)=70^{\circ}\right)$ in terms of $a, b$, and $c$.
\\
\textbf{Solution} :
918. Let us denote the events $x=70^{\circ}, y=70^{\circ}, \max \left(x^{\circ}, y^{\circ}\right)=70^{\circ}, \min \left(x^{\circ}, y^{\circ}\right)=70^{\circ}$ by $A, B, C, D$, respectively. We see that $A \cup B=C \cup D$ and $A \cap B=C \cap D$. Hence

$P(A)+P(B)=P(A \cup B)+P(A \cap B)=P(C \cup D)+P(C \cap D)=P(C)+P(D)$.

Therefore, $P(D)=P(A)+P(B)-P(C)$, that is, 

$$
\begin{aligned}
P\left(\min \left(x^{\circ}, y^{\circ}\right)\right.&\left.=70^{\circ}\right)=P\left(x^{\circ}=70^{\circ}\right)+P\left(y^{\circ}=70^{\circ}\right)-P\left(\max \left(x^{\circ}, y^{\circ}\right)=70^{\circ}\right) \\
&=a+b-c .
\end{aligned}
$$

(29th W.L. Putnam Mathematical Competition, 1968)
\\
\textbf{Topic} :Probability\\
\textbf{Book} :Putnam and Beyond\\
\textbf{Final Answer} :\\


\textbf{Problem Statement} :
920. Three independent students took an exam. The random variable $X$, representing the students who passed, has the distribution

$$
\left(\begin{array}{cccc}
0 & 1 & 2 & 3 \\
\frac{2}{5} & \frac{13}{30} & \frac{3}{20} & \frac{1}{60}
\end{array}\right) .
$$

Find each student's probability of passing the exam.
\\
\textbf{Solution} :
920. First solution: Denote by $p_{1}, p_{2}, p_{3}$ the three probabilities. By hypothesis,

$$
\begin{aligned}
&P(X=0)=\prod_{i}\left(1-p_{i}\right)=1-\sum_{i} p_{i}+\sum_{i \neq j} p_{i} p_{j}-p_{1} p_{2} p_{3}=\frac{2}{5}, \\
&P(X=1)=\sum_{\{i, j, k\}=\{1,2,3\}} p_{i}\left(1-p_{j}\right)\left(1-p_{k}\right)=\sum_{i} p_{i}-2 \sum_{i \neq j} p_{i} p_{j}+3 p_{1} p_{2} p_{3}=\frac{13}{30}, \\
&P(X=2)=\sum_{\{i, j, k\}=\{1,2,3\}} p_{i} p_{j}\left(1-p_{k}\right)=\sum_{i \neq j} p_{i} p_{j}-3 p_{1} p_{2} p_{3}=\frac{3}{20}, \\
&P(X=3)=p_{1} p_{2} p_{3}=\frac{1}{60} .
\end{aligned}
$$

This is a linear system in the unknowns $\sum_{i} p_{i}, \sum_{i \neq j} p_{i} p_{j}$, and $p_{1} p_{2} p_{3}$ with the solution

$$
\sum_{i} p_{i}=\frac{47}{60}, \quad \sum_{i \neq j} p_{i} p_{j}=\frac{1}{5}, \quad p_{1} p_{2} p_{3}=\frac{1}{60} .
$$

It follows that $p_{1}, p_{2}, p_{3}$ are the three solutions to the equation

$$
x^{3}-\frac{47}{60} x^{2}+\frac{1}{5} x-\frac{1}{60}=0 .
$$

Searching for solutions of the form $\frac{1}{q}$ with $q$ dividing 60, we find the three probabilities to be equal to $\frac{1}{3}, \frac{1}{4}$, and $\frac{1}{5}$. Second solution: Using the Poisson scheme

$$
\left(p_{1} x+1-p_{1}\right)\left(p_{2} x+1-p_{2}\right)\left(p_{3} x+1-p_{3}\right)=\frac{2}{5}+\frac{13}{30} x+\frac{3}{20} x^{2}+\frac{1}{60} x^{3},
$$

we deduce that $1-\frac{1}{p_{i}}, i=1,2,3$, are the roots of $x^{3}+9 x^{2}+26 x+24=0$ and $p_{1} p_{2} p_{3}=\frac{1}{60}$. The three roots are $-2,-3,-4$, which again gives $p_{i}$ 's equal to $\frac{1}{3}, \frac{1}{4}$, and $\frac{1}{5}$.

(N. Negoescu, Probleme cu... Probleme (Problems with... Problems), Editura Facla, 1975)
\\
\textbf{Topic} :Probability\\
\textbf{Book} :Putnam and Beyond\\
\textbf{Final Answer} :\\


\textbf{Problem Statement} :
921. Given the independent events $A_{1}, A_{2}, \ldots, A_{n}$ with probabilities $p_{1}, p_{2}, \ldots, p_{n}$, find the probability that an odd number of these events occurs.
\\
\textbf{Solution} :
921. Set $q_{i}=1-p_{i}, i=1,2, \ldots, n$, and consider the generating function

$$
Q(x)=\prod_{i=1}^{n}\left(p_{i} x+q_{i}\right)=Q_{0}+Q_{1} x+\cdots+Q_{n} x^{n} .
$$

The probability for exactly $k$ of the independent events $A_{1}, A_{2}, \ldots, A_{n}$ to occur is equal to the coefficient of $x^{k}$ in $Q(x)$, hence to $Q_{k}$. The probability $P$ for an odd number of events to occur is thus equal to $Q_{1}+Q_{3}+\cdots$. Let us compute this number in terms of $p_{1}, p_{2}, \ldots, p_{n}$.

We have

$$
Q(1)=Q_{0}+Q_{1}+\cdots+Q_{n} \quad \text { and } \quad Q(-1)=Q_{0}-Q_{1}+\cdots+(-1)^{n} Q_{n} \text {. }
$$

Therefore,

$$
P=\frac{Q(1)-Q(-1)}{2}=\frac{1}{2}\left(1-\prod_{i=1}^{n}\left(1-2 p_{i}\right)\right) .
$$

(Romanian Mathematical Olympiad, 1975)
\\
\textbf{Topic} :Probability\\
\textbf{Book} :Putnam and Beyond\\
\textbf{Final Answer} :\\


\textbf{Problem Statement} :
922. Out of every batch of 100 products of a factory, 5 are quality checked. If one sample does not pass the quality check, then the whole batch of one hundred will be rejected. What is the probability that a batch is rejected if it contains $5 \%$ faulty products.
\\
\textbf{Solution} :
922. It is easier to compute the probability of the contrary event, namely that the batch passes the quality check. Denote by $A_{i}$ the probability that the $i$ th checked product has the desired quality standard. We then have to compute $P\left(\cap_{i=1}^{5} A_{i}\right)$. The events are not independent, so we use the formula

$$
\begin{aligned}
P\left(\cap_{i=1}^{5} A_{i}\right)=& P\left(A_{1}\right) P\left(A_{2} / A_{1}\right)\left(A_{3} / A_{1} \cap A_{2}\right) P\left(A_{4} / A_{1} \cap A_{2} \cap A_{3}\right) \\
& \times P\left(A_{5} / A_{1} \cap A_{2} \cap A_{3} \cap A_{4}\right) .
\end{aligned}
$$

We find successively $P\left(A_{1}\right)=\frac{95}{100}, P\left(A_{2} / A_{1}\right)=\frac{94}{99}$ (because if $A_{1}$ occurs then we are left with 99 products out of which 94 are good $), P\left(A_{3} / A_{1} \cap A_{2}\right)=\frac{93}{98}, P\left(A_{4} / A_{1} \cap A_{2} \cap A_{3}\right)=$ $\frac{92}{97}, P\left(A_{5} / A_{1} \cap A_{2} \cap A_{3} \cap A_{4}\right)=\frac{91}{96}$. The answer to the problem is 

$$
1-\frac{95}{100} \cdot \frac{94}{99} \cdot \frac{93}{98} \cdot \frac{92}{97} \cdot \frac{91}{96} \approx 0.230 .
$$
\\
\textbf{Topic} :Probability\\
\textbf{Book} :Putnam and Beyond\\
\textbf{Final Answer} :\\


\textbf{Problem Statement} :
923. There are two jet planes and a propeller plane at the small regional airport of Gauss City. A plane departs from Gauss City and arrives in Eulerville, where there were already five propeller planes and one jet plane. Later, a farmer sees a jet plane flying out of Eulerville. What is the probability that the plane that arrived from Gauss City was a propeller plane, provided that all events are equiprobable?
\\
\textbf{Solution} :
923. We apply Bayes' formula. Let $B$ be the event that the plane flying out of Eulerville is a jet plane and $A_{1}$, respectively, $A_{2}$, the events that the plane flying between the two cities is a jet, respectively, a propeller plane. Then

$$
P\left(A_{1}\right)=\frac{2}{3}, \quad P\left(A_{2}\right)=\frac{1}{3}, \quad P\left(B / A_{1}\right)=\frac{2}{7}, \quad P\left(B / A_{2}\right)=\frac{1}{7} .
$$

Bayes formula gives

$$
P\left(A_{2} / B\right)=\frac{P\left(A_{2}\right) P\left(B / A_{2}\right)}{P\left(A_{1}\right) P\left(B / A_{1}\right)+P\left(A_{2}\right) P\left(B / A_{2}\right)}=\frac{\frac{1}{3} \cdot \frac{1}{7}}{\frac{2}{3} \cdot \frac{2}{7}+\frac{1}{3} \cdot \frac{1}{7}}=\frac{1}{5} .
$$

Thus the answer to the problem is $\frac{1}{5}$.

Remark. Without the farmer seeing the jet plane flying out of Eulerville, the probability would have been $\frac{1}{3}$. What you know affects your calculation of probabilities.
\\
\textbf{Topic} :Probability\\
\textbf{Book} :Putnam and Beyond\\
\textbf{Final Answer} :\\


\textbf{Problem Statement} :
924. A coin is tossed $n$ times. What is the probability that two heads will turn up in succession somewhere in the sequence?
\\
\textbf{Solution} :
924. We find instead the probability $P(n)$ for no consecutive heads to appear in $n$ throws. We do this recursively. If the first throw is tails, which happens with probability $\frac{1}{2}$, then the probability for no consecutive heads to appear afterward is $P(n-1)$. If the first throw is heads, the second must be tails, and this configuration has probability $\frac{1}{4}$. The probability that no consecutive heads appear later is $P(n-2)$. We obtain the recurrence

$$
P(n)=\frac{1}{2} P(n-1)+\frac{1}{4} P(n-2),
$$

with $P(1)=1$, and $P(2)=\frac{3}{4}$. Make this relation more homogeneous by substituting $x_{n}=2^{n} P(n)$. We recognize the recurrence for the Fibonacci sequence $x_{n+1}=x_{n}+x_{n-1}$, with the remark that $x_{1}=F_{3}$ and $x_{2}=F_{4}$. It follows that $x_{n}=F_{n+2}, P(n)=\frac{F_{n+2}}{2^{n}}$, and the probability required by the problem is $P(n)=1-\frac{F_{n+2}}{2^{n}}$.

(L.C. Larson, Problem-Solving Through Problems, Springer-Verlag, 1990)
\\
\textbf{Topic} :Probability\\
\textbf{Book} :Putnam and Beyond\\
\textbf{Final Answer} :\\


\textbf{Problem Statement} :
925. Two people, $A$ and $B$, play a game in which the probability that $A$ wins is $p$, the probability that $B$ wins is $q$, and the probability of a draw is $r$. At the beginning, $A$ has $m$ dollars and $B$ has $n$ dollars. At the end of each game, the winner takes a dollar from the loser. If $A$ and $B$ agree to play until one of them loses all his/her money, what is the probability of $A$ winning all the money?
\\
\textbf{Solution} :
925. Fix $N=m+n$, the total amount of money, and vary $m$. Denote by $P(m)$ the probability that $A$ wins all the money when starting with $m$ dollars. Clearly, $P(0)=0$ and $P(N)=1$. We want a recurrence relation for $P(m)$.

Assume that $A$ starts with $k$ dollars. During the first game, $A$ can win, lose, or the game can be a draw. If $A$ wins this game, then the probability of winning all the money afterward is $P(k+1)$. If $A$ loses, the probability of winning in the end is $P(k-1)$. Finally, if the first game is a draw, nothing changes, so the probability of $A$ winning in the end remains equal to $P(k)$. These three situations occur with probabilities $p, q, r$, respectively; hence 

$$
P(k)=p P(k+1)+q P(k-1)+r P(k) .
$$

Taking into account that $p+q+r=1$, we obtain the recurrence relation

$$
p P(k+1)-(p+q) P(k)+q P(k-1)=0 .
$$

The characteristic equation of this recurrence is $p \lambda^{2}-(p+q) \lambda+q=0$. There are two cases. The simpler is $p=q$. Then the equation has the double root $\lambda=1$, in which case the general term is a linear function in $k$. Since $P(0)=0$ and $P(N)=1$, it follows that $P(m)=\frac{m}{N}=\frac{m}{n+m}$. If $p \neq q$, then the distinct roots of the equation are $\lambda_{1}=1$ and $\lambda_{2}=\frac{q}{p}$, and the general term must be of the form $P(k)=c_{1}+c_{2}\left(\frac{q}{p}\right)^{k}$. Using the known values for $k=0$ and $N$, we compute

$$
c_{1}=-c_{2}=\frac{1}{1-\left(\frac{q}{p}\right)^{N}} .
$$

Hence the required probability is

$$
\frac{m}{m+n} \text { if } p=q \quad \text { and } \quad \frac{1-\left(\frac{q}{p}\right)^{m}}{1-\left(\frac{q}{p}\right)^{m+n}} \text { if } p \neq q .
$$

(K.S. Williams, K. Hardy, The Red Book of Mathematical Problems, Dover, Mineola, NY, 1996)
\\
\textbf{Topic} :Probability\\
\textbf{Book} :Putnam and Beyond\\
\textbf{Final Answer} :\\


\textbf{Problem Statement} :
927. What is the probability that the sum of two randomly chosen numbers in the interval $[0,1]$ does not exceed 1 and their product does not exceed $\frac{2}{9}$ ?
\\
\textbf{Solution} :
927. Let $x$ and $y$ be the two numbers. The set of all possible outcomes is the unit square

$$
D=\{(x, y) \mid 0 \leq x \leq 1,0 \leq y \leq 1\} .
$$

The favorable cases consist of the region

$$
D_{f}=\left\{(x, y) \in D \mid x+y \leq 1, x y \leq \frac{2}{9}\right\} .
$$

This is the set of points that lie below both the line $f(x)=1-x$ and the hyperbola $g(x)=\frac{2}{9 x}$. equal to

The required probability is $P=\frac{\operatorname{Area}\left(D_{f}\right)}{\operatorname{Area}(D)}$. The area of $D$ is 1 . The area of $D_{f}$ is

$$
\int_{0}^{1} \min (f(x), g(x)) d x .
$$

The line and the hyperbola intersect at the points $\left(\frac{1}{3}, \frac{2}{3}\right)$ and $\left(\frac{2}{3}, \frac{1}{3}\right)$. Therefore,

$$
\operatorname{Area}\left(D_{f}\right)=\int_{0}^{1 / 3}(1-x) d x+\int_{1 / 3}^{2 / 3} \frac{2}{9 x} d x+\int_{2 / 3}^{1}(1-x) d x=\frac{1}{3}+\frac{2}{9} \ln 2 .
$$

We conclude that $P=\frac{1}{3}+\frac{2}{9} \ln 2 \approx 0.487$.

(C. Reischer, A. Sâmboan, Culegere de Probleme de Teoria Probabilitătilor şi Statistică Matematica (Collection of Problems of Probability Theory and Mathematical Statistics), Editura Didactică şi Pedagogică, Bucharest, 1972)
\\
\textbf{Topic} :Probability\\
\textbf{Book} :Putnam and Beyond\\
\textbf{Final Answer} :\\


\textbf{Problem Statement} :
929. A husband and wife agree to meet at a street corner between 4 and 5 o'clock to go shopping together. The one who arrives first will await the other for 15 minutes, and then leave. What is the probability that the two meet within the given time interval, assuming that they can arrive at any time with the same probability? 
\\
\textbf{Solution} :
929. Denote by $x$, respectively, $y$, the fraction of the hour when the husband, respectively, wife, arrive. The configuration space is the square 

$$
D=\{(x, y) \mid 0 \leq x \leq 1,0 \leq y \leq 1\} .
$$

In order for the two people to meet, their arrival time must lie inside the region

$$
D_{f}=\left\{(x, y)|| x-y \mid \leq \frac{1}{4}\right\} .
$$

The desired probability is the ratio of the area of this region to the area of the square.

The complement of the region consists of two isosceles right triangles with legs equal to $\frac{3}{4}$, and hence of areas $\frac{1}{2}\left(\frac{3}{4}\right)^{2}$. We obtain for the desired probability

$$
1-2 \cdot \frac{1}{2} \cdot\left(\frac{3}{4}\right)^{2}=\frac{7}{16} \approx 0.44 .
$$

\section{(B.V. Gnedenko)}\
\textbf{Topic} :Probability\\
\textbf{Book} :Putnam and Beyond\\
\textbf{Final Answer} :\\


\textbf{Problem Statement} :
932. Let $n \geq 4$ be given, and suppose that the points $P_{1}, P_{2}, \ldots, P_{n}$ are randomly chosen on a circle. Consider the convex $n$-gon whose vertices are these points. What is the probability that at least one of the vertex angles of this polygon is acute?
\\
\textbf{Solution} :
932. The angle at the vertex $P_{i}$ is acute if and only if all other points lie on an open semicircle facing $P_{i}$. We first deduce from this that if there are any two acute angles at all, they must occur consecutively. Otherwise, the two arcs that these angles subtend would overlap and cover the whole circle, and the sum of the measures of the two angles would exceed $180^{\circ}$.

So the polygon has either just one acute angle or two consecutive acute angles. In particular, taken in counterclockwise order, there exists exactly one pair of consecutive angles the second of which is acute and the first of which is not.

We are left with the computation of the probability that for one of the points $P_{j}$, the angle at $P_{j}$ is not acute, but the following angle is. This can be done using integrals. But there is a clever argument that reduces the geometric probability to a probability with a finite number of outcomes. The idea is to choose randomly $n-1$ pairs of antipodal points, and then among these to choose the vertices of the polygon. A polygon with one vertex at $P_{j}$ and the other among these points has the desired property exactly when $n-2$ vertices lie on the semicircle to the clockwise side of $P_{j}$ and one vertex on the opposite semicircle. Moreover, the points on the semicircle should include the counterclockwisemost to guarantee that the angle at $P_{j}$ is not acute. Hence there are $n-2$ favorable choices of the total $2^{n-1}$ choices of points from the antipodal pairs. The probability for obtaining a polygon with the desired property is therefore $(n-2) 2^{-n+1}$.

Integrating over all choices of pairs of antipodal points preserves the ratio. The events $j=1,2, \ldots, n$ are independent, so the probability has to be multiplied by $n$. The answer to the problem is therefore $n(n-2) 2^{-n+1}$.

(66th W.L. Putnam Mathematical Competition, 2005, solution by C. Lin) 
\\
\textbf{Topic} :Probability\\
\textbf{Book} :Putnam and Beyond\\
\textbf{Final Answer} :\\\end{document}